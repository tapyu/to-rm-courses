\documentclass{article}

% Language setting
% Replace `english' with e.g. `spanish' to change the document language
\usepackage[english]{babel}

% math
\usepackage{amsfonts}
\newcommand{\trans}{\mathsf{T}}
\newcommand{\hermit}{\mathsf{H}}
\newcommand{\mc}[1]{\ensuremath{\mathcal{#1}}}
\newcommand{\mbb}[1]{\ensuremath{\mathbb{#1}}}
\newcommand{\Natural}{\mathbb{N}}
\newcommand{\Integer}{\mathbb{Z}}
\newcommand{\Irrational}{\mathbb{I}}
\newcommand{\Rational}{\mathbb{Q}}
\newcommand{\Real}{\mathbb{R}}
\newcommand{\Complex}{\mathbb{C}}
\newcommand{\obs}[1]{\textcolor{red}{(#1)}}

% |a| -> absolute value of a, which is a scalar
\newcommand\abs[1]{\left\lvert#1\right\rvert}

% Set page size and margins
% Replace `letterpaper' with`a4paper' for UK/EU standard size
\usepackage[letterpaper,top=2cm,bottom=2cm,left=3cm,right=3cm,marginparwidth=1.75cm]{geometry}

% Useful packages
\usepackage{amsmath}
\usepackage{graphicx}
\usepackage[colorlinks=true, allcolors=blue]{hyperref}

% title packages
\usepackage{authblk}

% biblatex
\usepackage[backend=bibtex, sorting=none, style=numeric-comp, defernumbers=true]{biblatex} % using biblatex
\addbibresource{refs.bib} % add reference file
% for each cited reference create a category named "cited"
\DeclareBibliographyCategory{cited}
\AtEveryCitekey{\addtocategory{cited}{\thefield{entrykey}}}

% strike out texts
\usepackage{soul}

% begin
\title{\textbf{Quick-but-comprehensive guide to Matrix Differentiation}  \vspace{-.3cm}}
\author{Rubem Vasconcelos Pacelli\\
  {\tt rubem.engenharia@gmail.com}}
\affil{Department of Teleinformatics Engineering, Federal University of Ceará.\\Fortaleza, Ceará, Brazil. \vspace{-.5cm}}

\begin{document}
\maketitle
\tableofcontents

\section{Introduction}
Since my Master's degree, I've been struggling with matrix differentiation as I could not find good references that cover it nicely. The bibliographies I found at that time were books from Economics (see references), and they use a \st{weird} distinct notation.

After delving a lot, I finally found a good reference of the \href{https://atmos.washington.edu/~dennis/MatrixCalculus.pdf}{Professor Randal} (honorable mention for \href{https://www.math.uwaterloo.ca/~hwolkowi/matrixcookbook.pdf}{the Matrix Cookbook} too). However, to my surprise, when I tried to apply theses matrix differentiation propositions, I got ``wrong'' answers! Only in my Doctorate, I discovered what was going on: \emph{there are different ways to represent a derivative of a vector}. As far as I know, there are two ways. If you do not select the author's representation, you will end up with the same result, but in a row vector (Jacobian formulation) instead of a column vector (Hessian formulation), and vice-versa. For the cases where the resulting derivative is a matrix, you will get its transpose.

Due to the lack of references and the need to get my own guide, I decided to make this quick guide. I will use the notation that most Engineers might be used to. Moreover, I will only cover the Jacobian formulation since this is the one that matches the derivative results I find in my books. If you are looking for the Hessian formulation, I highly recommend Professor Randal's notes, which use this representation. The unique drawback is that he does not use complex numbers.

Some of the proposals here were collected from class notes, while others I derived by myself. Obviously, this guide may have errors (I hope not). If you find it, feel free to reach me out through email.

\section{Notation and nomeclature}

Let
\begin{align}
    \mathbf{A} = \begin{bmatrix}
        a_{11} & a_{12} & \cdots & a_{1n} \\
        a_{21} & a_{22} & \cdots & a_{2n} \\
        \vdots &        &        & \vdots \\
        a_{m1} & \cdots & \cdots & a_{mn}
    \end{bmatrix} \in \mathbb{C}^{m \times n}
\end{align}
be a complex matrix with dimension equal to \(m \times n\), where \(a_{ij} \in \mathbb{C}\) is its element in the position \((i,j)\). Similarly, a complex vector is defined by
\begin{align}
    \mathbf{x} = \begin{bmatrix}
        x_1 \\
        x_2 \\
        \vdots \\
        x_n
    \end{bmatrix}  \in \mathbb{C}^{n}.
\end{align}

Nonbold Romain and Greek alphabets represent scalars, while bold uppercases and bold lowercases represent matrices and vectors, respectively. The operators \(\cdot^{\trans}\), \(\cdot^{\hermit}\), \(\text{tr}(\cdot)\), \(\textnormal{adj}(\cdot)\), and \(\abs{\cdot}\) represent, respectively, the transpose, the hermitian, the trace, the adjoint, and the determinant (or absolute value when the operand is a scalar).

\subsection{Jacobian formulation}

Consider two vectors \(\mathbf{x} \in \\mathbb{C}^n\) and \(\mathbf{y} \in \mathbb{C}^m\). In the Jacobian formulation (also called Denominator Layout), the partial derivative of each element in \(\mathbf{y}\) by each element in \(\mathbf{x}\) is represented as
\begin{align}
    \left[ \frac{\partial \mathbf{y}}{\partial \mathbf{x}} \right]_{\textnormal{Den}} = 
\end{align}

\section{Matrix Differentiation}

\nocite{*}
% \printbibheading
% \printbibliography[category=cited,heading=subbibliography,title={Cited papers}]
\printbibliography[notcategory=cited] % title={Supplementary papers},heading=subbibliography

\end{document}