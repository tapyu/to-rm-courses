\documentclass{article}
\pagenumbering{gobble}

% redefine \maketitle
\makeatletter % changes the catcode of @ to 11
\def\@maketitle{%
  \newpage
  \null
  \vskip 2em%
  \begin{center}%
  \let \footnote \thanks
    {\LARGE \@title \par}%
    \vskip 1.5em%
    {\large
      \lineskip .5em%
      \begin{tabular}[t]{c}%
        \@author\\
      \end{tabular}\par}%
    \vskip 1em%
    {\large {\tt Version:}\@date}%
  \end{center}%
  \par
  \vskip 1.5em}
  \makeatother % changes the catcode of @ back to 12


%%%%%
% commands
\usepackage{xparse} % make command behave differently depending on the number of arguments (see \diff command)

% items, tables and figs
\usepackage{enumitem}
\usepackage{tabularx}
\usepackage{tikz}
\usepackage{float}
\usepackage{standalone} % place tikz environments or other material in own source files

% math
\usepackage{amsmath}
\usepackage{mathtools} % is an extension package to amsmath
\usepackage{amsfonts}
\usepackage{amssymb}
\usepackage{newtxmath} % for Greek variants (bold, nonitalic, etc...)
\usepackage{bm} % for writing tensor, e.g., $\bm{\mathcal{A}}$
\usepackage{IEEEtrantools}
\usepackage{siunitx} % typesetting units and unitless numbers SI (Système International d’Unités) units
\DeclareMathAlphabet{\mathcal}{OMS}{cmsy}{m}{n} % correct the big-O notation
\SetMathAlphabet{\mathcal}{bold}{OMS}{cmsy}{b}{n}

% newcommands - operators
\DeclareMathOperator*{\argmax}{arg\,max} % argument min
\DeclareMathOperator*{\argmin}{arg\,min} % argument max
\newcommand{\tr}[1]{\ensuremath{\textnormal{tr}\left(#1\right)}} % trace
\newcommand{\adj}[1]{\ensuremath{\textnormal{adj}\left(#1\right)}} % adjugate matrix
\renewcommand{\dim}[1]{\ensuremath{\textnormal{dim}\left(#1\right)}} % dimension of a set
\newcommand{\nullspace}[1]{\ensuremath{\textnormal{N}\left(#1\right)}} % nullspace or kernel of the matrix
\newcommand{\nullity}[1]{\ensuremath{\textnormal{nullity}\left(#1\right)}} % nullity=dim(N(A))
\renewcommand{\span}[1]{\ensuremath{\textnormal{span}\left\{#1\right\}}} % span of a set of vectors
\newcommand{\range}[1]{\ensuremath{\textnormal{C}\left(#1\right)}} % range or column space of a matrix=span(a1,a2, ..., an), where ai is the ith column vector of the matrix A
\newcommand{\rank}[1]{\ensuremath{\textnormal{rank}\left(#1\right)}} % rank = dim(C(A))
\newcommand{\diag}[1]{\ensuremath{\textnormal{diag}\left(#1\right)}} % diagonal operator
\DeclareDocumentCommand{\vec}{om}{\ensuremath{\textnormal{vec}\IfValueT{#1}{_{\textnormal{#1}}}\left(#2\right)}} % vectorize operator, e.g., \vec[]{\mathbf{A}}
\newcommand{\unvec}[1]{\ensuremath{\textnorma/l{unvec}\left(#1\right)}} % unvectorize operator
\newcommand{\inner}[2]{\ensuremath{\langle#1,#2\rangle}} % inner product
\makeatletter % changes the catcode of @ to 11

\DeclarePairedDelimiter\abs{\lvert}{\rvert} % |x| -> absolute value or determinant
\let\oldabs\abs
\def\abs{\@ifstar{\oldabs}{\oldabs*}} % swap the asterist and the nonasterisk behaviors

\DeclarePairedDelimiter\ceil{\lceil}{\rceil} % ⌈x⌉
\let\oldceil\ceil
\def\ceil{\@ifstar{\oldceil}{\oldceil*}}

\DeclarePairedDelimiter\floor{\lfloor}{\rfloor} % ⌊x⌋
\let\oldfloor\floor
\def\floor{\@ifstar{\oldfloor}{\oldfloor*}}

\DeclarePairedDelimiter\norm{\lVert}{\rVert} % ||x|| -> l2-norm, 2-norm, Euclidean norm
\let\oldnorm\norm
\def\norm{\@ifstar{\oldnorm}{\oldnorm*}}

\makeatother % changes the catcode of @ back to 12
\newcommand{\frob}[1]{\ensuremath{\norm{#1}_\textrm{F}}} % Frobenius norm
\newcommand{\eval}[2]{\left.#1\right|_{#2}} % x|_x=a -> evaluation bar
\newcommand{\dom}[1]{\ensuremath{\textnormal{dom}\left(#1\right)}} % domain of the function
\newcommand{\intersection}{\bigcap\limits} % intersection operator
\DeclareDocumentCommand{\diff}{o}{\mathop{}\!\mathrm{d}\IfValueT{#1}{^{#1}}} % derivative(for high-order, use the optional argument)

% newcommands - comments or corrections
\newcommand{\obs}[1]{\textcolor{red}{(#1)}} % comment
\newcommand{\sizecorr}[1]{\makebox[0cm]{\phantom{$\displaystyle #1$}}} % Used to seize the height of equation
\newcommand{\ensureoperation}{\negmedspace {}} % To ensure that a new line symbol is an operation instead of a sign

\begin{document}
\title{List of Symbols}
\maketitle

\section{Font notation}
\begin{table}[htbp]
\begin{tabular}{ll}
  $a,b,c, \dots, A, B, C, \dots$ & Scalars \\
  $\mathbf{a}, \mathbf{b}, \mathbf{c}, \dots$ & Vectors \\
  $\mathbf{A}, \mathbf{B}, \mathbf{C}, \dots$ & Matrices \\
  $\bm{\mathcal{A}}, \bm{\mathcal{B}}, \bm{\mathcal{C}}, \dots$ & Tensors \\
  $A, B, C, \dots, \mathcal{A}, \mathcal{B}, \mathcal{C}, \dots$ &  Sets\\
\end{tabular}
\end{table}

\section{Usual symbols}
\begin{table}[H]
    \begin{tabularx}{\textwidth}{XX}
      \(\boldsymbol{\nabla}f, \mathbf{g}\) & Gradient vector \\
      \(\boldsymbol{\nabla}_{x}f, \mathbf{g}_{x}\) & Gradient vector with respect \(x\)\\
      \(\mathbf{g}\) (or \(\hat{\mathbf{g}}\) if the gradient vector is \(\mathbf{g}\)) & Stochastic approximation of the gradient vector \\
      \(\mathbf{w}, \boldsymbol{\thetaup}\) & Parameters/coefficients/weights vector \\
      \(\boldsymbol{\muup}_x, \mathbf{m}_x\) & Mean vector\\
      \(\hat{\boldsymbol{\muup}}_x, \hat{\mathbf{m}}_x\) & Sample mean vector\\
      \(\mathbf{R}_\mathbf{x}\) & Correlation matrix of \(\mathbf{x}\) \\
      \(R_x(\tau), r_x(\tau)\) & Autocorrelation function of the sinal \(x(t)\) or \(x[n]\)\\
      \(\delta(t)\) & Delta function\\
      \(\delta[n]\) & Kronecker function\\
      \(\hat{R}_x(\tau), \hat{r}_x(\tau)\) & Estimate autocorrelation function of the sinal \(x(t)\) or \(x[n]\)\\
      \(\hat{\mathbf{R}}_\mathbf{x}\) & Sample correlation matrix \\
      \(\mathbf{K}_\mathbf{x}, \mathbf{C}_x\) & Covariance matrix of \(\mathbf{x}\) \\
      \(\hat{\mathbf{K}}_\mathbf{x}, \hat{\mathbf{C}}_x\) & Sample covariance matrix \\
      \(\mathbf{W}, \mathbf{D}\) & Diagonal matrix \\
      \(\mathbf{P}\) & Projection matrix; Permutation matrix \\
      \(\boldsymbol{\Lambdaup}\) & Eigenvalue matrix \\
      \(\boldsymbol{\Sigmaup}\) & Singular value matrix\\
      \(\mathbf{U}\) & Upper matrix; Left singular vectors\\
      \(\mathbf{L}\) & Lower matrix\\
      \(\mathbf{V}\) & Right singular vectors\\
      \(\mathbf{J}\) & Jordan matrix; Jacobian matrix\\
      \(\mathbf{S}\) & Symmetric matrix\\
      \(\mathbf{Q}\) & Orthogonal matrix\\
      \(\mathbf{I}_N\) & \(N\times N\)-dimensional identity matrix\\
      \(\mathbf{0}_{M\times N}\) & \(M\times N\)-dimensional null matrix\\
      \(j\)& \(\sqrt{-1}\)
    \end{tabularx}
\end{table}

\section{Linear Algebra operations}
\begin{table}[H]
    \begin{tabularx}{\textwidth}{XX}
        \(\mathbf{A}^{-1}\) & Inverse matrix\\
        \(\mathbf{A}^+, \mathbf{A}^{\dagger}\) & Moore-Penrose pseudoinverse\\
        \(\mathbf{A}^\mathsf{T}\) & Transpose\\
        \(\mathbf{A}^*\) & Conjugate\\
        \(\mathbf{A}^\mathsf{H}\) & Hermitian\\
        \(\norm{\cdot}\) & \(l_1\) norm, 1-norm, or Manhatan norm\\
        \(\norm{\cdot}, \norm{\cdot}_2\) & \(l_2\) norm, 2-norm, or Euclidean norm\\
        \(\norm{\cdot}_p\) & \(l_p\) norm, \(p\)-norm, or Minkowski norm\\
        \(\norm{\cdot}_\infty\) & \(l_\infty\) norm, \(\infty\)-norm, or Chebyshev norm\\
        \(\frob{\cdot}\)& Frobenius norm \\
        \(\abs{\mathbf{A}}, \textnormal{det}\left( \mathbf{A} \right)\) & Determinant\\
        \(\nullspace{\mathbf{A}}\) & Nullspace (or kernel)\\
        \(\range{\mathbf{A}}\) & Columnspace (or range), i.e., the space \(\span{\mathbf{a}_1,\mathbf{a}_2, \dots, \mathbf{a}_n}\), where \(\mathbf{a}_i\) is the ith column vector of the matrix \(\mathbf{A}\)\\
        \(\diag{\mathbf{a}}, \diag{\mathbf{A}}\) & Diagonalization: a square, diagonal matrix with entries given by the vector \(\mathbf{a}\) or the elements in the diagonal of \(\mathbf{A}\) \\
        \(\vec[]{\mathbf{A}}\) &  Vectorization: stacks the columns of the matrix \(\mathbf{A}\) into a long column vector\\
        \(\vec[d]{\mathbf{A}}\) &  Extracts the diagonal elements of a square matrix and returns them
        in a column vector\\
        \(\vec[l]{\mathbf{A}}\) & Extracts the elements strictly below the main diagonal of a square matrix in a column-wise manner and returns them into a column vector\\
        \(\vec[u]{\mathbf{A}}\) & Extracts the elements strictly above the main diagonal of a square matrix in a column-wise manner and returns them into a column vector\\
        \(\vec[b]{\mathbf{A}}\) & Block vectorization operator: stacks square block matrices of the input into a long block column matrix\\
        \(\span{\mathbf{a}_1, \mathbf{a}_2, \dots, \mathbf{a}_n}\) & Space spanned by the argument vectors\\
        \(\rank{\mathbf{A}}\) & Rank, that is, \(\dim{\span{\mathbf{a}_1,\mathbf{a}_2, \dots, \mathbf{a}_n}} = \dim{\range{\mathbf{A}}}\), where \(\mathbf{a}_i\) is the ith column vector of the matrix \(\mathbf{A}\)\\
        \(\tr{\mathbf{A}}\)& trace\\
        \(\otimes\) & Kronecker product\\
        \(\circ\) & Outer product\\
        \(\odot\) & Hadamard (elementwise) product\\
        \(\diamond\) & Khatri-Rao product\\
        \(\times_n\) & \(n\)-mode product\\
        \(\inner{\cdot}{\cdot}\) & Inner product
    \end{tabularx}
\end{table}
\section{Sets}
\begin{table}[H]
    \begin{tabularx}{\textwidth}{XX}
        \(A \backslash B\) & Set subtraction, i.e., the set containing the elements of \(A\) that are not in \(B\)\\
        \(A \cup B\) & Set of union\\
        \(A \cap B\) & Set of intersection\\
        \(a \in A\)& \(a\) is element of \(A\) \\
        \(a \notin A\)& \(a\) is not element of \(A\) \\
        \(\left\{ 1,2, \dots, n \right\}\) & Discrete set containing the integer elements \(1,2, \dots, n\)\\
        \(\mathbb{R}\) & Set of real numbers\\
        \(\mathbb{C}\)& Set of complex numbers\\
        \(\mathbb{Z}\) & Set of integer number\\
        \(\mathbb{B} = \left\{ 0, 1 \right\}\) & Boolean set (?)\\ % Circuit Complexity and Neural Networks - Ian Parberry; Further Improvements in the Boolean Domain
        \(\mathbb{N}\) & Set of natural numbers\\
        \(\mathbb{K} \in \left\{ \mathbb{R}, \mathbb{C} \right\}\) & ???\\
        \([a, b]\) & Closed interval of a real set from \(a\) to \(b\)\\
        \((a, b)\) & Closed interval of a real set from \(a\) to \(b\)\\
        \([a, b), (a, b]\) & Half-open intervals of a real set from \(a\) to \(b\)\\
    \end{tabularx}
\end{table}

\section{Signals and functions operations and indexing}
\begin{table}[H]
    \begin{tabularx}{\textwidth}{XX}
        \(f: A \rightarrow B\)& \\
        \(f \circ g\) & Composition of the functions \(f\) and \(g\)\\
        \(*\) & Convolution\\
        \(x(t)\) & Continuous-time \(t\)\\
        \(x[n], x[k], x[m], x[i], \dots\) & Discrete-time \(n, k, m, i, \dots\)\\
        \(x(n), x(k), x(m), x(i), \dots\) & Discrete-time \(n, k, m, i, \dots\) (it should be used only if there are no continuous-time signals in the context to avoid ambiguity)\\
        \(\tilde{x}(t)\) or \(\tilde{x}[n]\) & Estimate of \(x(t)\) or \(x[n]\); the Hilbert transform of \(x(t)\) or \(x[n]\)\\
        \(x_I(t)\) or \(x_I[n]\) & Real or in-phase part of \(x(t)\) or \(x[n]\)\\
        \(x_Q(t)\) or \(x_Q[n]\) & Imaginary or quadrature part of \(x(t)\) or \(x[n]\)\\
        \(X(s)\) & Laplace transform of \(x(t)\)\\
        \(X(f)\) & Fourier transform (in linear frequency, \(\unit{\Hz}\)) of \(x(t)\)\\
        \(X(j\omega)\) & Fourier transform (in angular frequency, \(\unit{\radian\per\sec}\)) of \(x(t)\)\\
        \(X(z)\) & Z-transform of \(x[n]\)
    \end{tabularx}
\end{table}

\section{Probability and stochastic processes}
\begin{table}[H]
    \begin{tabularx}{\textwidth}{XX}
        \(E\left[ \cdot \right]\) & Statistical expectation\\
        \(E_u\left[ \cdot \right]\) & Statistical expectation with respect to \(u\)\\
        \(\textnormal{Var}(x)\) & Variance of the random variable \(x\)\\
        \(\textnormal{erfc}(\cdot)\) & Complementary error function\\
        \(P(A)\) & Probability of the event or set \(A\)\\
        \(p(a)\) or \(p(\mathbf{a})\) & Probability density function of the random variable \(a\) or random vector \(\mathbf{a}\)\\
        \(p(x\mid A)\) & Conditional probability density function\\
        \(a\sim P\) & Random variable \(a\) with distribution \(P\)
    \end{tabularx}
\end{table}

\section{General notations}
\begin{table}[H]
    \begin{tabularx}{\textwidth}{XX}
        \(a \wedge b\) & Logical AND of \(a\) and \(b\)\\
        \(a \vee b\) & Logical OR of \(a\) and \(b\)\\
        \(\exists\) & There exists\\
        \(\nexists\) & There does not exist\\
        \(\exists!\) & There exist an unique\\
        \(\forall\) & For all\\
        \(\mid\) & Such that\\
        \(\triangleq\) & Equal by definition\\
        \(\infty\) & Infinity\\
        \(\abs{a}\) & Absolute value of \(a\)\\
        \(\log\) & Base-10 logarithm or decimal logarithm\\
        \(\ln\) & Natual logarithm\\
        \(\textnormal{Re}\left\{ x \right\}\) & Real part of \(x\)\\
        \(\textnormal{Im}\left\{ x \right\}\) & Imaginary part of \(x\)\\
        \(\mathcal{O}(\cdot)\) & big-O notation\\
        \(\ceil{\cdot}\) & Ceiling operation\\
        \(\floor{\cdot}\) & Floor operation\\
        \(\angle\cdot\) & phase (complex argument)
    \end{tabularx}
\end{table}

\section{Abbreviations}
\begin{table}[H]
    \begin{tabularx}{\textwidth}{XX}
        wrt. & With respect to\\
        st. & Subject to\\
        iff. & If and only if
    \end{tabularx}
\end{table}
\end{document}