\documentclass{article}
\pagenumbering{gobble}

% redefine \maketitle
\makeatletter % changes the catcode of @ to 11
\def\@maketitle{%
  \newpage
  \null
  \vskip 2em%
  \begin{center}%
  \let \footnote \thanks
    {\LARGE \@title \par}%
    \vskip 1.5em%
    {\large
      \lineskip .5em%
      \begin{tabular}[t]{c}%
        \@author\\
      \end{tabular}\par}%
    \vskip 1em%
    {\large {\tt Version:}\@date}%
  \end{center}%
  \par
  \vskip 1.5em}
  \makeatother % changes the catcode of @ back to 12


%%%%%
% commands
\usepackage{xparse} % make command behave differently depending on the number of arguments (see \diff command)

% items, tables and figs
\usepackage{enumitem}
\usepackage{xltabular}
\usepackage{tikz}
\usepackage{float}
\usepackage{standalone} % place tikz environments or other material in own source files

% math
\usepackage{amsmath}
\usepackage{mathtools} % is an extension package to amsmath
\usepackage{amsfonts}
\usepackage{amssymb}
% \usepackage{newcomputermodern}
\usepackage{newtxmath} % for Greek variants (bold, nonitalic, etc...)
\usepackage{bm} % for writing tensor, e.g., $\bm{\mathcal{A}}$
\usepackage{IEEEtrantools}
\usepackage{siunitx} % typesetting units and unitless numbers SI (Système International d’Unités) units
\DeclareMathAlphabet{\mathcal}{OMS}{cmsy}{m}{n} % correct the big-O notation
\SetMathAlphabet{\mathcal}{bold}{OMS}{cmsy}{b}{n}

% newcommands - operators
\let\oldemptyset\emptyset
\let\emptyset\varnothing
\DeclareMathOperator*{\argmax}{arg\,max} % argument min
\DeclareMathOperator*{\argmin}{arg\,min} % argument max
\newcommand{\tr}[1]{\ensuremath{\textnormal{tr}\left(#1\right)}} % trace
\newcommand{\adj}[1]{\ensuremath{\textnormal{adj}\left(#1\right)}} % adjugate matrix
\renewcommand{\dim}[1]{\ensuremath{\textnormal{dim}\left(#1\right)}} % dimension of a set
\newcommand{\nullspace}[1]{\ensuremath{\textnormal{N}\left(#1\right)}} % nullspace or kernel of the matrix
\newcommand{\nullity}[1]{\ensuremath{\textnormal{nullity}\left(#1\right)}} % nullity=dim(N(A))
\renewcommand{\span}[1]{\ensuremath{\textnormal{span}\left(#1\right)}} % span of a set of vectors
\newcommand{\range}[1]{\ensuremath{\textnormal{C}\left(#1\right)}} % range or column space of a matrix=span(a1,a2, ..., an), where ai is the ith column vector of the matrix A
\newcommand{\rank}[1]{\ensuremath{\textnormal{rank}\left(#1\right)}} % rank = dim(C(A))
\newcommand{\diag}[1]{\ensuremath{\textnormal{diag}\left(#1\right)}} % diagonal operator
\DeclareDocumentCommand{\vec}{om}{\ensuremath{\textnormal{vec}\IfValueT{#1}{_{\textnormal{#1}}}\left(#2\right)}} % vectorize operator, e.g., \vec[]{\mathbf{A}}
\newcommand{\unvec}[1]{\ensuremath{\textnormal{unvec}\left(#1\right)}} % unvectorize operator
\newcommand{\inner}[2]{\ensuremath{\langle#1,#2\rangle}} % inner product
\makeatletter % changes the catcode of @ to 11

\DeclarePairedDelimiter\abs{\lvert}{\rvert} % |x| -> absolute value or determinant
\let\oldabs\abs
\def\abs{\@ifstar{\oldabs}{\oldabs*}} % swap the asterist and the nonasterisk behaviors

\DeclarePairedDelimiter\ceil{\lceil}{\rceil} % ⌈x⌉
\let\oldceil\ceil
\def\ceil{\@ifstar{\oldceil}{\oldceil*}}

\DeclarePairedDelimiter\floor{\lfloor}{\rfloor} % ⌊x⌋
\let\oldfloor\floor
\def\floor{\@ifstar{\oldfloor}{\oldfloor*}}

\DeclarePairedDelimiter\norm{\lVert}{\rVert} % ||x|| -> l2-norm, 2-norm, Euclidean norm
\let\oldnorm\norm
\def\norm{\@ifstar{\oldnorm}{\oldnorm*}}

\makeatother % changes the catcode of @ back to 12
\newcommand{\frob}[1]{\ensuremath{\norm{#1}_\textrm{F}}} % Frobenius norm
\newcommand{\eval}[2]{\left.#1\right|_{#2}} % x|_x=a -> evaluation bar
\newcommand{\dom}[1]{\ensuremath{\textnormal{dom}\left(#1\right)}} % domain of the function
\newcommand{\intersection}{\bigcap\limits} % intersection operator
\DeclareDocumentCommand{\diff}{o}{\mathop{}\!\mathrm{d}\IfValueT{#1}{^{#1}}} % derivative(for high-order, use the optional argument)

% newcommands - comments or corrections
\newcommand{\obs}[1]{\textcolor{red}{(#1)}} % comment
\newcommand{\sizecorr}[1]{\makebox[0cm]{\phantom{$\displaystyle #1$}}} % Used to seize the height of equation
\newcommand{\ensureoperation}{\negmedspace {}} % To ensure that a new line symbol is an operation instead of a sign

\begin{document}
\title{List of Symbols}
\maketitle

\section{Font notation}
\begin{xltabular}{\textwidth}{XX}
  $a,b,c, \dots, A, B, C, \dots$ & Scalars \\
  $\mathbf{a}, \mathbf{b}, \mathbf{c}, \dots$ & Vectors \\
  $\mathbf{A}, \mathbf{B}, \mathbf{C}, \dots$ & Matrices \\
  $\bm{\mathcal{A}}, \bm{\mathcal{B}}, \bm{\mathcal{C}}, \dots$ & Tensors \\
  $A, B, C, \dots, \mathcal{A}, \mathcal{B}, \mathcal{C}, \dots$ &  Sets\\
\end{xltabular}

\section{Common symbols}
\begin{xltabular}{\textwidth}{XX}
    \(\boldsymbol{\nabla}f, \mathbf{g}\) & Gradient vector \\
    \(\boldsymbol{\nabla}_{x}f, \mathbf{g}_{x}\) & Gradient vector with respect \(x\)\\
    \(\mathbf{g}\) (or \(\hat{\mathbf{g}}\) if the gradient vector is \(\mathbf{g}\)) & Stochastic approximation of the gradient vector \\
    \(J(\cdot)\) & Cost-function or objective function\\
    \(\Lambda(\cdot)\) & Likelihood function\\
    \(\Lambda_l(\cdot)\) & Log-likelihood function\\
    \(\mathcal{O}(\cdot), O(\cdot)\) & big-O notation\\
    \(\boldsymbol{\muup}_x, \mathbf{m}_x\) & Mean vector\\
    \(\hat{\boldsymbol{\muup}}_x, \hat{\mathbf{m}}_x\) & Sample mean vector\\
    \(r_x(\tau), R_x(\tau)\) & Autocorrelation function of the signal \(x(t)\) or \(x[n]\)\\
    \(\hat{r}_x(\tau), \hat{R}_x(\tau)\) & Estimated autocorrelation function of the signal \(x(t)\) or \(x[n]\)\\
    \(\mathbf{R}_\mathbf{x}\) & (Auto)correlation matrix of \(\mathbf{x}\) \\
    \(\hat{\mathbf{R}}_\mathbf{x}\) & Sample (auto)correlation matrix \\
    \(r_{x,d}(\tau), R_{x,d}(\tau)\) & Cross-correlation between \(x[n]\) and \(d[n]\) or \(x(t)\) and \(d(t)\)\\
    \(\hat{r}_{x,d}(\tau), \hat{R}_{x,d}(\tau)\) & Estimated cross-correlation between \(x[n]\) and \(d[n]\) or \(x(t)\) and \(d(t)\)\\
    \(\mathbf{R}_\mathbf{xy}\) & Cross-correlation matrix of \(\mathbf{x}\) and \(\mathbf{y}\)\\
    \(\hat{\mathbf{R}}_\mathbf{xy}\) & Sample cross-correlation matrix of \(\mathbf{R}_\mathbf{xy}\) \\
    \(\mathbf{p}_{\mathbf{x}d}\)& Cross-correlation vector\\
    \(\rho_{x,y}\) & Pearson correlation coefficient between \(x\) and \(y\)\\
    \(\hat{\rho}_{x,y}\) & Estimated Pearson correlation coefficient between \(x\) and \(y\)\\
    \(c_x(\tau), C_x(\tau)\) & Autocovariance function of the signal \(x(t)\) or \(x[n]\)\\
    \(\hat{c}_x(\tau), \hat{C}_x(\tau)\) & Estimated autocovariance function of the signal \(x(t)\) or \(x[n]\)\\
    \(\mathbf{C}_\mathbf{x}, \mathbf{K}_\mathbf{x}, \boldsymbol{\Sigmaup}_\mathbf{x}\) & (Auto)covariance matrix of \(\mathbf{x}\) \\
    \(\hat{\mathbf{C}}_\mathbf{x}, \hat{\mathbf{K}}_\mathbf{x}, \hat{\boldsymbol{\Sigmaup}}_\mathbf{x}\) & Sample (auto)covariance matrix \\
    \(c_{xy}(\tau), C_{xy}(\tau)\) & Cross-covariance function of the signal \(x(t)\) or \(x[n]\)\\
    \(\hat{c}_{xy}(\tau), \hat{C}_{xy}(\tau)\) & Estimated cross-covariance function of the signal \(x(t)\) or \(x[n]\)\\
    \(\mathbf{C}_{\mathbf{xy}}, \mathbf{K}_{\mathbf{xy}}, \boldsymbol{\Sigmaup}_{\mathbf{xy}}\) & Cross-covariance matrix of \(\mathbf{x}\) \\
    \(\hat{\mathbf{C}}_{\mathbf{xy}}, \hat{\mathbf{K}}_{\mathbf{xy}}, \hat{\boldsymbol{\Sigmaup}}_{\mathbf{xy}}\) & Sample cross-covariance matrix \\
    \(\delta(t)\) & Delta function\\
    \(\delta[n]\) & Kronecker function\\
    \(h(t), h[n]\) & Impulse response (continuous and discrete time)\\
    \(\mathbf{C}\) & Cofactor matrix\\
    \(\mathbf{W}, \mathbf{D}\) & Diagonal matrix \\
    \(\mathbf{w}, \boldsymbol{\thetaup}\) & Parameters, coefficients, or weights vector \\
    \(\mathbf{w}_o, \mathbf{w}^{\star}, \boldsymbol{\thetaup}_o, \boldsymbol{\thetaup}^{\star}\) & Optimum value of the parameters, coefficients, or weights vector \\
    \(\mathbf{W}\) & Matrix of the weights \\
    \(\mathbf{P}\) & Projection matrix; Permutation matrix \\
    \(\boldsymbol{\Lambdaup}\) & Eigenvalue matrix \\
    \(\boldsymbol{\Sigmaup}\) & Singular value matrix\\
    \(\mathbf{U}\) & Upper matrix; Left singular vectors\\
    \(\mathbf{L}\) & Lower matrix\\
    \(\mathbf{V}\) & Right singular vectors\\
    \(\mathbf{J}\) & Jordan matrix; Jacobian matrix\\
    \(\mathbf{S}\) & Symmetric matrix\\
    \(\mathbf{Q}\) & Orthogonal matrix\\
    \(\mathbf{I}_N\) & \(N\times N\)-dimensional identity matrix\\
    \(\mathbf{0}_{M\times N}\) & \(M\times N\)-dimensional null matrix\\
    \(\mathbf{0}_{N}\) & \(N\)-dimensional null vector\\
    \(\mathbf{0}\) & Null matrix, vector, or tensor (dimensionality understood by context)\\
    \(\mathbf{1}_{M\times N}\) & \(M\times N\)-dimensional ones matrix\\
    \(\mathbf{1}_{N}\) & \(N\)-dimensional ones vector\\
    \(\mathbf{1}\) & Ones matrix, vector, or tensor (dimensionality understood by context)\\
    \(j\)& \(\sqrt{-1}\)
\end{xltabular}

\section{Linear Algebra operations}
\begin{xltabular}{\textwidth}{XX}
    \(\mathbf{A}^{-1}\) & Inverse matrix\\
    \(\mathbf{A}^+, \mathbf{A}^{\dagger}\) & Moore-Penrose pseudoinverse\\
    \(\mathbf{A}^\top\) & Transpose\\
    \(\mathbf{A}^*\) & Conjugate\\
    \(\mathbf{A}^\mathsf{H}\) & Hermitian\\
    \(\frob{\mathbf{A}}\)& Frobenius norm \\
    \(\norm{\mathbf{A}}\) & Matrix norm\\
    \(\norm{\mathbf{a}}\) & \(l_1\) norm, 1-norm, or Manhatan norm\\
    \(\norm{\mathbf{a}}, \norm{\mathbf{a}}_2\) & \(l_2\) norm, 2-norm, or Euclidean norm\\
    \(\norm{\mathbf{a}}_p\) & \(l_p\) norm, \(p\)-norm, or Minkowski norm\\
    \(\norm{\mathbf{a}}_\infty\) & \(l_\infty\) norm, \(\infty\)-norm, or Chebyshev norm\\
    \(\abs{\mathbf{A}}, \textnormal{det}\left( \mathbf{A} \right)\) & Determinant\\
    \(\diag{\mathbf{a}}, \diag{\mathbf{A}}\) & Diagonalization: a square, diagonal matrix with entries given by the vector \(\mathbf{a}\) or the elements in the diagonal of \(\mathbf{A}\) \\
    \(\vec[]{\mathbf{A}}\) &  Vectorization: stacks the columns of the matrix \(\mathbf{A}\) into a long column vector\\
    \(\vec[d]{\mathbf{A}}\) &  Extracts the diagonal elements of a square matrix and returns them
    in a column vector\\
    \(\vec[l]{\mathbf{A}}\) & Extracts the elements strictly below the main diagonal of a square matrix in a column-wise manner and returns them into a column vector\\
    \(\vec[u]{\mathbf{A}}\) & Extracts the elements strictly above the main diagonal of a square matrix in a column-wise manner and returns them into a column vector\\
    \(\vec[b]{\mathbf{A}}\) & Block vectorization operator: stacks square block matrices of the input into a long block column matrix\\
    \(\unvec{\mathbf{A}}\)& Reshapes a column vector into a matrix\\
    \(\left[ \mathbf{A}, \mathbf{B}, \mathbf{C}, \dots \right]\) & CANDECOMP/PARAFAC (CP) decomposition of the tensor \(\bm{\mathcal{X}}\) from the outer product of column vectors of \(\mathbf{A}\), \(\mathbf{B}\), \(\mathbf{C}, \dots\) \obs{TODO: change the square brackets to the double one by using the commented commands} \\ % \lBrack/\mathbf{A}, \mathbf{B}, \mathbf{C}, \dots\rBrack\) and \usepackage{newcomputermodern}
    \(\left[ \boldsymbol{\lambdaup}; \mathbf{A}, \mathbf{B}, \mathbf{C}, \dots \right]\) & Normalized CANDECOMP/PARAFAC (CP) decomposition of the tensor \(\bm{\mathcal{X}}\) from the outer product of column vectors of \(\mathbf{A}\), \(\mathbf{B}\), \(\mathbf{C}, \dots\) \obs{TODO: change the square brackets to the double one by using the commented commands} \\ % \lBrack/\mathbf{A}, \mathbf{B}, \mathbf{C}, \dots\rBrack\) and \usepackage{newcomputermodern}
    \(\nullspace{\mathbf{A}}, \mathrm{nullspace}(\mathbf{A}), \mathrm{kernel}(\mathbf{A})\) & Nullspace (or kernel)\\
    \(\range{\mathbf{A}}, \mathrm{columnspace}(\mathbf{A}), \mathrm{range}(\mathbf{A})\) & Columnspace (or range), i.e., the space \(\span{\mathbf{a}_1,\mathbf{a}_2, \dots, \mathbf{a}_n}\), where \(\mathbf{a}_i\) is the ith column vector of the matrix \(\mathbf{A}\)\\
    \(\span{\mathbf{a}_1, \mathbf{a}_2, \dots, \mathbf{a}_n}\) & Space spanned by the argument vectors\\
    \(\span{\mathbf{A}}\) & Space spanned by the column vectors of \(\mathbf{A}\)\\
    \(\rank{\mathbf{A}}\) & Rank, that is, \(\dim{\span{\mathbf{a}_1,\mathbf{a}_2, \dots, \mathbf{a}_n}} = \dim{\range{\mathbf{A}}}\), where \(\mathbf{a}_i\) is the ith column vector of the matrix \(\mathbf{A}\)\\
    \(\nullity{\mathbf{A}}\) & Nullity of \(\mathbf{A}\), i.e., \(\dim{\nullspace{\mathbf{A}}}\)\\
    \(\tr{\mathbf{A}}\)& trace\\
    \(\mathbf{a} \perp \mathbf{b}\) & \(\mathbf{a}\) is orthogonal to \(\mathbf{b}\)\\
    \(\mathbf{a} \not\perp \mathbf{b}\) & \(\mathbf{a}\) is not orthogonal to \(\mathbf{b}\)\\
    \(\inner{\mathbf{a}}{\mathbf{b}}\) & Inner product, i.e., \(\mathbf{a}^\top\mathbf{b}\)\\
    \(\mathbf{a} \circ \mathbf{b}\) & Outer product, i.e., \(\mathbf{a}\mathbf{b}^\top\)\\
    \(\otimes\) & Kronecker product\\
    \(\odot\) & Hadamard (elementwise) product\\
    \(\diamond\) & Khatri-Rao product\\
    \(\otimes\) & Kronecker Product\\
    \(\times_n\) & \(n\)-mode product\\
    \(\mathbf{X}_{(n)}\) & \(n\)-mode matricization of the tensor \(\bm{\mathcal{X}}\)\\
    \(\bm{\mathcal{X}} \leq 0\) & Nonnegative tensor\\
    \(\mathbf{a} \preceq_K \mathbf{b}\) & Generalized inequality meaning that \(\mathbf{b}-\mathbf{a}\) belongs to the conic subset \(K\) in the space \(\mathbb{R}^{n}\)\\
    \(\mathbf{a} \prec_K \mathbf{b}\) & Strict generalized inequality meaning that \(\mathbf{b}-\mathbf{a}\) belongs to the interior of the conic subset \(K\) in the space \(\mathbb{R}^{n}\)\\
    \(\mathbf{a} \preceq \mathbf{b}\) & Generalized inequality meaning that \(\mathbf{b}-\mathbf{a}\) belongs to the nonnegative orthant conic subset, \(\mathbb{R}_{+}^{n}\), in the space \(\mathbb{R}^{n}\)\\
    \(\mathbf{a} \prec \mathbf{b}\) & Strict generalized inequality meaning that \(\mathbf{b}-\mathbf{a}\) belongs to the positive orthant conic subset, \(\mathbb{R}_{++}^{n}\), in the space \(\mathbb{R}^{n}\)\\
    \(\mathbf{A} \preceq_K \mathbf{B}\) & Generalized inequality meaning that \(\mathbf{B}-\mathbf{A}\) belongs to the conic subset \(K\) in the space \(\mathcal{S}^{n}\)\\
    \(\mathbf{A} \prec_K \mathbf{B}\) & Strict generalized inequality meaning that \(\mathbf{B}-\mathbf{A}\) belongs to the interior of the conic subset \(K\) in the space \(\mathcal{S}^{n}\)\\
    \(\mathbf{A} \preceq \mathbf{B}\) & Generalized inequality meaning that \(\mathbf{B}-\mathbf{A}\) belongs to the positive semidefinite conic subset, \(\mathcal{S}_{+}^{n}\), in the space \(\mathcal{S}^{n}\)\\
    \(\mathbf{A} \prec \mathbf{B}\) & Strict generalized inequality meaning that \(\mathbf{B}-\mathbf{A}\) belongs to the positive orthant conic subset, \(\mathcal{S}_{++}^{n}\), in the space \(\mathcal{S}^{n}\)
\end{xltabular}

\subsection{Indexing}
\begin{xltabular}[l]{\linewidth}{XX}
    \(x_{i_1,i_2, \dots, i_N}\) & Element in the position \((i_1,i_2, \dots, i_N)\) of the tensor \(\bm{\mathcal{X}}\)\\
    \(\bm{\mathcal{X}}^{(n)}\) & \(n\)th tensor in a nontemporal sequence\\
    \(\left[ \bm{\mathcal{X}} \right]_{i_1,i_2, \dots, i_N}\) & Element \(x_{i_1,i_2, \dots, i_N}\)\\
    \(\mathbf{x}_{j}, \mathbf{x}_{:j}\) & \(j\)th column of the matrix \(X\)\\
    \(\mathbf{x}_{j:}\) & \(j\)th row of the matrix \(X\)\\
    \(\mathbf{x}_{i_1,\dots,i_{j-1}, :, i_{j+1},\dots, i_N}\) & Mode-\(j\) fiber of the tensor \(\bm{\mathcal{X}}\)\\
    \(\mathbf{x}_{:,i_2,i_3}\) & Column fiber (mode-\(1\) fiber) of the thrid-order tensor \(\bm{\mathcal{X}}\)\\
    \(\mathbf{x}_{i_1,:,i_3}\) & Row fiber (mode-\(2\) fiber) of the thrid-order tensor \(\bm{\mathcal{X}}\)\\
    \(\mathbf{x}_{i_1,i_2,:}\) & Tube fiber (mode-\(3\) fiber) of the thrid-order tensor \(\bm{\mathcal{X}}\)\\
    \(\mathbf{X}_{i_1,:,:}\) & Horizontal slice of the thrid-order tensor \(\bm{\mathcal{X}}\)\\
    \(\mathbf{X}_{:,i_2,:}\) & Lateral slices slice of the thrid-order tensor \(\bm{\mathcal{X}}\)\\
    \(\mathbf{X}_{i_3}, \mathbf{X}_{:,:,i_3}\) & Frontal slices slice of the thrid-order tensor \(\bm{\mathcal{X}}\)
\end{xltabular}


\section{Sets}
\begin{xltabular}{\textwidth}{XX}
    \(A \setminus B\) & Set subtraction, i.e., the set containing the elements of \(A\) that are not in \(B\)\\
    \(A \cup B\) & Set of union\\
    \(A \cap B\) & Set of intersection\\
    \(A \times B\) & Cartesian product\\
    \(A \oplus B\) & Direct sum, e.g., \(\range{A^{\top}} \oplus \range{A^{\top}}^{\perp} = \mathbb{R}^{n}\)\\
    \(A^{\perp}\) & Orthogonal complement\\
    \(A^{c}\) & Complement\\
    \(\#A\) & Cardinality\\
    \(a \in A\)& \(a\) is element of \(A\) \\
    \(a \notin A\)& \(a\) is not element of \(A\) \\
    \(\left\{ 1,2, \dots, n \right\}\) & Discrete set containing the integer elements \(1,2, \dots, n\)\\
    \(\mathbb{R}\) & Set of real numbers\\
    \(\mathbb{C}\)& Set of complex numbers\\
    \(\mathbb{Z}\) & Set of integer number\\
    \(\mathbb{B} = \left\{ 0, 1 \right\}\) & Boolean set\\ % Circuit Complexity and Neural Networks - Ian Parberry; Further Improvements in the Boolean Domain
    \(\emptyset\) & Empty set\\
    \(\mathbb{N}\) & Set of natural numbers\\
    \(\mathbb{K} \in \left\{ \mathbb{R}, \mathbb{C} \right\}\) & ???\\
    \(\mathbb{K}^{I_1\times I_2 \times \dots \times I_N}\) & \(I_1\times I_2 \times \dots \times I_N\)-dimensional real (or complex) space\\
    \(\mathbb{K}_{+}\) & Nonnegative real (or complex) space\\
    \(\mathbb{K}_{++}\) & Positive real (or complex) space, i.e., \(\mathbb{K}_{++} = \mathbb{K}_{+}\setminus\left\{ \mathbf{0} \right\}\)\\
    \(\mathbb{S}^{n}, \mathcal{S}^{n}\) & Conic set of the symmetric matrices in \(\mathbb{R}^{n\times n}\)\\
    \(\mathbb{S}_{+}^{n}, \mathcal{S}_{+}^{n}\) & Conic set of the symmetric positive semidefinite matrices in \(\mathbb{R}^{n\times n}\)\\
    \(\mathbb{S}_{++}^{n}, \mathcal{S}_{++}^{n}\) & Conic set of the symmetric positive definite matrices in \(\mathbb{R}^{n\times n}\), i.e., \(\mathbb{S}_{++}^{n} = \mathbb{S}_{+}^{n}\setminus \left\{ \mathbf{0} \right\}\)\\
    \(\mathbb{H}^{n}\) & Set of all hermitian matrices in \(\mathbb{C}^{n\times n}\)\\
    \([a, b]\) & Closed interval of a real set from \(a\) to \(b\)\\
    \((a, b)\) & Opened interval of a real set from \(a\) to \(b\)\\
    \([a, b), (a, b]\) & Half-opened intervals of a real set from \(a\) to \(b\)\\
\end{xltabular}

\section{Signals and functions operations and indexing}
\begin{xltabular}{\textwidth}{XX}
    \(f: A \rightarrow B\)& A function \(f\) whose domain is \(A\) and codomain is \(B\)\\
    \(f^{\left( n \right)}\) & \(n\)th derivative of the function \(f\)\\
    \(f \circ g\) & Composition of the functions \(f\) and \(g\)\\
    \(\underset{\mathbf{y} \in \mathcal{A}}{\textnormal{inf }} g(\mathbf{x},\mathbf{y})\) & Infimum\\
    \(\underset{\mathbf{y} \in \mathcal{A}}{\textnormal{sup }} g(\mathbf{x},\mathbf{y})\) & Supremum\\
    \(*\) & Convolution\\
    \(x(t)\) & Continuous-time \(t\)\\
    \(x[n], x[k], x[m], x[i], \dots\) & Discrete-time \(n, k, m, i, \dots\)\\
    \(x(n), x(k), x(m), x(i), \dots\) & Discrete-time \(n, k, m, i, \dots\) (it should be used only if there are no continuous-time signals in the context to avoid ambiguity)\\
    \(\tilde{x}(t)\) or \(\tilde{x}[n]\) & Estimate of \(x(t)\) or \(x[n]\); the Hilbert transform of \(x(t)\) or \(x[n]\)\\
    \(x_I(t)\) or \(x_I[n]\) & Real or in-phase part of \(x(t)\) or \(x[n]\)\\
    \(x_Q(t)\) or \(x_Q[n]\) & Imaginary or quadrature part of \(x(t)\) or \(x[n]\)\\
    \(X(s)\) & Laplace transform of \(x(t)\)\\
    \(X(f)\) & Fourier transform (in linear frequency, \(\unit{\Hz}\)) of \(x(t)\)\\
    \(X(j\omega)\) & Fourier transform (in angular frequency, \(\unit{\radian\per\sec}\)) of \(x(t)\)\\
    \(S_x(f)\) & Power spectral density of \(x(t)\) in linear frequency\\
    \(S_x(j\omega)\) & Power spectral density of \(x(t)\) in angular frequency\\
    \(X(z)\) & Z-transform of \(x[n]\)  
\end{xltabular}

\section{Probability and stochastic processes}
\begin{xltabular}{\textwidth}{XX}
    \(E\left[ \cdot \right]\) & Statistical expectation\\
    \(E_u\left[ \cdot \right]\) & Statistical expectation with respect to \(u\)\\
    \(\textnormal{var}(x)\) & Variance of the random variable \(x\)\\
    \(\textnormal{erfc}(\cdot)\) & Complementary error function\\
    \(P(A)\) & Probability of the event or set \(A\)\\
    \(p(\cdot)\) & Probability density function\\
    \(p(x\mid A)\) & Conditional probability density function\\
    \(a\sim P\) & Random variable \(a\) with distribution \(P\)\\
    \(\mathcal{N}(\mu, \sigma^2)\) & Gaussian distribution of a random variable with mean \(\mu\) and variance \(\sigma^{2}\)\\
    \(\mathcal{CN}(\mu, \sigma^2)\) & Complex Gaussian distribution of a random variable with mean \(\mu\) and variance \(\sigma^{2}\)\\
    \(\mathcal{N}(\boldsymbol{\muup}, \boldsymbol{\Sigmaup})\) & Gaussian distribution of a vector random variable with mean \(\boldsymbol{\muup}\) and covariance matrix \(\boldsymbol{\Sigmaup}\)\\
    \(\mathcal{CN}(\boldsymbol{\muup}, \boldsymbol{\Sigmaup})\) & Complex Gaussian distribution of a vector random variable with mean \(\boldsymbol{\muup}\) and covariance matrix \(\boldsymbol{\Sigmaup}\)\\
    \(\mathcal{U}(a,b)\) & Uniform distribution from \(a\) to \(b\)
\end{xltabular}

\section{General notations}
\begin{xltabular}{\textwidth}{XX}
    \(a \wedge b\) & Logical AND of \(a\) and \(b\)\\
    \(a \vee b\) & Logical OR of \(a\) and \(b\)\\
    \(\lnot a\) & Logical negation of \(a\)\\
    \(\exists\) & There exists\\
    \(\nexists\) & There does not exist\\
    \(\exists!\) & There exist an unique\\
    \(\forall\) & For all\\
    \(\mid\) & Such that\\
    \(\therefore\) & Therefore\\
    \(\iff\) & Logical equivalence\\
    \(\triangleq\) & Equal by definition\\
    \(\neq\) & Not equal\\
    \(\infty\) & Infinity\\
    \(\abs{a}\) & Absolute value of \(a\)\\
    \(\log\) & Base-10 logarithm or decimal logarithm\\
    \(\ln\) & Natual logarithm\\
    \(\textnormal{Re}\left\{ x \right\}\) & Real part of \(x\)\\
    \(\textnormal{Im}\left\{ x \right\}\) & Imaginary part of \(x\)\\
    \(\ceil{\cdot}\) & Ceiling operation\\
    \(\floor{\cdot}\) & Floor operation\\
    \(\angle\cdot\) & phase (complex argument)\\
    \(x\;\mathrm{mod}\;y\) & Remainder, i.e., \(x-y\floor{x/y}\)\\
    \(\mathrm{frac}\left(x\right)\) & Fractional part, i.e., \(x\;\mathrm{mod}\;1\)
\end{xltabular}

\section{Abbreviations}
    \begin{xltabular}{\textwidth}{XX}
        wrt. & With respect to\\
        st. & Subject to\\
        iff. & If and only if\\
        EVD & Eigenvalue decomposition, or eigendecomposition\\
        SVD & Singular value decomposition\\
        CP & CANDECOMP/PARAFAC\\
    \end{xltabular}
\end{document}