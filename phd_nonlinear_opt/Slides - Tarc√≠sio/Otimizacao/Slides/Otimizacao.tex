\documentclass[brazil,xcolor={svgnames},10pt,aspectratio=1610]{beamer}
\mode<handout>{
  % No modo handout o beamer vai colocar mais do que um slide por página. Dá
  % para fazer mudanças como mudar o tema ou uma cor também.
  \usepackage{pgfpages}
  \pgfpagesuselayout{2 on 1}[a4paper,border shrink=5mm]
}

\usepackage[brazil]{babel}
\usepackage[none]{hyphenat}

% Disable beamer solution environment to support exsheets
\let\solution\relax
\usepackage{exsheets}
\SetupExSheets{headings-format={\bfseries},counter-format={:},headings=runin,solution/print=true,no-skip-below=true}

\usepackage{pifont}
\usepackage[default,scaled=0.95]{sourcesanspro}
\usepackage[scaled=0.95]{inconsolata}
\usepackage{fontawesome}
%%% FVExtra bug fix %%%
\makeatletter
\def\mdseries@tt{m}
\makeatother
%%%%%%%%%%%%%%%%%%%%%%%
\usepackage[utf8x]{inputenc}
\usepackage[T1]{fontenc}

\usepackage{newtxmath}
\usepackage{mathtools}
\usepackage{icomma}
\usepackage{centernot}

\usepackage{booktabs}
\usepackage{tabularkv}
\usepackage{tabularx}
\newcolumntype{C}{>{\centering\arraybackslash}X}
\newcolumntype{R}{>{\flushright\arraybackslash}X}
\newcolumntype{L}{>{\flushleft\arraybackslash}X}
\usepackage{colortbl}

\usepackage{multicol}

\usepackage{calc}

\usepackage[ddmmyy]{datetime}
\ddmmyyyydate

\usepackage{calc}

\hypersetup{
  pdfauthor = {Tarcisio F. Maciel}, %
  pdftitle = {Métodos numéricos para engenharias},%
  pdfsubject = {Disciplina de métodos numéricos para engenharias do CT-UFC},%
  pdfkeywords = {métodos numéricos, programação computacional},%
  colorlinks=false,%
}

\usepackage{graphicx}
\graphicspath{{.}{./Figs/}}
\usepackage{tikz}
\usetikzlibrary{arrows}
\usetikzlibrary{positioning}
\usetikzlibrary{shapes.misc}
\usetikzlibrary{shapes.symbols}
\usetikzlibrary{shapes.geometric}
\usetikzlibrary{circuits.ee.IEC}
\usepackage{pgf}

%%%%%%%%%%%%%%%%%%%%%%%%%%%%%%%%%
\usepackage[cache=true,newfloat=true]{minted}
% Run the command "pygmentize -L styles" to see available styles
\setminted{fontseries=b}

% C++
\newminted{cpp}{autogobble}  % Define ambiente 'cppcode' para c++
\newmintedfile[inputmintedcpp]{c++}{}  % Define versão customizada de inputminted para c++
\newmintinline[cppinline]{c++}{}  % Define nova versão customizada do
% mintinline chamada 'cppinline'

% Bash
\newminted{bash}{autogobble}  % Define ambiente 'bashcode' para bash
\newmintedfile[inputmintedbash]{bash}{}  % Define versão customizada de inputminted para bash
\newmintinline[bashinline]{bash}{}  % Define nova versão customizada do
% mintinline chamada 'bashinline'

% Octave
\newminted{octave}{autogobble}  % Define ambiente 'cppcode' para c++
\newmintedfile[inputmintedoctave]{octave}{}  % Define versão customizada de inputminted para c++
\newmintinline[octaveinline]{octave}{}  % Define nova versão customizada do
% mintinline chamada 'octaveinline'

% \usemintedstyle{emacs}
\SetupFloatingEnvironment{listing}{name=Listagem} % para newfloat=true no minted
%%%%%%%%%%%%%%%%%%%%%%%%%%%%%%%%%

%%%%% CAPTION / SUBFIG / BEAMER BUG FIX
\makeatletter
\let\@@magyar@captionfix\relax
\makeatother
\usepackage{caption}
\captionsetup[listing]{belowskip=0pt,aboveskip=0pt}  % This will impact minted with the newfloat option
\renewcommand\theFancyVerbLine{\tiny\arabic{FancyVerbLine}}
%%%%%%%%%%%%%%%%%%%%%%%%%%%%%%%%%

\usepackage{hyperref}
\hypersetup{
  pdfauthor = {Tarcisio Ferreira Maciel},%
  pdftitle = {Otimização não-linear},%
  pdfsubject = {Disciplina de otimização não-linear para os cursos do PPGETI},%
  pdfkeywords = {programação matemática, otimização linear, otimização não-linear},%
}

\usepackage{graphicx}
\graphicspath{{.}{./Figs/}}

\usepackage{tikz}
\usetikzlibrary{arrows}
\usetikzlibrary{positioning}
\usetikzlibrary{shapes.misc}
\usetikzlibrary{shapes.symbols}
\usetikzlibrary{shapes.geometric}
\usetikzlibrary{shapes.gates.logic.US}
\usetikzlibrary{fit}
\usetikzlibrary{circuits.ee.IEC}

\usetikzlibrary{external}
%\tikzexternalize
\tikzsetexternalprefix{Figs/}

\usepackage{pgf}
\usepackage{pgfplots}
\usepgfplotslibrary{patchplots}

%%%%%%%%%%%%%%%%%%%%%%%%%%%%%%

% xxxxxxxxxxRedefine o template para a "partpage" xxxxxxxxxxxxxxxxxxxxxxxxx
%\setbeamertemplate{part page}{
%  \begin{beamercolorbox}[sep=8pt,center,wd=\textwidth]{part title}
%    \usebeamerfont{part title}\insertpart\par
%  \end{beamercolorbox}
%}
% xxxxxxxxxxxxxxxxxxxxxxxxxxxxxxxxxxxxxxxxxxxxxxxxxxxxxxxxxxxxxxxxxxxxxxxxx

\usecolortheme{structure}
\useoutertheme{infolines}
\useinnertheme[shadow]{rounded}
\usefonttheme{structurebold}
\usefonttheme[onlymath]{serif}

%%% Printing
%\setbeamercolor{structure}{bg=White,fg=Black}
%\setbeamercolor{alerted text}{fg=Black}

%%% Screen
\setbeamercolor{structure}{bg=White,fg=DarkGreen}
\setbeamercolor{palette primary}{fg=Black,bg=structure.bg}
\setbeamercolor{palette secondary}{fg=Black,bg=structure.fg!30!White}
\setbeamercolor{palette tertiary}{fg=structure.bg,bg=structure.fg!90!White}
\setbeamercolor{title}{fg=structure.fg,bg=structure.bg}
\setbeamercolor{part page}{fg=structure.bg,bg=structure.fg}
\setbeamercolor{frametitle}{fg=structure.bg,bg=structure.fg}
\setbeamercolor{block title}{fg=structure.bg,bg=structure.fg}
\setbeamercolor{block body}{fg=Black,bg=structure.fg!05!White}
\setbeamercolor{alerted text}{fg=Red}

\setbeamerfont{part page}{size=\Large,series=\bfseries}
\setbeamerfont{title}{size=\Large,series=\bfseries}
\setbeamerfont{frametitle}{size=\large,series=\bfseries}
\setbeamerfont{block title}{size=\small,series=\bfseries}
\setbeamerfont{block body}{size=\footnotesize}
\setbeamerfont{section in head/foot}{series=\bfseries,size=\tiny}
\setbeamerfont{subsection in head/foot}{series=\bfseries,size=\tiny}
\setbeamerfont{institute in head/foot}{series=\bfseries,size=\tiny}
\setbeamerfont{author in head/foot}{series=\bfseries,size=\tiny}
\setbeamerfont{date in head/foot}{series=\bfseries,size=\tiny}
\setbeamerfont{footnote}{size=\tiny}
\setbeamerfont{bibliography entry author}{size=\scriptsize,series=\bfseries}
\setbeamerfont{bibliography entry title}{size=\scriptsize,series=\bfseries}
\setbeamerfont{bibliography entry journal}{size=\scriptsize,series=\bfseries}
\setbeamerfont{bibliography entry note}{size=\scriptsize,series=\bfseries}

\setbeamertemplate{bibliography item}[text]
\setbeamertemplate{theorems}[numbered]
\newtheorem{teorema}[theorem]{Teorema}
\newtheorem{propriedade}[theorem]{Propriedade}
\newtheorem{corolario}[theorem]{Corolário}

\setbeamertemplate{frametitle continuation}{(\insertcontinuationcount)}
\setbeamertemplate{navigation symbols}{}



\setlength{\leftmargini}{1em}
\setlength{\leftmarginii}{1em}
\setlength{\leftmarginiii}{1em}

\setlength{\abovedisplayskip}{-\baselineskip}

\newlength{\AuxWidth}

% Math stuff
% Alphabets (new fonts)
\DeclareMathAlphabet{\mathppl}{T1}{ppl}{m}{it}
\DeclareMathAlphabet{\mathphv}{T1}{phv}{m}{it}
\DeclareMathAlphabet{\mathpzc}{T1}{pzc}{m}{it}
% Operators
\DeclareMathOperator{\abs}{abs}
\DeclareMathOperator{\blockdiag}{blockdiag}
\DeclareMathOperator{\card}{card}
\DeclareMathOperator{\conv}{conv}
\DeclareMathOperator{\diag}{diag}
\DeclareMathOperator{\rank}{rank}
\DeclareMathOperator{\re}{Re}
\DeclareMathOperator{\subto}{\text{s.t.: }}
\DeclareMathOperator{\SubTo}{\text{subject to: }}
\DeclareMathOperator{\Sujeito}{\text{sujeito a}}
\DeclareMathOperator{\Vect}{vec}
\DeclareMathOperator{\tr}{tr}
\DeclareMathOperator{\adj}{adj}
\DeclareMathOperator{\nullity}{nullity}
\DeclareMathOperator{\sen}{sen}
\DeclareMathOperator{\dom}{dom}
\DeclareMathOperator{\epi}{epi}
\DeclareMathOperator{\hypo}{hypo}
\DeclareMathOperator{\minimize}{\text{minimize}}
\DeclareMathOperator{\maximize}{\text{maximize}}
\DeclareMathOperator{\setspan}{span}
% Paired delimiters
\DeclarePairedDelimiterX{\Abs}[1]{\lvert}{\rvert}{#1}
\DeclarePairedDelimiterX{\Norm}[1]{\lVert}{\rVert}{#1}
\DeclarePairedDelimiterX{\Floor}[1]{\lfloor}{\rfloor}{#1}
\DeclarePairedDelimiterX{\Ceil}[1]{\lceil}{\rceil}{#1}
\DeclarePairedDelimiterX{\InnerProd}[2]{\langle}{\rangle}{{#1},{#2}}
\DeclarePairedDelimiterX{\Round}[1]{\lceil}{\rfloor}{#1}
% Commands
\newcommand{\ArgMaxMin}[2]{\underset{#1}{\arg\max\!.\!\min\!.\!}\left\{#2\right\}}
\newcommand{\ArgMax}[2]{\underset{#1}{\arg\max\!.\!}\left\{#2\right\}}
\newcommand{\ArgMinMax}[2]{\underset{#1}{\arg\min\!.\!\max\!.\!}\left\{#2\right\}}
\newcommand{\ArgMin}[2]{\underset{#1}{\arg\min\!.\!}\left\{#2\right\}}
\newcommand{\Adj}[1]{\adj\left( #1 \right)}
\newcommand{\Avg}[1]{\overline{#1}}
\newcommand{\Card}[1]{\card\left\{#1\right\}}
\newcommand{\Conv}[1]{\conv\left\{#1\right\}}
\newcommand{\Conj}[1]{{#1}^*}
\newcommand{\Cos}[1]{\cos\left( #1 \right)}
\newcommand{\Deg}[1]{\deg\left(#1\right)}
\newcommand{\Det}[1]{\det\left(#1\right)}
\newcommand{\Diag}[1]{\mathcal{D}\left\{ #1 \right\}}
\newcommand{\Dim}[1]{\dim\left(#1 \right)}
\newcommand{\Elem}[2]{\left[{#1}\right]_{#2}}
\newcommand{\Exp}[1]{\text{e}^{#1}}
\newcommand{\Field}[1]{\mathbb{\uppercase{#1}}}
\newcommand{\First}{1^{\text{st}}}
\newcommand{\Herm}[1]{{#1}^{\mathrm{H}}}
\newcommand{\HInv}[1]{{#1}^{-\mathrm{H}}}
\newcommand{\Inv}[1]{{#1}^{-1}}
\newcommand{\TInv}[1]{{#1}^{-\mathrm{T}}}
\newcommand{\Lagrange}{\calL}
\newcommand{\LogTen}{\log_{10}}
\newcommand{\LogTwo}{\log_{2}}
\newcommand{\Max}[1]{\max\left\{ #1 \right\}}
\newcommand{\Maximize}[2]{\underset{#1}{\maximize}\left\{#2\right\}}
\newcommand{\Mean}[1]{\mathcal{E}\left\{ #1 \right\}}
\newcommand{\Min}[1]{\min\left\{ #1 \right\}}
\newcommand{\Minimize}[2]{\underset{#1}{\minimize}\;#2}
\newcommand{\Mod}[1]{\left\vert #1 \right\vert}
\newcommand{\MOD}[1]{\vert #1 \vert}
\newcommand{\Mt}[1]{\boldsymbol{#1}}
% \newcommand{\Norm}[1]{\left\Vert #1 \right\Vert}
\newcommand{\NormF}[1]{\Norm{#1}_{\mathrm{F}}}
\newcommand{\NormInf}[1]{\Norm{#1}_{\infty}}
\newcommand{\NormOne}[1]{\Norm{#1}_{1}}
\newcommand{\NormP}[1]{\Norm{#1}_{p}}
\newcommand{\NormTwo}[1]{\Norm{#1}_{2}}
\newcommand{\NORM}[1]{\Vert #1 \Vert}
\newcommand{\NORMF}[1]{\NORM{#1}_{\mathrm{F}}}
\newcommand{\NORMINF}[1]{\NORM{#1}_{\infty}}
\newcommand{\NORMONE}[1]{\NORM{#1}_{1}}
\newcommand{\NORMP}[1]{\NORM{#1}_{\text{p}}}
\newcommand{\NORMTWO}[1]{\NORM{#1}_{2}}
\newcommand{\Null}[1]{\calN\left(#1\right)}
\newcommand{\Nullity}[1]{\nullity\left(#1\right)}
\newcommand{\Order}[1]{\mathcal{O}\!\left(#1\right)}
\newcommand{\Ord}[1]{{#1}^{\text{th}}}
\newcommand{\PInv}[1]{{#1}^{\dagger}}
\newcommand{\Range}[1]{\calR\left(#1\right)}
\newcommand{\Rank}[1]{\rank\left(#1\right)}
\newcommand{\Real}[1]{\re\left\{#1\right\}}
\newcommand{\Set}[1]{\mathcal{\uppercase{#1}}}
\newcommand{\Second}{2^{\text{nd}}}
\newcommand{\Sin}[1]{\sin\left( #1 \right)}
\newcommand{\Span}[1]{\setspan\left\{ #1 \right\}}
\newcommand{\Third}{3^{\text{rd}}}
\newcommand{\Trace}[1]{\tr\left(#1\right)}
\newcommand{\Transp}[1]{#1^{\mathrm{T}}}
\newcommand{\Vector}[1]{\Vect\left\{#1\right\}}
\newcommand{\Vt}[1]{\boldsymbol{\lowercase{#1}}}
% Matrices
\newcommand{\mtA}{\Mt{A}}
\newcommand{\mtB}{\Mt{B}}
\newcommand{\mtC}{\Mt{C}}
\newcommand{\mtD}{\Mt{D}}
\newcommand{\mtE}{\Mt{E}}
\newcommand{\mtF}{\Mt{F}}
\newcommand{\mtG}{\Mt{G}}
\newcommand{\mtH}{\Mt{H}}
\newcommand{\mtI}{\Mt{I}}
\newcommand{\mtJ}{\Mt{J}}
\newcommand{\mtK}{\Mt{K}}
\newcommand{\mtL}{\Mt{L}}
\newcommand{\mtM}{\Mt{M}}
\newcommand{\mtN}{\Mt{N}}
\newcommand{\mtO}{\Mt{O}}
\newcommand{\mtP}{\Mt{P}}
\newcommand{\mtQ}{\Mt{Q}}
\newcommand{\mtR}{\Mt{R}}
\newcommand{\mtS}{\Mt{S}}
\newcommand{\mtT}{\Mt{T}}
\newcommand{\mtU}{\Mt{U}}
\newcommand{\mtV}{\Mt{V}}
\newcommand{\mtW}{\Mt{W}}
\newcommand{\mtX}{\Mt{X}}
\newcommand{\mtY}{\Mt{Y}}
\newcommand{\mtZ}{\Mt{Z}}
% Transposed matrices
\newcommand{\mtAt}{\Transp{\mtA}}
\newcommand{\mtBt}{\Transp{\mtB}}
\newcommand{\mtCt}{\Transp{\mtC}}
\newcommand{\mtDt}{\Transp{\mtD}}
\newcommand{\mtEt}{\Transp{\mtE}}
\newcommand{\mtFt}{\Transp{\mtF}}
\newcommand{\mtGt}{\Transp{\mtG}}
\newcommand{\mtHt}{\Transp{\mtH}}
\newcommand{\mtIt}{\Transp{\mtI}}
\newcommand{\mtJt}{\Transp{\mtJ}}
\newcommand{\mtKt}{\Transp{\mtK}}
\newcommand{\mtLt}{\Transp{\mtL}}
\newcommand{\mtMt}{\Transp{\mtM}}
\newcommand{\mtNt}{\Transp{\mtN}}
\newcommand{\mtOt}{\Transp{\mtP}}
\newcommand{\mtPt}{\Transp{\mtP}}
\newcommand{\mtQt}{\Transp{\mtQ}}
\newcommand{\mtRt}{\Transp{\mtR}}
\newcommand{\mtSt}{\Transp{\mtS}}
\newcommand{\mtTt}{\Transp{\mtT}}
\newcommand{\mtUt}{\Transp{\mtU}}
\newcommand{\mtVt}{\Transp{\mtV}}
\newcommand{\mtWt}{\Transp{\mtW}}
\newcommand{\mtXt}{\Transp{\mtX}}
\newcommand{\mtYt}{\Transp{\mtY}}
\newcommand{\mtZt}{\Transp{\mtZ}}
% Hermitian matrices
\newcommand{\mtAh}{\Herm{\mtA}}
\newcommand{\mtBh}{\Herm{\mtB}}
\newcommand{\mtCh}{\Herm{\mtC}}
\newcommand{\mtDh}{\Herm{\mtD}}
\newcommand{\mtEh}{\Herm{\mtE}}
\newcommand{\mtFh}{\Herm{\mtF}}
\newcommand{\mtGh}{\Herm{\mtG}}
\newcommand{\mtHh}{\Herm{\mtH}}
\newcommand{\mtIh}{\Herm{\mtI}}
\newcommand{\mtJh}{\Herm{\mtJ}}
\newcommand{\mtKh}{\Herm{\mtK}}
\newcommand{\mtLh}{\Herm{\mtL}}
\newcommand{\mtMh}{\Herm{\mtM}}
\newcommand{\mtNh}{\Herm{\mtN}}
\newcommand{\mtOh}{\Herm{\mtP}}
\newcommand{\mtPh}{\Herm{\mtP}}
\newcommand{\mtQh}{\Herm{\mtQ}}
\newcommand{\mtRh}{\Herm{\mtR}}
\newcommand{\mtSh}{\Herm{\mtS}}
\newcommand{\mtTh}{\Herm{\mtT}}
\newcommand{\mtUh}{\Herm{\mtU}}
\newcommand{\mtVh}{\Herm{\mtV}}
\newcommand{\mtWh}{\Herm{\mtW}}
\newcommand{\mtXh}{\Herm{\mtX}}
\newcommand{\mtYh}{\Herm{\mtY}}
\newcommand{\mtZh}{\Herm{\mtZ}}
% Inverse matrices
\newcommand{\mtAi}{\Inv{\mtA}}
\newcommand{\mtBi}{\Inv{\mtB}}
\newcommand{\mtCi}{\Inv{\mtC}}
\newcommand{\mtDi}{\Inv{\mtD}}
\newcommand{\mtEi}{\Inv{\mtE}}
\newcommand{\mtFi}{\Inv{\mtF}}
\newcommand{\mtGi}{\Inv{\mtG}}
\newcommand{\mtHi}{\Inv{\mtH}}
\newcommand{\mtIi}{\Inv{\mtI}}
\newcommand{\mtJi}{\Inv{\mtJ}}
\newcommand{\mtKi}{\Inv{\mtK}}
\newcommand{\mtLi}{\Inv{\mtL}}
\newcommand{\mtMi}{\Inv{\mtM}}
\newcommand{\mtNi}{\Inv{\mtN}}
\newcommand{\mtOi}{\Inv{\mtP}}
\newcommand{\mtPi}{\Inv{\mtP}}
\newcommand{\mtQi}{\Inv{\mtQ}}
\newcommand{\mtRi}{\Inv{\mtR}}
\newcommand{\mtSi}{\Inv{\mtS}}
\newcommand{\mtTi}{\Inv{\mtT}}
\newcommand{\mtUi}{\Inv{\mtU}}
\newcommand{\mtVi}{\Inv{\mtV}}
\newcommand{\mtWi}{\Inv{\mtW}}
% \newcommand{\mtXi}{\Inv{\mtX}} % Conflicts with the greek letter matrix \mtXi
\newcommand{\mtYi}{\Inv{\mtY}}
\newcommand{\mtZi}{\Inv{\mtZ}}
% Bar matrices
\newcommand{\mtAb}{\bar{\mtA}}
\newcommand{\mtBb}{\bar{\mtB}}
\newcommand{\mtCb}{\bar{\mtC}}
\newcommand{\mtDb}{\bar{\mtD}}
\newcommand{\mtEb}{\bar{\mtE}}
\newcommand{\mtFb}{\bar{\mtF}}
\newcommand{\mtGb}{\bar{\mtG}}
\newcommand{\mtHb}{\bar{\mtH}}
\newcommand{\mtIb}{\bar{\mtI}}
\newcommand{\mtJb}{\bar{\mtJ}}
\newcommand{\mtKb}{\bar{\mtK}}
\newcommand{\mtLb}{\bar{\mtL}}
\newcommand{\mtMb}{\bar{\mtM}}
\newcommand{\mtNb}{\bar{\mtN}}
\newcommand{\mtOb}{\bar{\mtP}}
\newcommand{\mtPb}{\bar{\mtP}}
\newcommand{\mtQb}{\bar{\mtQ}}
\newcommand{\mtRb}{\bar{\mtR}}
\newcommand{\mtSb}{\bar{\mtS}}
\newcommand{\mtTb}{\bar{\mtT}}
\newcommand{\mtUb}{\bar{\mtU}}
\newcommand{\mtVb}{\bar{\mtV}}
\newcommand{\mtWb}{\bar{\mtW}}
\newcommand{\mtXb}{\bar{\mtX}}
\newcommand{\mtYb}{\bar{\mtY}}
\newcommand{\mtZb}{\bar{\mtZ}}
% Underlined matrices
\newcommand{\mtAu}{\underline{\mtA}}
\newcommand{\mtBu}{\underline{\mtB}}
\newcommand{\mtCu}{\underline{\mtC}}
\newcommand{\mtDu}{\underline{\mtD}}
\newcommand{\mtEu}{\underline{\mtE}}
\newcommand{\mtFu}{\underline{\mtF}}
\newcommand{\mtGu}{\underline{\mtG}}
\newcommand{\mtHu}{\underline{\mtH}}
\newcommand{\mtIu}{\underline{\mtI}}
\newcommand{\mtJu}{\underline{\mtJ}}
\newcommand{\mtKu}{\underline{\mtK}}
\newcommand{\mtLu}{\underline{\mtL}}
\newcommand{\mtMu}{\underline{\mtM}}
\newcommand{\mtNu}{\underline{\mtN}}
\newcommand{\mtOu}{\underline{\mtP}}
\newcommand{\mtPu}{\underline{\mtP}}
\newcommand{\mtQu}{\underline{\mtQ}}
\newcommand{\mtRu}{\underline{\mtR}}
\newcommand{\mtSu}{\underline{\mtS}}
\newcommand{\mtTu}{\underline{\mtT}}
\newcommand{\mtUu}{\underline{\mtU}}
\newcommand{\mtVu}{\underline{\mtV}}
\newcommand{\mtWu}{\underline{\mtW}}
\newcommand{\mtXu}{\underline{\mtX}}
\newcommand{\mtYu}{\underline{\mtY}}
\newcommand{\mtZu}{\underline{\mtZ}}
% Dotted matrices
\newcommand{\mtAd}{\dot{\mtA}}
\newcommand{\mtBd}{\dot{\mtB}}
\newcommand{\mtCd}{\dot{\mtC}}
\newcommand{\mtDd}{\dot{\mtD}}
\newcommand{\mtEd}{\dot{\mtE}}
\newcommand{\mtFd}{\dot{\mtF}}
\newcommand{\mtGd}{\dot{\mtG}}
\newcommand{\mtHd}{\dot{\mtH}}
\newcommand{\mtId}{\dot{\mtI}}
\newcommand{\mtJd}{\dot{\mtJ}}
\newcommand{\mtKd}{\dot{\mtK}}
\newcommand{\mtLd}{\dot{\mtL}}
\newcommand{\mtMd}{\dot{\mtM}}
\newcommand{\mtNd}{\dot{\mtN}}
\newcommand{\mtOd}{\dot{\mtP}}
\newcommand{\mtPd}{\dot{\mtP}}
\newcommand{\mtQd}{\dot{\mtQ}}
\newcommand{\mtRd}{\dot{\mtR}}
\newcommand{\mtSd}{\dot{\mtS}}
\newcommand{\mtTd}{\dot{\mtT}}
\newcommand{\mtUd}{\dot{\mtU}}
\newcommand{\mtVd}{\dot{\mtV}}
\newcommand{\mtWd}{\dot{\mtW}}
\newcommand{\mtXd}{\dot{\mtX}}
\newcommand{\mtYd}{\dot{\mtY}}
\newcommand{\mtZd}{\dot{\mtZ}}
% Double dotted matrices
\newcommand{\mtAdd}{\ddot{\mtA}}
\newcommand{\mtBdd}{\ddot{\mtB}}
\newcommand{\mtCdd}{\ddot{\mtC}}
\newcommand{\mtDdd}{\ddot{\mtD}}
\newcommand{\mtEdd}{\ddot{\mtE}}
\newcommand{\mtFdd}{\ddot{\mtF}}
\newcommand{\mtGdd}{\ddot{\mtG}}
\newcommand{\mtHdd}{\ddot{\mtH}}
\newcommand{\mtIdd}{\ddot{\mtI}}
\newcommand{\mtJdd}{\ddot{\mtJ}}
\newcommand{\mtKdd}{\ddot{\mtK}}
\newcommand{\mtLdd}{\ddot{\mtL}}
\newcommand{\mtMdd}{\ddot{\mtM}}
\newcommand{\mtNdd}{\ddot{\mtN}}
\newcommand{\mtOdd}{\ddot{\mtP}}
\newcommand{\mtPdd}{\ddot{\mtP}}
\newcommand{\mtQdd}{\ddot{\mtQ}}
\newcommand{\mtRdd}{\ddot{\mtR}}
\newcommand{\mtSdd}{\ddot{\mtS}}
\newcommand{\mtTdd}{\ddot{\mtT}}
\newcommand{\mtUdd}{\ddot{\mtU}}
\newcommand{\mtVdd}{\ddot{\mtV}}
\newcommand{\mtWdd}{\ddot{\mtW}}
\newcommand{\mtXdd}{\ddot{\mtX}}
\newcommand{\mtYdd}{\ddot{\mtY}}
\newcommand{\mtZdd}{\ddot{\mtZ}}

% Special matrices
\newcommand{\mtLambda}{\Mt{\Lambda}}
\newcommand{\mtPhi}{\Mt{\Phi}}
\newcommand{\mtPsi}{\Mt{\Psi}}
\newcommand{\mtSigma}{\Mt{\Sigma}}
\newcommand{\mtGamma}{\Mt{\Gamma}}
\newcommand{\mtXi}{\Mt{\Xi}}
\newcommand{\mtZero}{\Mt{0}}
\newcommand{\mtOne}{\Mt{1}}
\newcommand{\mtUpsilon}{\Mt{\Upsilon}}
% Vectors
\newcommand{\vtA}{\Vt{A}}
\newcommand{\vtB}{\Vt{B}}
\newcommand{\vtC}{\Vt{C}}
\newcommand{\vtD}{\Vt{D}}
\newcommand{\vtE}{\Vt{E}}
\newcommand{\vtF}{\Vt{F}}
\newcommand{\vtG}{\Vt{G}}
\newcommand{\vtH}{\Vt{H}}
\newcommand{\vtI}{\Vt{I}}
\newcommand{\vtJ}{\Vt{J}}
\newcommand{\vtK}{\Vt{K}}
\newcommand{\vtL}{\Vt{L}}
\newcommand{\vtM}{\Vt{M}}
\newcommand{\vtN}{\Vt{N}}
\newcommand{\vtO}{\Vt{P}}
\newcommand{\vtP}{\Vt{P}}
\newcommand{\vtQ}{\Vt{Q}}
\newcommand{\vtR}{\Vt{R}}
\newcommand{\vtS}{\Vt{S}}
\newcommand{\vtT}{\Vt{T}}
\newcommand{\vtU}{\Vt{U}}
\newcommand{\vtV}{\Vt{V}}
\newcommand{\vtW}{\Vt{W}}
\newcommand{\vtX}{\Vt{X}}
\newcommand{\vtY}{\Vt{Y}}
\newcommand{\vtZ}{\Vt{Z}}
% Transposed vectors
\newcommand{\vtAt}{\Transp{\vtA}}
\newcommand{\vtBt}{\Transp{\vtB}}
\newcommand{\vtCt}{\Transp{\vtC}}
\newcommand{\vtDt}{\Transp{\vtD}}
\newcommand{\vtEt}{\Transp{\vtE}}
\newcommand{\vtFt}{\Transp{\vtF}}
\newcommand{\vtGt}{\Transp{\vtG}}
\newcommand{\vtHt}{\Transp{\vtH}}
\newcommand{\vtIt}{\Transp{\vtI}}
\newcommand{\vtJt}{\Transp{\vtJ}}
\newcommand{\vtKt}{\Transp{\vtK}}
\newcommand{\vtLt}{\Transp{\vtL}}
\newcommand{\vtMt}{\Transp{\vtM}}
\newcommand{\vtNt}{\Transp{\vtN}}
\newcommand{\vtOt}{\Transp{\vtP}}
\newcommand{\vtPt}{\Transp{\vtP}}
\newcommand{\vtQt}{\Transp{\vtQ}}
\newcommand{\vtRt}{\Transp{\vtR}}
\newcommand{\vtSt}{\Transp{\vtS}}
\newcommand{\vtTt}{\Transp{\vtT}}
\newcommand{\vtUt}{\Transp{\vtU}}
\newcommand{\vtVt}{\Transp{\vtV}}
\newcommand{\vtWt}{\Transp{\vtW}}
\newcommand{\vtXt}{\Transp{\vtX}}
\newcommand{\vtYt}{\Transp{\vtY}}
\newcommand{\vtZt}{\Transp{\vtZ}}
% Hermitian vectors
\newcommand{\vtAh}{\Herm{\vtA}}
\newcommand{\vtBh}{\Herm{\vtB}}
\newcommand{\vtCh}{\Herm{\vtC}}
\newcommand{\vtDh}{\Herm{\vtD}}
\newcommand{\vtEh}{\Herm{\vtE}}
\newcommand{\vtFh}{\Herm{\vtF}}
\newcommand{\vtGh}{\Herm{\vtG}}
\newcommand{\vtHh}{\Herm{\vtH}}
\newcommand{\vtIh}{\Herm{\vtI}}
\newcommand{\vtJh}{\Herm{\vtJ}}
\newcommand{\vtKh}{\Herm{\vtK}}
\newcommand{\vtLh}{\Herm{\vtL}}
\newcommand{\vtMh}{\Herm{\vtM}}
\newcommand{\vtNh}{\Herm{\vtN}}
\newcommand{\vtOh}{\Herm{\vtP}}
\newcommand{\vtPh}{\Herm{\vtP}}
\newcommand{\vtQh}{\Herm{\vtQ}}
\newcommand{\vtRh}{\Herm{\vtR}}
\newcommand{\vtSh}{\Herm{\vtS}}
\newcommand{\vtTh}{\Herm{\vtT}}
\newcommand{\vtUh}{\Herm{\vtU}}
\newcommand{\vtVh}{\Herm{\vtV}}
\newcommand{\vtWh}{\Herm{\vtW}}
\newcommand{\vtXh}{\Herm{\vtX}}
\newcommand{\vtYh}{\Herm{\vtY}}
\newcommand{\vtZh}{\Herm{\vtZ}}
% Bar vectors
\newcommand{\vtAb}{\bar{\vtA}}
\newcommand{\vtBb}{\bar{\vtB}}
\newcommand{\vtCb}{\bar{\vtC}}
\newcommand{\vtDb}{\bar{\vtD}}
\newcommand{\vtEb}{\bar{\vtE}}
\newcommand{\vtFb}{\bar{\vtF}}
\newcommand{\vtGb}{\bar{\vtG}}
\newcommand{\vtHb}{\bar{\vtH}}
\newcommand{\vtIb}{\bar{\vtI}}
\newcommand{\vtJb}{\bar{\vtJ}}
\newcommand{\vtKb}{\bar{\vtK}}
\newcommand{\vtLb}{\bar{\vtL}}
\newcommand{\vtMb}{\bar{\vtM}}
\newcommand{\vtNb}{\bar{\vtN}}
\newcommand{\vtOb}{\bar{\vtP}}
\newcommand{\vtPb}{\bar{\vtP}}
\newcommand{\vtQb}{\bar{\vtQ}}
\newcommand{\vtRb}{\bar{\vtR}}
\newcommand{\vtSb}{\bar{\vtS}}
\newcommand{\vtTb}{\bar{\vtT}}
\newcommand{\vtUb}{\bar{\vtU}}
\newcommand{\vtVb}{\bar{\vtV}}
\newcommand{\vtWb}{\bar{\vtW}}
\newcommand{\vtXb}{\bar{\vtX}}
\newcommand{\vtYb}{\bar{\vtY}}
\newcommand{\vtZb}{\bar{\vtZ}}
% Vectors underlined
\newcommand{\vtAu}{\underline{\Vt{A}}}
\newcommand{\vtBu}{\underline{\Vt{B}}}
\newcommand{\vtCu}{\underline{\Vt{C}}}
\newcommand{\vtDu}{\underline{\Vt{D}}}
\newcommand{\vtEu}{\underline{\Vt{E}}}
\newcommand{\vtFu}{\underline{\Vt{F}}}
\newcommand{\vtGu}{\underline{\Vt{G}}}
\newcommand{\vtHu}{\underline{\Vt{H}}}
\newcommand{\vtIu}{\underline{\Vt{I}}}
\newcommand{\vtJu}{\underline{\Vt{J}}}
\newcommand{\vtKu}{\underline{\Vt{K}}}
\newcommand{\vtLu}{\underline{\Vt{L}}}
\newcommand{\vtMu}{\underline{\Vt{M}}}
\newcommand{\vtNu}{\underline{\Vt{N}}}
\newcommand{\vtOu}{\underline{\Vt{P}}}
\newcommand{\vtPu}{\underline{\Vt{P}}}
\newcommand{\vtQu}{\underline{\Vt{Q}}}
\newcommand{\vtRu}{\underline{\Vt{R}}}
\newcommand{\vtSu}{\underline{\Vt{S}}}
\newcommand{\vtTu}{\underline{\Vt{T}}}
\newcommand{\vtUu}{\underline{\Vt{U}}}
\newcommand{\vtVu}{\underline{\Vt{V}}}
\newcommand{\vtWu}{\underline{\Vt{W}}}
\newcommand{\vtXu}{\underline{\Vt{X}}}
\newcommand{\vtYu}{\underline{\Vt{Y}}}
\newcommand{\vtZu}{\underline{\Vt{Z}}}
% Vectors dotted
\newcommand{\vtAd}{\dot{\Vt{A}}}
\newcommand{\vtBd}{\dot{\Vt{B}}}
\newcommand{\vtCd}{\dot{\Vt{C}}}
\newcommand{\vtDd}{\dot{\Vt{D}}}
\newcommand{\vtEd}{\dot{\Vt{E}}}
\newcommand{\vtFd}{\dot{\Vt{F}}}
\newcommand{\vtGd}{\dot{\Vt{G}}}
\newcommand{\vtHd}{\dot{\Vt{H}}}
\newcommand{\vtId}{\dot{\Vt{I}}}
\newcommand{\vtJd}{\dot{\Vt{J}}}
\newcommand{\vtKd}{\dot{\Vt{K}}}
\newcommand{\vtLd}{\dot{\Vt{L}}}
\newcommand{\vtMd}{\dot{\Vt{M}}}
\newcommand{\vtNd}{\dot{\Vt{N}}}
\newcommand{\vtOd}{\dot{\Vt{P}}}
\newcommand{\vtPd}{\dot{\Vt{P}}}
\newcommand{\vtQd}{\dot{\Vt{Q}}}
\newcommand{\vtRd}{\dot{\Vt{R}}}
\newcommand{\vtSd}{\dot{\Vt{S}}}
\newcommand{\vtTd}{\dot{\Vt{T}}}
\newcommand{\vtUd}{\dot{\Vt{U}}}
\newcommand{\vtVd}{\dot{\Vt{V}}}
\newcommand{\vtWd}{\dot{\Vt{W}}}
\newcommand{\vtXd}{\dot{\Vt{X}}}
\newcommand{\vtYd}{\dot{\Vt{Y}}}
\newcommand{\vtZd}{\dot{\Vt{Z}}}
% Vectors double dotted
\newcommand{\vtAdd}{\ddot{\Vt{A}}}
\newcommand{\vtBdd}{\ddot{\Vt{B}}}
\newcommand{\vtCdd}{\ddot{\Vt{C}}}
\newcommand{\vtDdd}{\ddot{\Vt{D}}}
\newcommand{\vtEdd}{\ddot{\Vt{E}}}
\newcommand{\vtFdd}{\ddot{\Vt{F}}}
\newcommand{\vtGdd}{\ddot{\Vt{G}}}
\newcommand{\vtHdd}{\ddot{\Vt{H}}}
\newcommand{\vtIdd}{\ddot{\Vt{I}}}
\newcommand{\vtJdd}{\ddot{\Vt{J}}}
\newcommand{\vtKdd}{\ddot{\Vt{K}}}
\newcommand{\vtLdd}{\ddot{\Vt{L}}}
\newcommand{\vtMdd}{\ddot{\Vt{M}}}
\newcommand{\vtNdd}{\ddot{\Vt{N}}}
\newcommand{\vtOdd}{\ddot{\Vt{P}}}
\newcommand{\vtPdd}{\ddot{\Vt{P}}}
\newcommand{\vtQdd}{\ddot{\Vt{Q}}}
\newcommand{\vtRdd}{\ddot{\Vt{R}}}
\newcommand{\vtSdd}{\ddot{\Vt{S}}}
\newcommand{\vtTdd}{\ddot{\Vt{T}}}
\newcommand{\vtUdd}{\ddot{\Vt{U}}}
\newcommand{\vtVdd}{\ddot{\Vt{V}}}
\newcommand{\vtWdd}{\ddot{\Vt{W}}}
\newcommand{\vtXdd}{\ddot{\Vt{X}}}
\newcommand{\vtYdd}{\ddot{\Vt{Y}}}
\newcommand{\vtZdd}{\ddot{\Vt{Z}}}
% Special vectors
\newcommand{\vtAlpha}{\Vt{\boldsymbol{\alpha}}}
\newcommand{\vtBeta}{\Vt{\boldsymbol{\beta}}}
\newcommand{\vtDelta}{\Vt{\boldsymbol{\Delta}}}
\newcommand{\vtEta}{\Vt{\boldsymbol{\eta}}}
\newcommand{\vtLambda}{\Vt{\boldsymbol{\lambda}}}
\newcommand{\vtMy}{\Vt{\boldsymbol{\mu}}}
\newcommand{\vtNy}{\Vt{\boldsymbol{\nu}}}
\newcommand{\vtOne}{\Vt{1}}
\newcommand{\vtPsi}{\Vt{\boldsymbol{\psi}}}
\newcommand{\vtSigma}{\Vt{\boldsymbol{\sigma}}}
\newcommand{\vtTau}{\Vt{\boldsymbol{\tau}}}
\newcommand{\vtZero}{\Vt{0}}
% Fields
\newcommand{\fdA}{\Field{A}}
\newcommand{\fdB}{\Field{B}}
\newcommand{\fdC}{\Field{C}}
\newcommand{\fdD}{\Field{D}}
\newcommand{\fdE}{\Field{E}}
\newcommand{\fdF}{\Field{F}}
\newcommand{\fdG}{\Field{G}}
\newcommand{\fdH}{\Field{H}}
\newcommand{\fdI}{\Field{I}}
\newcommand{\fdJ}{\Field{J}}
\newcommand{\fdK}{\Field{K}}
\newcommand{\fdL}{\Field{L}}
\newcommand{\fdM}{\Field{M}}
\newcommand{\fdN}{\Field{N}}
\newcommand{\fdO}{\Field{O}}
\newcommand{\fdP}{\Field{P}}
\newcommand{\fdQ}{\Field{Q}}
\newcommand{\fdR}{\Field{R}}
\newcommand{\fdS}{\Field{S}}
\newcommand{\fdT}{\Field{T}}
\newcommand{\fdU}{\Field{U}}
\newcommand{\fdV}{\Field{V}}
\newcommand{\fdW}{\Field{W}}
\newcommand{\fdX}{\Field{X}}
\newcommand{\fdY}{\Field{Y}}
\newcommand{\fdZ}{\Field{Z}}
% Sets
\newcommand{\stA}{\Set{A}}
\newcommand{\stB}{\Set{B}}
\newcommand{\stC}{\Set{C}}
\newcommand{\stD}{\Set{D}}
\newcommand{\stE}{\Set{E}}
\newcommand{\stF}{\Set{F}}
\newcommand{\stG}{\Set{G}}
\newcommand{\stH}{\Set{H}}
\newcommand{\stI}{\Set{I}}
\newcommand{\stJ}{\Set{J}}
\newcommand{\stK}{\Set{K}}
\newcommand{\stL}{\Set{L}}
\newcommand{\stM}{\Set{M}}
\newcommand{\stN}{\Set{N}}
\newcommand{\stO}{\Set{O}}
\newcommand{\stP}{\Set{P}}
\newcommand{\stQ}{\Set{Q}}
\newcommand{\stR}{\Set{R}}
\newcommand{\stS}{\Set{S}}
\newcommand{\stT}{\Set{T}}
\newcommand{\stU}{\Set{U}}
\newcommand{\stV}{\Set{V}}
\newcommand{\stW}{\Set{W}}
\newcommand{\stX}{\Set{X}}
\newcommand{\stY}{\Set{Y}}
\newcommand{\stZ}{\Set{Z}}
% Alias for math fonts
%%%%%%%%%%%%%%%%%%%%%%%%%%%%%%%
% rm
%%%%%%%%%%%%%%%%%%%%%%%%%%%%%%%
\newcommand{\rmA}{\mathrm{A}}
\newcommand{\rmB}{\mathrm{B}}
\newcommand{\rmC}{\mathrm{C}}
\newcommand{\rmD}{\mathrm{D}}
\newcommand{\rmE}{\mathrm{E}}
\newcommand{\rmF}{\mathrm{F}}
\newcommand{\rmG}{\mathrm{G}}
\newcommand{\rmH}{\mathrm{H}}
\newcommand{\rmI}{\mathrm{I}}
\newcommand{\rmJ}{\mathrm{J}}
\newcommand{\rmK}{\mathrm{K}}
\newcommand{\rmL}{\mathrm{L}}
\newcommand{\rmM}{\mathrm{M}}
\newcommand{\rmN}{\mathrm{N}}
\newcommand{\rmO}{\mathrm{O}}
\newcommand{\rmP}{\mathrm{P}}
\newcommand{\rmQ}{\mathrm{Q}}
\newcommand{\rmR}{\mathrm{R}}
\newcommand{\rmS}{\mathrm{S}}
\newcommand{\rmT}{\mathrm{T}}
\newcommand{\rmU}{\mathrm{U}}
\newcommand{\rmV}{\mathrm{V}}
\newcommand{\rmW}{\mathrm{W}}
\newcommand{\rmX}{\mathrm{X}}
\newcommand{\rmY}{\mathrm{Y}}
\newcommand{\rmZ}{\mathrm{Z}}
%%%%%%%%%%%%%%%%%%%%%%%%%%%%%%%
\newcommand{\rma}{\mathrm{a}}
\newcommand{\rmb}{\mathrm{b}}
\newcommand{\rmc}{\mathrm{c}}
\newcommand{\rmd}{\mathrm{d}}
\newcommand{\rme}{\mathrm{e}}
\newcommand{\rmf}{\mathrm{f}}
\newcommand{\rmg}{\mathrm{g}}
\newcommand{\rmh}{\mathrm{h}}
\newcommand{\rmi}{\mathrm{i}}
\newcommand{\rmj}{\mathrm{j}}
\newcommand{\rmk}{\mathrm{k}}
\newcommand{\rml}{\mathrm{l}}
\newcommand{\rmm}{\mathrm{m}}
\newcommand{\rmn}{\mathrm{n}}
\newcommand{\rmo}{\mathrm{o}}
\newcommand{\rmp}{\mathrm{p}}
\newcommand{\rmq}{\mathrm{q}}
\newcommand{\rmr}{\mathrm{r}}
\newcommand{\rms}{\mathrm{s}}
\newcommand{\rmt}{\mathrm{t}}
\newcommand{\rmu}{\mathrm{u}}
\newcommand{\rmv}{\mathrm{v}}
\newcommand{\rmw}{\mathrm{w}}
\newcommand{\rmx}{\mathrm{x}}
\newcommand{\rmy}{\mathrm{y}}
\newcommand{\rmz}{\mathrm{z}}
%%%%%%%%%%%%%%%%%%%%%%%%%%%%%%%
%%%%%%%%%%%%%%%%%%%%%%%%%%%%%%%
% sf
%%%%%%%%%%%%%%%%%%%%%%%%%%%%%%%
\newcommand{\sfA}{\mathsf{A}}
\newcommand{\sfB}{\mathsf{B}}
\newcommand{\sfC}{\mathsf{C}}
\newcommand{\sfD}{\mathsf{D}}
\newcommand{\sfE}{\mathsf{E}}
\newcommand{\sfF}{\mathsf{F}}
\newcommand{\sfG}{\mathsf{G}}
\newcommand{\sfH}{\mathsf{H}}
\newcommand{\sfI}{\mathsf{I}}
\newcommand{\sfJ}{\mathsf{J}}
\newcommand{\sfK}{\mathsf{K}}
\newcommand{\sfL}{\mathsf{L}}
\newcommand{\sfM}{\mathsf{M}}
\newcommand{\sfN}{\mathsf{N}}
\newcommand{\sfO}{\mathsf{O}}
\newcommand{\sfP}{\mathsf{P}}
\newcommand{\sfQ}{\mathsf{Q}}
\newcommand{\sfR}{\mathsf{R}}
\newcommand{\sfS}{\mathsf{S}}
\newcommand{\sfT}{\mathsf{T}}
\newcommand{\sfU}{\mathsf{U}}
\newcommand{\sfV}{\mathsf{V}}
\newcommand{\sfW}{\mathsf{W}}
\newcommand{\sfX}{\mathsf{X}}
\newcommand{\sfY}{\mathsf{Y}}
\newcommand{\sfZ}{\mathsf{Z}}
%%%%%%%%%%%%%%%%%%%%%%%%%%%%%%%
%\newcommand{\sfa}{\mathsf{a}}
%\newcommand{\sfb}{\mathsf{b}}
%\newcommand{\sfc}{\mathsf{c}}
%\newcommand{\sfd}{\mathsf{d}}
%\newcommand{\sfe}{\mathsf{e}}
%\newcommand{\sff}{\mathsf{f}}
%\newcommand{\sfg}{\mathsf{g}}
%\newcommand{\sfh}{\mathsf{h}}
%\newcommand{\sfi}{\mathsf{i}}
%\newcommand{\sfj}{\mathsf{j}}
%\newcommand{\sfk}{\mathsf{k}}
%\newcommand{\sfl}{\mathsf{l}}
%\newcommand{\sfm}{\mathsf{m}}
%\newcommand{\sfn}{\mathsf{n}}
%\newcommand{\sfo}{\mathsf{o}}
%\newcommand{\sfp}{\mathsf{p}}
%\newcommand{\sfq}{\mathsf{q}}
%\newcommand{\sfr}{\mathsf{r}}
%\newcommand{\sfs}{\mathsf{s}}
%\newcommand{\sft}{\mathsf{t}}
%\newcommand{\sfu}{\mathsf{u}}
%\newcommand{\sfv}{\mathsf{v}}
%\newcommand{\sfw}{\mathsf{w}}
%\newcommand{\sfx}{\mathsf{x}}
%\newcommand{\sfy}{\mathsf{y}}
%\newcommand{\sfz}{\mathsf{z}}
%%%%%%%%%%%%%%%%%%%%%%%%%%%%%%%
%%%%%%%%%%%%%%%%%%%%%%%%%%%%%%%
% bf
%%%%%%%%%%%%%%%%%%%%%%%%%%%%%%%
\newcommand{\bfA}{\mathbf{A}}
\newcommand{\bfB}{\mathbf{B}}
\newcommand{\bfC}{\mathbf{C}}
\newcommand{\bfD}{\mathbf{D}}
\newcommand{\bfE}{\mathbf{E}}
\newcommand{\bfF}{\mathbf{F}}
\newcommand{\bfG}{\mathbf{G}}
\newcommand{\bfH}{\mathbf{H}}
\newcommand{\bfI}{\mathbf{I}}
\newcommand{\bfJ}{\mathbf{J}}
\newcommand{\bfK}{\mathbf{K}}
\newcommand{\bfL}{\mathbf{L}}
\newcommand{\bfM}{\mathbf{M}}
\newcommand{\bfN}{\mathbf{N}}
\newcommand{\bfO}{\mathbf{O}}
\newcommand{\bfP}{\mathbf{P}}
\newcommand{\bfQ}{\mathbf{Q}}
\newcommand{\bfR}{\mathbf{R}}
\newcommand{\bfS}{\mathbf{S}}
\newcommand{\bfT}{\mathbf{T}}
\newcommand{\bfU}{\mathbf{U}}
\newcommand{\bfV}{\mathbf{V}}
\newcommand{\bfW}{\mathbf{W}}
\newcommand{\bfX}{\mathbf{X}}
\newcommand{\bfY}{\mathbf{Y}}
\newcommand{\bfZ}{\mathbf{Z}}
%%%%%%%%%%%%%%%%%%%%%%%%%%%%%%%
\newcommand{\bfa}{\mathbf{a}}
\newcommand{\bfb}{\mathbf{b}}
\newcommand{\bfc}{\mathbf{c}}
\newcommand{\bfd}{\mathbf{d}}
\newcommand{\bfe}{\mathbf{e}}
\newcommand{\bff}{\mathbf{f}}
\newcommand{\bfg}{\mathbf{g}}
\newcommand{\bfh}{\mathbf{h}}
\newcommand{\bfi}{\mathbf{i}}
\newcommand{\bfj}{\mathbf{j}}
\newcommand{\bfk}{\mathbf{k}}
\newcommand{\bfl}{\mathbf{l}}
\newcommand{\bfm}{\mathbf{m}}
\newcommand{\bfn}{\mathbf{n}}
\newcommand{\bfo}{\mathbf{o}}
\newcommand{\bfp}{\mathbf{p}}
\newcommand{\bfq}{\mathbf{q}}
\newcommand{\bfr}{\mathbf{r}}
\newcommand{\bfs}{\mathbf{s}}
\newcommand{\bft}{\mathbf{t}}
\newcommand{\bfu}{\mathbf{u}}
\newcommand{\bfv}{\mathbf{v}}
\newcommand{\bfw}{\mathbf{w}}
\newcommand{\bfx}{\mathbf{x}}
\newcommand{\bfy}{\mathbf{y}}
\newcommand{\bfz}{\mathbf{z}}
%%%%%%%%%%%%%%%%%%%%%%%%%%%%%%%
%%%%%%%%%%%%%%%%%%%%%%%%%%%%%%%
% it
%%%%%%%%%%%%%%%%%%%%%%%%%%%%%%%
\newcommand{\itA}{\mathit{A}}
\newcommand{\itB}{\mathit{B}}
\newcommand{\itC}{\mathit{C}}
\newcommand{\itD}{\mathit{D}}
\newcommand{\itE}{\mathit{E}}
\newcommand{\itF}{\mathit{F}}
\newcommand{\itG}{\mathit{G}}
\newcommand{\itH}{\mathit{H}}
\newcommand{\itI}{\mathit{I}}
\newcommand{\itJ}{\mathit{J}}
\newcommand{\itK}{\mathit{K}}
\newcommand{\itL}{\mathit{L}}
\newcommand{\itM}{\mathit{M}}
\newcommand{\itN}{\mathit{N}}
\newcommand{\itO}{\mathit{O}}
\newcommand{\itP}{\mathit{P}}
\newcommand{\itQ}{\mathit{Q}}
\newcommand{\itR}{\mathit{R}}
\newcommand{\itS}{\mathit{S}}
\newcommand{\itT}{\mathit{T}}
\newcommand{\itU}{\mathit{U}}
\newcommand{\itV}{\mathit{V}}
\newcommand{\itW}{\mathit{W}}
\newcommand{\itX}{\mathit{X}}
\newcommand{\itY}{\mathit{Y}}
\newcommand{\itZ}{\mathit{Z}}
%%%%%%%%%%%%%%%%%%%%%%%%%%%%%%%
\newcommand{\ita}{\mathit{a}}
\newcommand{\itb}{\mathit{b}}
\newcommand{\itc}{\mathit{c}}
\newcommand{\itd}{\mathit{d}}
\newcommand{\ite}{\mathit{e}}
\newcommand{\itf}{\mathit{f}}
\newcommand{\itg}{\mathit{g}}
\newcommand{\ith}{\mathit{h}}
\newcommand{\iti}{\mathit{i}}
\newcommand{\itj}{\mathit{j}}
\newcommand{\itk}{\mathit{k}}
\newcommand{\itl}{\mathit{l}}
\newcommand{\itm}{\mathit{m}}
\newcommand{\itn}{\mathit{n}}
\newcommand{\ito}{\mathit{o}}
\newcommand{\itp}{\mathit{p}}
\newcommand{\itq}{\mathit{q}}
\newcommand{\itr}{\mathit{r}}
\newcommand{\its}{\mathit{s}}
\newcommand{\itt}{\mathit{t}}
\newcommand{\itu}{\mathit{u}}
\newcommand{\itv}{\mathit{v}}
\newcommand{\itw}{\mathit{w}}
\newcommand{\itx}{\mathit{x}}
\newcommand{\ity}{\mathit{y}}
\newcommand{\itz}{\mathit{z}}
%%%%%%%%%%%%%%%%%%%%%%%%%%%%%%%
%%%%%%%%%%%%%%%%%%%%%%%%%%%%%%%
% frak
%%%%%%%%%%%%%%%%%%%%%%%%%%%%%%%
\newcommand{\fkA}{\mathfrak{A}}
\newcommand{\fkB}{\mathfrak{B}}
\newcommand{\fkC}{\mathfrak{C}}
\newcommand{\fkD}{\mathfrak{D}}
\newcommand{\fkE}{\mathfrak{E}}
\newcommand{\fkF}{\mathfrak{F}}
\newcommand{\fkG}{\mathfrak{G}}
\newcommand{\fkH}{\mathfrak{H}}
\newcommand{\fkI}{\mathfrak{I}}
\newcommand{\fkJ}{\mathfrak{J}}
\newcommand{\fkK}{\mathfrak{K}}
\newcommand{\fkL}{\mathfrak{L}}
\newcommand{\fkM}{\mathfrak{M}}
\newcommand{\fkN}{\mathfrak{N}}
\newcommand{\fkO}{\mathfrak{O}}
\newcommand{\fkP}{\mathfrak{P}}
\newcommand{\fkQ}{\mathfrak{Q}}
\newcommand{\fkR}{\mathfrak{R}}
\newcommand{\fkS}{\mathfrak{S}}
\newcommand{\fkT}{\mathfrak{T}}
\newcommand{\fkU}{\mathfrak{U}}
\newcommand{\fkV}{\mathfrak{V}}
\newcommand{\fkW}{\mathfrak{W}}
\newcommand{\fkX}{\mathfrak{X}}
\newcommand{\fkY}{\mathfrak{Y}}
\newcommand{\fkZ}{\mathfrak{Z}}
%%%%%%%%%%%%%%%%%%%%%%%%%%%%%%%
\newcommand{\fka}{\mathfrak{a}}
\newcommand{\fkb}{\mathfrak{b}}
\newcommand{\fkc}{\mathfrak{c}}
\newcommand{\fkd}{\mathfrak{d}}
\newcommand{\fke}{\mathfrak{e}}
\newcommand{\fkf}{\mathfrak{f}}
\newcommand{\fkg}{\mathfrak{g}}
\newcommand{\fkh}{\mathfrak{h}}
\newcommand{\fki}{\mathfrak{i}}
\newcommand{\fkj}{\mathfrak{j}}
\newcommand{\fkk}{\mathfrak{k}}
\newcommand{\fkl}{\mathfrak{l}}
\newcommand{\fkm}{\mathfrak{m}}
\newcommand{\fkn}{\mathfrak{n}}
\newcommand{\fko}{\mathfrak{o}}
\newcommand{\fkp}{\mathfrak{p}}
\newcommand{\fkq}{\mathfrak{q}}
\newcommand{\fkr}{\mathfrak{r}}
\newcommand{\fks}{\mathfrak{s}}
\newcommand{\fkt}{\mathfrak{t}}
\newcommand{\fku}{\mathfrak{u}}
\newcommand{\fkv}{\mathfrak{v}}
\newcommand{\fkw}{\mathfrak{w}}
\newcommand{\fkx}{\mathfrak{x}}
\newcommand{\fky}{\mathfrak{y}}
\newcommand{\fkz}{\mathfrak{z}}
%%%%%%%%%%%%%%%%%%%%%%%%%%%%%%%
% Eufrak matrices
\newcommand{\mtkA}{\boldsymbol{\fkA}}
\newcommand{\mtkB}{\boldsymbol{\fkB}}
\newcommand{\mtkC}{\boldsymbol{\fkC}}
\newcommand{\mtkD}{\boldsymbol{\fkD}}
\newcommand{\mtkE}{\boldsymbol{\fkE}}
\newcommand{\mtkF}{\boldsymbol{\fkF}}
\newcommand{\mtkG}{\boldsymbol{\fkG}}
\newcommand{\mtkH}{\boldsymbol{\fkH}}
\newcommand{\mtkI}{\boldsymbol{\fkI}}
\newcommand{\mtkJ}{\boldsymbol{\fkJ}}
\newcommand{\mtkK}{\boldsymbol{\fkK}}
\newcommand{\mtkL}{\boldsymbol{\fkL}}
\newcommand{\mtkM}{\boldsymbol{\fkM}}
\newcommand{\mtkN}{\boldsymbol{\fkN}}
\newcommand{\mtkO}{\boldsymbol{\fkO}}
\newcommand{\mtkP}{\boldsymbol{\fkP}}
\newcommand{\mtkQ}{\boldsymbol{\fkQ}}
\newcommand{\mtkR}{\boldsymbol{\fkR}}
\newcommand{\mtkS}{\boldsymbol{\fkS}}
\newcommand{\mtkT}{\boldsymbol{\fkT}}
\newcommand{\mtkU}{\boldsymbol{\fkU}}
\newcommand{\mtkV}{\boldsymbol{\fkV}}
\newcommand{\mtkW}{\boldsymbol{\fkW}}
\newcommand{\mtkX}{\boldsymbol{\fkX}}
\newcommand{\mtkY}{\boldsymbol{\fkY}}
\newcommand{\mtkZ}{\boldsymbol{\fkZ}}
%%%%%%%%%%%%%%%%%%%%%%%%%%%%%%%
%%%%%%%%%%%%%%%%%%%%%%%%%%%%%%%
% ppl
%%%%%%%%%%%%%%%%%%%%%%%%%%%%%%%
\newcommand{\pplA}{\mathppl{A}}
\newcommand{\pplB}{\mathppl{B}}
\newcommand{\pplC}{\mathppl{C}}
\newcommand{\pplD}{\mathppl{D}}
\newcommand{\pplE}{\mathppl{E}}
\newcommand{\pplF}{\mathppl{F}}
\newcommand{\pplG}{\mathppl{G}}
\newcommand{\pplH}{\mathppl{H}}
\newcommand{\pplI}{\mathppl{I}}
\newcommand{\pplJ}{\mathppl{J}}
\newcommand{\pplK}{\mathppl{K}}
\newcommand{\pplL}{\mathppl{L}}
\newcommand{\pplM}{\mathppl{M}}
\newcommand{\pplN}{\mathppl{N}}
\newcommand{\pplO}{\mathppl{O}}
\newcommand{\pplP}{\mathppl{P}}
\newcommand{\pplQ}{\mathppl{Q}}
\newcommand{\pplR}{\mathppl{R}}
\newcommand{\pplS}{\mathppl{S}}
\newcommand{\pplT}{\mathppl{T}}
\newcommand{\pplU}{\mathppl{U}}
\newcommand{\pplV}{\mathppl{V}}
\newcommand{\pplW}{\mathppl{W}}
\newcommand{\pplX}{\mathppl{X}}
\newcommand{\pplY}{\mathppl{Y}}
\newcommand{\pplZ}{\mathppl{Z}}
%%%%%%%%%%%%%%%%%%%%%%%%%%%%%%%
\newcommand{\ppla}{\mathppl{a}}
\newcommand{\pplb}{\mathppl{b}}
\newcommand{\pplc}{\mathppl{c}}
\newcommand{\ppld}{\mathppl{d}}
\newcommand{\pple}{\mathppl{e}}
\newcommand{\pplf}{\mathppl{f}}
\newcommand{\pplg}{\mathppl{g}}
\newcommand{\pplh}{\mathppl{h}}
\newcommand{\ppli}{\mathppl{i}}
\newcommand{\pplj}{\mathppl{j}}
\newcommand{\pplk}{\mathppl{k}}
\newcommand{\ppll}{\mathppl{l}}
\newcommand{\pplm}{\mathppl{m}}
\newcommand{\ppln}{\mathppl{n}}
\newcommand{\pplo}{\mathppl{o}}
\newcommand{\pplp}{\mathppl{p}}
\newcommand{\pplq}{\mathppl{q}}
\newcommand{\pplr}{\mathppl{r}}
\newcommand{\ppls}{\mathppl{s}}
\newcommand{\pplt}{\mathppl{t}}
\newcommand{\pplu}{\mathppl{u}}
\newcommand{\pplv}{\mathppl{v}}
\newcommand{\pplw}{\mathppl{w}}
\newcommand{\pplx}{\mathppl{x}}
\newcommand{\pply}{\mathppl{y}}
\newcommand{\pplz}{\mathppl{z}}
%%%%%%%%%%%%%%%%%%%%%%%%%%%%%%%
%%%%%%%%%%%%%%%%%%%%%%%%%%%%%%%
% phv
%%%%%%%%%%%%%%%%%%%%%%%%%%%%%%%
\newcommand{\phvA}{\mathphv{A}}
\newcommand{\phvB}{\mathphv{B}}
\newcommand{\phvC}{\mathphv{C}}
\newcommand{\phvD}{\mathphv{D}}
\newcommand{\phvE}{\mathphv{E}}
\newcommand{\phvF}{\mathphv{F}}
\newcommand{\phvG}{\mathphv{G}}
\newcommand{\phvH}{\mathphv{H}}
\newcommand{\phvI}{\mathphv{I}}
\newcommand{\phvJ}{\mathphv{J}}
\newcommand{\phvK}{\mathphv{K}}
\newcommand{\phvL}{\mathphv{L}}
\newcommand{\phvM}{\mathphv{M}}
\newcommand{\phvN}{\mathphv{N}}
\newcommand{\phvO}{\mathphv{O}}
\newcommand{\phvP}{\mathphv{P}}
\newcommand{\phvQ}{\mathphv{Q}}
\newcommand{\phvR}{\mathphv{R}}
\newcommand{\phvS}{\mathphv{S}}
\newcommand{\phvT}{\mathphv{T}}
\newcommand{\phvU}{\mathphv{U}}
\newcommand{\phvV}{\mathphv{V}}
\newcommand{\phvW}{\mathphv{W}}
\newcommand{\phvX}{\mathphv{X}}
\newcommand{\phvY}{\mathphv{Y}}
\newcommand{\phvZ}{\mathphv{Z}}
%%%%%%%%%%%%%%%%%%%%%%%%%%%%%%%
\newcommand{\phva}{\mathphv{a}}
\newcommand{\phvb}{\mathphv{b}}
\newcommand{\phvc}{\mathphv{c}}
\newcommand{\phvd}{\mathphv{d}}
\newcommand{\phve}{\mathphv{e}}
\newcommand{\phvf}{\mathphv{f}}
\newcommand{\phvg}{\mathphv{g}}
\newcommand{\phvh}{\mathphv{h}}
\newcommand{\phvi}{\mathphv{i}}
\newcommand{\phvj}{\mathphv{j}}
\newcommand{\phvk}{\mathphv{k}}
\newcommand{\phvl}{\mathphv{l}}
\newcommand{\phvm}{\mathphv{m}}
\newcommand{\phvn}{\mathphv{n}}
\newcommand{\phvo}{\mathphv{o}}
\newcommand{\phvp}{\mathphv{p}}
\newcommand{\phvq}{\mathphv{q}}
\newcommand{\phvr}{\mathphv{r}}
\newcommand{\phvs}{\mathphv{s}}
\newcommand{\phvt}{\mathphv{t}}
\newcommand{\phvu}{\mathphv{u}}
\newcommand{\phvv}{\mathphv{v}}
\newcommand{\phvw}{\mathphv{w}}
\newcommand{\phvx}{\mathphv{x}}
\newcommand{\phvy}{\mathphv{y}}
\newcommand{\phvz}{\mathphv{z}}
%%%%%%%%%%%%%%%%%%%%%%%%%%%%%%%
%%%%%%%%%%%%%%%%%%%%%%%%%%%%%%%
% pzc
%%%%%%%%%%%%%%%%%%%%%%%%%%%%%%%
\newcommand{\pzcA}{\mathpzc{A}}
\newcommand{\pzcB}{\mathpzc{B}}
\newcommand{\pzcC}{\mathpzc{C}}
\newcommand{\pzcD}{\mathpzc{D}}
\newcommand{\pzcE}{\mathpzc{E}}
\newcommand{\pzcF}{\mathpzc{F}}
\newcommand{\pzcG}{\mathpzc{G}}
\newcommand{\pzcH}{\mathpzc{H}}
\newcommand{\pzcI}{\mathpzc{I}}
\newcommand{\pzcJ}{\mathpzc{J}}
\newcommand{\pzcK}{\mathpzc{K}}
\newcommand{\pzcL}{\mathpzc{L}}
\newcommand{\pzcM}{\mathpzc{M}}
\newcommand{\pzcN}{\mathpzc{N}}
\newcommand{\pzcO}{\mathpzc{O}}
\newcommand{\pzcP}{\mathpzc{P}}
\newcommand{\pzcQ}{\mathpzc{Q}}
\newcommand{\pzcR}{\mathpzc{R}}
\newcommand{\pzcS}{\mathpzc{S}}
\newcommand{\pzcT}{\mathpzc{T}}
\newcommand{\pzcU}{\mathpzc{U}}
\newcommand{\pzcV}{\mathpzc{V}}
\newcommand{\pzcW}{\mathpzc{W}}
\newcommand{\pzcX}{\mathpzc{X}}
\newcommand{\pzcY}{\mathpzc{Y}}
\newcommand{\pzcZ}{\mathpzc{Z}}
%%%%%%%%%%%%%%%%%%%%%%%%%%%%%%%
\newcommand{\pzca}{\mathpzc{a}}
\newcommand{\pzcb}{\mathpzc{b}}
\newcommand{\pzcc}{\mathpzc{c}}
\newcommand{\pzcd}{\mathpzc{d}}
\newcommand{\pzce}{\mathpzc{e}}
\newcommand{\pzcf}{\mathpzc{f}}
\newcommand{\pzcg}{\mathpzc{g}}
\newcommand{\pzch}{\mathpzc{h}}
\newcommand{\pzci}{\mathpzc{i}}
\newcommand{\pzcj}{\mathpzc{j}}
\newcommand{\pzck}{\mathpzc{k}}
\newcommand{\pzcl}{\mathpzc{l}}
\newcommand{\pzcm}{\mathpzc{m}}
\newcommand{\pzcn}{\mathpzc{n}}
\newcommand{\pzco}{\mathpzc{o}}
\newcommand{\pzcp}{\mathpzc{p}}
\newcommand{\pzcq}{\mathpzc{q}}
\newcommand{\pzcr}{\mathpzc{r}}
\newcommand{\pzcs}{\mathpzc{s}}
\newcommand{\pzct}{\mathpzc{t}}
\newcommand{\pzcu}{\mathpzc{u}}
\newcommand{\pzcv}{\mathpzc{v}}
\newcommand{\pzcw}{\mathpzc{w}}
\newcommand{\pzcx}{\mathpzc{x}}
\newcommand{\pzcy}{\mathpzc{y}}
\newcommand{\pzcz}{\mathpzc{z}}
%%%%%%%%%%%%%%%%%%%%%%%%%%%%%%%
%%%%%%%%%%%%%%%%%%%%%%%%%%%%%%%
% bb
%%%%%%%%%%%%%%%%%%%%%%%%%%%%%%%
\newcommand{\bbA}{\mathbb{A}}
\newcommand{\bbB}{\mathbb{B}}
\newcommand{\bbC}{\mathbb{C}}
\newcommand{\bbD}{\mathbb{D}}
\newcommand{\bbE}{\mathbb{E}}
\newcommand{\bbF}{\mathbb{F}}
\newcommand{\bbG}{\mathbb{G}}
\newcommand{\bbH}{\mathbb{H}}
\newcommand{\bbI}{\mathbb{I}}
\newcommand{\bbJ}{\mathbb{J}}
\newcommand{\bbK}{\mathbb{K}}
\newcommand{\bbL}{\mathbb{L}}
\newcommand{\bbM}{\mathbb{M}}
\newcommand{\bbN}{\mathbb{N}}
\newcommand{\bbO}{\mathbb{O}}
\newcommand{\bbP}{\mathbb{P}}
\newcommand{\bbQ}{\mathbb{Q}}
\newcommand{\bbR}{\mathbb{R}}
\newcommand{\bbS}{\mathbb{S}}
\newcommand{\bbT}{\mathbb{T}}
\newcommand{\bbU}{\mathbb{U}}
\newcommand{\bbV}{\mathbb{V}}
\newcommand{\bbW}{\mathbb{W}}
\newcommand{\bbX}{\mathbb{X}}
\newcommand{\bbY}{\mathbb{Y}}
\newcommand{\bbZ}{\mathbb{Z}}
%%%%%%%%%%%%%%%%%%%%%%%%%%%%%%%
\newcommand{\bba}{\mathbb{a}}
\newcommand{\bbb}{\mathbb{b}}
\newcommand{\bbc}{\mathbb{c}}
\newcommand{\bbd}{\mathbb{d}}
\newcommand{\bbe}{\mathbb{e}}
\newcommand{\bbf}{\mathbb{f}}
\newcommand{\bbg}{\mathbb{g}}
\newcommand{\bbh}{\mathbb{h}}
\newcommand{\bbi}{\mathbb{i}}
\newcommand{\bbj}{\mathbb{j}}
\newcommand{\bbk}{\mathbb{k}}
\newcommand{\bbl}{\mathbb{l}}
\newcommand{\bbm}{\mathbb{m}}
\newcommand{\bbn}{\mathbb{n}}
\newcommand{\bbo}{\mathbb{o}}
\newcommand{\bbp}{\mathbb{p}}
\newcommand{\bbq}{\mathbb{q}}
\newcommand{\bbr}{\mathbb{r}}
\newcommand{\bbs}{\mathbb{s}}
\newcommand{\bbt}{\mathbb{t}}
\newcommand{\bbu}{\mathbb{u}}
\newcommand{\bbv}{\mathbb{v}}
\newcommand{\bbw}{\mathbb{w}}
\newcommand{\bbx}{\mathbb{x}}
\newcommand{\bby}{\mathbb{y}}
\newcommand{\bbz}{\mathbb{z}}
%%%%%%%%%%%%%%%%%%%%%%%%%%%%%%%
%%%%%%%%%%%%%%%%%%%%%%%%%%%%%%%
% sc
%%%%%%%%%%%%%%%%%%%%%%%%%%%%%%%
\newcommand{\scA}{\mathscr{A}}
\newcommand{\scB}{\mathscr{B}}
\newcommand{\scC}{\mathscr{C}}
\newcommand{\scD}{\mathscr{D}}
\newcommand{\scE}{\mathscr{E}}
\newcommand{\scF}{\mathscr{F}}
\newcommand{\scG}{\mathscr{G}}
\newcommand{\scH}{\mathscr{H}}
\newcommand{\scI}{\mathscr{I}}
\newcommand{\scJ}{\mathscr{J}}
\newcommand{\scK}{\mathscr{K}}
\newcommand{\scL}{\mathscr{L}}
\newcommand{\scM}{\mathscr{M}}
\newcommand{\scN}{\mathscr{N}}
\newcommand{\scO}{\mathscr{O}}
\newcommand{\scP}{\mathscr{P}}
\newcommand{\scQ}{\mathscr{Q}}
\newcommand{\scR}{\mathscr{R}}
\newcommand{\scS}{\mathscr{S}}
\newcommand{\scT}{\mathscr{T}}
\newcommand{\scU}{\mathscr{U}}
\newcommand{\scV}{\mathscr{V}}
\newcommand{\scW}{\mathscr{W}}
\newcommand{\scX}{\mathscr{X}}
\newcommand{\scY}{\mathscr{Y}}
\newcommand{\scZ}{\mathscr{Z}}
%%%%%%%%%%%%%%%%%%%%%%%%%%%%%%%
%%%%%%%%%%%%%%%%%%%%%%%%%%%%%%%
% cal
%%%%%%%%%%%%%%%%%%%%%%%%%%%%%%%
\newcommand{\calA}{\mathcal{A}}
\newcommand{\calB}{\mathcal{B}}
\newcommand{\calC}{\mathcal{C}}
\newcommand{\calD}{\mathcal{D}}
\newcommand{\calE}{\mathcal{E}}
\newcommand{\calF}{\mathcal{F}}
\newcommand{\calG}{\mathcal{G}}
\newcommand{\calH}{\mathcal{H}}
\newcommand{\calI}{\mathcal{I}}
\newcommand{\calJ}{\mathcal{J}}
\newcommand{\calK}{\mathcal{K}}
\newcommand{\calL}{\mathcal{L}}
\newcommand{\calM}{\mathcal{M}}
\newcommand{\calN}{\mathcal{N}}
\newcommand{\calO}{\mathcal{O}}
\newcommand{\calP}{\mathcal{P}}
\newcommand{\calQ}{\mathcal{Q}}
\newcommand{\calR}{\mathcal{R}}
\newcommand{\calS}{\mathcal{S}}
\newcommand{\calT}{\mathcal{T}}
\newcommand{\calU}{\mathcal{U}}
\newcommand{\calV}{\mathcal{V}}
\newcommand{\calW}{\mathcal{W}}
\newcommand{\calX}{\mathcal{X}}
\newcommand{\calY}{\mathcal{Y}}
\newcommand{\calZ}{\mathcal{Z}}
%%%%%%%%%%%%%%%%%%%%%%%%%%%%%%%
%%%%%%%%%%%%%%%%%%%%%%%%%%%%%%%
% txt
%%%%%%%%%%%%%%%%%%%%%%%%%%%%%%%
\newcommand{\txtA}{\text{A}}
\newcommand{\txtB}{\text{B}}
\newcommand{\txtC}{\text{C}}
\newcommand{\txtD}{\text{D}}
\newcommand{\txtE}{\text{E}}
\newcommand{\txtF}{\text{F}}
\newcommand{\txtG}{\text{G}}
\newcommand{\txtH}{\text{H}}
\newcommand{\txtI}{\text{I}}
\newcommand{\txtJ}{\text{J}}
\newcommand{\txtK}{\text{K}}
\newcommand{\txtL}{\text{L}}
\newcommand{\txtM}{\text{M}}
\newcommand{\txtN}{\text{N}}
\newcommand{\txtO}{\text{O}}
\newcommand{\txtP}{\text{P}}
\newcommand{\txtQ}{\text{Q}}
\newcommand{\txtR}{\text{R}}
\newcommand{\txtS}{\text{S}}
\newcommand{\txtT}{\text{T}}
\newcommand{\txtU}{\text{U}}
\newcommand{\txtV}{\text{V}}
\newcommand{\txtW}{\text{W}}
\newcommand{\txtX}{\text{X}}
\newcommand{\txtY}{\text{Y}}
\newcommand{\txtZ}{\text{Z}}
%%%%%%%%%%%%%%%%%%%%%%%%%%%%%%%
\newcommand{\txta}{\text{a}}
\newcommand{\txtb}{\text{b}}
\newcommand{\txtc}{\text{c}}
\newcommand{\txtd}{\text{d}}
\newcommand{\txte}{\text{e}}
\newcommand{\txtf}{\text{f}}
\newcommand{\txtg}{\text{g}}
\newcommand{\txth}{\text{h}}
\newcommand{\txti}{\text{i}}
\newcommand{\txtj}{\text{j}}
\newcommand{\txtk}{\text{k}}
\newcommand{\txtl}{\text{l}}
\newcommand{\txtm}{\text{m}}
\newcommand{\txtn}{\text{n}}
\newcommand{\txto}{\text{o}}
\newcommand{\txtp}{\text{p}}
\newcommand{\txtq}{\text{q}}
\newcommand{\txtr}{\text{r}}
\newcommand{\txts}{\text{s}}
\newcommand{\txtt}{\text{t}}
\newcommand{\txtu}{\text{u}}
\newcommand{\txtv}{\text{v}}
\newcommand{\txtw}{\text{w}}
\newcommand{\txtx}{\text{x}}
\newcommand{\txty}{\text{y}}
\newcommand{\txtz}{\text{z}}
%%%%%%%%%%%%%%%%%%%%%%%%%%%%%%%
%%%%%%%%%%%%%%%%%%%%%%%%%%%%%%%
% tt
%%%%%%%%%%%%%%%%%%%%%%%%%%%%%%%
\newcommand{\ttA}{\mathtt{A}}
\newcommand{\ttB}{\mathtt{B}}
\newcommand{\ttC}{\mathtt{C}}
\newcommand{\ttD}{\mathtt{D}}
\newcommand{\ttE}{\mathtt{E}}
\newcommand{\ttF}{\mathtt{F}}
\newcommand{\ttG}{\mathtt{G}}
\newcommand{\ttH}{\mathtt{H}}
\newcommand{\ttI}{\mathtt{I}}
\newcommand{\ttJ}{\mathtt{J}}
\newcommand{\ttK}{\mathtt{K}}
\newcommand{\ttL}{\mathtt{L}}
\newcommand{\ttM}{\mathtt{M}}
\newcommand{\ttN}{\mathtt{N}}
\newcommand{\ttO}{\mathtt{O}}
\newcommand{\ttP}{\mathtt{P}}
\newcommand{\ttQ}{\mathtt{Q}}
\newcommand{\ttR}{\mathtt{R}}
\newcommand{\ttS}{\mathtt{S}}
\newcommand{\ttT}{\mathtt{T}}
\newcommand{\ttU}{\mathtt{U}}
\newcommand{\ttV}{\mathtt{V}}
\newcommand{\ttW}{\mathtt{W}}
\newcommand{\ttX}{\mathtt{X}}
\newcommand{\ttY}{\mathtt{Y}}
\newcommand{\ttZ}{\mathtt{Z}}
%%%%%%%%%%%%%%%%%%%%%%%%%%%%%%%
\newcommand{\tta}{\mathtt{a}}
\newcommand{\ttb}{\mathtt{b}}
\newcommand{\ttc}{\mathtt{c}}
\newcommand{\ttd}{\mathtt{d}}
\newcommand{\tte}{\mathtt{e}}
\newcommand{\ttf}{\mathtt{f}}
\newcommand{\ttg}{\mathtt{g}}
\newcommand{\tth}{\mathtt{h}}
\newcommand{\tti}{\mathtt{i}}
\newcommand{\ttj}{\mathtt{j}}
\newcommand{\ttk}{\mathtt{k}}
\newcommand{\ttl}{\mathtt{l}}
\newcommand{\ttm}{\mathtt{m}}
\newcommand{\ttn}{\mathtt{n}}
\newcommand{\tto}{\mathtt{o}}
\newcommand{\ttp}{\mathtt{p}}
\newcommand{\ttq}{\mathtt{q}}
\newcommand{\ttr}{\mathtt{r}}
\newcommand{\tts}{\mathtt{s}}
\newcommand{\ttt}{\mathtt{t}}
\newcommand{\ttu}{\mathtt{u}}
\newcommand{\ttv}{\mathtt{v}}
\newcommand{\ttw}{\mathtt{w}}
\newcommand{\ttx}{\mathtt{x}}
\newcommand{\tty}{\mathtt{y}}
\newcommand{\ttz}{\mathtt{z}}
%%%%%%%%%%%%%%%%%%%%%%%%%%%%%%%


% Local Variables:
% ispell-local-dictionary: "en_US"
% End:

%\input{Acronimos.tex}

\title{Otimização não-linear}
\author[DRAFT]{
  \textbf{Prof. Tarcisio F. Maciel, Dr.-Eng.} \\
  \small\textbf{Colaboradores:} Diego A. Sousa, M.-Eng., José Mairton B. da Silva Jr., M.-Eng., \\
  Francisco Hugo C. Neto, M.-Eng., e Yuri Victor L. de Melo, M.-Eng.
}
\institute[Em desenvolvimento]{
  \normalsize Universidade Federal do Ceará \\
  Centro de Tecnologia \\
  Programa de Pós-Graduação em Engenharia de Teleinformática
}
\date{\today}

% Command to start a new part and also create a frame with the part page
\newcommand{\createpartandshowpage}[1]{
  \part{#1}
  \begin{frame}
    \partpage{}
  \end{frame}
}

\begin{document}

\begin{frame}
	\titlepage
\end{frame}

% \AtBeginPart{
% \begin{frame}<handout:0>
% 	\begin{block}{\centering\large\bfseries Parte \insertpartnumber}
% 		\centering\large\insertpart
% 	\end{block}
% \end{frame}
%
% \begin{frame}<handout:0>
% 	\frametitle{Conteúdo}
% 	\tableofcontents[hideallsubsections]
% \end{frame}
% }
%
% \AtBeginLecture{
% \begin{frame}<handout:0>
% 	\begin{block}{\centering\large\bfseries Tema da aula}
% 		\centering\large\insertlecture
% 	\end{block}
% \end{frame}
% }

\AtBeginSection{
\begin{frame}
	\frametitle{Conteúdo}
	\tableofcontents[currentsection,hideothersubsections]
\end{frame}
}

\AtBeginSubsection{
\begin{frame}<handout:0>
	\frametitle{Conteúdo}
	\tableofcontents[currentsection,subsectionstyle=show/shaded/hide]
\end{frame}
}

\createpartandshowpage{Introdução}
% !TeX root = Otimizacao.tex
% !TeX encoding = UTF-8
% !TeX spellcheck = pt_BR
% !TeX program = pdflatex
\section{Introdução}

\subsection{O que é otimização?}

\begin{frame}{Introdução~\cite{Nocedal2006}}
  \begin{itemize}\addtolength{\itemsep}{\baselineskip}
    \item Pessoas otimizam
    \begin{itemize}\setlength{\AuxWidth}{\widthof{Engenheiros}}
    \item \makebox[\AuxWidth][l]{Investidores} \ding{220} criam portfólios que minimizam riscos e atingem uma certa taxa
      de retorno
    \item \makebox[\AuxWidth][l]{Fabricantes} \ding{220} maximizam eficiência no projeto e operação de suas plantas
      produtivas
    \item \makebox[\AuxWidth][l]{Engenheiros} \ding{220} ajustam parâmetros para reduzir custos e aumentar a eficiência
      de seus projetos
    \end{itemize}

    \item A natureza otimiza
    \begin{itemize}\setlength{\AuxWidth}{\widthof{Sistemas físicos}}
      \item \makebox[\AuxWidth][l]{Sistemas físicos} \ding{220} tendem ao estado de mínima energia
      \item \makebox[\AuxWidth][l]{Raios de luz} \ding{220} percorrem o caminho de menor tempo de percurso
      \item \makebox[\AuxWidth][l]{Moléculas} \ding{220} se acomodam para minimizar a energia potencial dos elétrons
    \end{itemize}

    \item Otimização é uma ferramenta importante para tomada de decisões e análise de sistemas físicos
  \end{itemize}
\end{frame}

\begin{frame}{Introdução~\cite{Nocedal2006}}
  \begin{itemize}\addtolength{\itemsep}{\baselineskip}
    \item O que é otimização?
    \begin{itemize}
      \item Dar a algo um rendimento ótimo, criando-lhe as condições mais favoráveis ou tirando o melhor partido possível; tornar algo ótimo ou ideal \cite{HolandaFerreira2010}
    \end{itemize}
  
    \item Porque otimizar?
    \begin{itemize}
      \item Com otimização é possível melhorar o desempenho de um sistema, ou seja, deixar o sistema mais rápido e eficiente \cite{Nocedal2006}
    \end{itemize}
    
    \item Como otimizar?
    \begin{itemize}
      \item Para otimizar é preciso definir o \alert{objetivo}, uma medida que quantifica o desempenho do sistema
      \item O objetivo depende de certos de certas características do sistema, chamadas de \alert{variáveis} que otimizam o sistema
      \item Por fim é frequente o uso de \alert{restrições} que descrevem situações do sistema consideradas não-desejáveis \cite{Nocedal2006}
    \end{itemize}
  \end{itemize}
\end{frame}

\subsection{Problemas de otimização}

\begin{frame}{Elementos de um problema de otimização}
  \begin{itemize}
    \item Uso da ferramenta de otimização \ding{220} identificar alguns elementos
    \begin{itemize}\setlength{\AuxWidth}{\widthof{Restrições}}
      \item \makebox[\AuxWidth][l]{Objetivo} \ding{220} medida quantitativa do desempenho do sistema em estudo \ding{220} lucro, tempo, energia, entre outros, ou uma combinação de fatores resultando em um escalar
      \item \makebox[\AuxWidth][l]{Variáveis} \ding{220} parâmetros cujos valores podem ser ajustados e dos quais depende o desempenho do sistema 
      \item \makebox[\AuxWidth][l]{Restrições} \ding{220} condições às quais os valores da variáveis estão sujeitos
    \end{itemize}
    
    \item Modelagem de problemas de otimização \ding{220} Processo de identificação do objetivo, variáveis e restrições
    \begin{itemize}\setlength{\AuxWidth}{\widthof{Construção de um modelo apropriado}}
      \item \makebox[\AuxWidth][l]{Construção de um modelo apropriado} \ding{220} \alert{algumas vezes é o passo mais importante}
      \item \makebox[\AuxWidth][l]{Excessivamente simplista} \ding{220} não provê informação suficiente
      \item \makebox[\AuxWidth][l]{Complexo demais} \ding{220} difícil de resolver
    \end{itemize}

    \item Após modelado \ding{220} solucionado usando um algoritmo de otimização

    \item Não há algoritmo universal \ding{220} coleção de algoritmos especializados para cada tipo de problema

    \item<alert@1> Maiores benefícios \ding{220} surgem quando o tipo do problema é conhecido

    \item Resultado do algoritmo \ding{220} validado utilizando as condições de optimalidade
  \end{itemize}
\end{frame}

\begin{frame}{Modelo matématico de um problema de otimização}
  \begin{itemize}
    \item \alert{Matematicamente, otimização é a maximização ou minimização de uma função objetivo sujeita a restrições sobre suas variáveis de otimização}
    \item Em nossos modelos matemáticos, tipicamente:
    \begin{itemize}\setlength{\AuxWidth}{\widthof{$f_i(\vtX) : \fdR^n \rightarrow \fdR$}}
      \item \makebox[\AuxWidth][l]{$\vtX \in \fdR^{n}$} \ding{220} vetor de \alert{variáveis de otimização}
      \item \makebox[\AuxWidth][l]{$f(\vtX) : \fdR^n \rightarrow \fdR$} \ding{220} \alert{função objetivo} que se deseja maximizar ou minimizar
      \item \makebox[\AuxWidth][l]{$f_i(\vtX) : \fdR^n \rightarrow \fdR$} \ding{220} as \alert{restrições} de igualdade e desigualdade que $\vtX$ deve satisfazer
    \end{itemize}
    \item Logo, pode-se escrever um problema de otimização como
    \begin{subequations}\label{eq_prob_otimizacao}
      \begin{align}
        \vtX^\star &= \Minimize{\vtX \in \fdR^n}{f(\vtX)} \\
        \Sujeito \quad f_i(\vtX) &= 0, \quad i = 1, 2, \ldots, I, \\
        f_j(\vtX) &\geq 0, \quad j = 1, 2, \ldots, J,
      \end{align}
    \end{subequations}
    onde $i$ e $j$ são os índices para as restrições de igualdade e desigualdade, respectivamente, e $ \vtX^\star $ é uma \alert{solução ótima}
  \end{itemize}
\end{frame}

\begin{frame}{Exemplo de problema de otimização: transporte de produtos~\cite{Nocedal2006}}
  \begin{itemize}\footnotesize
    \item Uma companhia possui $ 02 $ fábricas $ F_1 $ e $ F_2 $ e $ 12 $ lojas $ R_1, R_2, \ldots, R_{12} $. Cada
      fábrica $ F_i $ pode produzir $ a_i $ toneladas de um produto ($ a_i $ é a capacidade de produção da planta) e cada loja possui uma demanda semanal de $ b_j $ toneladas do produto. O custo de transporte da fábrica $ F_i $ para a loja $ R_j $ de uma tonelada do produto é $ c_{i,j} $. O problema é determinar as quantidades $ x_{i,j} \in \fdR_{+} $ do produto que devem ser transportadas de cada fábrica para cada loja de modo a atender a todos os requisitos e minimizar o custo total. % 
    \uncover<2->{ % Begin uncover
    Esse problema pode ser formulado como segue: %
    \begin{subequations}
      \begin{align}
        \{ x^\star_{i,j} \} = \Minimize{ \{x_{i,j}\} }{ & \sum\limits_{i = 1}^{2}\sum\limits_{j = 1}^{12}  c_{i,j}x_{i,j} } \\
        \Sujeito \quad & \sum\limits_{j = 1}^{12} x_{i,j} \leq a_i, \quad i = 1, 2 \\
        & \sum\limits_{i = 1}^{2} x_{i,j} \geq b_j, \quad j = 1, 2, \ldots, 12 \\
        & x_{i,j} \geq 0, \quad i = 1, 2 \quad \text{e} \quad j = 1, 2, \ldots, 12
      \end{align}
    \end{subequations}
    \item Este problema é um \alert{problema de otimização linear} \ding{220} função custo e todas as restrições são funções lineares das variáveis do problema
    \item<alert@2->[\faBook] Reescreva o problema acima em forma vetorial/matricial
  } % End uncover
  \end{itemize}
\end{frame}

\begin{frame}{\normalsize Exemplo de problema de otimização: minimização da soma das correlações espaciais~\cite{Maciel2006}}
	\begin{itemize}\footnotesize
  \item A correlação espacial $ \rho_{i,j} $ entre os canais $ \vtH_i = \begin{bmatrix} h_{i,1} & h_{i,2} & \ldots &
      h_{i,N} \end{bmatrix}$ e $ \vtH_j = \begin{bmatrix} h_{j,1} & h_{j,2} & \ldots & h_{j,N} \end{bmatrix}$, com
    $\vtH_i, \vtH_j \in \fdC^N $ do enlace direto de uma estação rádio base com $ N $ antenas para os terminais móveis $
    i, j $ é dada por $ \rho_{i,j} = \dfrac{\Abs{\vtH_i \vtHh_j}}{\NormTwo{\vtH_i} \NormTwo{\vtH_j}} $. Sabendo que
    existem $ K $ terminais móveis, selecione $ G \leq N $ terminais móveis tal que a soma das correlações entre eles
    dois-a-dois seja mínima, ou seja, selecione os $ G $ terminais móveis com os canais menos correlacionados.
    \uncover<2->{ % Begin uncover
			Esse problema pode ser formulado como segue: %
			\begin{subequations}
				\begin{align}
					\vtX^\star = \Minimize{ \vtX }{& \frac{1}{2}\vtXt \mtR \vtX }, \\
					\Sujeito \quad & \Transp{\vtOne}\vtX = G, \\
					& \vtX \in \fdB^K,
				\end{align}
			\end{subequations} %
			onde $ \mtR = {[\rho_{i,j}]}_{i,j}, \quad i, j \in  \{1, 2, \ldots, K\} $.
			\item Este problema é um \alert{problema de otimização binário quadrático} \ding{220} função custo quadrática com variáveis de otimização binárias
			\item<alert@2->[\faBook] Reescreva o problema acima utilizando somatórios
		} % End uncover
	\end{itemize}
\end{frame}

\subsection{Classes de problemas de otimização}

\begin{frame}{Classes de problemas de otimização}
  \begin{itemize}\addtolength{\itemsep}{0.5\baselineskip}\setlength{\AuxWidth}{\widthof{Máximo/mínimo global}}
    \item Natureza das variáveis de otimização, da função objetivo, e das restrições \ding{220} diferentes tipos de problemas de otimização e algoritmos de otimização

    \item Variáveis de otimização
    \begin{itemize}\setlength{\AuxWidth}{\widthof{$ x_1, \ldots, x_k \in \fdR $ e $ x_{k+1}, \ldots, x_n \in \fdZ $}}
      \item \makebox[\AuxWidth][l]{$\vtX \in \fdR^{n}$} \ding{220} otimização contínua (mais fácil de resolver)
      \item \makebox[\AuxWidth][l]{$\vtX \in \fdZ^{n}$} \ding{220} otimização inteira (pode requerer relaxações contínuas)
      \item \makebox[\AuxWidth][l]{$ x_1, \ldots, x_k \in \fdR $ e $ x_{k+1}, \ldots, x_n \in \fdZ $} \ding{220} otimização inteira mista
    \end{itemize}

    \item Função objetivo e restrições
    \begin{itemize}\setlength{\AuxWidth}{\widthof{$ f(\vtX) $ e $ f_i(\vtX) $ convexas, $ f_j(\vtX)  $ lineares}}
      \item \makebox[\AuxWidth][l]{$ f(\vtX) $, $ f_i(\vtX) $ e $ f_j(\vtX)  $ lineares} \ding{220} Otimização linear
      \item \makebox[\AuxWidth][l]{$ f(\vtX) $ e $ f_i(\vtX) $ convexas, $ f_j(\vtX)  $ lineares} \ding{220} Otimização convexa
    \end{itemize}

    \item \makebox[\AuxWidth][l]{Máximo/mínimo local} \ding{220} $f( \vtX^{\star} )$ é um máximo/mínimo local de $f(\vtX)$ se existe um sub-espaço aberto $\fdA \subset \fdR^n$ contendo $\vtX^{\star}$ tal que $f(\vtX) \lesseqgtr f(\vtX^{\star}), \forall \vtX \in \fdA$.

		\item \makebox[\AuxWidth][l]{Máximo/mínimo global} \ding{220} $f( \vtX^{\star} )$ é um máximo/mínimo global de $f(\vtX)$ se $f(\vtX) \lessgtr f(\vtX^{\star})$,   $ \forall \vtX \in \fdR^n$
  \end{itemize}
\end{frame}

\begin{frame}{Classes de problemas de otimização}
  \begin{itemize}
    \item Algoritmos de otimização \ding{220} especializados para cada tipo de problema
    \begin{itemize}\addtolength{\itemsep}{0.5\baselineskip}\setlength{\AuxWidth}{\widthof{Otimização sem restrições}}
      \item \makebox[\AuxWidth][l]{Otimização linear} \ding{220} método Simplex
      \item \makebox[\AuxWidth][l]{Otimização convexa} \ding{220} método dos pontos interiores
      \item \makebox[\AuxWidth][l]{Otimização linear inteira} \ding{220} método \textit{branch-and-bound}
      \item \makebox[\AuxWidth][l]{Otimização sem restrições} \ding{220} método de busca direta e métodos do gradiente
      \item \makebox[\AuxWidth][l]{Otimização com restrições} \ding{220} métodos dos pontos interiores
      \item \makebox[\AuxWidth][l]{Otimização determinística} \ding{220} restrições e parâmetros dados por funções bem definidas
      \item \makebox[\AuxWidth][l]{Otimização estocástica} \ding{220} restrições ou parâmetros dependem de variáveis aleatórias
    \end{itemize}
  \end{itemize}
\end{frame}

\begin{frame}{Exemplo de problema de otimização: mínimos quadrados~\cite{Boyd2004}}
  \begin{itemize}\small
    \item Considere o problema de minimizar a soma dos erros quadráticos entre as componentes de um vetor $ \vtY = \mtA \vtX $ e um vetor de referência $ \vtB $. Esse é um problema de mínimos quadrados sem restrições que pode ser escrito como
    \begin{equation}
      \vtX^\star = \Minimize{\vtX}{\NormTwo{\mtA \vtX - \vtB}^2}
    \end{equation}
    \item<2-> Note que
    \begin{equation}
      \begin{split}
        \NormTwo{\mtA \vtX - \vtB}^2 &= \Transp{(\mtA \vtX - \vtB)}(\mtA \vtX - \vtB) = (\Transp{\vtX} \Transp{\mtA} - \Transp{\vtB})(\mtA \vtX - \vtB) \\ 
        &= \Transp{\vtX} \Transp{\mtA}\mtA \vtX - \Transp{\vtX} \Transp{\mtA}\vtB -   \Transp{\vtB}\mtA \vtX - \Transp{\vtB}\vtB = \Transp{\vtX} \Transp{\mtA}\mtA \vtX - 2\Transp{\vtB}\mtA \vtX - \Transp{\vtB}\vtB
      \end{split}
    \end{equation}
    \item<2-> Derivando a equação acima em relação a $ \vtX $ e igualando a $ \vtZero $ temos
    \begin{equation}
      \frac{d}{d\vtX}\left(\Transp{\vtX}\Transp{\mtA}\mtA\vtX - 2\Transp{\vtB}\mtA \vtX - \Transp{\vtB}\vtB\right) = 0 \Rightarrow 2\Transp{\mtA}\mtA\vtX - 2\Transp{\mtA}\vtB = 0 \Rightarrow \boxed{\vtX^\star = \Inv{(\Transp{\mtA}\mtA)}\Transp{\mtA}\vtB}
    \end{equation}
    \uncover<2->{\item<alert@2>[\faBook] Liste as classes de problemas de otimização às quais esse problema pertence}
  \end{itemize}
\end{frame}

% Local Variables:
% fill-column: 120
% ispell-local-dictionary: "pt_BR"
% TeX-master: "Slides"
% End:

\createpartandshowpage{Conjuntos convexos e funções convexas}
% !TeX root = Otimizacao.tex
% !TeX encoding = UTF-8
% !TeX spellcheck = pt_BR
% !TeX program = pdflatex
\section{Conjuntos convexos}

\subsection{Conjuntos afins e convexos}

\begin{frame}{Conjuntos afins}
  \begin{columns}
    \begin{column}{0.63\linewidth}
      \begin{itemize}\addtolength{\itemsep}{\baselineskip}
        \item Linha e segmentos
        \begin{itemize}
        \item Sejam os pontos $ \vtX_1, \vtX_2 \in \fdR^n $ com $ \vtX_1 \neq \vtX_2 $ \ding{220} pontos $ \vtY $ da
          forma $ \vtY = \theta\vtX_1 + (1-\theta)\vtX_2 $ com $ \theta \in \fdR$, $ 0 \leq \theta \leq 1 $, formam um
          segmento fechado ligando $ \vtX_1 $ a $ \vtX_2 $
          \item Representando $ \vtY = \vtX_2 + \theta(\vtX_1 - \vtX_2) $ \ding{220} Ponto base $ \vtX_2 $ e direção $ 
          (\vtX_1 - \vtX_2) $ apontando de $ \vtX_2 $ para $ \vtX_1 $ e escalonada por $ \theta $
        \end{itemize}
        
        \item Conjuntos afins 
        \begin{itemize}
          \item Um conjunto $ \fdA\subset \fdR^n$ é afim \ding{220} para qualquer $ \vtX_1, \vtX_2 \in \fdA\subset 
          \fdR$ e $ \theta \in \fdR $, o segmento de reta $ \vtY =  \theta\vtX_1 + (1-\theta)\vtX_2 \in \fdA$
          \item Combinação afim \ding{220} $ \vtY = \theta_1 \vtX_1 + \theta_2 \vtX_2 + \ldots + \theta_k \vtX_k \in 
          \fdA\subset \fdR $, com $ \theta_1 + \theta_2 + \ldots + \theta_k = 1 $
        \end{itemize}
      \end{itemize}
    \end{column}
    \begin{column}{0.33\linewidth}
      \centering
      \begin{tikzpicture}[thick,scale=0.8]
        \draw[help lines] (0,0) grid (5,4);
        \coordinate (x1) at (1,1);
        \node at (x1) [above] {$ \vtX_1 $};
        \coordinate (x2) at (4,3);
        \node at (x2) [above] {$ \vtX_2 $};
        \draw[black] (x1) -- (x2);
        \fill[black] ($(x1)$) circle (2pt);
        \fill[black] ($(x2)$) circle (2pt);
        \fill[red] ($(x1)!.4!(x2)$) circle (2pt);
        \node at ($(x1)!.4!(x2)$) [below right] {$ \vtY $};
      \end{tikzpicture}
      
      \vspace*{\baselineskip}
      
      \begin{tikzpicture}[thick,scale=0.8]
        \draw[help lines] (0,0) grid (5,4);
        \coordinate (x1) at (1,1);
        \node at (x1) [above] {$ \vtX_1 $};
        \coordinate (x2) at (4,3);
        \node at (x2) [above] {$ \vtX_2 $};
        \draw[red] ($(x1)!-.3!(x2)$) -- ($(x1)!1.3!(x2)$);
        \draw[black] (x1) -- (x2);
        \fill[black] ($(x1)$) circle (2pt);
        \fill[black] ($(x2)$) circle (2pt);
        \fill[red] ($(x1)!.4!(x2)$) circle (2pt);
        \node at ($(x1)!.4!(x2)$) [below right] {$ \vtY $};
      \end{tikzpicture}
    \end{column}
  \end{columns}
\end{frame}

\begin{frame}{Conjuntos convexos}
  \begin{columns}
    \begin{column}{0.63\linewidth}
      \begin{itemize}\addtolength{\itemsep}{\baselineskip}
        \item Conjuntos convexos
        \begin{itemize}
        \item Um conjunto $ \fdA \subset \fdR^n$ é convexo caso um segmento entre $\vtX_{1}, \vtX_{2} \in \fdA$ e $ 0
          \leq \theta \leq 1 $, pertença a $ \fdA$, ou seja, $ \theta \vtX_{1}+(1-\theta) \vtX_{2} \in \fdA$
          \item Em termos geométricos, um conjunto convexo é um conjunto sem buracos ou reentrâncias
        \end{itemize}
        
        \item Invólucro convexo (\textit{convex hull})
        \begin{itemize}
          \item Dado um conjunto $ \fdA$, o invólucro convexo de $ \fdA $ é o menor conjunto convexo que engloba $ \fdA 
          $, sendo denotado por $ \Conv{\fdA} = \{ \theta_{1} \vtX_{1} + \cdots + \theta_{k} \vtX_{k} \mid \vtX_{i} \in 
          \fdA, \theta_{i} \geq 0, i=1, \cdots, k, \theta_{1} + \cdots + \theta_{k} = 1\} $
        \end{itemize}
      \end{itemize}
    \end{column}
    \begin{column}{0.33\linewidth}
      \begin{tikzpicture}[thick,scale=0.8]
        \draw[help lines] (0,0) grid (5,4);
        \coordinate (x1) at (1,1);
        \coordinate (x2) at (3,0.5);
        \coordinate (x3) at (4,2);
        \coordinate (x4) at (3,3);
        \coordinate (x5) at (1.5,2);
        \path[draw,color=Blue!50,fill=Blue!25] (x1) -- (x2) -- (x3) -- (x4) -- (x5) -- cycle;
        \fill[draw=none,fill=Blue!50] (x1) circle (2pt);
        \fill[draw=none,fill=Blue!50] (x2) circle (2pt);
        \fill[draw=none,fill=Blue!50] (x3) circle (2pt);
        \fill[draw=none,fill=Blue!50] (x4) circle (2pt);
        \fill[draw=none,fill=Blue!50] (x5) circle (2pt);
        \fill[draw=none,fill=Blue!50] (2,1) circle (2pt);
        \fill[draw=none,fill=Blue!50] (3,1.5) circle (2pt);
        \fill[draw=none,fill=Blue!50] (3.5,2) circle (2pt);
        \fill[draw=none,fill=Blue!50] (2, 1.8) circle (2pt);
        \fill[draw=none,fill=Blue!50] (3, 2.3) circle (2pt);
      \end{tikzpicture}

      \vspace*{\baselineskip}

      \begin{tikzpicture}[thick,scale=0.8]
        \draw[help lines] (0,0) grid (5,4);
        \coordinate (x1) at (1,1);
        \coordinate (x2) at (3,0.5);
        \coordinate (x3) at (3,1.5);
        \coordinate (x4) at (4,2);
        \coordinate (x5) at (3,3);
        \coordinate (x6) at (1.5,2);
        \path[draw,color=Blue!50,fill=Blue!25] (x1) -- (x2) -- (x3) -- (x4) -- (x5) -- (x6) -- cycle;
        \fill[draw=none,fill=Blue!50] (x1) circle (2pt);
        \fill[draw=none,fill=Blue!50] (x2) circle (2pt);
        \fill[draw=none,fill=Blue!50] (x3) circle (2pt);
        \fill[draw=none,fill=Blue!50] (x4) circle (2pt);
        \fill[draw=none,fill=Blue!50] (x5) circle (2pt);
        \fill[draw=none,fill=Blue!50] (x6) circle (2pt);
        \fill[draw=none,fill=Blue!50] (2,1) circle (2pt);
        \fill[draw=none,fill=Blue!50] (2.5,0.8) circle (2pt);
        \fill[draw=none,fill=Blue!50] (3.5,2) circle (2pt);
        \fill[draw=none,fill=Blue!50] (2, 1.8) circle (2pt);
        \fill[draw=none,fill=Blue!50] (3, 2.3) circle (2pt);
      \end{tikzpicture}
    \end{column}
  \end{columns}
\end{frame}

\subsection{Operações sobre conjuntos que preservam convexidade}

\begin{frame}{Operações sobre conjuntos que preservam convexidade}
  \vspace*{-0.5\baselineskip}
  \begin{itemize}\footnotesize\setlength{\AuxWidth}{\widthof{Multiplicação por escalar}}
    \item \makebox[\AuxWidth][l]{Interseção} \ding{220} se $ \fdA_{\alpha}$ é convexo para todo $\alpha \in \fdR$, então %
    \begin{equation}\label{eq_oper_inter}
      \bigcap_{\alpha \in \fdR} \fdA_{\alpha} \quad \quad \quad \text{também é convexo}
    \end{equation} %

    \item \makebox[\AuxWidth][l]{Multiplicação por escalar} \ding{220} se $ \fdA \subseteq \fdR^n$ é convexo e $\alpha \in \fdR$, então %
    \begin{equation}\label{eq_oper_escalar}
      \alpha \fdA = \{ \alpha \vtX \mid \vtX \in \fdA \} \quad \quad \quad \text{também é convexo}
    \end{equation} 
    
    \item \makebox[\AuxWidth][l]{Translação} \ding{220} se $ \fdA \subseteq \fdR^n$ é convexo e $\vtAlpha \in \fdR^n$, então %
    \begin{equation}\label{eq_oper_trans}
      \fdA + \vtAlpha = \{ \vtX + \vtAlpha \mid \vtX \in \fdA \} \quad \quad \quad \text{também é convexo}
    \end{equation} %

    \item \makebox[\AuxWidth][l]{Soma} \ding{220} se $\fdA_{1}$ e $\fdA_{2}$ são convexos, então %
    \begin{equation}\label{eq_oper_soma}
      \fdA_{1} + \fdA_{2} = \{ \vtX + \vtY \mid \vtX \in \fdA_{1}, \vtY \in \fdA_{2}\} \quad \quad \quad \text{também é convexo}
    \end{equation} %
      
    \item \makebox[\AuxWidth][l]{Produto cartesiano} \ding{220} se $\fdA_{1}$ e $\fdA_{2}$ são convexos, então %
    \begin{equation}\label{eq_oper_pcart}
      \fdA_{1} \times \fdA_{2} = \{ (\vtX, \vtY) \mid \vtX \in \fdA_{1}, \vtY \in \fdA_{2}\}\quad \quad \quad \text{também é convexo}
    \end{equation} %
  \end{itemize}
\end{frame}

\subsection{Hiperplanos, cones e desigualdades generalizadas}

\begin{frame}{Hiperplanos, cones e desigualdades generalizadas}
  \begin{columns}
    \begin{column}{0.48\linewidth}
      \begin{itemize}
        \item Hiperplano \ding{220} conjunto de pontos, podendo ser escrito como
        \begin{equation}\label{eq_hiperplano}
        \{ \vtX \mid \Transp{\vtA}(\vtX-\vtX_{0}) = 0) \}
        \end{equation}
        onde $ \vtA \in \fdR^n$, $ \vtA \neq \vtZero$, e $\vtX_{0}$ determina o offset do hiperplano. Um hiperplano divide o espaço em dois semi-espaços
        
        \item Semi-espaço \ding{220} conjunto da forma  
        \begin{equation}\label{eq_halfspace}
        \{ \vtX \mid \Transp{\vtA}(\vtX-\vtX_{0}) \leq 0) \}
        \end{equation}
        onde $ \vtA \in \fdR^n$, $ \vtA \neq \vtZero$
      \end{itemize} 
    \end{column}

    \hfill

    \begin{column}{0.48\linewidth}
      % TODO: Adicionar figuras
      \begin{block}{TODO}
        Adicionar figuras
      \end{block}
    \end{column}
  \end{columns}
\end{frame}

\begin{frame}{Hiperplanos, cones e desigualdades generalizadas}
  \begin{columns}
    \begin{column}{0.48\linewidth}
      \begin{itemize}
        \item Cone é conjunto de pontos tais que $ \forall \vtX \in \fdA \subset \fdR^n$ e $ \theta \geq 0, \theta \in \fdR_{+}$
        \begin{equation}\label{eq_cone}
          \theta \vtX \in \fdA, \forall \vtX \in \fdA
        \end{equation}
       
        \item Cone Convexo é um conjunto que é simultaneamente um cone e convexo, ou seja, para qualquer $\vtX_{1}, \vtX_{2} \in \fdA \subset \fdR^n$ e $ \theta_{1}, \theta_{2} \geq 0, \theta_1$
        \begin{equation}\label{eq_cone_conv}
          \theta_{1} \vtX_{1} + \theta_{2} \vtX_{2} \in \fdA
        \end{equation}
      \end{itemize}
    \end{column}

    \hfill

    \begin{column}{0.48\linewidth}
      % TODO: Adicionar figuras
      \begin{block}{TODO}
        Adicionar figuras
      \end{block}
    \end{column}
  \end{columns}
\end{frame}

\begin{frame}{Conjuntos Convexos}
  \begin{columns}
    \begin{column}{0.48\linewidth}
      \begin{itemize}
        \item A bola Euclidiana com o centro em $\vtX_{c}$ e raio $r$ é representada por
        \begin{equation}\label{eq_ball}
          B(\vtX_{c}, r) = \{ \vtX_{c} + r\vtU \mid \NormTwo{\vtU} \leq 1 \}
        \end{equation}
        
        \item O elipsoide com o centro em $\vtX_{c}$ é um conjunto representado na forma
        \begin{equation}\label{eq_elipsoide}
          \calE = \{ \vtX \mid \Transp{(\vtX-\vtX_{c})} \Inv{\mtP} (\vtX-\vtX_{c}) \leq 1) \}
        \end{equation}
        onde $ \mtP \in \fdA \subset \fdS^n_{++}$, ou seja, $ \mtP $ é uma matriz simétrica positiva definida
      \end{itemize}
    \end{column}

    \hfill

    \begin{column}{0.48\linewidth}
      % TODO: Adicionar figuras
      \begin{block}{TODO}
        Adicionar figuras
      \end{block}
    \end{column}
  \end{columns}
\end{frame}

\begin{frame}{Conjuntos Convexos}
  \begin{itemize}
    \item Poliedro
    \begin{itemize}
      \item O poliedro é definido como um conjunto de igualdades lineares e inequações.
      
      \begin{equation}\label{eq_poliedro}
      A x \preceq b, C x = d
      \end{equation}
      onde $\preceq$ é o símbolo que representa desiguadade componente a componente entre vetores.
      
      \item Dessa forma, a representação do poliedro pode escrita como: 
      
      \begin{equation}\label{eq_poliedro_2}
      P = \{ x \mid A x \preceq b, C x = d \}
      \end{equation}
    \end{itemize}
    
  \end{itemize}
\end{frame}


\begin{frame}{Conjuntos Convexos}
  
  \begin{itemize}
    \item Norma da Bola com centro $x_{c}$ e raio $r$ é representado por:
    
    \begin{equation}\label{eq_norm_ball}
    \{ x \mid \Norm{x-x_c} \leq r \}
    \end{equation}
    onde $\Norm{\cdot}$ é a norma.
    
    \item A norma do cone pode ser definido como:
    
    \begin{equation}\label{eq_norm_cone}
    \{ (x,t) \in  \fdR^n+1 \mid \Norm{x} \leq t \}
    \end{equation}
  \end{itemize}
  
\end{frame}



\begin{frame}{Conjuntos Convexos}
  
  \begin{itemize}
    \item PSD (Cone positivo semidefinido) é um cone formado a parti das matrizes positivas semidefinidas que utiliza a notação:
    
    \begin{equation}\label{eq_psd}
    \fdA^n_+ = \{ X \in \fdA^n \mid X \succ 0 \}
    \end{equation}
    onde $\fdA^n$ denota o conjunto de matrizes simetricas. Dessa forma o conjunto $\fdA^n_+$ é um cone positivo 
    semidefinido, se $ \theta_{1}, \theta_{2} \geqslant 0$ e $A, B \in \fdA^n_+$, assim $ \theta_{1} A + \theta_{2} B 
    \in \fdA^n_+$.  
  \end{itemize}
\end{frame}



\begin{frame}{Conjuntos Convexos}
  
  \begin{itemize} 
    \item Desigualdade generalizadas
    \begin{itemize}
      \item Um cone pode ser usado para definir a desigualdade generalizada, que é similar a relação de ordem apresentada em $\fdR$. Desta forma, para um cone $ K $ onde $ K \subseteq \fdR^n$, as desigualdades não estritas e estritas $ \preceq_{K}$ e $\prec_{K}$, respectivamente são definidas da seguinte maneira:
      
      \begin{subequations}
        \begin{align}
        x \preceq_{K} y \Leftrightarrow y - x \in K\\
        x \prec_{K} y \Leftrightarrow y - x \in \textbf{int(}K\textbf{)}
        \end{align}
      \end{subequations}
      
      onde $\textbf{int($K$)}$ representa o interior do conjunto $K$.
      
      \item Propriedades
      \begin{itemize}
        \item Se $x \preceq_{K} y$ e $u \preceq_{K} v$, então $ x + u \preceq_{K} y + v$
        \item  Se $x \preceq_{K} y$ e $y \preceq_{K} u$, então $ x \preceq_{K} u$
        \item  Se $x \preceq_{K} y$ e $y \preceq_{K} x$, então $ x = y$
        \item  Se $x \prec_{K} y$ e $u \preceq_{K} v$, então $ x + u \prec_{K} y + v$
      \end{itemize}
      
    \end{itemize}
    
  \end{itemize}
\end{frame}

\section{Propriedades básicas}

\subsection{Propriedades básicas e exemplos}

\begin{frame}{Propriedades básicas}
  \begin{itemize}
    \item Uma função $f: \bbR^n \rightarrow \bbR^n$ é convexa somente se o domínio de $f$ ($\dom f$) for convexo e 
    $\forall x, y \in \dom f$, e $0 \leq \theta \leq 1$ \eqref{eq:condConvexidade} for verdade.
    \begin{equation}
      f(\theta x + (1-\theta) y) \leq \theta f(x) + (1-\theta) f(y)
      \label{eq:condConvexidade}
    \end{equation}
    \item Observando a condição \eqref{eq:condConvexidade}, temos que em uma função $f(\cdot)$ convexa, qualquer 
    segmento de reta entre os pontos $(x, f(x))$ e $(y, f(y))$ estará acima de $f(\cdot)$ entre os valores $x$ e $y$, 
    como mostrado na figura abaixo.
    \begin{figure}
      \centering
%      \includegraphics[scale=0.2]{convexFunction}
    \end{figure}
  \end{itemize}
\end{frame}

\begin{frame}
  \frametitle{Propriedades básicas}
  \begin{itemize}
    \item Dizemos que uma função $f(\cdot)$ é estritamente convexa se a desigualdade ocorrer sempre para $0 < \theta < 
    1$ e $x \neq y$, ou seja,
    \begin{equation}
      f(\theta x + (1-\theta) y) < \theta f(x) + (1-\theta) f(y)
      \label{eq:condConvexidadeEstrita}
    \end{equation}
    \item Define-se uma função $f(\cdot)$ como côncava, se $-f(\cdot)$ for convexa.
    \item Funções afins são convexas e côncavas, visto que
    \begin{equation}
      f(\theta x + (1-\theta) y) = \theta f(x) + (1-\theta) f(y)
      \label{eq:condFuncaoAfim}
    \end{equation}
    para $0 \leq \theta \leq 1$.
  \end{itemize}
\end{frame}

\begin{frame}
  \frametitle{Propriedades básicas}
  \begin{itemize}
    \item Uma função $f(\cdot)$ é convexa se e somente se ela é convexa quando restrita a qualquer linha que intercepta 
    seu domínio. Ou seja, $f(x)$ é convexa se e somente se para todo $x \in \dom f$ e para todo $v$, a função $g(t) = 
    f(x + tv)$ é convexa no domínio $\{t | x + tv \in \dom f\}$.
    \item Esta propriedade é muito importante, pois pode-se avaliar a convexidade de uma função, restringindo-a a uma 
    linha.
  \end{itemize}
\end{frame}

\begin{frame}
  \frametitle{Extensão}
  \begin{itemize}
    \item As vezes é conveniente que o domínio da função $f(\cdot)$ seja estendido para todo $\bbR^n$, atribuindo o 
    valor $\infty$ para $f(x)$ quando $x \notin \dom f$.
    \item Assim, a extensão de $f(x)$, $\tilde{f}: \bbR^n \rightarrow \bbR \cup \infty$ é definida como:
    \begin{equation}
      \tilde{f}(x) = \left\{\begin{array}{ll}
        f(x) & x \in \dom f \\
        \infty & x \notin \dom f
      \end{array} \right.
    \end{equation}
    \item Essa extensão é útil pelo fato de simplificar a notação e de não haver a necessidade de se explicitar o seu 
    domínio.
    \item De modo análogo, podemos estender funções côncavas adicionando $-\infty$ a seu domínio.
  \end{itemize}
\end{frame}

\begin{frame}{Condições de primeira ordem}
  \begin{itemize}
    \item Seja $f(\cdot)$ uma função diferenciável, então $f(\cdot)$ é convexo se e somente se o $\dom f$ for convexo e 
    \begin{equation}
      f(y) \geq f(x) + \nabla \Transp{f(x)} (y-x) \,\,\,\,\,\, \forall x,y \in \dom f
      \label{eq:condPrimeiraOrdem}
    \end{equation} 
    \item Note que o segundo termo de \eqref{eq:condPrimeiraOrdem} é uma função afim em relação a $y$ e que é igual a 
    aproximação de primeira ordem de Taylor de $f(y)$ em torno de $x$.
    \begin{figure}
      \centering
%      \includegraphics[scale=0.3]{firstOrderCondition}
    \end{figure}
  \end{itemize}
\end{frame}

\begin{frame}{Condições de primeira ordem}
  \begin{itemize}
    \item Assim, se a aproximação de primeira ordem de Taylor de uma função sempre representar um subestimador global, 
    a função $f(\cdot)$ é convexa.
    \item De \eqref{eq:condPrimeiraOrdem}, podemos ver que a partir da informação local ($x$ e $\nabla f(x)$) nós 
    podemos derivar a informação global (subestimador global) de $f(\cdot)$.
    \item Podemos ainda deduzir de \eqref{eq:condPrimeiraOrdem} que se $\nabla f(x) = 0$, então $f(y) \geq f(x)$, o que 
    implica dizer que o ponto $(x, f(x))$ representa o mínimo global de $f(\cdot)$.
    \item Para que $f(\cdot)$ seja estritamente convexo, temos que a desigualdade em \eqref{eq:condPrimeiraOrdem} tem 
    que ser estrita, ou seja,
    \begin{equation}
      f(y) > f(x) + \nabla \Transp{f(x)} (y-x) \,\,\,\,\,\, \forall x,y \in \dom f \text{ e } x \neq y
      \label{eq:condPrimeiraOrdemEstrita}
    \end{equation} 
  \end{itemize}
\end{frame}

\begin{frame}{Condições de segunda ordem}
  \begin{itemize}
    \item Seja $f(\cdot)$ duas vezes diferenciável. $f(\cdot)$ é convexo se e somente se o $\dom f$ é convexo e sua 
    Hessiana é semi-positiva definida $\forall x \in \dom f$, ou seja,
    \begin{equation}
      \nabla^2 f(x) \succeq 0.
      \label{eq:condSegundaOrdem}
    \end{equation}
    \item No caso das funções reais \eqref{eq:condSegundaOrdem} se reduz a $f''(x) \geq 0$, o que implica dizer que 
    $f(\cdot)$ é não decrescente.
    \item Geograficamente, a condição \eqref{eq:condSegundaOrdem} pode ser interpretada como o gráfico da função 
    possuir curvatura sempre positiva.
    \item De modo análogo, temos que $f(\cdot)$ é côncava se e somente se
    \begin{equation}
      \nabla^2 f(x) \preceq 0.
      \label{eq:condSegundaOrdemConcava}
    \end{equation}
    \item Se $\nabla^2 f(x) \succ 0$, então $f(\cdot)$ é estritamente convexa, contudo, \alert{a recíproca não é 
    verdadeira!}
  \end{itemize}
\end{frame}

\begin{frame}{Conjunto de subníveis}
  \begin{itemize}
    \item O conjunto de subnível $\alpha$ de uma função $f: \bbR^n \rightarrow \bbR$ é definido como:
    \begin{equation}
      C_\alpha = \{x \in \dom f | f(x) \leq \alpha\}
      \label{eq:conjuntoSubnivel}
    \end{equation}
    \item Conjuntos de subnível de uma função convexa são convexos para qualquer $\alpha$.
    \item Prova:
    Seja $x, y \in \fdA_\alpha$, então $f(x) \leq \alpha$ e $f(y) \leq \alpha$. Assim, de \eqref{eq:condConvexidade}, 
    temos que:
    \begin{align*}
      f(\theta x + (1-\theta) y) & \leq \theta f(x) + (1-\theta) f(y) \\
      & \leq \theta \alpha + (1-\theta) \alpha \leq \alpha 
    \end{align*}
    Logo, $\theta x + (1-\theta) y \in \fdA_\alpha$
    \item \alert{A recíproca não é verdadeira!} Uma função pode ter todos seus subníveis convexos e não ser convexa.
  \end{itemize}
\end{frame}

\begin{frame}
  \frametitle{\normalsize Epígrafo e Hipografo}
  \begin{itemize}
    \item O grafo de uma função $f: \bbR^n \rightarrow \bbR$ é definido como
    \begin{equation}
      \{ (x, f(x)) | x \in \dom f \},
    \end{equation}
    que é um subconjunto de $\bbR^{n+1}$.
    \item O epígrafo de $f(\cdot)$ é definido como:
    \begin{equation}
      \epi f = \{ (x, t) | x \in \dom f, f(x) \leq t \}
    \end{equation}
    que é um subconjunto de $\bbR^{n+1}$.
    \begin{figure}
      \centering
%      \includegraphics[scale=0.2]{epigraph}
    \end{figure}
  \end{itemize}
\end{frame}

\begin{frame}
  \frametitle{\normalsize Epígrafo e Hipografo}
  \begin{itemize}
    \item O epígrafo faz a ligação entre funções e conjuntos convexos.
    \item Uma função $f(\cdot)$ é convexa se e somente se seu epígrafo representar um conjunto convexo.
    \item O hipografo de uma função é definido como sendo
    \begin{equation}
      \hypo f = \{ (x, t) | x \in \dom f, t \leq f(x) \}
    \end{equation}
    \item A função $f(\cdot)$ é côncava se e somente seu hipografo for \alert{convexo}.
  \end{itemize}
\end{frame}

\begin{frame}
  \frametitle{\normalsize Desigualdade de Jensen e extensões}
  \begin{itemize}
    \item 
  \end{itemize}
\end{frame}

\subsection{Operações que preservam convexidade}

\subsection{Função conjugada}

\subsection{Funções quasi-convexas}

\subsection{Funções log-côncavas e log-convexas}

\subsection{Convexidade e inequações generalizadas}

% Local Variables:
% TeX-master: "Otimizacao.tex"
% End:

\createpartandshowpage{Problemas de otimização convexa}
% !TeX root = Otimizacao.tex
% !TeX encoding = UTF-8
% !TeX spellcheck = pt_BR
% !TeX program = pdflatex
\section{Problemas de otimização convexa}

\subsection{Otimização convexa}

\subsection{Otimização Linear}

\subsection{Otimização Quadrática}

\subsection{Programação Geométrica}

\subsection{Programação Semidefinida}

\subsection{Restrições por inequações generalizadas}

\subsection{Otimização Vetorial}


\createpartandshowpage{Aplicações}
% !TeX root = Otimizacao.tex
% !TeX encoding = UTF-8
% !TeX spellcheck = pt_BR
% !TeX program = pdflatex
\section{Problemas de otimização}

\subsection{Algoritmo de {\itshape water filling}}

\begin{frame}{Algoritmo de \textit{water filling}}
  \begin{itemize}
    \item O objetivo do water filling é distribuir potências entre os links de uma sistema telecomunicação objetivando a maximizaçao da vazão de dados, uma vez que a qualidade do link dependente da potência utilizada na comunicação, dessa forma:
    \begin{equation}
    \min - \sum\limits_{i = 1}^n \log(\alpha_{i} + x_{i}) \text{sujeito a}\begin{cases}
    x  \succeq 0, \\
    \textbf{1}^{T}x = 1
    \end{cases}
    \end{equation}
    onde $\alpha_{i} > 0$, $x_{i}$ representa a potência alocada ao canal $i$.
  \end{itemize}
\end{frame}

\begin{frame}{Algoritmo de \textit{water filling}}
  
  \begin{itemize}
    \item Adotando o multiplicador lagrangeano $ \lambda^{*} \in \fdR^n$ para $x \succeq 0$ e o multiplicador $v^{*} \in \fdR$ para $\textbf{1}^{T}x = 1$, o lagrangeano apresenta a seguinte forma:
    \begin{equation}
    H(x^{*},\lambda^{*}, v^{*}) = - \sum\limits_{i = 1}^n \log(\alpha_{i} + x_{i}) - \lambda^{*} x + v^{*}(\textbf{1}^{T}x - 1)
    \end{equation}
    
    \item Derivando em seguida utilizando as KKT: 
    
    \begin{subequations}
      \begin{align}		
      \frac{H(x^{*}, \lambda^{*}, v^{*})}{\partial x^{*}} = - \frac{1}{(\alpha_{i}+x_{i})} - \lambda^{*} + v^{*} = 0\\
      \frac{H(x^{*}, \lambda^{*}, v^{*})}{\partial \lambda^{*}} = x = 0\\
      \frac{H(x^{*}, \lambda^{*}, v^{*})}{\partial v^{*}} = \textbf{1}^{T}x - 1 = 0
      \end{align}
    \end{subequations}
    
    
  \end{itemize}
\end{frame}

\begin{frame}{Algoritmo de \textit{water filling}}
  
  \begin{itemize}
    \item Utilizando as KKT:
    \begin{subequations}
      \begin{align}		
      x_{*} \succeq 0, \\
      \textbf{1}^{T}x^{*} = 1, \\
      \lambda^{*} \succeq 0,\\
      \lambda^{*}_{i}x^{*}_{i} = 0, \\
      - \frac{1}{(\alpha_{i}+x^{*}_{i})} - \lambda^{*}_{i} + v^{*} = 0
      \end{align}
    \end{subequations}
    
    \item Isolando $\lambda^{*}_{i}$ e substituindo em $\lambda^{*}_{i}x^{*}_{i}$:
    
    \begin{subequations}
      \begin{align}		
      x^{*}_{i}( v^{*} - \frac{1}{\alpha_{i} + x^{*}_{i}}) = 0, \\
      v^{*} - \frac{1}{\alpha_{i} + x^{*}_{i}} \succeq 0
      \end{align}
    \end{subequations}
    
    
  \end{itemize}
\end{frame}

\begin{frame}{Algoritmo de \textit{water filling}}
  
  \begin{itemize}
    \item Isolando $x^{*}_{i}$ da expressão $v^{*} \succeq  \frac{1}{\alpha_{i} + x^{*}_{i}}$ temos:
    
    \begin{equation}
    x^{*}_{i} \geq \frac{1}{v_{*}} - \alpha_{i} 
    \end{equation}
    
    
    \item Entretando $x^{*}_{i}$ não pode ser negativo no caso $ v^{*} > \frac{1}{\alpha_{i}}$, assim:
    
    \begin{equation}\label{eq_vec_orthonormal}
    x^{*}_{i} = \begin{cases}
    \frac{1}{v^{*}} - \alpha_{i}, & v^{*} < \frac{1}{\alpha_{i}} \\
    0, & v^{*} \geq \frac{1}{\alpha_{i}}
    \end{cases}
    \end{equation}
    
  \end{itemize}
\end{frame}

\subsection{Regressão}

\begin{frame}
	\frametitle{Regressão}
	\begin{itemize}
		\item Sejam dois vetores $\vtX = \Transp{\left[x_1 \,\, x_2 \,\, \dots \,\, x_N \right]}$ e $\vtY = \Transp{\left[y_1 \,\, y_2 \,\, \dots \,\, y_N \right]}$, que representam respectivamente a entrada e a saída de um determinado sistema, em que $\vtX, \vtY \in \mathbb{R}^{N \times 1}$;
		\item A análise de regressão visa estimar relacionamento entre $\vtX$ e $\vtY$.
		\item Em um caso geral, a análise de regressão visa encontrar os $M$ parâmetros $\vtAlpha = \Transp{\left[\alpha_1 \,\, \alpha_2 \,\, \dots \,\, \alpha_M \right]}$ de uma função $f(\cdot)$ qualquer, de modo que $f(\vtX, \vtAlpha)$ seja o mais próximo de $\vtY$, ou seja,
		\[
			\vtY \approx f(\vtX, \vtAlpha)
		\]
		\item Em outras palavras, para se estimar os parâmetros $\vtAlpha$, podemos minimizar erro quadrático médio entre $\vtY$ e $f(\vtX, \vtAlpha)$;
		\item Assim, temos que:
		\[
			e = \dfrac{1}{N}\Norm{\vtY - f(\vtX, \vtAlpha)}^2
		\]
	\end{itemize}
\end{frame}

\begin{frame}
	\frametitle{Regressão}
	\begin{itemize}		
		\item Assim, escrevendo o problema de otimização, temos que:
		\begin{align*}
			\dot{\vtAlpha} &= \ArgMin{\vtAlpha}{e} \\
			&= \ArgMin{\vtAlpha}{\dfrac{1}{N}\Norm{\vtY - f(\vtX, \vtAlpha)}^2}
		\end{align*}
		\item Como $N$ é uma constante positiva, então o problema de otimização pode ser simplificado como
		\[
			\dot{\vtAlpha} = \ArgMin{\vtAlpha}{\Norm{\vtY - f(\vtX, \vtAlpha)}^2}
		\]
		\item Como a norma ao quadrado é uma função convexa, então, desde que a função $f(\vtX, \vtAlpha)$ seja convexa, o problema de otimização em questão pode ser resolvido pelo método \textit{Least-squares}.
		\item Na literatura existem vários modelos de regressão, para diversos casos.
	\end{itemize}
\end{frame}

\subsection{Regressão polinomial}
\begin{frame}
	\frametitle{Regressão polinomial}
	\begin{itemize}
		\item Um dos modelos de regressão mais utilizados é a polinomial;
		\item É um caso geral da regressão linear;
		\item Neste caso, seja um polinômio de ordem $M$, então
		\begin{equation}
			\vtAlpha = \Transp{\left[\alpha_0 \,\, \alpha_1 \,\, \dots \,\, \alpha_M \right]}
			\label{eq:mtARegPol}
		\end{equation}
		 é o vetor de coeficientes com $M+1$ elementos;
		\item Temos que a função $f(\vtX, \vtAlpha)$ é definida como:
		\[
			f(\vtX, \vtAlpha) = \mtV \vtAlpha
		\]
		em que $\mtV \in \mathbb{R}^{N \times M}$ é uma matriz de Vandermonde, gerada pelos elementos de $\vtX$
		{\scriptsize
		\begin{equation}
			\mtV = \begin{bmatrix} 
				1 & x_1 & x_1^2 & \dots & x_1^M \\
				1 & x_2 & x_2^2 & \dots & x_2^M \\
				\vdots & \vdots & \vdots & \ddots & \vdots \\
				1 & x_N & x_N^2 & \dots & x_N^M
			\end{bmatrix}
			\label{eq:mtVRegPol}
		\end{equation}}
	\end{itemize}
\end{frame}

\begin{frame}
	\frametitle{Regressão polinomial}
	\begin{itemize}
		\item Nesse caso, $\mtV \vtAlpha$ é convexo.
		\item Assim, denotando o resultado da função objetivo por $L(\vtAlpha)$, temos que:
		\begin{align*}
			L(\vtAlpha) &= \Norm{\vtY - \mtV \vtAlpha}^2 \\
			&= \Transp{(\vtY - \mtV \vtAlpha)}(\vtY - \mtV \vtAlpha) \\
			&= (\Transp{\vtY} - \Transp{\vtAlpha} \Transp{\mtV})(\vtY - \mtV \vtAlpha) \\
			&= \Transp{\vtY}\vtY - \Transp{\vtY}\mtV\vtAlpha - \Transp{\vtAlpha}\Transp{\mtV}\vtY + \Transp{\vtAlpha}\Transp{\mtV}\mtV \vtAlpha
		\end{align*}
		\item Assim, para encontrar o valor de $\vtAlpha$ que minimiza o valor de $L(\vtAlpha)$, temos que \alert{$\dfrac{\partial L(\vtAlpha)}{\partial \vtAlpha} = 0$}, logo:
		\begin{align*}
			\dfrac{\partial L(\vtAlpha)}{\partial \vtAlpha} &= - \Transp{\mtV}\vtY - \Transp{\mtV}\vtY + (\mtV\Transp{\mtV} + \Transp{\mtV}\mtV)\vtAlpha = 0
		\end{align*}
	\end{itemize}
\end{frame}

\begin{frame}
	\frametitle{Regressão polinomial}
	\begin{itemize}
		\item Como $\mtV\Transp{\mtV}$ é simétrico, então, temos que:
		\[
			2\Transp{\mtV}\mtV\vtAlpha - 2\Transp{\mtV}\vtY = 0
		\]
		\[
			\Transp{\mtV}\mtV\vtAlpha = \Transp{\mtV}\vtY
		\]
		\[
			\vtAlpha = \Inv{\left(\Transp{\mtV}\mtV\right)}\Transp{\mtV}\vtY
		\]
		\[
			\boxed{\vtAlpha = \PInv{\mtV}\vtY}
		\]
	\end{itemize}
\end{frame}

\subsection{Regressão polinomial multivariada}
\begin{frame}
	\frametitle{Regressão polinomial multivariada}
	\begin{itemize}
		\item Em alguns casos, a variável de saída depende de uma associação de $K$ de valores de entrada;
		\item Matematicamente, podemos escrever que cada elemento do vetor de saída $\vtY = \Transp{\left[y_1 \,\, y_2 \,\, \dots \,\, y_N \right]}$ depende de uma linha da matriz de dados de entrada
		\[
			\vtX = \begin{bmatrix} 
				x_{11} & x_{12} & \dots & x_{1K} \\
				x_{21} & x_{22} & \dots & x_{2K} \\
				\vdots & \vdots & \ddots & \vdots \\
				x_{N1} & x_{N2} & \dots & x_{NK}
			 \end{bmatrix}
		\]
		\item Podemos reescrever nosso problema como um problema de regressão polinomial normal, alterando apenas a matriz $\mtV$.
	\end{itemize}
\end{frame}

\begin{frame}
	\frametitle{Regressão polinomial multivariada}
	\begin{itemize}
		\item No caso da regressão polinomial multivariada, seja um polinômio de grau $M$, o vetor de coeficientes de $KM + 1$ elementos e é dado por:
		\begin{equation}
			\vtAlpha = \Transp{\left[\alpha_0 \,\, \alpha_{11} \,\, \dots \,\, \alpha_{M1} \,\, \dots \,\, \alpha_{12} \,\, \dots \,\, \alpha_{M2} \,\, \dots \,\, \alpha_{1K} \,\, \dots \,\, \alpha_{MK} \right]}
			\label{eq:mtARegPolMV}
		\end{equation}
		\item Temos que a matriz $\mtV$ é dada por:
		\begin{equation}
			{\setcounter{MaxMatrixCols}{20}
			\mtV = \begin{bmatrix}
				1 & x_{11} & x_{11}^2 & \dots & x_{11}^M & x_{12} & x_{12}^2 & \dots & x_{12}^M & \dots & x_{1K} & \dots & x_{1K}^M \\
				%
				1 & x_{21} & x_{21}^2 & \dots & x_{21}^M & x_{22} & x_{22}^2 & \dots & x_{22}^M & \dots & x_{2K} & \dots & x_{2K}^M \\
				%
				\vdots & \vdots & \vdots & \ddots & \vdots & \vdots & \vdots & \ddots & \vdots & \vdots & \vdots & \ddots & \vdots \\
				%
				1 & x_{N1} & x_{N1}^2 & \dots & x_{N1}^M & x_{N2} & x_{N2}^2 & \dots & x_{N2}^M & \dots & x_{NK} & \dots & x_{NK}^M 
			\end{bmatrix}}
			\label{eq:mtVRegPolMV}
		\end{equation}
		\item A solução do problema de otimização se dá da mesma forma que no caso monovariável.
	\end{itemize}
\end{frame}

\subsection{Regressão logística}
\begin{frame}{Regressão logística}
	\begin{itemize}
		\item Considere agora que a variável de saída $\vtY$ é qualitativa dicotômica (sucesso (1) ou falha (0));
		\item Nosso objetivo é saber qual a probabilidade de sucesso de $y_i$ dado um conjunto de $K$ dados de entrada $\vtX_i = [x_{i1} \,\, x_{2i} \,\, \dots \,\, x_{iK}]$, para $i \in (1, N)$;
		\item Como $y_i$ possui uma distribuição binomial de probabilidades, podemos calcular o valor da probabilidade de sucesso $(y_i = 1)$ a partir da média de várias amostras de $y_i$ sob as mesmas condições $\vtX_i$.
		\item Assim temos que o vetor de probabilidades de sucesso é dado por:
		\[
			\vtP_y = \begin{bmatrix}
				P(y_1 | \vtX_1) \\
				P(y_2 | \vtX_2) \\
				\vdots \\
				P(y_N | \vtX_N)
			\end{bmatrix} = 
			\begin{bmatrix}
				\Mean{y_1 | \vtX_1} \\
				\Mean{y_2 | \vtX_2} \\
				\vdots \\
				\Mean{y_N | \vtX_N}
			\end{bmatrix}
		\]
	\end{itemize}
\end{frame}

\begin{frame}
	\frametitle{Regressão logística}
	\begin{itemize}
		\item Da maneira como os dados de saída estão, o modelo de regressão polinomial não se mostra adequado, visto que o mesmo pode inferir valores de probabilidade maiores que 1 ou menores que 0, o que seria incorreto!
		\item Para utilizar a regressão polinomial seria necessário aplicar uma função de mapeamento em $\vtP_y$ de forma a gerar $\vtQ_y$, que pudesse admitir quaisquer valores reais. Ou seja,
		\begin{equation}
			\vtP_y \xrightarrow{f} \vtQ_y, \,\,\,\,\, \vtP_y \in (0,1) \textrm{ e } \vtQ_y \in \mathbb{R}
		\end{equation}
		\item Uma função que resolve este problema é a função logística, dada por:
		\begin{equation}
			f(p) = \ln\left(\dfrac{p}{1-p}\right)
		\end{equation}
		\item Assim, temos que:
		\begin{equation}
			\vtQ_y = \ln\left(\dfrac{\vtP_y}{1 - \vtP_y}\right)
		\end{equation}
	\end{itemize}
\end{frame}

\begin{frame}
	\frametitle{Regressão logística}
	\begin{itemize}
		\item Agora, podemos aplicar uma regressão polinomial considerando os dados de entrada $\mtX = \Transp{\left[\vtX_1 \,\, \vtX_2 \,\, \dots \,\, \vtX_K\right]}$ e a saída $\vtQ_y$. Logo,
		\begin{equation}
			\vtQ_y = \mtV \vtAlpha,
		\end{equation}
		em que $\vtAlpha$ e $\mtV$ são definidos de acordo com \eqref{eq:mtARegPol} e \eqref{eq:mtVRegPol} se for utilizada a regressão polinomial monovariada $(K=1)$ ou \eqref{eq:mtARegPolMV} e \eqref{eq:mtVRegPolMV} se for utilizada a regressão polinomial multivariada.
		\item Assim, temos que os coeficientes da regressão polinominal são dados por:
		\begin{equation}
			\vtAlpha = \PInv{\mtV} \vtQ_y \,\,\,\, \Longrightarrow \,\,\,\, \vtAlpha = \PInv{\mtV} \ln\left(\dfrac{\vtP_y}{1 - \vtP_y}\right)
		\end{equation}
		\item Assim, temos que a função de regressão que relaciona $\mtX$ com $\vtP_y$ é dada por:
		\begin{equation}
			\boxed{\vtP_y = \dfrac{\exp\left(\mtV \vtAlpha\right)}{1 + \exp\left(\mtV \vtAlpha\right)}}
		\end{equation}
	\end{itemize}
\end{frame}
% % !TeX root = Otimizacao.tex
% !TeX encoding = UTF-8
% !TeX spellcheck = pt_BR
% !TeX program = pdflatex
\subsection{Beamforming Adaptativo e Robusto}

% Introducao
\begin{frame}{Introdução}
		\begin{itemize}
		\item Uma das abordagens recentes para aumentar a capacidade e desempenho de sistemas celulares é a utilização de antenas inteligentes (\textit{smart}) em [type=exam];
		\item Considerando essa abordagem, a tecnologia SDMA tornou-se recentemente uma dos conceitos chave da $3^a$ e subsequente gerações de sistemas celulares;
		\item Especificamente, \textit{beamforming} adaptativo no UL utilizando diversas tem se mostrado capaz de eficientemente mitigar a interferência co-canal e melhorar consideravelmente as características do sistema em termos de:
		\begin{itemize}
		  \item Capacidade;
		  \item Desempenho;
		  \item Cobertura.
		\end{itemize}
		\end{itemize}
\end{frame}

\begin{frame}{O que é \textit{beamforming} adaptativo ?}
	\begin{itemize}
		\item Cada sensor em uma série (\textit{array}) de antenas possui um coeficiente de peso ajustável;
		\item Esses coeficientes podem ser utilizados pela BS para colocar o feixe principal de seu padrão de radiação na direção do usuário de interesse, enquanto rejeita a interferência co-canal dos outros usuários com um nulo em suas direções;
		\item Para emparelhar de forma ótima o feixe principal com o usuário de interesse, a BS necessita de um conhecimento preciso da \textbf{assinatura espacial do canal};
		\item Devido à dificuldade de obter essa informação com precisão, é necessária a utilização de um \textit{beamforming} adaptativo robusto tanto aos \textcolor{red}{\textbf{erros no canal}} como também à \textcolor{red}{\textbf{contaminação por interferência}}~\cite[Cap. 6]{Trees2002} \cite[Cap.9]{Kaiser2005}.
	\end{itemize}
\end{frame}

\begin{frame}{Beamforming adaptativo - Intro.}
	\begin{itemize}
		\item Um esquema básico com 6 antenas é ilustrado na Figura \ref{fig:basic_beamforming}
		\begin{figure}
			\centering
      \includegraphics[width=0.60\linewidth]{Figs/basic_beamforming}\label{fig:basic_beamforming}
	  \end{figure}
	  \item A saída do beamformer é:
	  \[ y(k) = \Herm{\vtW}\vtX(k),
	  \]
	  onde $k$ é o índice temporal, $\vtW = [w_1,\ldots,w_M]^T$ é o vetor complexo $M\times 1$ de peso da antena, $\vtX(k)=[x_1(k),\ldots,x_M(k)]^T$ é o vetor $M\times 1$ de entrada e $M$ é o número de antenas na base (no array).
	\end{itemize}
\end{frame}

\begin{frame}{Beamforming adaptativo - Intro.}
	\begin{itemize}
		\item O vetor de entrada $\vtX(k)$ é dado por
		\[
			\vtX(k) = \vtS_s(k) + \vtI(k) + \vtN(k),
		\]
		onde $\vtS_s(k),\vtI(k), \vtN(k)$ são componentes estatisticamente independentes do sinal desejado, da interferência de múltiplo acesso e do ruído, respectivamente;
		\item Em cenários com desvanecimento lento, o vetor $\vtS_s(k)$ pode ser modelado como 
		\[
		\vtS_s(k)=s(k)\vtA_s,
		\]
		onde $s(k)$ é a forma de onda complexa do usuário desejado e $\vtA_s$ é a assinatura espacial $M\times 1$ que especifica a frente de onda do usuário;
	\end{itemize}
\end{frame}

\begin{frame}{Beamforming adaptativo - Intro.}
	\begin{itemize}
		\item As matrizes de covariância do sinal e da interferência mais ruído são dadas por
		\begin{align}
		  \mtR_s &= \fdE\{\vtS_s(k)\Herm{\vtS}_s(k)\},\\
		  \mtR_{i+n} &= \fdE\left\{(\vtI(k)+\vtN(k))\Herm{(\vtI(k)+\vtN(k))}\right\};
		\end{align}
		\item Considerando desvanecimento lento, temos então:
		\begin{equation}
		  \mtR_s = \sigma^2 \vtA_s \Herm{\vtA}_s,
		\end{equation}
		onde $\sigma^2 = \fdE\{\|s(k)\|^2\}$;
	\end{itemize}
\end{frame}

\begin{frame}{Beamforming adaptativo - Intro.}
	\begin{itemize}
		\item A relação sinal-interferência mais ruído (SINR) pode então ser definida como:
		\begin{align}
		  SINR &= \frac{\fdE\{\|\Herm{\vtW}\vtS_s(k)\|^2\}}{\fdE\{\|\Herm{\vtW}(\vtI(k) + \vtN(k)))\|^2\}};\\
		  SINR &= \frac{\Herm{\vtW}\mtR_s\vtW}{\Herm{\vtW}\mtR_{i+n}\vtW};
		\end{align}
		\item Considerando desvanecimento lento, temos então:
		\begin{equation}
		  SINR = \frac{\sigma^2\|\Herm{\vtW}\vtA_a\|^2}{\Herm{\vtW}\mtR_{i+n}\vtW};
		\end{equation}
	\end{itemize}
\end{frame}

\begin{frame}{Beamforming adaptativo - MVDR }
        \begin{itemize}
                \item Um critério importante é o da não distorção , onde é requerido que na ausência de ruido,
                \begin{align}
                  \Herm{\vtW}\mtR_s\vtW &= 1\\
                  \text{ou}\nonumber\\
                  \Herm{\vtW}\vtA_s &= 1;
                \end{align}
                \item Outro critério importante é o da miníma variância da saída com a presença do ruido, dado por:
                \begin{align}
                  \fdE\{\|y(k)\|^2\} = \Herm{\vtW}\mtR_{i+n}\vtW;
                \end{align}
                \item Ao considerar ambos os critérios, temos o chamado beamforming MVDR.
        \end{itemize}
\end{frame}

\begin{frame}{Beamforming adaptativo - MVDR}
	\begin{itemize}
		\item O problema de otimização pode ser escrito como:
		\begin{equation}
			\label{eq:util_max}
			\begin{array}{ll}
				\minimize_\vtW & \Herm{\vtW}\mtR_{i+n}\vtW, \\
				\text{suj. a} & \Herm{\vtW}\mtR_{s}\vtW = 1;
			\end{array}
		\end{equation}
		\item O beamforming MVDR também pode ser compreendido como a maximização da SINR com o critério da não distorção;
		\item O lagrangeano pode ser escrito como:
		\begin{equation}
		  \stL(\vtW,\lambda) = \Herm{\vtW}\mtR_{i+n}\vtW - \lambda\left(\Herm{\vtW}\mtR_{s}\vtW - 1\right)
		\end{equation}
	\end{itemize}
\end{frame}

\begin{frame}{Beamforming adaptativo - MVDR}
	\begin{itemize}
		\item As condições de KKT do problema são dadas por:
		\begin{align}
		  \frac{\partial\stL(\vtW,\lambda)}{\partial\vtW} &= 0,\\
		  \lambda\left(\Herm{\vtW}\mtR_{s}\vtW - 1\right) &= 0;
		\end{align}
		\item Temos portanto:
		\begin{align}
		  \frac{\partial\stL(\vtW,\lambda)}{\partial\vtW} &= \mtR_{i+n}\vtW - \lambda\mtR_s\vtW\\
		  \mtR_{i+n}\vtW &= \lambda\mtR_s\vtW\label{seq:gen_eigen};
		\end{align}
		\item A equação \eqref{seq:gen_eigen} pode ser interpretada como um problema de autovalor generalizado, onde $\lambda$ é o autovalor e  $\vtW$ o autovetor.
	\end{itemize}
\end{frame}

\begin{frame}{Beamforming adaptativo - MVDR}
	\begin{itemize}
		\item A solução da equação \eqref{seq:gen_eigen} corresponde ao menor autovalor generalizado ($\lambda$) das matrizes $\{\mtR_{i+n},\mtR_s\}$ e ao autovetor ($\vtW$) relacionado ao maior autovalor de $\Inv{\mtR}_{i+n}\mtR_s$;
		\item 
	\end{itemize}
\end{frame}
% !TeX root = Otimizacao.tex
% !TeX encoding = UTF-8
% !TeX spellcheck = pt_BR
% !TeX program = pdflatex
\section{Problemas de Otimização}

\subsection{Projeto de Arranjo de Antenas}

\begin{frame}
\frametitle{Motivação}
\begin{itemize}
\item Em um arranjo de antenas, as sa\'idas de in\'umeros elementos emissores s\~ao linearmente combinadas de modo a gerar um padr\~ao de radiação resultante.

\item O arranjo resultante tem um padr\~ao direcional que depende dos pesos relativos (fatores de escala) usados no processo de combinação.

\item O objetivo do projeto de fatores de escala é escolher os pesos de modo a gerar um diagrama de irradiação desejado.

\item O diagrama de irradiação pode ser ajustado de modo a aumentar o ganho na direção de um usuário ou reduzir o ganho na direção de maior interfer\^encia.

% Inserir figura de um padrão diagrama de irradiação aleatório.

\end{itemize}

\end{frame}

\subsection{Modelagem Matemática}
% osciladores harmônicos
\begin{frame}
\frametitle{Osciladores Harmônicos}
\begin{itemize}

% \item A unidade básica de uma antena transmissora é o oscilador harmônico isotrópico, que emite ondas eletromagnéticas de frequ\^encia $\omega$ e comprimento de onda $\lambda$.

\item Osciladores harmônicos isotrópicos constituem a unidade básica das antenas transmissoras. Estes elementos emitem ondas eletromagnéticas com frequ\^encia $\omega$ e comprimento de onda $\lambda$.

\item O campo elétrico gerado por um oscilador em um ponto $P$ do espa\c{c}o, localizado a uma distância $d$ da antena, é dado por

\begin{equation}
\frac{1}{d}\cdot\mathbf{Re}\left[z \cdot \exp \left( j\omega t - \frac{2\pi d}{\lambda} \right) \right]
\end{equation}

\item Observe que $z \in \mathbb{C}$,  é um parâmetro de projeto denominado peso da antena. Este fator dimensiona a magnitude e a fase do campo elétrico.

\end{itemize}
\end{frame}

\begin{frame}
\frametitle{Arranjo de osciladores}
\begin{itemize}

% \item Um \'unico elemento de irradiação n\~ao é capaz de atender a todos os requisitos técnicos necessários a uma determinada aplicação, sendo assim necessária a combinação de vários elementos osciladores, arranjados nos espa\c{c}o e interconectados entre si.

\item Arranjo de osciladores é uma combinação de vários elementos de irradiação, com uma determinada distribuição espacial e interconectados entre si.

\item Considere que colocamos $n$ osciladores nas posi\c{c}\~oes $p_{k} \in \mathbb{R}^{3}$, $k = 1, \ldots, n$. Cada oscilador é associado a um peso complexo $z_{k}$. Assim, o campo elétrico total recebido em um ponto $p \in \mathbb{R}^{3}$ é dado por

\begin{equation}
E = \mathfrak{Re}\left[ \exp\left( j\omega t \right) \cdot \sum_{k = 1}^{n} \frac{1}{d_{k}} z_{k} \cdot \exp \left( - \frac{2 \pi j d_{k}}{\lambda}  \right) \right]
\end{equation}

\item onde $d_{k} = \Norm{p - p_{k}}$ é a distância entre os pontos $p$ e $p_{k}$, onde $ k = 1, \ldots, n$.

\end{itemize}
\end{frame}


% diagrama de um arranjo linear
\begin{frame}
\frametitle{Diagrama de um Arranjo Linear}
\begin{itemize}
%\item Assumindo que os osciladores formam um arranjo linear, isto é,  eles s\~ao colocados em pontos equidistantes ao longo do eixo x, suas posi\c{c}\~oes s\~ao dadas por $p_k = k\cdot e_{1}$, onde $k = 1, \ldots, n$ e $e_{1} = \left(1, 0, 0 \right)$ é o primeiro vetor unitário. 

\item Considera-se que o ponto $p$ em análise está muito distante do arranjo de osciladores. Desse modo, sua posição será dada por $p = ru$, onde $u \in \mathbb{R}^{3}$ é um vetor unitário que determina a direção  e $r$ é um escalar que especifica a distância da origem.

\item Para um arranjo linear, o campo elétrico $E$ depende apenas do ângulo $\phi$ entre o vetor e o ponto distante $p$

\begin{equation}
E \approx \mathfrak{Re} \left(\frac{1}{r}\cdot \exp \left( j\omega t - \frac{2\pi r}{\lambda}\right) \right) \cdot D_{z}\left( \phi \right)
\end{equation}

\item observe que $\phi$ é o â ngulo entre os vetores $u$ e $e_{1}$.

\item A função $D_z : [0, 2\pi]  \to \mathbb{C}$ é chamada de diagrama da antena. Esta função depende da escolha do vetor de pesos complexos $z = \left(z_{1}, \ldots, z_{n} \right)$

\begin{equation}
D_{z} \left( \phi \right) = \sum_{k = 1}^{n} z_{k} \cdot \exp \left( \frac{2 \pi j k \cos{\phi}}{\lambda} \right)
\end{equation}

\end{itemize}
\end{frame}

\subsection{Modelagem do Diagrama de Antenas}

\begin{frame}
\frametitle{Modelagem do Diagrama de Antenas}
\begin{itemize}
\item O quadrado do módulo do diagrama de antena, $|D_{z}\left( \phi \right)|$, é proporcional a direção da densidade de energia emitida pela antena.

\item \'E de grande interesse modelar a magnitude do diagrama $\| D_{z}\left( \cdot \right) |$, por meio da escolha do vetor de pesos, $z$,  de modo a atender os requisitos de diretividade do sistema.

\item Um requerimento t\'ipico estabelece que a antena transmite bem ao longo de uma determinada direção. Ou seja, a energia é concentrada ao longo de direção alvo, $\phi_{alvo}$, enquanto é reduzida numa outra regi\~ao.

\item Outro requerimento clássico considera a minimização da pot\^encia de rui\'ido térmico gerado pela antena.

% Inserir figura indicando as regiões de incidência

\end{itemize}
\end{frame}

\begin{frame}
\frametitle{Normalização}
\begin{itemize}

\item A energia enviada ao longo da direção alvo deve ser normalizada. 

\begin{equation}
\mathfrak{Re} \left( D_{z}\left( \phi_{alvo} \right) \right) \geq 1
\end{equation}

\item  N\~ao se modifica a direção da distribuição de energia ao multiplicar-se todos os pesos por um n\'umero complexo n\~ao nulo. 

\item Esta é uma restrição afim nas partes reais e imaginárias da variável $z \in \mathbb{C}^{n}$

\end{itemize}
\end{frame}

\begin{frame}
\frametitle{Atenuação do Lóbulo Lateral}
\begin{itemize}

\item Define-se como banda de passagem o intervalo angular $\left[ - \Phi, \Phi \right]$ onde se pretende concentrar a energia; a banda de parada, corresponde aos pontos fora deste intervalo.

\item Para garantir o cumprimento do requerimento de concentração de energia, estabelece-se que

\begin{equation}
|\phi| \geq \Phi \iff |D_{z}\left( \phi \right)| \leq \delta
\end{equation}

\item onde $\delta$ é o n\'ivel desejado de atenuação na banda de parada ou como o n\'ivel do lóbulo lateral.

\item Em vez de considerar um intervalo cont\'inuo, é feita uma discretização dessa restrição

\begin{equation}
|D_{z}\left( \phi_{i} \right)| \leq \delta, i = 1, \ldots, N
\end{equation}

\item os angulos $\phi_{1}, \ldots, \phi_{N}$ s\~ao regularmente espa\c{c}ados na banda de parada.

\end{itemize}
\end{frame}

\begin{frame}
\frametitle{Limitação da Pot\^encia do Ru\'ido Térmico}

\begin{itemize}
\item Em algumas situa\c{c}\~oes é importante controlar a pot\^encia do ru\'ido térmico gerado pelas antenas.

\item Verifica-se que esta pot\^encia é proporcional \`a norma Euclidiana do vetor complexo $z$

\begin{equation}
\Gamma \propto \sqrt{\sum_{i = 1}^{n} |z_{i}|^{2}}
\end{equation}

\end{itemize}

\end{frame}

\begin{frame}
\frametitle{Dilema entre Atenuação do Lóbulo Lateral e a Pot\^encia do Ru\'ido Térmico}

Um problema de otimização t\'ipico envolveria

\begin{itemize}
\item uma restrição de normalização, que determina um valor unitário no diagrama de magnitude em uma direção espec\'ifica :
\begin{equation}
\mathfrak{Re} \left( D_{z}\left( \phi_{alvo}\right) \geq 1 \right)
\end{equation}
\item uma restrição quanto ao n\'ivel de atenuação do lóbulo lateral :
\begin{equation}
|D_{z} \left( \phi_{z} \right) | \leq \delta, i = 1, \ldots, N
\end{equation}
\item uma restrição na pot\^encia do ru\'ido térmico:
\begin{equation}
\Norm{z}_{2} \leq \gamma
\end{equation}
\end{itemize}
\end{frame}


\subsection{Método dos M\'inimos Quadrados}

\begin{frame}

\frametitle{Formulação do problema de otimização}

\begin{itemize}

\item Uma curva de trade-off t\'ipica pode ser obtida comparando-se o n\'ivel de ru\'ido térmico $\gamma$ para um dado valor de atenuação do lóbulo lateral $\delta$.

\item Cada ponto da curva $\left( \delta, \gamma \right)$ pode ser obtido solucionando-se o problema de otimização

\begin{equation*}
\begin{aligned}
& \underset{z}{\text{minimize}}
&& \Norm{z}_{2} \\
& \text{subject to}
&& \mathfrak{Re}(D_{z}(\phi_{alvo})) \geq 1 \\
&
&& |D_{z}(\phi_{i})| \leq \delta, i = 1, \ldots, N
\end{aligned}
\end{equation*}

\end{itemize}

\end{frame}

\begin{frame}
\frametitle{Projeto de Arranjo de Antenas usando o Método dos M\'inimos Quadrados}
\begin{itemize}
\item A ideia básica para a resolução deste problema consiste em penalizar as restri\c{c}\~oes, isto é, definir um parâmetro de trade-off, $\mu$ e se reescrever o problema como

\begin{equation*}
\begin{aligned}
& \underset{z}{\text{minimize}}
&& \Norm{z}_{2}^{2} + \mu \sum_{i = 1}^{n}|D_{z}(\phi_{i})|^{2} \\
& \text{subject to}
&& \mathfrak{Re}(D_{z}(\phi_{alvo})) \geq 1 \\
\end{aligned}
\end{equation*}

\item Lembrando que a função é linear em $z$, pode-se afirmar que se trata de um problema de m\'inimos quadrados com restri\c{c}\~oes lineares. 
\end{itemize}
\end{frame}

\begin{frame}
\frametitle{Implementação no CVX}

\begin{table}[!tb]
\centering
\caption{Parâmetros de Simulação}
\centering
\small
\begin{tabular}{l|c}
\hline
\hline
\textbf{Parâmetro} & \textbf{Valor}\\
\hline
N\'umero de antenas & 10\\
\hline
Comprimento de onda & 8\\
\hline
\^Angulo alvo & $0$\\
\hline
Banda de Passagem & $\left[- \pi / 6, \pi / 6\right]$\\
\hline
Parâmetro de Discretização & 90\\
\hline
Parâmetro de Trade-off & 0,5\\
\hline
\hline
\end{tabular}
\end{table}

\end{frame}

% \begin{frame}
% \frametitle{Implementação no CVX}
%\lstinputlisting[language=Matlab]{Codes/beamOpt_LS.m}
% \end{frame}

\begin{frame}

\frametitle{Diagrama de Irradiação}
\begin{figure}
\centering
\includegraphics[width = 0.6\columnwidth]{Figs/pattern_LS}
\end{figure}

\end{frame}

\begin{frame}
\frametitle{Curva $(\delta, \gamma)$}
\begin{figure}
\centering
\includegraphics[width = 0.6\columnwidth]{Figs/tradeoffCurve}
\end{figure}
\end{frame}

\begin{frame}
\frametitle{Projeto de Arranjo de Antenas usando SOCP}
O método Second Order Cone Programming (SOCP) permite realizar o projeto de arranjo de antenas de dois modos diferentes

\begin{itemize}
\item Minimização do ru\'ido térmico para um dado n\'ivel dos lóbulos laterais
\item Minimização da atenuação do lóbulo lateral 
\end{itemize}

\end{frame}

\begin{frame}
\frametitle{Minimização do ru\'ido térmico para um dado n\'ivel dos lóbulos laterais}

O problema de minimização da pot\^encia do ru\'ido térmico submetido a uma restrição do n\'ivel dos lóbulos laterais pode ser escrito como

\begin{equation*}
\begin{aligned}
& \underset{z \in \mathbf{C}^{n},\delta}{\text{minimize}}
&& \sum_{i = 1}^{n} \Norm{z_{i}}_{2} \\
& \text{subject to}
&& \mathfrak{Re}(D_{z}(\phi_{alvo})) \geq 1 \\
&
&& |D_{z}(\phi_{i})| \leq \delta, i = 1, \ldots, m
\end{aligned}
\end{equation*}
\\

As N restri\c{c}\~oes representam cones de segunda ordem em função das variáveis de decis\~ao, uma vez que eles envolvem restrição de magnitude em um vetor complexo que depende de modo afim dessas variáveis.

\end{frame}

\begin{frame}
\frametitle{Restrição de Magnitude em Vetores Complexos Afins}
\begin{itemize}
\item Muitos problemas que envolvem variáveis complexas e restri\c{c}\~oes de magnitude podem ser solucionados usando SOCP.

\item A ideia básica consiste em escrever a magnitude de um n\'umero complexo como uma norma euclidiana

\begin{equation*}
|z| = \sqrt{z_{R}^{2} + z_{I}^{2}} = \Norm{\dfrac{z_{R}}{z_{I}}}_{2}
\end{equation*}

\item Por exemplo, considere o n\'umero complexo f(x), onde $x \in \mathbf{R}^{n}$ é uma variável de projeto e que a função $f: \mathbf{R}^{n} \rightarrow \mathbf{C}$ é afim. Os valores dessa função podem ser escritos como $f(x) = (a_{R}^{T}x + b_{R}) + j (a_{I}^{T}x + b_{I})$.

\item Uma restrição de magnitude da forma $|f(x)| \leq t$ pode ser escrita como um cone de segunda ordem em (x,t)

\begin{equation*}
\Norm{\dfrac{a_{R}^{T}x + b_{R}}{a_{I}^{T}x + b_{I}}}_{2} \leq t
\end{equation*}
\end{itemize}
\end{frame}

% \begin{frame}
% \frametitle{Implementação no CVX}
%\lstinputlisting[language=Matlab]{Codes/beamOpt_SOCP.m}
% \end{frame}

\begin{frame}
\frametitle{Resultado}
\begin{figure}
\centering
\includegraphics[width = 0.6\columnwidth]{Figs/result_SOCP}
\end{figure}

Padr\~ao de radiação resultante para a pot\^encia de ruido térmico m\'inima dado um limite de 0,4 no lóbulo lateral considerando N = 90 pontos.
\end{frame}

\begin{frame}
\frametitle{ Minimização da atenuação do lóbulo lateral}
Este problema tem como objetivo minimizar o n\'ivel de atenuação dos lóbulos laterais, $\delta$, dado o requerimento de normalização $\mathfrak{Re}(D_{z}(0)) \geq 1$. 

Esta análise pode ser realizada a partir do seguinte  formula\c{a}\~ao de um SOCP 

\begin{equation*}
\begin{aligned}
& \underset{z \in \mathbf{C}^{n},\delta}{\text{minimize}}
&&\delta \\
& \text{subject to}
&& \mathfrak{Re}(D_{z}(0)) \geq 1 \\
&
&& |D_{z}(\phi_{i})| \leq \delta, i = 1, \ldots, m
\end{aligned}
\end{equation*}
\end{frame}

%{\begin{frame}
%\frametitle{Implementação no CVX}
%\lstinputlisting[language=Matlab]{Codes/beamOpt_SOCP_02.m}
%\end{frame}}

\begin{frame}
\frametitle{Resutado}
\begin{figure}
\centering
\includegraphics[width = 0.6\columnwidth]{Figs/result_SOCP_02}
\end{figure}
\end{frame}

% Local Variables:
% TeX-master: "Otimizacao.tex"
% End:

\createpartandshowpage{Revisão de ferramentas matemática}
% !TeX root = Otimizacao.tex
% !TeX encoding = UTF-8
% !TeX spellcheck = pt_BR
\section{Revisão de álgebra linear}

\subsection{Vetores e operações com vetores}

\begin{frame}{Campo de um espaço vetorial~\cite{Horn2012}}
  \begin{itemize}
    \item O campo escalar subjacente a um espaço vetorial é o conjunto de escalares onde os elementos do vetor são definidos
    \begin{itemize}
      \item Normalmente o campo dos números reais $ \fdR $ ou complexos $ \fdC $
      \item Alternativamente poderia ser o dos racionais, inteiros módulo algum número primo, etc.
    \end{itemize}
    \item Um campo precisa ser fechado sob duas operações binárias (i.e., que recebem dois operandos)
    \begin{itemize}
      \item Adição
      \item Multiplicação
    \end{itemize}
    \item As operações precisam ser associativas, comutativas e possuir um elemento neutro
    \item Elementos inversos precisam existir para todos elemento sob a adição e multiplicação, exceto para a identidade sobre a multiplicação
    \item Multiplicação precisa ser distributiva sobre adição
  \end{itemize}
\end{frame}


\begin{frame}
  \frametitle{Vetores e (sub)espaços vetoriais}
  \begin{itemize}
    \item O \textbf{\alert{espaço vetorial linear}} $\fdR^n$ é o conjunto de todos os vetores $\vtX$ de dimensão $n\times 1$ juntamente com as operações de adição de vetores e multiplicação por um escalar \cite[cap. 2]{Strang1988}

    \item Um \textbf{\alert{vetor}} $\vtX \in \fdR^n$ é normalmente representado como
    \begin{equation}
      \vtX = \left[\begin{matrix} x_1 \\ x_2 \\ \vdots \\ x_n\end{matrix}\right], \text{ com } \Transp{\vtX} = \left[\begin{matrix} x_1 & x_2 & \ldots & x_n\end{matrix}\right]
    \end{equation}
    

    \item Embora a maior parte dos conceitos se estenda facilmente para vetores complexos, consideraremos apenas vetores reais (exceto se explicitamente mencionado)
  \end{itemize}
\end{frame}

\begin{frame}{Vetores e (sub)espaços vetoriais}
  \begin{itemize}
    \item Há oito propriedades que precisam ser satisfeitas por um espaço vetorial\footnote{Relações similares podem ser definidas para números complexos em $\fdC^n$} \cite[Ex. 2.1.5]{Strang1988}
    {\scriptsize \begin{subequations}\label{eq_vector_space}
      \begin{align}
        \vtX + \vtY &= \vtY + \vtX & \quad \text{\scriptsize(Comutatividade da adição)} \\
        \vtX + (\vtY + \vtZ) &= (\vtX + \vtY) + \vtZ & \quad \text{\scriptsize(Associatividade da adição)} \\
        \exists \vtZero \Rightarrow \vtX + \vtZero &= \vtX & \quad \text{\scriptsize(Elemento neutro da adição)} \\
        \exists -\vtX \Rightarrow \vtX + (-\vtX) &= \vtZero & \quad \text{\scriptsize(Elemento simétrico)} \\
        1\cdot\vtX &= \vtX & \quad \text{\scriptsize(Elemento neutro da multiplicação)} \\
        (\alpha_1\cdot\alpha_2)\cdot\vtX &= \alpha_1\cdot(\alpha_2\cdot\vtX) & \quad \text{\scriptsize(Associatividade da multiplicação)} \\
        \alpha_1\cdot(\vtX + \vtY) &= \alpha_1\cdot\vtX + \alpha_1\cdot\vtY & \quad \text{\scriptsize(Distributividade)} \\
        (\alpha_1 + \alpha_2)\cdot\vtX &= \alpha_1\cdot\vtX + \alpha_2\cdot\vtX & \quad \text{\scriptsize(Distributividade)}
      \end{align}
    \end{subequations}}
    
    \item Um \textbf{\alert{subespaço vetorial linear}} $\fdS \subset \fdR^n$ é um conjunto não-vazio que satisfaz:
    \begin{itemize}
      \item $\forall \vtX, \vtY \in \fdS \Rightarrow \vtX + \vtY \in \fdS$
      \item $\forall \vtX \in \fdS$ e $\alpha \in \fdR \Rightarrow \alpha\vtX \in \fdS$
    \end{itemize}
    
    \item Um subespaço é portanto um subconjunto \alert{fechado} sob as operações de adição e multiplicação por escalar
  \end{itemize}
\end{frame}

\begin{frame}{Vetores e (sub)espaços vetoriais}
  \begin{itemize}
    \item Considerando o espaço vetorial $ \fdR^3 $, alguns exemplos simples de subespaços vetoriais $ \fdS \subset \fdR^3 $ são \alert{retas} e \alert{planos}
    
    \item Um vetor $ \vtV \in \fdR^3 $ age como suporte de uma reta que representa um subespaço $ \fdS_1 \subset \fdR^3 $
    \begin{itemize}
      \item Quaisquer vetores $ \vtV_1, \vtV_2 \in \fdS_1$ encontram-se sobre a reta e, portanto, podem ser escritos como $ \vtV_1 = \alpha \vtV $ e $ \vtV_2 = \beta \vtV $, para algum $ \alpha, \beta \in \fdR $
      \item Além disso, $ \vtV_3= \vtV_1 + \vtV_2 = (\alpha + \beta)\vtV $ está também sobre a reta
    \end{itemize}
  \end{itemize}
  \begin{columns}
    \begin{column}{0.48\linewidth}
      \centering
      \begin{tikzpicture}[
        font=\scriptsize,
        >=stealth',]
        \begin{axis}[
          scale=0.75,
          xlabel=$ x $,ylabel=$ y $,zlabel=$ z $,
          xtick=\empty,
          ytick=\empty,
          ztick=\empty,
          mesh/interior colormap={bluewhite}{color=(Gray) color=(White)},
          colormap/blackwhite,]      

          \addplot3[surf,domain=0:1] {x + y};
          \addplot3 [->,DarkGreen] plot coordinates {(0,0,0) (1,1,2)};
          \addplot3 [->,DarkRed] plot coordinates {(0.5,0.5,1) (0.8,0.8,1.6)};
          \addplot3 [->,DarkCyan] plot coordinates {(0.5,0.5,1) (0.3,0.3,0.6)};
          \addplot3 [->,DarkMagenta] plot coordinates {(0,0,0) (0.1,0.1,0.2)};
        \end{axis}
      \end{tikzpicture}
    \end{column}
    \hfill
    \begin{column}{0.48\linewidth}
      \centering
      \begin{tikzpicture}[
        font=\scriptsize,
        >=stealth',]
        \begin{axis}[
          scale=0.75,
          xlabel=$ x $,ylabel=$ y $,zlabel=$ z $,
          xtick=\empty,
          ytick=\empty,
          ztick=\empty,
          mesh/interior colormap={bluewhite}{color=(Gray) color=(White)},
          colormap/blackwhite,]      

          \addplot3[surf,domain=0:1] {x + y};
          \addplot3 [->,DarkGreen] plot coordinates {(0,0,0) (0.2,0.8,1)};
          \addplot3 [->,DarkGreen] plot coordinates {(0,0,0) (0.8,0.2,1)};
        \end{axis}
      \end{tikzpicture}
    \end{column}
  \end{columns}
\end{frame}

\begin{frame}
  \frametitle{Norma de vetores}
  \begin{itemize}
    \item A \textbf{\alert{norma}} de um vetor é uma generalização do conceito de comprimento ou magnitude de um vetor
    \item A \textbf{\alert{norma}} $\Norm{\vtX}$ de $\vtX \in \fdR^n$ é uma função $f : \fdR^n \rightarrow \fdR$ que satisfaz \cite[pág. 46]{Chen1999}:
    {\small\begin{subequations}
      \begin{align}
        \Norm{\vtX} &\geq 0 & \text{(Não-negatividade)} \\
        \Norm{\vtX} &= 0 \Leftrightarrow \vtX = \vtZero & \text{(Elemento neutro)} \\
        \Norm{\alpha\vtX} &= \Abs{\alpha}\Norm{\vtX}, \forall \alpha \in \fdR & \text{(Escalabilidade)} \\
        \Norm{\vtX + \vtY} &\leq \Norm{\vtX} + \Norm{\vtY}, \forall \vtX, \vtY \in \fdR^n & \text{(Desigualdade triangular)}
      \end{align}
    \end{subequations}}
    \item Uma família de normas (norma-$p$) que atende às propriedades é dada por
    \begin{equation}\label{eq_norm_p}
      \NormP{\vtX} = \left(\sum\limits_{i = 1}^n \Abs{x_i}^p\right)^{\frac{1}{p}}, \quad p \in \fdN_+
    \end{equation}
    \item Um caso caso de interesse é \textbf{\alert{norma-1}} (ou norma $\ell_1$) onde $p = 1$ em \eqref{eq_norm_p}
    \item Outro caso de interesse é \textbf{\alert{norma euclidiana}} (ou norma-2 ou ainda norma $\ell_2$) onde $p = 2$ em \eqref{eq_norm_p}
  \end{itemize}
\end{frame}

\begin{frame}
  \frametitle{Norma euclidiana e produto interno~\cite[cap. 1]{Rugh1996}}
  \begin{itemize}\small
    \item A norma euclidiana atende ainda as seguintes propriedades:
    \begin{itemize}
      \item $\Abs{\Transp{\vtX}\vtY} \leq \NormTwo{\vtX}\NormTwo{\vtY}$ (Desigualdade de Cauchy-Schwartz)
      \item $\underset{1 \leq i \leq n}{\max} \Abs{x_i} \leq \NormTwo{\vtX} \leq \sqrt{n} \underset{1 \leq i \leq n}{\max}\Abs{x_i}$
    \end{itemize}
    \item O \textbf{\alert{produto interno}} ou \textbf{\alert{produto escalar}} $\InnerProd{\vtX}{\vtY}$ entre dois vetores $\vtX$ e $\vtY \in \fdR^n$ é escrito como
    \begin{equation}\label{eq_inner_prod}
      \InnerProd{\vtX}{\vtY} = \Transp{\vtX}\vtY = \sum\limits_{i=1}^n x_i y_i = \Transp{\vtY}\vtX = \InnerProd{\vtY}{\vtX}
    \end{equation}
    \item A norma euclidiana guarda as seguintes relações com o produto interno
    \begin{subequations}
      \begin{align}
        \InnerProd{\vtX}{\vtX} = \Transp{\vtX}\vtX &= \sum\limits_{i=1}^n x_i x_i = \sum\limits_{i=1}^n x^2_i = \sum\limits_{i=1}^n \Abs{x_i}^2 = \NormTwo{\vtX}^2 \\
        \InnerProd{\vtX}{\vtY} = \Transp{\vtX}\vtY &= \NormTwo{\vtX}\NormTwo{\vtY}\cos\theta_{\angle^{\vtX}_{\vtY}}
      \end{align}
    \end{subequations}
    \item Pare vetores pertencentes a $\fdC$, $\Transp{(\cdot)}$ é substituído por $\Herm{(\cdot)}$ que representa o conjugado-transposto de um vetor com a conjugação denotada por $\Conj{(\cdot)}$
  \end{itemize}
\end{frame}

\subsection{Independência linear, bases e representações}

\begin{frame}
  \frametitle{Independência linear \cite[cap. 3]{Chen1999}}
  \begin{itemize}
    \item Os vetores $\vtX_1, \vtX_2, \ldots, \vtX_m \in \fdR^n$ são ditos \textbf{\alert{linearmente independentes}} (L.I.) se e somente se, para $\alpha_1, \alpha_2, \ldots, \alpha_m \in \fdR$,
    \begin{equation}\label{eq_indep_lin}
      \alpha_1\vtX_1 + \alpha_2\vtX_2 + \ldots + \alpha_m\vtX_m = \vtZero \Leftrightarrow \alpha_1 = \alpha_2 = \ldots = \alpha_m = 0,
    \end{equation}
    caso contrário $\vtX_1, \vtX_2, \ldots, \vtX_m$ são ditos \textbf{\alert{linearmente dependentes}} (L.D.)
    \item Se $\vtX_1, \vtX_2, \ldots, \vtX_m$ são L.D., então existe pelo menos um $\alpha_i \neq 0$, tal que é possível escrever
    {\small\begin{equation}
      \vtX_i = -\frac{1}{\alpha_i}\left[\alpha_1\vtX_1 + \alpha_2\vtX_2 + \ldots + \alpha_{i-1}\vtX_{i-1} + \alpha_{i+1}\vtX_{i+1} + \ldots + \alpha_m\vtX_m\right]
    \end{equation}}
    \item A \textbf{\alert{dimensão}} de um (sub)espaço vetorial linear é dado pelo máximo número de vetores L.I. neste (sub)espaço
  \end{itemize}
\end{frame}

\begin{frame}
  \frametitle{Bases, representações e ortonormalização}
  \begin{itemize}
    \item Um conjunto de $n$ vetores L.I. pertencentes a um (sub)espaço vetorial é chamado uma \textbf{\alert{base}} para esse (sub)espaço
    \item Dada uma base para um (sub)espaço vetorial, todo vetor desse subespaço pode ser escrito como uma combinação linear única dos vetores que formam a base
    \item Sejam $\left\{\vtB_1, \vtB_2, \ldots, \vtB_n\right\} \in \fdR^n$ um conjunto de vetores L.I. que formam uma base para $\fdR^n$
    \item Todo vetor $\vtX \in \fdR^n$ pode ser representado como
    \begin{equation}
      \vtX = \alpha_1\vtB_1+\alpha_2\vtB_2+\ldots+\alpha_n\vtB_n = \underbrace{\left[\begin{matrix}\vtB_1 &\vtB_2 &\ldots & \vtB_n\end{matrix}\right]}_{\mtB}\underbrace{\left[\begin{matrix}\alpha_1 \\ \alpha_2 \\ \vdots \\ \alpha_n\end{matrix}\right]}_{\vtAlpha} = \mtB\vtAlpha
    \end{equation}
  \end{itemize}
\end{frame}

\begin{frame}
  \frametitle{Bases, representações e ortonormalização}
  \begin{itemize}
    \item O vetor $\vtAlpha = \Transp{\left[\begin{matrix}\alpha_1 & \alpha_2 & \ldots & \alpha_n\end{matrix}\right]}$ é chamado de \textbf{\alert{representação}} do vetor $\vtX$ na base $\left\{\vtB_1, \vtB_2, \ldots, \vtB_n\right\}$ (ou ainda na base $\mtB$)
    \item A cada $\fdR^n$ é associada uma \textbf{\alert{base canônica}} $\left\{\vtI_1, \vtI_2, \ldots, \vtI_n\right\}$ onde $\vtI_i$ é a $i$-ésima coluna de uma matriz identidade $\mtI_n$ de dimensão $n \times n$
    \item Note que, na base canônica, um vetor $\vtX \in \fdR^n$ é representado como
    \begin{equation}
      \vtX = x_1\vtI_1+x_2\vtI_2+\ldots+x_n\vtI_n = \left[\begin{matrix}x_1 \\ x_2 \\ \vdots \\ x_n\end{matrix}\right]
    \end{equation}
  \end{itemize}
\end{frame}

\begin{frame}
  \frametitle{Vetores normalizados, ortogonais e ortonormais}
  \begin{itemize}
    \item Um vetor $\vtX$ é dito \textbf{\alert{normalizado}} se sua norma euclidiana $\NormTwo{\vtX} = 1$
    \item Um vetor unitário $\vtU$ na direção de um vetor $\vtX$ é obtido normalizando o vetor $\vtX$ como
    \begin{equation}
      \vtU_{\vtX} = \frac{\vtX}{\NormTwo{\vtX}}
    \end{equation}
    \item Dois vetores $\vtX_1$ e $\vtX_2$ são ditos \textbf{\alert{ortogonais}} se o produto interno entre eles é nulo, i.e., se $\Transp{\vtX}_1\vtX_2 = \Transp{\vtX}_2\vtX_1 = 0$
    \item Um conjunto de vetores $\vtX_1, \vtX_2, \ldots, \vtX_n$ é dito \textbf{\alert{ortonormal}} se
    \begin{equation}\label{eq_vec_orthonormal}
      \Transp{\vtX}_i\vtX_j = \begin{cases}
        1, & i = j \\
        0, & i \neq j
      \end{cases}
    \end{equation}
    \item Um conjunto de vetores L.I. $\left\{\vtX_1, \vtX_2, \ldots, \vtX_n\right\} \in \fdR^n$ que atende a \eqref{eq_vec_orthonormal} formam uma \textbf{\alert{base ortonormal}}
  \end{itemize}
\end{frame}

\begin{frame}
  \frametitle{Componente ortogonal e paralela}
  \begin{itemize}
    \item O comprimento $\Abs{P_{\parallel}(\vtX_i, \vtX_j)}$ da componente paralela $P_{\parallel}(\vtX_i, \vtX_j)$ de um vetor $\vtX_i$ na direção de um vetor $\vtX_j$ é dado pelo produto escalar do primeiro vetor com o vetor unitário na direção do segundo vetor, i.e.,
    \begin{equation}\label{eq_proj_parallel}
      \Abs{P_{\parallel}(\vtX_i, \vtX_j)} = \frac{\Transp{\vtX}_i\vtX_j}{\NormTwo{\vtX_j}} = \Transp{\vtX}_i\frac{\vtX_j}{\NormTwo{\vtX_j}} = \Transp{\vtX}_i\vtU_j, \text{ onde } \vtU_j = \frac{\vtX_j}{\NormTwo{\vtX_j}}
    \end{equation}
    \item Usando \eqref{eq_proj_parallel}, a \textbf{\alert{componente paralela}} $P_{\parallel}(\vtX_i, \vtX_j)$ e a \textbf{\alert{componente ortogonal}} $P_{\perp}(\vtX_i, \vtX_j)$ de um vetor $\vtX_i$ em relação a um vetor $\vtX_j$ são dadas respectivamente por
    \begin{subequations}
      \begin{align}
        P_{\parallel}(\vtX_i, \vtX_j) &= (\Transp{\vtX}_i\vtU_j)\vtU_j \\
        P_{\perp}(\vtX_i, \vtX_j) &= \vtX_i - P_{\parallel}(\vtX_i, \vtX_j) = \vtX_i - (\Transp{\vtX}_i\vtU_j)\vtU_j
      \end{align}
    \end{subequations}
  \end{itemize}
\end{frame}

\begin{frame}
  \frametitle{Processo de ortonormalização de Gram-Schmidt}
  \begin{itemize}
    \item O \textbf{\alert{processo de ortonormalização de Gram-Schmidt}} permite construir uma base ortonormal a partir de um conjunto de vetores L.I.
    \item Segundo esse processo, um conjunto $\left\{\vtX_1, \vtX_2, \ldots, \vtX_n\right\} \in \fdR^n$ de vetores L.I. produz a base ortonormal $\left\{\vtB_1, \vtB_2, \ldots, \vtB_n\right\}$ como
    \begin{align*}
      \vtV_1 & = \vtX_1, & \vtB_1 = \vtV_1/\NormTwo{\vtV_1} \\
      \vtV_2 & = \vtX_2 - (\Transp{\vtX}_2\vtB_1)\vtB_1, & \vtB_2 = \vtV_2/\NormTwo{\vtV_2} \\
      \vtV_3 & = \vtX_3 - (\Transp{\vtX}_3\vtB_1)\vtB_1 - (\Transp{\vtX}_3\vtB_2)\vtB_2, & \vtB_3 = \vtV_3/\NormTwo{\vtV_3} \\
      \vtV_4 & = \vtX_4 - (\Transp{\vtX}_4\vtB_1)\vtB_1 - (\Transp{\vtX}_4\vtB_2)\vtB_2 - (\Transp{\vtX}_4\vtB_3)\vtB_3, & \vtB_4 = \vtV_4/\NormTwo{\vtV_4} \\
      \ldots \\
      \vtV_n & = \vtX_n - \sum\limits_{i=1}^{n-1}(\Transp{\vtX}_n\vtB_i)\vtB_i, & \vtB_n = \vtV_n/\NormTwo{\vtV_n} \\
    \end{align*}
  \end{itemize}
\end{frame}

\subsection{Matrizes e operações com matrizes}

\begin{frame}
  \frametitle{Matrizes}
  \begin{itemize}
    \item Uma matriz $\mtA$ de dimensão $m \times n$ pertencente ao $\fdR^{m \times n}$ e sua transposta $\Transp{\mtA} \in \fdR^{n \times m}$ são denotadas por
    \begin{equation}\label{eq_mat_transp}
      \begin{split}
        \mtA &= \left[\begin{matrix}
          a_{1,1} & a_{1,2} & \ldots & a_{1,n} \\
          a_{2,1} & a_{2,2} & \ldots & a_{2,n} \\
          \vdots & \vdots & \ddots & \vdots \\
          a_{m,1} & a_{m,2} & \ldots & a_{m,n}
        \end{matrix}\right] = \left[\begin{matrix} \vtA_1 & \vtA_2 & \ldots & \vtA_n\end{matrix}\right] \text{ e } \\
        \Transp{\mtA} &= \left[\begin{matrix}
          a_{1,1} & a_{2,1} & \ldots & a_{m,1} \\
          a_{1,2} & a_{2,2} & \ldots & a_{m,2} \\
          \vdots & \vdots & \ddots & \vdots \\
          a_{1,n} & a_{2,n} & \ldots & a_{m,n}
        \end{matrix}\right] = \left[\begin{matrix} \Transp{\vtA}_1 \\ \Transp{\vtA}_2 \\ \vdots \\ \Transp{\vtA}_n\end{matrix}\right],
      \end{split}
    \end{equation}
    onde $\vtA_i, 1 \leq i \leq n$ denota a $i$-ésima coluna de $\mtA$
    \item Assume-se aqui conhecimento sobre as operações de adição de matrizes, multiplicação por escalar, e multiplicação de matrizes
  \end{itemize}
\end{frame}

\begin{frame}
  \frametitle{Algumas operações com matrizes}
  \begin{itemize}
    \item Para uma matriz $\mtA \in \fdC^{m\times n}$, a \textbf{\alert{matrix conjugada}} $\Conj{\mtA}$ é obtida conjugando cada elemento $a_{i,j}$ de $\mtA$
    \item De forma similar, a \textbf{\alert{matriz conjugada-transposta}} $\Herm{\mtA}$ de $\mtA$ é obtida conjugando a matriz transposta $\Transp{\mtA}$  de $\mtA$, i.e., $\Herm{\mtA} = \Conj{\left(\Transp{\mtA}\right)}$
    \item A \textbf{\alert{matriz inversa}} $\Inv{\mtA}$ de uma matriz $\mtA$ de dimensão $n\times n$ é a matriz que satisfaz $\Inv{\mtA}\mtA = \mtI_n$
    \item Algumas propriedades relevantes envolvendo matrizes são \cite{Petersen2008}:
    {\footnotesize\begin{subequations}
      \begin{align}
        \Inv{(\mtA\mtB)} &= \Inv{\mtB}\Inv{\mtA} \\
        \Transp{(\mtA\mtB)} &= \Transp{\mtB}\Transp{\mtA} \\
        \Herm{(\mtA\mtB)} &= \Herm{\mtB}\Herm{\mtA} \\
        \Inv{\left(\Transp{\mtA}\right)} &= \Transp{\left(\Inv{\mtA}\right)} \\
        \Inv{\left(\Herm{\mtA}\right)} &= \Herm{\left(\Inv{\mtA}\right)} \\
        \Transp{(\mtA+\mtB)} &= \Transp{\mtA} + \Transp{\mtB} \\
        \Herm{(\mtA+\mtB)} &= \Herm{\mtA} + \Herm{\mtB}
      \end{align}
    \end{subequations}}
  \end{itemize}
\end{frame}

\begin{frame}
  \frametitle{Traço de uma matriz}
  \begin{itemize}
    \item O \textbf{\alert{traço}} $\Trace{\mtA}$ de uma matriz $\mtA$ de dimensão $n \times n$ é definido como a soma dos elementos da diagonal da mesma, i.e.,
    \begin{equation}\label{eq_trace}
      \Trace{\mtA} = \sum\limits_{i=1}^n a_{i,i}
    \end{equation}
    \item Entre as propriedades do $\Trace{\cdot}$ temos \cite{Petersen2008}:
    \begin{subequations}
      \begin{align}
        \Trace{\mtA} &= \Trace{\Transp{\mtA}} \\
        \Trace{\mtA\mtB} &= \Trace{\mtB\mtA} \\
        \Trace{\mtA\mtB\mtC} &= \Trace{\mtC\mtA\mtB} = \Trace{\mtB\mtC\mtA} \\
        \Trace{\alpha\mtA + \beta\mtB} &= \alpha\Trace{\mtA} + \beta\Trace{\mtB}
      \end{align}
    \end{subequations}
  \end{itemize}
\end{frame}

\begin{frame}
  \frametitle{Determinante uma matriz}
  \begin{itemize}\small
    \item O \textbf{\alert{determinante}} $\Det{\mtA}$ de uma matriz $\mtA \in \fdR^{n\times n}$ pode ser escrito através da \alert{expansão de Laplace} sobre uma linha ou uma coluna da matriz como
    \begin{equation}\label{eq_det}
      \Det{\mtA} = \sum\limits_{i = 1}^n a_{i,j}c_{i,j} = \sum\limits_{j = 1}^n a_{i,j}c_{i,j},
    \end{equation}
    onde $c_{i,j}$ é o \textbf{\alert{cofator}} associado a $a_{i,j}$, o qual é dado por
    \begin{equation}\label{eq_cofactor}
      c_{i,j} = (-1)^{i+j}\Det{\tilde{\mtA}_{i,j}},
    \end{equation}
    onde a matriz $\tilde{\mtA}_{i,j} \in \fdR^{(n-1)\times(n-1)}$ é formada a partir de $\mtA$ pela exclusão de sua $i$-ésima linha e $j$-ésima coluna
    \item A \textbf{\alert{matriz adjunta}} $\Adj{\mtA}$ de $\mtA$ é a matriz transposta dos cofatores de $\mtA$
    \item Uma matriz $\mtA$ é dita \textbf{\alert{não-singular}} e possui inversa $\Inv{\mtA}$ se $\Det{\mtA} \neq 0$
    \item A matriz inversa pode ser calculada como
    \begin{equation}\label{eq_adjoint}
      \Inv{\mtA} = \frac{1}{\Det{\mtA}}\Adj{\mtA}
    \end{equation}
  \end{itemize}
\end{frame}

\begin{frame}
  \frametitle{Exemplo para inversa de uma matriz}
  \begin{itemize}
    \item Se a matriz $\mtA$ é dada por
    \begin{equation*}
      \mtA = \left[\begin{matrix}
          a_{1,1} & a_{1,2} \\
          a_{2,1} & a_{2,2}
      \end{matrix}\right]
    \end{equation*}
    temos que
    \begin{equation*}
      \begin{split}
        \Inv{\mtA} &= \frac{1}{\Det{\mtA}}\Adj{\mtA} = \frac{1}{a_{1,1}a_{2,2} - a_{1,2}a_{2,1}}\left[\begin{matrix}
          c_{1,1} & c_{2,1} \\
          c_{1,2} & c_{2,2}
        \end{matrix}\right] \\
        &= \frac{1}{a_{1,1}a_{2,2} - a_{1,2}a_{2,1}}\left[\begin{matrix}
          a_{2,2} & -a_{1,2} \\
          -a_{2,1} & a_{1,1}
        \end{matrix}\right]
      \end{split}
    \end{equation*}
  \end{itemize}
\end{frame}

\subsection{Transformações lineares}

\begin{frame}
  \frametitle{Transformação linear}
  \begin{itemize}
    \item Um função $f(\cdot) : \fdR^n \rightarrow \fdR^m$ é um \textbf{\alert{operador linear}} se e somente se
    \begin{equation}\label{eq_lin_operator}
      f(\alpha_1\vtX_1+\alpha_2\vtX_2) = \alpha_1 f(\vtX_1)+\alpha_2 f(\vtX_2), \forall \vtX_1, \vtX_2 \in \fdR^n, \alpha_1, \alpha_2 \in \fdR
    \end{equation}
    \item Sejam $\fdX \subset \fdR^n$ e $\fdY \subset \fdR^m$ são dois espaços vetoriais e sejam:
    \begin{itemize}
      \item $L(\cdot) : \fdX \rightarrow \fdY$ um operador linear
      \item $\vtX_1, \vtX_2, \ldots, \vtX_n$ um conjunto de $n$ vetores L.I. em $\fdX$
      \item $\vtY_i = L(\vtX_i), i = 1, 2, \ldots, n$ um conjunto de $n$ vetores em $\fdY$
    \end{itemize}
    \item Então, podemos afirmar que:
    \begin{itemize}
      \item O operador linear $L(\cdot)$ é unicamente determinado pelos $n$ pares $(\vtX_i, \vtY_i), i = 1, 2, \ldots, n$
      \item Se $\{\vtP_1, \vtP_2, \ldots, \vtP_n\}$ e $\{\vtQ_1, \vtQ_2, \ldots, \vtQ_m\}$ são bases para $\fdX$ e $\fdY$, respectivamente, então o operador linear $L(\cdot)$ pode ser representado por uma matriz $\mtT$ de dimensão $m\times n$
      \item A $i$-ésima coluna da matriz $\mtT$ é a representação de $\vtY_i$ na base $\mtP = \left[\begin{matrix}\vtP_1 & \vtP_2 & \ldots & \vtP_n\end{matrix}\right]$
    \end{itemize}
  \end{itemize}
\end{frame}

\begin{frame}
  \frametitle{Transformações lineares: mudança de base}
  \begin{itemize}
    \item Exemplos de transformações lineares comuns no $\fdR^2$
    \begin{itemize}
      \item \textbf{Mudança de escala}: $\mtT = \alpha\mtI$ (mesma escala nos eixos $x$ e $y$) ou $\mtT = \left[\begin{matrix}\alpha_1 & 0 \\ 0 & \alpha_2 \end{matrix}\right]$ (escalas diferentes para os eixos $x$ e $y$)
      \item \textbf{Rotação}: $\mtT = \left[\begin{matrix}\cos(\theta) & -\sin(\theta) \\ \sin(\theta) & \cos(\theta) \end{matrix}\right]$
    \end{itemize}
    \item Outra transformação linear de interesse refere-se à mudança de base em um espaço vetorial
    \item Se $\vtAlpha$ e $\vtBeta$ são respectivamente as representações de $\vtX \in \fdX \subset \fdR^n$ nas bases $\mtA = \left[\begin{matrix}\vtA_1 & \vtA_2 & \ldots & \vtA_n\end{matrix}\right]$ e $\mtB = \left[\begin{matrix}\vtB_1 & \vtB_2 & \ldots & \vtB_n\end{matrix}\right]$, temos
    \begin{equation}\label{eq_base_change}
      \vtX = \underbrace{\left[\begin{matrix}\vtA_1 & \vtA_2 & \ldots & \vtA_n\end{matrix}\right]}_{\mtA}\underbrace{\left[\begin{matrix}\alpha_1 \\ \alpha_2 \\ \vdots \\ \alpha_n\end{matrix}\right]}_{\vtAlpha} = \underbrace{\left[\begin{matrix}\vtB_1 & \vtB_2 & \ldots & \vtB_n\end{matrix}\right]}_{\mtB}\underbrace{\left[\begin{matrix}\beta_1 \\ \beta_2 \\ \vdots \\ \beta_n\end{matrix}\right]}_{\vtBeta}
    \end{equation}
  \end{itemize}
\end{frame}

\begin{frame}
  \frametitle{Transformações lineares: mudança de base}
  \begin{itemize}
    \item Se $\vtP_i$ é a representação de $\vtA_i$ na base $\mtB$, então temos que
    \begin{equation}
      \vtA_i = \left[\begin{matrix}\vtB_1 & \vtB_2 & \ldots & \vtB_n\end{matrix}\right]\vtP_i = \mtB\vtP_i,\quad i = 1, 2, \ldots, n
    \end{equation}
    \item Juntando as $n$ equações acima em expressão matricial obtemos
    \begin{equation}\label{eq_base_change2}
      \mtA = \left[\begin{matrix}\vtB_1 & \vtB_2 & \ldots & \vtB_n\end{matrix}\right]\left[\begin{matrix}\vtP_1 & \vtP_2 & \ldots & \vtP_n\end{matrix}\right] = \mtB\mtP
    \end{equation}
    \item Substituindo \eqref{eq_base_change2} em \eqref{eq_base_change}, obtemos
    \begin{equation*}
        \mtA\vtAlpha = \mtB\vtBeta \Rightarrow \mtB\mtP\vtAlpha = \mtB\vtBeta \Rightarrow \Inv{\mtB}\mtB\mtP\vtAlpha = \Inv{\mtB}\mtB\vtBeta \Rightarrow \boxed{\mtP\vtAlpha = \vtBeta}
    \end{equation*}
    \item Logo, temos que $\mtP$ é a transformação linear que leva a representação $\vtAlpha$ de $\vtX$ na base $\mtA$ para sua representação $\vtBeta$ na base $\mtB$
  \end{itemize}
\end{frame}

\begin{frame}
  \frametitle{Transformações lineares: mudança de base}
  \begin{itemize}
    \item Um desenvolvimento análogo ao anterior provê a transformação linear $\mtQ$ que leva a representação $\vtBeta$ de $\vtX$ na base $\mtB$ para sua representação $\vtAlpha$ na base $\mtA$
    \item Se $\mtP$ e $\mtQ$ são conhecidas, então para um vetor $\vtX$ qualquer com representações $\vtAlpha$ na base $\mtA$ e $\vtBeta$ na base $\mtB$ temos
    \begin{equation*}
      \vtBeta = \mtP\vtAlpha \text{ e } \vtAlpha = \mtQ\vtBeta \Rightarrow \vtBeta = \mtP\mtQ\vtBeta \Rightarrow \mtQ\mtP = \mtI \Rightarrow \boxed{\Inv{\mtP} = \mtQ}
    \end{equation*}
  \end{itemize}
\end{frame}

\begin{frame}
  \frametitle{Transformações lineares: transformações de similaridade}
  \begin{itemize}
    \item Considere que:
    \begin{itemize}
      \item $\mtU = \left[\begin{matrix} \vtU_1 & \vtU_2 & \ldots & \vtU_n\end{matrix}\right]$ e $\bar{\mtU} = \left[\begin{matrix} \bar{\vtU}_1, \bar{\vtU}_2, \ldots, \bar{\vtU}_n\end{matrix}\right]$ são duas bases para um subespaço vetorial $\fdX$
      \item $L(\cdot)$ é um operador linear tal que $\vtY_i = L(\vtU_i)$ e $\bar{\vtY}_i = L(\bar{\vtU}_i)$, $i = 1, 2, \ldots, n$
      \item $\mtV = \left[\begin{matrix}\vtV_1, \vtV_2, \ldots, \vtV_n\end{matrix}\right]$ e $\bar{\mtV} = \left[\begin{matrix}\bar{\vtV}_1, \bar{\vtV}_2, \ldots, \bar{\vtV}_n\end{matrix}\right]$ são duas uma base para o espaço gerado por $\vtY_i$
      \item $\mtA = \left[\begin{matrix}\vtA_1, \vtA_2, \ldots, \vtA_n\end{matrix}\right]$ e $\bar{\mtA} = \left[\begin{matrix}\bar{\vtA}_1, \bar{\vtA}_2, \ldots, \bar{\vtA}_n\end{matrix}\right]$ são as representação de $\vtY_i$ e $\bar{\vtY}_i$ nas bases $\mtV$ e $\bar{\mtV}$, respectivamente
    \end{itemize}
    \item Como $\mtU$ e $\bar{\mtU}$ são base para $\fdX$, um vetor $\vtX \in \fdX$ pode ser escrito como combinação linear das colunas de $\mtU$ ou de $\bar{\mtU}$ e como $L(\vtX)$ é um operador linear temos
    \begin{subequations}\label{eq_transf_sim_lin_op}
      \begin{align}
        L(\vtX) &= L\left(\sum\limits_{i = 1}^n \alpha_i\vtU_i\right) = \sum\limits_{i = 1}^n \alpha_iL\left(\vtU_i\right) = \sum\limits_{i = 1}^n \alpha_i\vtY_i \label{eq_transf_sim_lin_op_a} \\
        L(\vtX) &= L\left(\sum\limits_{i = 1}^n \bar{\alpha}_i\bar{\vtU}_i\right) = \sum\limits_{i = 1}^n \bar{\alpha}_iL\left(\bar{\vtU}_i\right) = \sum\limits_{i = 1}^n \bar{\alpha}_i\bar{\vtY}_i \label{eq_transf_sim_lin_op_b}
      \end{align}
    \end{subequations}
  \end{itemize}
\end{frame}

\begin{frame}
  \frametitle{Transformações lineares: transformações de similaridade}
  \begin{itemize}\footnotesize
    \item Podemos escrever ainda que
    \begin{subequations}\label{eq_transf_sim_rep_y}
      \begin{align}
        L\left(\left[\begin{matrix} \vtU_1 & \vtU_2 & \ldots & \vtU_n\end{matrix}\right]\right) &= \left[\begin{matrix} \vtY_1 & \vtY_2 & \ldots & \vtY_n\end{matrix}\right] \nonumber \\
        &= \left[\begin{matrix} \vtV_1 & \vtV_2 & \ldots & \vtV_n\end{matrix}\right]\left[\begin{matrix} \vtA_1 & \vtA_2 & \ldots & \vtA_n\end{matrix}\right] = \mtV\mtA \label{eq_transf_sim_rep_y_a}\\
        L\left(\left[\begin{matrix} \bar{\vtU}_1 & \bar{\vtU}_2 & \ldots & \bar{\vtU}_n\end{matrix}\right]\right) &= \left[\begin{matrix} \bar{\vtY}_1 & \bar{\vtY}_2 & \ldots & \bar{\vtY}_n\end{matrix}\right]  \nonumber \\
        &= \left[\begin{matrix} \bar{\vtV}_1 & \bar{\vtV}_2 & \ldots & \bar{\vtV}_n\end{matrix}\right]\left[\begin{matrix} \bar{\vtA}_1 & \bar{\vtA}_2 & \ldots & \bar{\vtA}_n\end{matrix}\right] = \bar{\mtV}\bar{\mtA} \label{eq_transf_sim_rep_y_b}
      \end{align}
    \end{subequations}
    \item Usando \eqref{eq_transf_sim_lin_op} e \eqref{eq_transf_sim_rep_y}, para um vetor $\vtX$ com representação $\vtAlpha$ na base $\mtU$ e um vetor $\vtY = L(\vtX)$ com representação $\vtBeta$ na base $\mtV$ temos
    \begin{equation}\label{eq_beta_rep}
      \begin{split}
        \vtY &= L(\vtX) \Rightarrow \mtV\vtBeta = L(\mtU\vtAlpha) \Rightarrow \mtV\vtBeta = \sum\limits_{i=1}^n \alpha_iL\left(\vtU_i\right) \Rightarrow \mtV\vtBeta = \sum\limits_{i=1}^n \alpha_i\vtY_i \\
        & \Rightarrow \mtV\vtBeta = \mtV\mtA\vtAlpha \Rightarrow \boxed{\vtBeta  = \mtA\vtAlpha}
      \end{split}
    \end{equation}
    \item Analogamente para um vetor $\vtX$ com representação $\bar{\vtAlpha}$ na base $\bar{\mtU}$ e um vetor $\bar{\vtY} = L(\bar{\vtX})$ com representação $\bar{\vtBeta}$ na base $\bar{\mtV}$ obtemos
    \begin{equation}\label{eq_betabar_rep}
      \boxed{\bar{\vtBeta}  = \bar{\mtA}\bar{\vtAlpha}}
    \end{equation}
  \end{itemize}
\end{frame}

\begin{frame}
  \frametitle{Transformações lineares: transformações de similaridade}
  \begin{itemize}\small
    \item Seja $\mtP$ a transformação linear que leva a representação $\vtAlpha$ de $\vtX$ da base $\mtU$ para $\bar{\vtAlpha}$ na base $\bar{\mtU}$ e seja $\mtQ = \Inv{\mtP}$ a transformação linear que traz a representação $\vtAlpha$ de $\vtX$ na base $\bar{\mtU}$ de volta para a base $\mtU$
    \item Como há um representação única para o operador linear $L(\vtX)$ e $\vtY = L(\vtX)$, as transformações $\mtP$ e $\mtQ = \Inv{\mtP}$ também realizam as mudança de base de $\mtV$ para $\bar{\mtV}$ e vice-versa para um vetor $\vtY$
  \end{itemize}
  \vspace{-\baselineskip}
  \begin{columns}[t]
    \begin{column}{0.48\linewidth}
  \begin{center}
    \begin{tikzpicture}[
      thick,
      bend angle=30,
      font=\footnotesize,
      node distance=2cm,
      >=stealth',
      ]
      \node (x)  {$\vtX$};
      \node (y) [right=of x] {$\vtY = L(\vtX)$};
      \node (alpha) [below=0.3cm of x] {$\mtU\vtAlpha$};
      \node (alphabar) [below=of alpha] {$\bar{\mtU}\bar{\vtAlpha}$};
      \node (beta) [right=of alpha] {$\vtBeta = \mtA\vtAlpha$};
      \node (betabar) [right=of alphabar] {$\bar{\vtBeta} = \bar{\mtA}\bar{\vtAlpha}$};
      \node (pL) [coordinate,right=1cm of x.east,label=90:$L$] {};
      \node (pA) [coordinate,right=1cm of alpha.east,label=90:$\mtA$] {};
      \node (pAbar) [coordinate,right=1cm of alphabar.east,label=90:$\bar{\mtA}$] {};
      \node (pP1) [coordinate,below=1cm of alpha] {};
      \node (P1) [left=0.5cm of pP1] {$\mtP$};
      \node (Q1) [right=0.5cm of pP1] {$\Inv{\mtP}$};
      \node (pP2) [coordinate,below=1cm of beta] {};
      \node (P2) [left=0.5cm of pP2] {$\mtP$};
      \node (Q2) [right=0.5cm of pP2] {$\Inv{\mtP}$};
      \path (x) edge[->] (y);
      \path (alpha) edge[->] (beta);
      \path (alphabar) edge[->] (betabar);
      \path (alpha) edge[->,bend right] (alphabar);
      \path (alphabar) edge[->,bend right] (alpha);
      \path (beta) edge[->,bend right] (betabar);
      \path (betabar) edge[->,bend right] (beta);
    \end{tikzpicture}
  \end{center}
    \end{column}
    \hfill
    \begin{column}{0.48\linewidth}
      \begin{itemize}\small
        \item Com o auxílio do diagrama ao lado e usando \eqref{eq_beta_rep} e \eqref{eq_betabar_rep} temos que
        \begin{equation}\label{eq_transf_sim}
          \begin{split}
            \bar{\vtBeta} &= \bar{\mtA}\bar{\vtAlpha} \Rightarrow \bar{\mtA}\bar{\vtAlpha} = \mtP\vtBeta \\
            & \Rightarrow \bar{\mtA}\bar{\vtAlpha} = \mtP\mtA\vtAlpha \\
            & \Rightarrow \bar{\mtA}\bar{\vtAlpha} = \mtP\mtA\Inv{\mtP}\bar{\vtAlpha} \\
            & \Rightarrow \boxed{\bar{\mtA} = \mtP\mtA\Inv{\mtP}} \\
            & \Rightarrow \boxed{\mtA = \Inv{\mtP}\bar{\mtA}\mtP}
          \end{split}
        \end{equation}
      \end{itemize}
    \end{column}
  \end{columns}
\end{frame}

\begin{frame}
  \frametitle{Transformações lineares: transformações de similaridade}
  \begin{itemize}\small
    \item As transformações em $\mtP\bar{\mtA}\Inv{\mtP}$ e $\Inv{\mtP}\mtA\mtP$ mostradas em \eqref{eq_transf_sim} são chamadas \textbf{\alert{transformações de similaridade}}
    \item Matrizes $\mtA$ e $\bar{\mtA}$ que se relacionam conforme \eqref{eq_transf_sim} são ditas \textbf{\alert{matrizes similares}}
    \item Em particular, as matrizes $\mtA$ e $\bar{\mtA}$ são representações de um mesmo operador linear $L(\cdot)$ em duas bases distintas
    \item Todas as representações de um mesmo operador linear são similares
    \item Como um operador linear em uma dada base pode ser representado por uma matriz $\mtA$, essa matriz pode ser vista como o operador linear propriamente dito
    \item Sendo o operador linear $L(\cdot) : \fdR^n \rightarrow \fdR^n$ descrito pela matriz $\mtA \in \fdR^{n \times n}$, temos que
    \begin{equation}
      \begin{split}
        \vtY &= L(\vtX) = \mtA\left(\sum\limits_{i=1}^n \alpha_i\vtU_i\right) = \sum\limits_{i=1}^n \alpha_i\mtA\vtU_i = \sum\limits_{i=1}^n \alpha_i\vtY_i \\
        & \Rightarrow \vtY_i = \mtA\vtU_i, i = 1, 2, \ldots, n
      \end{split}
    \end{equation}
  \end{itemize}
\end{frame}


\subsection{Subespaços fundamentais}

\begin{frame}
  \frametitle{Sistema de equações lineares}
  \begin{itemize}\small
    \item Considere um sistema de $m$ equações lineares e $n$ variáveis $x_1, x_2, \ldots, x_n$,
    \begin{equation}\label{eq_sys}
      \begin{split}
        a_{1,1}\cdot x_1 + a_{1,2}\cdot x_2 + \ldots + a_{1,n}\cdot x_n &= y_1 \\
        a_{2,1}\cdot x_1 + a_{2,2}\cdot x_2 + \ldots + a_{2,n}\cdot x_n &= y_2 \\
        \ldots & \\
        a_{m,1}\cdot x_1 + a_{m,2}\cdot x_2 + \ldots + a_{m,n}\cdot x_n &= y_m
      \end{split}
    \end{equation}
    onde os coeficientes $a_{i,j}$ e $y_{i}$ são dados e $i = 1, 2, \ldots, m$ e $j = 1, 2, \ldots, n$
    \item Usando notação matricial, podemos definir
    \begin{equation}
      \mathbf{A} =
      \left[\begin{matrix}
        a_{1,1} & a_{1,2} & \ldots & a_{1,n} \\
        a_{2,1} & a_{2,2} & \ldots & a_{2,n} \\
        \vdots & \vdots & \ddots & \vdots \\
        a_{m,1} & a_{m,2} & \ldots & a_{m,n}
      \end{matrix}\right],
      \quad
      \mathbf{x} =
      \left[\begin{matrix}
        x_1 \\
        x_2 \\
        \vdots \\
        x_n
      \end{matrix}\right],
      \quad \text{e} \quad
      \vtY =
      \left[\begin{matrix}
        y_1 \\
        y_2 \\
        \vdots \\
        y_m
      \end{matrix}\right],
    \end{equation}
    e reescrever \eqref{eq_sys} como
    \begin{equation}\label{eq_sis_mat}
      \mathbf{A}\mathbf{x} = \vtY
    \end{equation}
  \end{itemize}
\end{frame}

\begin{frame}{Posto de uma matriz}
  \begin{itemize}
    \item O \textbf{\alert{posto}} ou \textbf{\alert{rank}} de uma matriz $\mtA = \left[\begin{matrix} \vtA_1 & \vtA_2 & \ldots & \vtA_n\end{matrix}\right]$ é denotado por $\Rank{\mtA}$ e pode ser definido como o número de colunas $\vtA_i$ de $\mtA$ que são L.I.
    \item O posto de uma matriz corresponde portanto à dimensão do espaço vetorial gerado pelas colunas da matriz
    \item Dada uma matriz $\mtA$ de dimensão $m\times n$, temos que
    \begin{equation}\label{eq_rank_transp}
      \Rank{\mtA} = \Rank{\Transp{\mtA}} \Rightarrow \Rank{\mtA} \leq \Min{m,n}
    \end{equation}
    e portanto os espaço vetoriais gerados pelas colunas e pelas linhas de $\mtA$ têm a mesma dimensão
    \item Para $\mtA$ com dimensão $m \times n$ e $\mtB$ com dimensão $n \times p$, a \textbf{\alert{desigualdade de Sylvester}} estabelece que
    \begin{equation}\label{eq_sylvester}
      \Rank{\mtA} + \Rank{\mtB} - n \leq \Rank{\mtA\mtB} \leq \Min{\Rank{\mtA},\Rank{\mtB}}
    \end{equation}
  \end{itemize}
\end{frame}

\begin{frame}
  \frametitle{Subespaços fundamentais: espaço coluna}
  \begin{itemize}\small
    \item O sistema \eqref{eq_sys} possui solução se $\vtY$ pode ser escrito como combinação linear das colunas de $\mtA$
    \item Nesse caso, cada vetor $\vtX$ (se existir) que satisfaz \eqref{eq_sys} é uma representação de $\vtY$ no subespaço gerado pelas colunas de $\mtA$
    \item Considere que $\{\vtB_1, \vtB_2, \ldots, \vtB_r\}$ formam uma base para o subespaço gerado pelas colunas de $\mtA$, onde $r = \Rank{\mtA}$
    \item O sistema \eqref{eq_sys} possui solução se e somente se $\vtY$ pertence ao subespaço $\Range{\mtA}$ gerado pela base $\{\vtB_1, \vtB_2, \ldots, \vtB_r\}$, ou seja, o subespaço gerado pelas colunas de $\mtA$
    \item O subespaço $\Range{\mtA}$ de $\mtA \in \fdR^{m \times n}$ é chamado \textbf{\alert{espaço coluna}} ou \textbf{\alert{espaço \textit{range}}} de $ \mtA $
    \begin{equation}\label{eq_range_space}
      \Range{\mtA} \triangleq \left\{\vtY = \mtA\vtX \vert \vtX \in \fdR^n \right\} \subset \fdR^m
    \end{equation}
    \item O $\Rank{\mtA}$ é a dimensão do espaço coluna $\Range{\mtA}$
    \item Um matriz $\mtA$ possui (pseudo-)inversa quando o seu $\Rank{\mtA}$ é máximo, i.e., $\Rank{\mtA} = \Min{m,n}$
    \item Para $ \mtA \in \fdR^{m \times p} $ e $ \mtB \in \fdR^{m \times q} $, temos que
    \begin{equation}
      \Range{\mtA} + \Range{\mtB} = \Range{\begin{bmatrix}\mtA & \mtB\end{bmatrix}}
    \end{equation}
  \end{itemize}
\end{frame}

\begin{frame}
  \frametitle{Subespaços fundamentais: espaço nulo}
  \begin{itemize}
    \item Quando $\vtY = \vtZero$ em \eqref{eq_sys} temos $\mtA\vtX = \vtZero$
    \item O conjunto de soluções de $\mtA\vtX = \vtZero$ define por si só um subespaço vetorial: o \textbf{\alert{espaço nulo}} $\Null{\mtA}$ de $\mtA$
    \begin{equation}\label{eq_null_space}
      \Null{\mtA} \triangleq \left\{\vtX \in \fdR^n \vert \mtA\vtX = 0\right\} \subset \fdR^n
    \end{equation}
    \item A dimensão do espaço nulo $\Null{\mtA}$ é chamada de \textbf{\alert{nulidade}} de $\mtA$ e é denotada por $\Nullity{\mtA}$
    \item Para $ \mtA \in \fdR^{p \times n} $ e $ \mtB \in \fdR^{q \times n} $, temos que
    \begin{equation}
      \Null{\mtA} + \Null{\mtB} = \Null{\begin{bmatrix}\mtA \\ \mtB\end{bmatrix}}
    \end{equation}
    \item Note que o espaço coluna e o espaço nulo de uma matriz $\mtA$ são duais tal que o $\Rank{\mtA}$ é o dual da $\Nullity{\mtA}$
    \item De fato, para $\mtA$ com dimensão $m \times n$ temos que
    \begin{align}
      \Rank{\mtA} + \Nullity{\mtA} &= n & \text{(Teorema do posto-nulidade)}
    \end{align}
  \end{itemize}
\end{frame}

\begin{frame}
  \frametitle{\normalsize Subespaços fundamentais: espaço linha e espaço nulo à esquerda}
  \begin{itemize}
    \item Associados a uma matriz $\mtA$ e seus subespaços $\Range{\mtA}$ e $\Null{\mtA}$ temos ainda dois outros subespaços:
    \begin{itemize}
      \item O \textbf{\alert{espaço linha}} de $\mtA$, denotado por $\Range{\Transp{\mtA}}$, que corresponde ao espaço coluna de $\Transp{\mtA}$ e é gerado pelas linhas de $\mtA$
      \item O \textbf{\alert{espaço nulo à esquerda}} de $\mtA$, denotado $\Null{\Transp{\mtA}}$, que corresponde ao espaço nulo de $\Transp{\mtA}$ e é o espaço que contém todos os vetores $\vtZ$ que satisfazem $\Transp{\mtA}\vtZ = 0$
    \end{itemize}
    \item De maneira similar à anterior, temos para $\mtA$ com dimensão $m \times n$ que
    \begin{align}
      \Rank{\Transp{\mtA}} + \Nullity{\Transp{\mtA}} &= m & \text{(Teorema do posto-nulidade)}
    \end{align}
    \item Além disso, temos ainda de acordo com \eqref{eq_rank_transp} que
    \begin{equation}
      \Rank{\mtA} = \Rank{\Transp{\mtA}} = r \leq \Min{m,n}
    \end{equation}
  \end{itemize}
\end{frame}

\begin{frame}{Caracterização do posto de uma matriz~\cite{Horn2012}}
  \begin{itemize}
    \item Na caracterização do posto de uma matriz $ \mtA \in \fdR^{m \times n} $ são equivalentes
    \begin{enumerate}\addtolength{\itemsep}{0.5\baselineskip}
      \item $ \Rank{\mtA} = r $
      \item $ r $, e não mais que $ r $, linhas de $ \mtA $ são linearmente independentes
      \item $ r $, e não mais que $ r $, colunas de $ \mtA $ são linearmente independentes
      \item Alguma submatriz $ r \times r $ de $ \mtA $ tem determinante não-nulo e toda submatriz $ (r+1) \times (r+1) $ de $ \mtA $ tem determinante nulo
      \item $ \Dim{\Range{ \mtA }} = r $
      \item Há $ r $, e não mais que $ r $, vetores $ \vtB_1, \ldots, \vtB_r $ tais que o sistema linear $ \mtA\vtX = \vtB_j $ é consistente para $ j = 1, \ldots, r $
      \item $ r = n - \Dim{\Null{\mtA}} $ (teorema do posto-nulidade)
      \item $ r = \Min{p : \mtA = \mtX \Transp{\mtY}} $ para algum $ \mtX \in \fdR^{m\times p}, \mtY \in \fdR^{n\times p} $
      \item $ r = \Min{p : \mtA = \vtX_1 \Transp{\vtY}_1 + \ldots + \vtX_p \Transp{\vtY}_p} $ para algum $ \vtX_1, \ldots, \vtX_p \in \fdR^m, \vtY_1, \ldots, \vtY_p \in \fdR^n $
    \end{enumerate}
  \end{itemize}
\end{frame}

\begin{frame}{Desigualdades do posto de uma matriz~\cite{Horn2012}}
  \begin{enumerate}\addtolength{\itemsep}{\baselineskip}
    \item Se $ \mtA \in \fdR^{m\times n} $, então $ \Rank{\mtA} \leq \Min{m, n} $
    \item Desigualdade de Sylvester: se $ \mtA \in \fdR^{m \times k} $ e $ \mtB \in \fdR^{k \times n} $, então $ (\Rank{\mtA} + \Rank{\mtB}) - k \leq \Rank{\mtA \mtB} \leq \Min{\Rank{\mtA}, \Rank{\mtB}} $ ()
    \item Desigualdade soma-posto: se $ \mtA, \mtB \in \fdR^{m \times k} $ então $\Abs{\Rank{\mtA}-\Rank{\mtB}} \leq \Rank{\mtA + \mtB} \leq \Rank{\mtA} + \Rank{\mtB} $ com igualdade na segunda desigualdade se e somente se $ \Range{\mtA} \cap \Range{\mtB} = \emptyset $
    \item Desigualdade de Frobenius: se $ \mtA \in \fdR^{m \times k} $, $ \mtB \in \fdR^{k \times p} $ e $ \mtC \in \fdR^{p \times n} $, então $ \Rank{\mtA\mtB} + \Rank{\mtB \mtC} \leq \Rank{\mtB} + \Rank{\mtA \mtB \mtC} $ com igualdade se e somente se $ \exists \mtX, \mtY : \mtB= \mtB \mtC \mtX + \mtY \mtA \mtB $
  \end{enumerate}
\end{frame}

\begin{frame}{Igualdades do posto de uma matriz~\cite{Horn2012}}
  \begin{enumerate}\addtolength{\itemsep}{0.5\baselineskip}
    \item Se $ \mtA \in \fdC^{m \times n} $, então $ \Rank{\mtA} = \Rank{\Transp{\mtA}} = \Rank{\Conj{\mtA}} = \Rank{\Herm{\mtA}}$
    
    \item Se $ \mtA \in \fdR^{m \times m} $ e $ \mtC \in \fdR^{n \times n} $ são matrizes não-singulares e $ \mtB \in \fdR^{m \times n} $, então $ \Rank{\mtA\mtB} = \Rank{\mtB} = \Rank{\mtB\mtC} = \Rank{\mtA\mtB\mtC} $, ou seja, multiplicações à direita ou esquerda por matrizes não-singulares não afetam o posto
    
    \item Se $ \mtA, \mtB \in \fdR^{m \times n} $  então $ \Rank{\mtA} = \Rank{\mtB} $ se e somente se $ \exists \mtX \in \fdR^{m \times m}, \mtY \in \fdR^{n \times n} $ não-singulares tais que $ \mtB = \mtX \mtA \mtY $

    \item Se $ \mtA \in \fdC^{m \times n} $ então $ \Rank{\Herm{\mtA}\mtA} = \Rank{\mtA} $
    
    \item Fatorização de posto completo: se $ \mtA \in \fdR^{m \times n} $, então $ \Rank{\mtA} = k $ se e somente se $ \mtA =  \mtX \Transp{\mtY} $ onde $\mtX \in \fdR^{m \times k}$  e $ \mtY \in \fdR^{n \times k} $ têm colunas independentes cada
    
    \item Se $ \mtA \in \fdR^{m \times n} $, $ \mtX \in \fdR^{n \times k} $, e $ \mtY \in \fdR^{m \times k}  $ e $ \mtW= \Transp{\mtY}\mtA \mtX $ é não-singular, então $ \Rank{\mtA - \mtA\mtX\Inv{\mtW}\Transp{\mtY}\mtA} = \Rank{\mtA} - \Rank{\mtA \mtX \Inv{\mtW} \Transp{\mtY} \mtA} $
  \end{enumerate}
\end{frame}

\subsection{Autovalores e autovetores}

\begin{frame}
  \frametitle{Definições básicas}
  \begin{itemize}
    \item Sejam uma matriz $\mtA \in \fdR^{n\times n}$ , um vetor $\vtV \in \fdR^{n}$ e um escalar $\lambda \in \fdR$ tais que %
    \begin{equation}\label{eq_auto}
      \mtA\vtV = \lambda\vtV
    \end{equation}
    \item A transformação linear expressa por $\mtA$ aplicada ao vetor $\vtV$ resulta em uma versão escalonada de $\vtV$, i.e., $\lambda\vtV$
    \item Nesse caso, o escalar $\lambda$ é denominado um {\alert{autovalor}} de $\mtA$ e $\vtV$ é chamado o {\alert{autovetor}} de $\mtA$ associado ao autovalor $\lambda$
    \item Autovalores e autovetores encontram diversas aplicações em engenharia
    \begin{itemize}
      \item Esforços e direções principais em mecânica dos materiais
      \item Estudos de momentos de inércia
      \item Frequências naturais e modos de vibração
      \item Alocação ótima de potência em comunicações co-canal
      \item Formação de feixe em sistemas de comunicação com antenas inteligentes
    \end{itemize}
  \end{itemize}
\end{frame}

\begin{frame}
  \frametitle{Equação característica}
  \begin{itemize}\small
    \item Sendo $\mtI$ uma matriz identidade e $\mathbf{0}$ um vetor de zeros, reescrevemos \eqref{eq_auto} como
    \begin{equation}
      (\mtA - \lambda\mtI)\vtV = \mathbf{0}
    \end{equation}
    \item Se a matriz $(\mtA - \lambda\mtI)$ não é singular \ding{220} $\exists \; (\mtA - \lambda\mtI)^{-1}$  \ding{220} $\vtV = (\mtA - \lambda\mtI)^{-1}\mathbf{0}$ (solução trivial)
    \item Se a matriz $(\mtA - \lambda\mtI)$ é singular, temos que
    \begin{equation}\label{eq_carac}
      \det(\mtA - \lambda\mtI) = 0,
    \end{equation}
    \item Expandindo \eqref{eq_carac} usando as regras para cálculo de determinantes \ding{220} equação polinomial de grau $n$
    \item De fato, \eqref{eq_carac} é chamada de {\alert{equação característica}} da matriz $\mtA$ e suas raízes $\lambda_1, \lambda_2, \ldots, \lambda_n$ são os \alert{autovalores} de $\mtA$
    \item O autovetor $\vtV_i$ associado a $\lambda_i, \; i = 1, 2, \ldots, n$, pode ser determinado substituindo-se $\lambda_i$ em \eqref{eq_auto}, ou seja,
    \begin{equation}\label{eq_lambda}
      \mtA\vtV_i = \lambda_i\vtV_i, \quad i = 1, 2, \ldots, n
    \end{equation}
    \item Por convenção, assume-se que $\Abs{\lambda_{\max}} = \Abs{\lambda_1} \geq \Abs{\lambda_2} \geq \ldots \geq \Abs{\lambda_n} = \Abs{\lambda_{\min}}$ e que $\NormTwo{\vtV_i} = 1, \quad i = 1, 2, \ldots, n$
  \end{itemize}
\end{frame}

\begin{frame}
  \frametitle{Propriedades de autovalores e autovetores}
  \begin{propriedade}
    Se $\lambda_1, \lambda_2, \ldots, \lambda_n$ são os autovalores da matriz $\mtA \in \fdR^{n \times n}$, então os autovalores de $\mtA^k$, $k > 0$, são $\lambda^k_1, \lambda^k_2, \ldots, \lambda^k_n$.
  \end{propriedade}
  \vfill
  \begin{proof}[Prova]
    $\mtA^k \vtV_i =  \mtA^{k-1} \mtA \vtV_i = \lambda_i \mtA^{k-1} \vtV_i = \lambda^2_i \mtA^{k-2} \vtV_i = \ldots = \lambda^{k-1}_i \mtA \vtV_i = \lambda^{k}_i \vtV_i$.
  \end{proof}
  \vfill
  \begin{itemize}
    \item Logo, todo autovetor $\vtV_i$ de $\mtA$ é autovetor de $\mtA^k$.
  \end{itemize}
  \vfill
  \begin{propriedade}
    Sejam  $\vtV_1, \vtV_2, \ldots, \vtV_n$ os autovetores da matriz $\mtA \in \fdR^{n \times n}$, correspondentes a autovalores distintos $\lambda_1, \lambda_2, \ldots, \lambda_n$, então $\vtV_1, \vtV_2, \ldots, \vtV_n$ são linearmente independentes.
  \end{propriedade}
\end{frame}

\begin{frame}
  \frametitle{Propriedades de autovalores e autovetores}
  \vspace{-0.5\baselineskip}
  \begin{proof}[Prova]
    \begin{itemize}
      \item  Se $\vtV_1, \vtV_2, \ldots, \vtV_n$, então existe $\alpha_i, i = 1, 2, \ldots, n$, não todos nulos, tal que $\sum\limits_{i=1}^n \alpha_i \vtV_i = 0$, que multiplicado repetidamente por $\mtA$ leva ao conjunto de $n$ equações
      \begin{equation}\label{eq_li_autovect2}
        \sum\limits_{i=1}^n \alpha_i \lambda^k_i\vtV_i = 0, \quad k = 1, 2, \ldots, n,
      \end{equation}
      o qual pode ser reescrito matricialmente como
      \begin{equation}\label{eq_li_autovect3}
        \begin{bmatrix}
          \alpha_1 \vtV_1 & \alpha_2 \vtV_2 & \alpha_3 \vtV_3 & \ldots & \alpha_n \vtV_n
        \end{bmatrix}
        \underbrace{\begin{bmatrix}
          1 & \lambda_1 & \lambda^2_1 & \ldots & \lambda^{n-1}_1 \\
          1 & \lambda_2 & \lambda^2_2 & \ldots & \lambda^{n-1}_2 \\
          \vdots & \vdots & \vdots & \ddots & \vdots \\
          1 & \lambda_n & \lambda^2_n & \ldots & \lambda^{n-1}_n
        \end{bmatrix}}_{\mtS} = \mtZero
      \end{equation}
    \end{itemize}
  \end{proof}
\end{frame}

\begin{frame}
  \frametitle{Propriedades de autovalores e autovetores}
  \vspace{-0.25\baselineskip}
  \begin{proof}[Prova]
    \begin{itemize}
      \item A matriz $\mtS$ em \eqref{eq_li_autovect3} é chamada \alert{matriz de Vandermonde} e para $\lambda_i$ distintos é não-singular. Logo
      \begin{equation}\label{eq_li_autovect4}
        \begin{bmatrix}
          \alpha_1 \vtV_1 & \alpha_2 \vtV_2 & \ldots & \alpha_n \vtV_n
        \end{bmatrix}
        \mtS = \mtZero \Rightarrow
        \begin{bmatrix}
          \alpha_1 \vtV_1 & \alpha_2 \vtV_2 & \ldots & \alpha_n \vtV_n
        \end{bmatrix}
        = \mtZero \Inv{\mtS} = \mtZero
      \end{equation}
      \item Logo, $\alpha_i$ precisam ser todos nulos, já que $\vtV_i$ são não-nulos e, portanto, $\vtV_i$ são L.I.
    \end{itemize}
  \end{proof}
  \vspace{-0.25\baselineskip}
  \begin{propriedade}
    Sejam $\lambda_1, \lambda_2, \ldots, \lambda_n$ os autovalores da matriz $\mtA \in \fdC^{n \times n}$, $\mtA = \Herm{\mtA}$, então $\lambda_1, \lambda_2, \ldots, \lambda_n \in \fdR$ e $\lambda_i \geq 0$.
  \end{propriedade}
  \vspace{-0.25\baselineskip}
  \begin{proof}[Prova]
    Temos que $\mtA \vtV_i = \lambda_i \vtV_i \Rightarrow \Herm{\vtV}_i \Herm{\mtA}  = \Conj{\lambda}_i \Herm{\vtV}_i \Rightarrow \Herm{\vtV}_i \mtA  = \Conj{\lambda}_i \Herm{\vtV}_i$. Logo \\
    $\mtA \vtV_i = \lambda_i \vtV_i \Rightarrow \Herm{\vtV}_i \mtA \vtV_i = \lambda_i \Herm{\vtV}_i \vtV_i \geq 0 \Rightarrow \Conj{\lambda}_i \Herm{\vtV}_i \vtV_i = \lambda_i \Herm{\vtV}_i \vtV_i \geq 0 \Rightarrow \lambda_i = \Conj{\lambda}_i \geq 0$
  \end{proof}
\end{frame}

\begin{frame}
  \frametitle{Propriedades de autovalores e autovetores}
  \vspace{-0.25\baselineskip}
  \begin{propriedade}
    Sejam  $\vtV_1, \vtV_2, \ldots, \vtV_n$ os autovetores da matriz $\mtA \in \fdC^{n \times n}, \mtA = \Herm{\mtA}$, correspondentes a autovalores distintos $\lambda_1, \lambda_2, \ldots, \lambda_n$, então $\vtV_1, \vtV_2, \ldots, \vtV_n$ são ortogonais.
  \end{propriedade}
  \vspace{-0.25\baselineskip}
  \begin{proof}[Prova]
    Temos que $\mtA \vtV_j = \lambda_j \vtV_j \Rightarrow \Herm{\vtV}_j \Herm{\mtA}  = \Conj{\lambda}_j \Herm{\vtV}_j \Rightarrow \Herm{\vtV}_j \mtA  = \lambda_j \Herm{\vtV}_j$. Logo \\
    $\mtA \vtV_i = \lambda_i \vtV_i \Rightarrow \Herm{\vtV}_j \mtA \vtV_i = \lambda_i \Herm{\vtV}_j \vtV_i \Rightarrow \lambda_j \Herm{\vtV}_j \vtV_i = \lambda_i \Herm{\vtV}_j \vtV_i \Rightarrow \Herm{\vtV}_j \vtV_i = 0$, pois $\lambda_i \neq \lambda_j$
  \end{proof}
  \begin{itemize}\footnotesize
    \item Os autovetores de uma matriz $\mtA$, são ortonormais se associados a autovalores distintos, i.e.,
      \begin{subequations}\label{u}
        \begin{align}
          \Transp{\vtV}_i\vtV_i &= 1, \quad \forall i, \quad i = 1,2,\ldots,n \label{eq_ortogonal}\\
          \Transp{\vtV}_i\vtV_j &= 0, \quad \forall i \neq j, \quad i,j = 1,2,\ldots,n \label{eq_normal}
        \end{align}
      \end{subequations}
    \item Formam uma base para o espaço coluna gerado pela matriz, i.e., qualquer vetor $\vtX$ nesse espaço pode ser escrito como uma combinação linear
    \begin{equation}\label{eq_comb_lin}
      \vtX = \alpha_1\vtV_1 + \alpha_2\vtV_1 + \cdots + \alpha_n\vtV_n
    \end{equation}
    onde $\alpha_i, \; i = 1,2,\ldots,n$, são constantes reais tais $\alpha_i = 0, \forall i \Leftrightarrow \vtX = \mathbf{0}$
  \end{itemize}
\end{frame}

\begin{frame}
  \frametitle{Propriedades de autovalores e autovetores}
  \vspace{-0.25\baselineskip}
  \begin{propriedade}
    Sejam $\vtV_1, \vtV_2, \ldots, \vtV_n$ os autovetores da matriz $\mtA$, correspondentes a autovalores distintos $\lambda_1, \lambda_2, \ldots, \lambda_n$, então $\Inv{\mtV} \mtA \mtV = \mtLambda, \mtA \in \fdR^{n\times n}$ e $\Herm{\mtV} \mtA \mtV = \mtLambda, \mtA \in \fdC^{n\times n}$ onde $\mtV = \begin{bmatrix} \vtV_1 & \vtV_2 & \ldots & \vtV_n\end{bmatrix}$ e $\mtLambda = \diag\{\lambda_1, \lambda_2, \ldots, \lambda_n\}$.
  \end{propriedade}
  \vspace{-0.25\baselineskip}
  \begin{proof}[Prova]
    A prova segue diretamente das propriedades de independência linear ($\exists \Inv{\mtV}$) e de ortogonalidade ($\Herm{\mtV}\mtV = \mtI$).
  \end{proof}
  \vspace{-0.25\baselineskip}
  \begin{propriedade}
    Sejam $\lambda_1, \lambda_2, \ldots, \lambda_n$ os autovalores da matriz $\mtA$, então o traço $\Trace{\mtA}$ de $\mtA$ é igual à soma dos autovalores de $\mtA$.
  \end{propriedade}
  \vspace{-0.25\baselineskip}
  \begin{proof}[Prova]
    Temos que $\Trace{\mtA} = \sum\limits_{i=1}^{n} a_{i,i} = \Trace{\mtV \mtLambda \Inv{\mtV}} = \Trace{\mtLambda \Inv{\mtV} \mtV} = \Trace{\mtLambda} = \sum\limits_{i=1}^{n} \lambda_i$.
  \end{proof}
\end{frame}

\begin{frame}
  \frametitle{Propriedades de autovalores e autovetores}
  \begin{propriedade}
    Sejam $\lambda_1, \lambda_2, \ldots, \lambda_n$ os autovalores da matriz $\mtA$, então o determinante $\Det{\mtA}$ de $\mtA$ é igual ao produto dos autovalores de $\mtA$.
  \end{propriedade}
  \vspace{-0.25\baselineskip}
  \begin{proof}[Prova]
    Temos que $\Det{\mtA} = \Det{\mtV \mtLambda \Inv{\mtV}} = \Det{\mtV} \Det{\mtLambda} \Det{\Inv{\mtV}} = \Det{\mtV} \Det{\mtLambda} \Det{\mtV}^{-1} = \Det{\mtLambda} = \prod\limits_{i=1}^{n} \lambda_i$.
  \end{proof}
\end{frame}

\begin{frame}
  \frametitle{Transformação de similaridade}
  \begin{itemize}
    \item Transformações de similaridade são úteis para transformar uma matriz associada a um problema em uma forma similar e de mais simples manipulação
    \item Se agruparmos as $n$ equações em \eqref{eq_lambda} em uma única equação podemos escrever
    \begin{equation}\label{eq_mat_auto}
      \begin{split}
        \mtA \underbrace{\left[\begin{matrix} \vtV_1 & \vtV_2 & \ldots & \vtV_n\end{matrix}\right]}_{\mtV} &=
        \underbrace{ \left[\begin{matrix} \vtV_1 & \vtV_2 & \ldots & \vtV_n\end{matrix}\right] }_{\mtV} \underbrace{\left[\begin{matrix}
          \lambda_1 & 0 & \ldots & 0 \\
          0 & \lambda_2 & \ldots & 0 \\
          \vdots & \vdots & \ddots & \vdots \\
          0 & 0 & \ldots & \lambda_n \\
        \end{matrix}\right]}_{\mtLambda} \\
        \mtA\mtV &= \mtV\mtLambda \Rightarrow \boxed{\mtA = \mtV\mtLambda\Inv{\mtV}}
      \end{split}
    \end{equation}
    \item As matrizes $\mtLambda$ e $\mtV$ são as matrizes dos autovalores e autovetores de $\mtA$, respectivamente
  \end{itemize}
\end{frame}

\begin{frame}
  \frametitle{Transformação de similaridade}
  \begin{itemize}\footnotesize
    \item Duas matrizes $\mtA$ e $\mtB$ de dimensão $n \times n$ são ditas \textbf{similares} se existe uma transformação $\mtT$ tal que
    \begin{equation}\label{eq_similaridade}
      \mtA = \mtT\mtB\Inv{\mtT} \Rightarrow \mtB = \Inv{\mtT}\mtA\mtT
    \end{equation}
    \item Matrizes similares possuem os mesmos autovalores e encontram várias aplicações em engenharia
    \item A partir de \eqref{eq_mat_auto} e \eqref{eq_similaridade}, podemos observar que
    \begin{equation}\label{eq_autosimilar}
      \mtA = \mtV\mtLambda\Inv{\mtV} \text { e } \mtLambda = \Inv{\mtV}\mtA\mtV,
    \end{equation}
    de modo que a matriz $\mtA$ é similar à matriz diagonal $\mtLambda$
    \item Observe ainda que há uma relação entre as inversas de matrizes similares dada por
    \begin{equation}\label{eq_inv_sim}
      \Inv{\mtA} = \left(\Inv{\mtV}\right)^{-1}\Inv{\mtLambda}\Inv{\mtV} = \mtV\Inv{\mtLambda}\Inv{\mtV},
    \end{equation}
    de modo que a inversa de $\mtA$ pode ser facilmente obtida se $\mtLambda$ e $\mtV$ forem conhecidas, pois $\mtLambda$ é diagonal
    \item Note ainda que $\Det{\Inv{\mtA}} = \Det{\Inv{\mtLambda}} = \prod\limits_{i=1}^n\lambda^{-1}_i$
  \end{itemize}
\end{frame}

\begin{frame}
  \frametitle{Forma de Jordan}
  \begin{itemize}
    \item ...
  \end{itemize}
\end{frame}

\subsection{Funções de matrizes quadradas}

\begin{frame}
  \frametitle{Polinômios de matrizes quadradas}
  \begin{itemize}
    \item Se $\mtA$ é uma matriz quadrada e $k$ é um número não-negativo, então
    \begin{equation}
      \mtA^k = \underbrace{\mtA\cdot\mtA\cdot\mtA\cdots\mtA}_{k \text{ vezes}} \quad \text{e} \quad \mtA^0 = \mtI
    \end{equation}
    \item Se $f(\lambda)$ é um polinômio qualquer e, por exemplo, $f_1(\lambda) = \lambda^3 + 2\lambda^2 - 6$ e $f_2(\lambda) = (\lambda+2)(4\lambda-3)$ então
    \begin{equation}
      f_1(\mtA) = \mtA^3 + 2\mtA^2 - 6\mtI \quad \text{e} \quad f_2(\mtA) = (\mtA+2\mtI)(4\mtA - 3\mtI)
    \end{equation}
    \item Em outros termos, polinômios podem ser aplicados diretamente a matrizes constituindo assim uma classe de funções de matrizes
  \end{itemize}
\end{frame}

\begin{frame}
  \frametitle{Teorema de Cayley-Hamilton}
  \begin{itemize}
    \item Se $\mtA$ é uma matriz quadrada, então seu polinômio característico $\Delta(\lambda) = \Det{\lambda\mtI - \mtA}$ é dado por
    \begin{equation}\label{eq_delta_char}
      \Delta(\lambda) = \lambda^n + \alpha_{n-1}\lambda^{n-1} + \alpha_{n-2}\lambda^{n-2} + \ldots + \alpha_{1}\lambda + \alpha_0
    \end{equation}
    \item O teorema de Cayley-Hamilton estabelece que a função polinomial \eqref{eq_delta_char} aplicada à matriz $\mtA$ é identicamente nula, i.e.,
    \begin{equation}\label{eq_cayley}
      \Delta(\mtA) = \mtA^n + \alpha_{n-1}\mtA^{n-1} + \alpha_{n-2}\mtA^{n-2} + \ldots + \alpha_{1}\mtA + \alpha_0\mtI = \mtZero,
    \end{equation}
    \item Logo, $\mtA$ satisfaz sua própria equação característica
    \item O teorema de Cayley-Hamilton é útil para calcular potências de $\mtA^{k}, k > n$ em função de $\mtA^n, \ldots, \mtA$, como por exemplo
    {\small\begin{equation}
      \begin{split}\label{eq_a_n_plus_one}
        \mtA^{n+1} &= \mtA\cdot(\mtA^n + \alpha_{n-1}\mtA^{n-1} + \alpha_{n-2}\mtA^{n-2} + \ldots + \alpha_{1}\mtA + \alpha_0\mtI) = \mtZero \Rightarrow \\
        \mtA^{n+1} &= -\alpha_{n-1}\mtA^{n} - \alpha_{n-2}\mtA^{n-1} - \ldots - \alpha_{1}\mtA^2 - \alpha_0\mtA
      \end{split}
    \end{equation}}
  \end{itemize}
\end{frame}

\begin{frame}
  \frametitle{Teorema de Cayley-Hamilton}
  \begin{itemize}
    \item De acordo com \eqref{eq_cayley}, $\mtA^{n}$ pode ser escrita como combinação linear de $\left\{\mtI, \mtA, \ldots, \mtA^{n-1}\right\}$
    \item De acordo com \eqref{eq_a_n_plus_one}, $\mtA^{n+1}$ pode ser escrita como combinação linear de $\left\{\mtA, \mtA^2, \ldots, \mtA^n\right\}$
    \item Logo, para qualquer função polinomial $f(\lambda)$ temos que $f(\mtA)$ pode ser escrita como
    \begin{equation}
      f(\mtA) = \beta_{0}\mtI + \beta_{1}\mtA + \ldots + \beta_{n-2}\mtA^{n-2} + \beta_{n-1}\mtA^{n-1}
    \end{equation}
    para algum conjunto de valores $\beta_0, \beta_1, \ldots, \beta_{n-1}$
    \item Se $\mtA = \mtT\bar{\mtA}\Inv{\mtT}$ então $\mtA^k = \mtT\bar{\mtA}^k\Inv{\mtT}$ e, portanto $ f(\mtA) = \mtT f(\bar{\mtA})\Inv{\mtT} $, pois
    \begin{multline*}
      f(\mtA) = f(\mtT\bar{\mtA}\Inv{\mtT}) \\ 
      = \beta_{0}\mtT \Inv{\mtT} + \beta_{1}\mtT \bar{\mtA} \Inv{\mtT} + \ldots + \beta_{n-2}\mtT\bar{\mtA}^{n-2}\Inv{\mtT} + \beta_{n-1}\mtT \bar{\mtA}^{n-1} \Inv{\mtT} \\
      = \mtT(\beta_{0}\mtI + \beta_{1}\mtA + \ldots + \beta_{n-2}\mtA^{n-2} + \beta_{n-1}\mtA^{n-1})\Inv{\mtT} \\
      = \mtT f(\bar{\mtA})\Inv{\mtT}
    \end{multline*}
  \end{itemize}
\end{frame}

\subsection{Norma de matrizes}

\begin{frame}
  \frametitle{Norma induzida}
  \begin{itemize}
    \item A {\alert{norma induzida}} $\Norm{\mtA}$ de uma matriz $\mtA$ pode ser definida através do problema de otimização
    \begin{subequations}\label{eq_induced_norm}
      \begin{align}
        \Norm{\mtA} &= \Max{\NormTwo{\mtA\vtX}} = \Max{(\Transp{\vtX}\Transp{\mtA}\mtA\vtX)^{\frac{1}{2}}}, \label{eq_induced_norm_a} \\
        \SubTo \NormTwo{\vtX} &= 1 \label{eq_induced_norm_b}
      \end{align}
    \end{subequations}
    \item Dado que o $\Range{\mtA} = \Range{\Transp{\mtA}\mtA}$ e que $\NormTwo{\vtX} = 1$ conforme \eqref{eq_induced_norm_b}, verificamos que a solução de \eqref{eq_induced_norm} corresponde à raiz quadrada do maior autovalor de $\Transp{\mtA}\mtA$
    \item A norma induzida acima pode ser definida em termos de outras normas que não a norma euclidiana
    \item Apenas a norma induzida pela norma euclidiana é também chamada {\alert{norma espectral}}
  \end{itemize}
\end{frame}

\begin{frame}
  \frametitle{Norma induzida}
  \begin{itemize}
    \item As normas induzidas para matrizes satisfazem as mesmas propriedades que as normas de vetores, tais como:
    \begin{subequations}
      \begin{align}
        \Norm{\mtA} &= \Norm{\Transp{\mtA}} \geq 0 & \text{(Não-negatividade)}\\
        \Norm{\mtA\mtB} &\leq \Norm{\mtA}\Norm{\mtB} & \text{(Desigualdade de Cauchy-Schwarz)} \\
        \Norm{\mtA} &= 0 \Leftrightarrow \mtA = \mtZero & \text{(Elemento neutro)} \\
        \Norm{\alpha\mtA} &= \Abs{\alpha}\Norm{\mtA}, \forall \alpha \in \fdR & \text{(Escalabilidade)} \\
        \Norm{\mtA + \mtB} &\leq \Norm{\mtA} + \Norm{\mtB}, \forall \mtA, \mtB \in \fdR^{n\times n} & \text{(Desigualdade triangular)} \\
        \underset{\substack{1 \leq i \leq m \\ 1 \leq j \leq n}}{\max} \Abs{a_{i,j}} &\leq \Norm{\mtA} \leq \sqrt{mn} \underset{\substack{1 \leq i \leq m \\ 1 \leq j \leq n}}{\max}\Abs{a_{i,j}}
      \end{align}
    \end{subequations}
    \item Para matrizes complexas, as matrizes transpostas são substituídas por matrizes hermitianas
  \end{itemize}
\end{frame}

\subsection{Cálculo com matrizes}

\begin{frame}
  \frametitle{Cálculo com matrizes}
  \begin{itemize}
    \item As regras de cálculo aplicadas a matrizes devem ser consistentes com as regras de cálculo utilizando escalares
    \item Os resultados sugerem apenas a reorganização dos termos entre formas escalares e matriciais
    \item Este princípio conduz à conclusão de que a derivada e a integral de matrizes pode ser definida elemento-a-elemento, i.e., o elemento $i,j$ das matrizes
    \begin{equation}
      \int_{0}^t \mtA(\tau)d\tau \quad \text{e} \quad \frac{d\mtA(t)}{dt} \quad \text{são} \quad \int_{0}^t a_{i,j}(\tau)d\tau \quad \text{e} \quad \frac{d a_{i,j}(t)}{dt},
    \end{equation}
    respectivamente
    \item O teorema fundamental do cálculo aplicado a matrizes leva a
    \begin{equation}
      \frac{d}{dt}\left(\int_{0}^t \mtA(\tau)d\tau\right) = \mtA(t)
    \end{equation}
  \end{itemize}
\end{frame}

\begin{frame}
  \frametitle{Cálculo com matrizes}
  \begin{itemize}
    \item Similarmente, a regra do produto para matrizes $\mtA(t)$ e $\mtB(t)$ resulta em
    \begin{equation}
      \frac{d}{dt}\left(\mtA(t)\mtB(t)\right) = \dot{\mtA}(t)\mtB(t) + \mtA(t)\dot{\mtB}(t)
    \end{equation}
    \item No entanto, é importante observar que
    \begin{equation}
      \frac{d\mtA^2(t)}{dt} = \dot{\mtA}(t)\mtA(t) + \mtA(t)\dot{\mtA}(t)
    \end{equation}
    \item Nesse contexto, um relação importante é a desigualdade triangular %
    \begin{equation}
      \Norm{\int_{t_0}^t \vtX(\tau)d\tau} \leq \Abs{\int_{t_0}^t \Norm{\vtX(\tau)}d\tau}
    \end{equation}
    onde para $ t \geq t_0 $ o módulo pode ser desconsiderado
  \end{itemize}
\end{frame}

\subsection{Forma quadrática e matriz positiva (semi)definida}

\begin{frame}
  \frametitle{Forma quadrática e matriz positiva (semi)definida}
  \begin{itemize}
    \item Seja $\mtQ$ uma matriz pertencente a $\fdR^{n\times n}$ e $\vtX$ um vetor pertencente a $\fdR^n$
    \item O produto $\Transp{\vtX}\mtQ\vtX$ é chamado {\alert{forma quadrática}} em $\vtX$
    \item A matriz $\mtQ$ pode ser considerada simétrica (i.e., $\mtQ = \Transp{\mtQ}$) no estudo de formas quadráticas pois
    \begin{equation}
      \Transp{\vtX}\left(\mtQ + \Transp{\mtQ}\right)\vtX = \Transp{\vtX}\mtQ\vtX + \Transp{\vtX}\Transp{\mtQ}\vtX = 2\Transp{\vtX}\mtQ\vtX
    \end{equation}
    e a forma quadrática não muda ao se substituir $\mtQ$ por $\dfrac{\mtQ + \Transp{\mtQ}}{2}$
    \item Um matriz $\mtQ$ é chamada:
    \begin{itemize}
      \item {\alert{Positiva definida}} se $\Transp{\vtX}\mtQ\vtX > 0, \forall \vtX \neq \vtZero$
      \item {\alert{Positiva semidefinida}} se $\Transp{\vtX}\mtQ\vtX \geq 0, \forall \vtX$
    \end{itemize}
    \item Um matriz $\mtQ$ é {\alert{negativa definida ou semidefinida}} se $-\mtQ$ é positiva definida ou semidefinida, respectivamente
  \end{itemize}
\end{frame}

\begin{frame}
  \frametitle{Forma quadrática e matriz positiva (semi)definida}
  \begin{itemize}
    \item A notação $\mtQ \succ 0$ e $\mtQ \succeq 0$ é normalmente utilizada para indicar que $\mtQ$ é positiva definida ou semidefinida, respectivamente
    \item A notação $\mtQ_1 \succ \mtQ_2$ e $\mtQ_1 \succeq \mtQ_2$ implicam $\mtQ_1 - \mtQ_2 \succ 0$ e $\mtQ_1 - \mtQ_2 \succeq 0$, respectivamente
    \item Todos os autovalores de uma matriz simétrica são reais
    \begin{itemize}
      \item Todos os autovalores de $\mtQ$ são reais e positivos se $\mtQ \succ 0$
      \item Todos os autovalores de $\mtQ$ são reais e não-negativos se $\mtQ \succeq 0$
    \end{itemize}
    \item Algumas relações importantes para formas quadráticas são:
    {\small \begin{subequations}
      \begin{align}
        \lambda_{\min}\Transp{\vtX}\vtX &\leq \Transp{\vtX}\mtQ\vtX \leq \lambda_{\max}\Transp{\vtX}\vtX, & \text{(Desigualdade de Rayleigh-Ritz)}\\
        \Norm{\mtQ} & \leq \Trace{\mtQ} \leq n \Norm{\mtQ}
      \end{align}
    \end{subequations}}
    \item Para matrizes e vetores complexos, a operação $\Transp{(\cdot)}$ é substituída por $\Herm{(\cdot)}$ e a forma quadrática em $\vtX$ torna-se $\Herm{\vtX}\mtQ\vtX$
  \end{itemize}
\end{frame}

\begin{frame}{Menores principais dominantes}
  \begin{itemize}\small
    \item Um {\alert{menor principal}} de uma matriz $\mtQ$ simétrica em $\fdR^{n\times n}$ é o determinante da submatriz formada pela remoção de $ r $ linhas e colunas de mesmo índice da matriz
    \item Para a mesma matriz $\mtQ$, o $p$-ésimo {\alert{menor principal dominante}} de $\mtQ$ é o determinante da submatriz $\mtQ_p$ superior esquerda compreendendo os elementos $q_{i,j}$ de $\mtQ$ para $i,j = 1, 2, \ldots, p$, de modo que o 1$^\text{o}$, 2$^\text{o}$ e 3$^\text{o}$ menores principais dominantes $\mtQ \in \fdR^{n\times n}$, $n \geq 3$, são
    {\footnotesize \begin{equation}
      \begin{split}
        \Det{\mtQ_1} &= \Det{\left[q_{1,1}\right]}, \quad \Det{\mtQ_2} = \Det{\left[\begin{matrix}q_{1,1} & q_{1,2} \\ q_{2,1} & q_{2,2}\end{matrix}\right]}, \text{ e } \\
        \Det{\mtQ_3} &= \Det{\left[\begin{matrix}q_{1,1} & q_{1,2} & q_{1,3}\\ q_{2,1} & q_{2,2} & q_{2,3} \\ q_{3,1} & q_{3,2} & q_{3,3}\end{matrix}\right]}
      \end{split}
    \end{equation}}
    \item A matriz $\mtQ$ é positiva definida se e somente se todos os seus menores principais dominantes são positivos, i.e., se $\Det{\mtQ_p} > 0,$ $\forall p = 1, 2, \ldots, n$
    \item A matriz $\mtQ$ é positiva semidefinida se e somente se todos os seus menores principais dominantes são não-negativos
  \end{itemize}
\end{frame}



\subsection{Fatoração QR}

\begin{frame}{Fatoração QR~\cite[cap. 4]{Gilat2008}}
  \footnotesize
  \begin{itemize}
    \item O método da fatoração QR faz uso de {\alert{transformações de Householder}} e de {\alert{transformações de similaridade}} para decompor uma matriz $\mtA$ em um produto de uma matriz ortogonal $\mtQ$ ($\mtQ\Transp{\mtQ} = \mtI$), por uma matriz triangular superior $\mtR$
    \item A fatoração QR se inicia com a matriz $\mtA^{(1)} = \mtA$ cujos autovalores devem ser determinados
    \item A matriz $\mtA^{(1)}$ é fatorada como
    \begin{equation}\label{eq_qr1}
      \mtA^{(1)} = \mtQ^{(1)}\mtR^{(1)},
    \end{equation}
    onde $\mtQ^{(1)}$ é uma matriz ortogonal, i.e., $\mtQ^{(1)}\Transp{\left(\mtQ^{(1)}\right)} = \mtI$, mas a matriz $\mtR^{(1)}$ não é ainda uma matriz triangular superior
    \item A matriz $\mtA^{(2)}$ é obtida multiplicando $\mtR^{(1)}$ à direita por $\mtQ^{(1)}$, i.e.,
    \begin{equation}\label{eq_qr2}
      \mtA^{(2)} = \mtR^{(1)}\mtQ^{(1)}
    \end{equation}
    \item Usando \eqref{eq_qr1}, temos que $\mtR^{(1)} = \left(\mtQ^{(1)}\right)^{-1}\mtA^{(1)}$, a qual substituida em \eqref{eq_qr2} resulta em
    \begin{equation}
      \mtA^{(2)} = \left(\mtQ^{(1)}\right)^{-1}\mtA^{(1)}\mtQ^{(1)}
    \end{equation}
    \item Logo, $\mtA^{(1)}$ e $\mtA^{(2)}$ são matrizes similares, i.e., possuem os mesmos autovalores
  \end{itemize}
\end{frame}

\begin{frame}
  \frametitle{Matriz de Householder}
  \begin{itemize}
    \item Para obter $\mtQ$ e $\mtR$, o método de decomposição QR utiliza matrizes de transformação de Householder
    \item Dado um vetor
    \begin{equation}
      \vtV = \mathbf{c} + \Vert \mathbf{c} \Vert_2\vtE,
    \end{equation}
    a {\alert{matriz de transformação de Householder}} $\mathbf{H}$ associada ao vetor $\vtV$ é definida como
    \begin{equation}\label{eq_householder}
      \mathbf{H} = \mtI - 2\frac{\vtV\Transp{\vtV}}{\Transp{\vtV}\vtV},
    \end{equation}
    onde $\mtI$ é a matriz identidade e $\vtE$ é um vetor com uma única componente igual a $\pm 1$ e todas as demais componentes igual a zero.
    \item Por conveniência, costuma-se definir ainda o vetor $\vtE$ como $\pm\vtI_i$, i.e., em termos da $i$-ésima coluna $\mtI$
    \item Note que a matriz de transformação de Householder é ortogonal, i.e., $\Transp{\mtH}\mtH = \mtI$
  \end{itemize}
\end{frame}

\begin{frame}
  \frametitle{Algoritmo de fatoração QR}
  \begin{itemize}
    \item O \textbf{passo 1} do algoritmo QR consiste em determinar $\mtQ^{(1)}$ e $\mtR^{(1)}$
    \item Para tanto, os vetores $\mathbf{c}^{(1)}$ e $\vtE^{(1)}$ geradores da matriz de Householder $\mathbf{H}^{(1)}$ são definidos como
    \begin{equation}\label{eq_hh1}
      \begin{split}
      \mathbf{c}^{(1)} &= \mathbf{a}_1, \text{ onde }
      \mtA = \left[\begin{matrix}
        \mathbf{a}_1 & \mathbf{a}_2 & \ldots & \mathbf{a}_n
      \end{matrix}\right] = \left[\begin{matrix}
        a_{1,1} & a_{1,2} & \ldots & a_{1,n} \\
        a_{2,1} & a_{2,2} & \ldots & a_{2,n} \\
        \vdots & \vdots & \vdots & \ddots & \vdots \\
        a_{n,1} & a_{n,2} & \ldots & a_{n,n}
      \end{matrix}\right], \text{ e }\\
      \vtE^{(1)} &=
      \begin{cases}
        \vtI_1, & a_{1,1} \geq 0 \\
        -\vtI_1, & a_{1,1} < 0
      \end{cases}, \text{ onde } \mtI = \left[\begin{matrix}
        \vtI_1 & \vtI_2 & \ldots & \vtI_n
      \end{matrix}\right]
      \end{split}
    \end{equation}
    \item Usando \eqref{eq_hh1} em \eqref{eq_householder}, as matrizes $\mtQ^{(1)}$ e $\mtR^{(1)}$ são definidas como
    \begin{equation}
      \mtQ^{(1)} = \mathbf{H}^{(1)} \quad \text{ e } \quad  \mtR^{(1)} = \mathbf{H}^{(1)}\mtA^{(1)}
    \end{equation}
  \end{itemize}
\end{frame}

\begin{frame}
  \frametitle{Algoritmo de fatoração QR}
  \begin{itemize}
    \item A matriz $\mtR^{(1)}$ obtida no \textbf{passo 1} é da forma
    \begin{equation}
      \mtR^{(1)} =
      \left[\begin{matrix}
        \mathbf{r}^{(1)}_{1,1} & \mathbf{r}^{(1)}_{1,2} & \mathbf{r}^{(1)}_{1,3} & \ldots & \mathbf{r}^{(1)}_{1,n} \\
        0 & \mathbf{r}^{(1)}_{2,2} & \mathbf{r}^{(1)}_{2,3} & \ldots & \mathbf{r}^{(1)}_{2,n} \\
        0 & \mathbf{r}^{(1)}_{3,2} & \mathbf{r}^{(1)}_{3,3} & \ldots & \mathbf{r}^{(1)}_{3,n} \\
        \vdots & \vdots & \ddots & \vdots \\
        0 & \mathbf{r}^{(1)}_{n,2} & \mathbf{r}^{(1)}_{n,3} & \ldots & \mathbf{r}^{(1)}_{n,n}
      \end{matrix}\right]
    \end{equation}
    \item No \textbf{passo 2} do algoritmo QR, aproximadamente o mesmo processo do \textbf{passo 1} é repetido
    \item No entanto, os vetores $\mathbf{c}^{(2)}$ e $\vtE^{(2)}$ geradores da matriz de Householder $\mathbf{H}^{(2)}$ são definidos como
    \begin{equation}\label{eq_hh2}
      \mathbf{c}^{(2)} = \left[\begin{matrix}
        0 & r^{(1)}_{2,2} & r^{(1)}_{3,2} & \ldots & r^{(1)}_{n,2}
      \end{matrix}\right]^{T}, \quad \text{e} \quad
      \vtE^{(2)} =
      \begin{cases}
        \vtI_2, & r^{(1)}_{2,2} \geq 0 \\
        -\vtI_2, & r^{(1)}_{2,2} < 0
      \end{cases}
    \end{equation}
  \end{itemize}
\end{frame}

\begin{frame}{Algoritmo de fatoração QR}
  \begin{itemize}\small
    \item Usando \eqref{eq_hh2} e \eqref{u}, as matrizes $\mtQ^{(2)}$ e $\mtR^{(2)}$ são obtidas como
    \begin{equation}
      \mtQ^{(2)} = \mtQ^{(1)}\mathbf{H}^{(2)} \quad \text{ e } \quad  \mtR^{(2)} = \mathbf{H}^{(2)}\mtR^{(1)}
    \end{equation}
    onde a matriz $\mtR^{(2)}$ tem a forma
    \begin{equation}
      \mtR^{(2)} =
      \left[\begin{matrix}
        \mathbf{r}^{(2)}_{1,1} & \mathbf{r}^{(2)}_{1,2} & \mathbf{r}^{(2)}_{1,3} & \ldots & \mathbf{r}^{(2)}_{1,n} \\
        0 & \mathbf{r}^{(2)}_{2,2} & \mathbf{r}^{(2)}_{2,3} & \ldots & \mathbf{r}^{(2)}_{2,n} \\
        0 & 0 & \mathbf{r}^{(2)}_{3,3} & \ldots & \mathbf{r}^{(2)}_{3,n} \\
        \vdots & \vdots & \ddots & \vdots \\
        0 & 0 & \mathbf{r}^{(2)}_{n,3} & \ldots & \mathbf{r}^{(2)}_{n,n}
      \end{matrix}\right]
    \end{equation}
    \item No \textbf{passo 3} aproximadamente o mesmo processo do \textbf{passo 2} é repetido, exceto que os vetores $\mathbf{c}^{(3)}$ e $\vtE^{(3)}$ geradores $\mathbf{H}^{(3)}$ são dados por
    \begin{equation}\label{eq_hh3}
      \mathbf{c}^{(3)} = \left[\begin{matrix}
        0 & 0 & r^{(2)}_{3,3} & \ldots & r^{(2)}_{n,3}
      \end{matrix}\right]^{T}, \quad \text{e} \quad
      \vtE^{(3)} =
      \begin{cases}
        \vtI_3, & r^{(2)}_{3,3} \geq 0 \\
        -\vtI_3, & r^{(2)}_{3,3} < 0
      \end{cases}
    \end{equation}
  \end{itemize}
\end{frame}

\begin{frame}
  \frametitle{Algoritmo de fatoração QR}
  \begin{itemize}
    \item Usando \eqref{eq_hh3} e \eqref{eq_householder}, as matrizes $\mtQ^{(3)}$ e $\mtR^{(3)}$ são obtidas como
    \begin{equation}
      \mtQ^{(3)} = \mtQ^{(2)}\mathbf{H}^{(3)} \quad \text{ e } \quad  \mtR^{(3)} = \mathbf{H}^{(3)}\mtR^{(2)}
    \end{equation}
    \item Em cada passo, o processo descrito anteriormente é repetido
    \item Observe ainda que a cada passo $i$, os elementos da $i$-ésima coluna da matriz $\mtR^{(i)}$ são zerados
    \item De forma geral, as iterações descritas anteriormente podem ser escritas como
    \begin{equation}
      \mtQ^{(i)} = \mtQ^{(i-1)}\mathbf{H}^{(i)} \quad \text{ e } \quad  \mtR^{(i)} = \mathbf{H}^{(i)}\mtR^{(i-1)}
    \end{equation}
    onde $\mtQ^{(0)} = \mtI$ e $\mtR^{(0)} = \mtA$
    \item Um total de $n-1$ passos é realizado até que a matriz $\mtR^{(n-1)} = \mathbf{H}^{(n-1)}\mtR^{(n-2)}$ obtida é triangular superior
    \item Esse processo pode ser repetido até que a matriz $ \mtA^{(n)} = \mtR^{(n-1)}\mtQ^{(n-1)} $ seja triangular. Nesse caso, os autovalores de $ \mtA $ serão os elementos da diagonal de $ \mtA^{(n)} $
  \end{itemize}
\end{frame}


% !TeX root = Otimizacao.tex
% !TeX encoding = UTF-8
% !TeX spellcheck = pt_BR
% !TeX program = pdflatex

\section{Conceitos de Derivadas Multivariável}

%\renewcommand*{\arraystretch}{1.8}

\subsection{Fundamentos Teóricos}

\begin{frame}
	\frametitle{\normalsize Derivadas Multivariável}
	\begin{itemize}
		\item Na resolução de problemas de otimização é bastante importante que se tenha uma base sólida sobre derivadas multivariável.
		\item Nesta seção, serão apresentados alguns casos, que são comumente encontrados na literatura.
	\end{itemize}
\end{frame}

\begin{frame}
	\frametitle{\normalsize Derivada Vetor-Escalar}
	\begin{itemize}
		\item Seja um vetor $\vtY = \Transp{\begin{bmatrix} y_1 & y_2 & \dots & y_N \end{bmatrix}} \in \mathbb{C}^{N \times 1}$ e um escalar $x$, temos que:
		\[\renewcommand{\arraystretch}{1.8}
			\dfrac{\partial \vtY}{\partial x} = \begin{bmatrix} 
				\dfrac{\partial y_1}{\partial x} \\
				\dfrac{\partial y_2}{\partial x} \\ 
				\vdots \\ 
				\dfrac{\partial y_N}{\partial x} 
			\end{bmatrix}
		\]
		\item Como se pode perceber, para se derivar um vetor em relação a um escalar, basta que derivar cada elemento do vetor $\vtY$ pelo escalar $x$.
		\item Neste caso, a dimensão de $\dfrac{\partial \vtY}{\partial x}$ é a mesma de $\vtY$.
	\end{itemize}
\end{frame}

\begin{frame}
	\frametitle{\normalsize Derivada Escalar-Vetor}
	\begin{itemize}
		\item Sejam agora o vetor $\vtX = \Transp{\begin{bmatrix} x_1 & x_2 & \dots & x_N \end{bmatrix}} \in \mathbb{C}^{N \times 1}$ e um escalar $y$, temos que:
		\[\renewcommand{\arraystretch}{1.8}
			\dfrac{\partial y}{\partial \vtX} = \begin{bmatrix} 
				\dfrac{\partial y}{\partial x_1} \\
				\dfrac{\partial y}{\partial x_2} \\ 
				\vdots \\ 
				\dfrac{\partial y}{\partial x_N} 
			\end{bmatrix}
		\]
		\item De modo análogo ao caso anterior, para se derivar um escalar em relação a um vetor, basta que derivar o escalar $y$ em relação a cada elemento do vetor $\vtX$.
		\item A dimensão de $\dfrac{\partial y}{\partial \vtX}$ é a mesma de $\vtX$.
	\end{itemize}
\end{frame}

\begin{frame}
	\frametitle{\normalsize Derivada Vetor-Vetor}
	\begin{itemize}
		\item Sejam os vetores $\vtX = \Transp{\begin{bmatrix} x_1 & x_2 & \dots & x_N \end{bmatrix}}$ e $\vtY = \Transp{\begin{bmatrix} y_1 & y_2 & \dots & y_M \end{bmatrix}}$, em que $\vtX \in \mathbb{C}^{N \times 1}$ e $\vtY \in \mathbb{C}^{M \times 1}$.
		\item Existem na literatura duas formas distintas de se representar a derivada de vetor em relação a outro vetor.
		\vspace*{-3ex}
		\begin{columns}
			\begin{column}[t]{0.4\paperwidth}
				\begin{itemize}
					\item ``\textit{Denominator Layout}'' ou formulação Jacobiana;
					\[\renewcommand{\arraystretch}{1.8}
						\left[ \dfrac{\partial \vtY}{\partial \vtX} \right]_{\textrm{Den}} = \begin{bmatrix} 
							\dfrac{\partial \vtY}{\partial x_1} \\
							\dfrac{\partial \vtY}{\partial x_2} \\ 
							\vdots \\ 
							\dfrac{\partial \vtY}{\partial x_N} 
						\end{bmatrix}
					\]
				\end{itemize}
			\end{column}
			\begin{column}[t]{0.4\paperwidth}
				\begin{itemize}
					\item ``\textit{Numerator Layout}'' ou formulação Hessiana;
					\[\renewcommand{\arraystretch}{1.8}
						\left[ \dfrac{\partial \vtY}{\partial \vtX} \right]_{\textrm{Num}} = \begin{bmatrix} 
							\dfrac{\partial \vtY}{\partial x_1} &
							\dfrac{\partial \vtY}{\partial x_2} & 
							\dots &
							\dfrac{\partial \vtY}{\partial x_N} 
						\end{bmatrix}
					\]
				\end{itemize}
			\end{column}
		\end{columns}
	\end{itemize}
\end{frame}

\begin{frame}
	\frametitle{\normalsize Derivada Vetor-Vetor}
	\begin{itemize}
		\item Como se pode perceber
		\[
			\left[ \dfrac{\partial \vtY}{\partial \vtX} \right]_{\textrm{Den}} = \Transp{\left[ \dfrac{\partial \vtY}{\partial \vtX} \right]_{\textrm{Num}}}
		\]
		\item Para nosso estudo, iremos utilizar a formulação Hessiana, por ser a mais comumente utilizada na literatura.
		\item Assim,
		\[\renewcommand{\arraystretch}{1.8}
			\dfrac{\partial \vtY}{\partial \vtX} = \begin{bmatrix}
				\dfrac{\partial y_1}{\partial x_1} & \dfrac{\partial y_2}{\partial x_1} & \dots & \dfrac{\partial y_M}{\partial x_1} \\
				\dfrac{\partial y_1}{\partial x_2} & \dfrac{\partial y_2}{\partial x_2} & \dots & \dfrac{\partial y_M}{\partial x_2} \\
				\vdots & \vdots & \ddots & \vdots \\
				\dfrac{\partial y_1}{\partial x_N} & \dfrac{\partial y_2}{\partial x_N} & \dots & \dfrac{\partial y_M}{\partial x_N} \\
			\end{bmatrix}
		\]
	\end{itemize}
\end{frame}

\begin{frame}
	\frametitle{\normalsize Derivada Matriz-Escalar}
	\begin{itemize}
		\item Segue a mesma lógica que a derivada vetor-escalar.
		\item Desta forma, seja uma matriz $\mtY \in \mathbb{C}^{N \times M}$ e um escalar $x$
		\item Assim,
		\[\renewcommand{\arraystretch}{1.8}
			\dfrac{\partial \mtY}{\partial x} = \begin{bmatrix}
				\dfrac{\partial y_{11}}{\partial x} & \dfrac{\partial y_{12}}{\partial x} & \dots & \dfrac{\partial y_{1M}}{\partial x} \\
				\dfrac{\partial y_{21}}{\partial x} & \dfrac{\partial y_{22}}{\partial x} & \dots & \dfrac{\partial y_{2M}}{\partial x} \\
				\vdots & \vdots & \ddots & \vdots \\
				\dfrac{\partial y_{N1}}{\partial x} & \dfrac{\partial y_{N2}}{\partial x} & \dots & \dfrac{\partial y_{NM}}{\partial x} \\
			\end{bmatrix}
		\]
	\end{itemize}
\end{frame}

\begin{frame}
	\frametitle{\normalsize Derivada Escalar-Matriz}
	\begin{itemize}
		\item Segue a mesma lógica que a derivada escalar-vetor.
		\item Desta forma, seja uma matriz $\mtX \in \mathbb{C}^{N \times M}$ e um escalar $y$
		\item Assim,
		\[\renewcommand{\arraystretch}{1.8}
			\dfrac{\partial y}{\partial \mtX} = \begin{bmatrix}
				\dfrac{\partial y}{\partial x_{11}} & \dfrac{\partial y}{\partial x_{12}} & \dots & \dfrac{\partial y}{\partial x_{1M}} \\
				\dfrac{\partial y}{\partial x_{21}} & \dfrac{\partial y}{\partial x_{22}} & \dots & \dfrac{\partial y}{\partial x_{2M}} \\
				\vdots & \vdots & \ddots & \vdots \\
				\dfrac{\partial y}{\partial x_{N1}} & \dfrac{\partial y}{\partial x_{N2}} & \dots & \dfrac{\partial y}{\partial x_{NM}} \\
			\end{bmatrix}
		\]
	\end{itemize}
\end{frame}

\subsection{Demonstrações de Derivadas Multivariável}
\subsubsection{Derivada de $\Herm{\vtA} \vtX$}

\begin{frame}
	\frametitle{\normalsize Derivada de $\Herm{\vtA} \vtX$}
	\begin{itemize}
		\item Sejam $\vtA, \vtX \in \mathbb{C}^{N \times 1}$, temos que:
		{\tiny
		\begin{align*}
			\dfrac{\partial \Herm{\vtA} \vtX}{\partial \vtX} &= \dfrac{\partial}{\partial \vtX} \left(
			\begin{bmatrix}
				\Conj{a}_1 & \Conj{a}_2 & \dots & \Conj{a}_N
			\end{bmatrix} \begin{bmatrix}
				x_{1} \\ x_{2} \\ \vdots \\ x_{N}
			\end{bmatrix} \right) 
			= \frac{\partial}{\partial \vtX} \left( \sum_{i = 1}^N \Conj{a}_ix_i \right) \\
		\end{align*}}
		\item Recaindo na formulação da derivada escalar-vetor, logo:
		{\tiny
		\begin{align*}
			\frac{\partial \Herm{\vtA} \vtX}{\partial \vtX} &= \begin{bmatrix}
				\dfrac{\partial}{\partial x_1} \left( \sum_{i = 1}^N \Conj{a}_ix_i \right) \\ \dfrac{\partial}{\partial x_2} \left( \sum_{i = 1}^N \Conj{a}_ix_i \right) \\ \vdots \\ \dfrac{\partial}{\partial x_N} \left( \sum_{i = 1}^N \Conj{a}_ix_i \right) 
			\end{bmatrix}
			= \begin{bmatrix}
				\Conj{a}_1 \\ \Conj{a}_2 \\ \vdots \\ \Conj{a}_N
			\end{bmatrix}
		\end{align*}}
		\[
			\boxed{\frac{\partial \Herm{\vtA} \vtX}{\partial \vtX} = \Conj{\vtA}}
		\]
	\end{itemize}
\end{frame}

\subsubsection{Derivada de $\Transp{\vtA} \vtX$}
\begin{frame}
	\frametitle{\normalsize Derivada de $\Transp{\vtA} \vtX$}
	\begin{itemize}
		\item Considerando os mesmo vetores $\vtA, \vtX \in \mathbb{C}^{N \times 1}$, temos que:
		{\tiny
		\begin{align*}
			\frac{\partial \Transp{\vtA} \vtX}{\partial \vtX} &= \frac{\partial}{\partial \vtX} \left(
			\begin{bmatrix}
				a_1 & a_2 & \dots & a_N
			\end{bmatrix} \begin{bmatrix}
				x_{1} \\ x_{2} \\ \vdots \\ x_{N}
			\end{bmatrix} \right)
			= \frac{\partial}{\partial \vtX} \left( \sum_{i = 1}^N a_ix_i \right)
		\end{align*}}
		\item De modo análogo ao caso anterior,
		{\tiny
		\begin{align*}
			\frac{\partial \Transp{\vtA} \vtX}{\partial \vtX} &= \begin{bmatrix}
				\dfrac{\partial}{\partial x_1} \left( \sum_{i = 1}^N a_ix_i \right) \\ \dfrac{\partial}{\partial x_2} \left( \sum_{i = 1}^N a_ix_i \right) \\ \vdots \\ \dfrac{\partial}{\partial x_N} \left( \sum_{i = 1}^N a_ix_i \right) 
			\end{bmatrix} 
			= \begin{bmatrix}
				a_1 \\ a_2 \\ \vdots \\ a_N
			\end{bmatrix}
		\end{align*}}
		\[
			\boxed{\frac{\partial \Transp{\vtA} \vtX}{\partial \vtX} = \vtA}
		\]
	\end{itemize}
\end{frame}

\subsubsection{Derivada de $\Herm{\vtX} \vtA$}
\begin{frame}
	\frametitle{\normalsize Derivada de $\Herm{\vtX} \vtA$}
	\begin{itemize}
		\item Considerando os mesmo vetores $\vtA, \vtX \in \mathbb{C}^{N \times 1}$, temos que:
		{\tiny
		\begin{align*}
			\frac{\partial \Herm{\vtX} \vtA}{\partial \vtX} &= \frac{\partial}{\partial \vtX} \left(
			\begin{bmatrix}
				x^*_1 & x^*_2 & \dots & x^*_N
			\end{bmatrix} \begin{bmatrix}
				a_{1} \\ a_{2} \\ \vdots \\ a_{N}
			\end{bmatrix} \right) 
			= \frac{\partial}{\partial \vtX} \left( \sum_{i = 1}^N x^*_ia_i \right)
		\end{align*}}
		\item De modo análogo aos casos anteriores e sabendo que $\color{red} \dfrac{dx^*}{dx} = 0$, temos
		{\tiny
		\begin{align*}
			\dfrac{\partial \Herm{\vtX} \vtA}{\partial \vtX} &= \begin{bmatrix}
					\dfrac{\partial}{\partial x_1} \left( \sum_{i = 1}^N x^*_ia_i \right) \\ \dfrac{\partial}{\partial x_2} \left( \sum_{i = 1}^N x^*_ia_i \right) \\ \vdots \\
					\dfrac{\partial}{\partial x_N} \left( \sum_{i = 1}^N x^*_ia_i \right) 
				\end{bmatrix} 
				= \begin{bmatrix}
					0 \\ 0 \\ \vdots \\ 0
				\end{bmatrix}
		\end{align*}}
		\[
			\boxed{\dfrac{\partial \Herm{\vtX} \vtA}{\partial \vtX} = \vtZero}
		\]
	\end{itemize}
\end{frame}

\subsubsection{Derivada de $\Transp{\vtX} \vtA$}
\begin{frame}
	\frametitle{\normalsize Derivada de $\Transp{\vtX} \vtA$}
	\begin{itemize}
		\item Considerando os mesmo vetores $\vtA, \vtX \in \mathbb{C}^{N \times 1}$, temos que:
		{\tiny
		\begin{align*}
			\dfrac{\partial \Transp{\vtX} \vtA}{\partial \vtX} &= \frac{\partial}{\partial \vtX} \left(
			\begin{bmatrix}
				x_1 & x_2 & \dots & x_N
			\end{bmatrix} \begin{bmatrix}
				a_{1} \\ a_{2} \\ \vdots \\ a_{N}
			\end{bmatrix} \right) 
			= \frac{\partial}{\partial \vtX} \left( \sum_{i = 1}^N x_ia_i \right) \\
			&= \begin{bmatrix}
				\dfrac{\partial}{\partial x_1} \left( \sum_{i = 1}^N x_ia_i \right) \\ \dfrac{\partial}{\partial x_2} \left( \sum_{i = 1}^N x_ia_i \right) \\ \vdots \\ \dfrac{\partial}{\partial x_N} \left( \sum_{i = 1}^N x_ia_i \right) 
			\end{bmatrix} 
			= \begin{bmatrix}
				a_1 \\ a_2 \\ \vdots \\ a_N
			\end{bmatrix}
		\end{align*}}
		\[
			\boxed{\frac{\partial \Transp{\vtX} \vtA}{\partial \vtX} = \vtA}
		\]
	\end{itemize}
\end{frame}

\subsubsection{Derivada de $\mtA \vtX$}
\begin{frame}
	\frametitle{\normalsize Derivada de $\mtA \vtX$}
	\begin{itemize}
		\item Considerando agora uma matriz $\mtA \in \mathbb{C}^{M \times N}$ e um vetor$\vtX \in \mathbb{C}^{N \times 1}$, temos que:
		{\tiny
		\begin{align*}
			\dfrac{\partial \mtA \vtX}{\partial \vtX} &= \dfrac{\partial}{\partial \vtX} \left(
				\begin{bmatrix}
					a_{11} & a_{12} & \dots & a_{1N} \\
					a_{21} & a_{22} & \dots & a_{2N} \\
					\vdots & \vdots & \ddots & \vdots \\
					a_{M1} & a_{M2} & \dots & a_{MN} \\
				\end{bmatrix} \begin{bmatrix}
					x_{1} \\ x_{2} \\ \vdots \\ x_{N}
				\end{bmatrix} \right) 
			%
			= \frac{\partial}{\partial \vtX} \left(\Transp{\begin{bmatrix} 
				\displaystyle \sum_{j = 1}^N a_{1j}x_j & \displaystyle \sum_{j = 1}^N a_{2j}x_j & \dots & \displaystyle \sum_{j = 1}^N a_{Mj}x_j
			\end{bmatrix}} \right) \\
			%
			&\hspace*{-10ex} = \begin{bmatrix}
				\displaystyle \dfrac{\partial}{\partial x_1} \left( \sum_{j = 1}^N a_{1j}x_j \right) & 
				\displaystyle \dfrac{\partial}{\partial x_1} \left( \sum_{j = 1}^N a_{21j}x_j \right) & 
				\dots & 
				\displaystyle \dfrac{\partial}{\partial x_1} \left( \sum_{j = 1}^N a_{Mj}x_j \right) \\
				\displaystyle \dfrac{\partial}{\partial x_2} \left( \sum_{j = 1}^N a_{1j}x_j \right) & 
				\displaystyle \dfrac{\partial}{\partial x_2} \left( \sum_{j = 1}^N a_{21j}x_j \right) & 
				\dots & 
				\displaystyle \dfrac{\partial}{\partial x_2} \left( \sum_{j = 1}^N a_{Mj}x_j \right) \\
				\vdots & \vdots & \ddots & \vdots \\
				\displaystyle \dfrac{\partial}{\partial x_N} \left( \sum_{j = 1}^N a_{1j}x_j \right) & 
				\displaystyle \dfrac{\partial}{\partial x_N} \left( \sum_{j = 1}^N a_{21j}x_j \right) & 
				\dots & 
				\displaystyle \dfrac{\partial}{\partial x_N} \left( \sum_{j = 1}^N a_{Mj}x_j \right) \\
			\end{bmatrix} 
			%
			= \begin{bmatrix}
				a_{11} & a_{21} & \dots & a_{N1} \\
				a_{12} & a_{22} & \dots & a_{N2} \\
				\vdots & \vdots & \ddots & \vdots \\
				a_{1M} & a_{2M} & \dots & a_{NM} \\
			\end{bmatrix}			
		\end{align*}}
		\[
			\boxed{\frac{\partial \mtA \vtX}{\partial \vtX} = \Transp{\mtA}}
		\]
	\end{itemize}
\end{frame}

\subsubsection{Derivada de $\Transp{\vtX} \mtA \vtX$}
\begin{frame}
	\frametitle{\normalsize Derivada de $\Transp{\vtX} \mtA \vtX$}
	\begin{itemize}
		\item Considerando a mesma matriz $\mtA \in \mathbb{C}^{M \times N}$ e um vetor$\vtX \in \mathbb{C}^{N \times 1}$ do caso anterior, temos que:
		{\tiny
		\begin{align*}
			\frac{\partial \Transp{\vtX} \mtA \vtX}{\partial \vtX} &= \frac{\partial}{\partial \vtX} \left(
				\begin{bmatrix}
					x_{1} & x_{2} & \dots & x_{N}
				\end{bmatrix}
				\begin{bmatrix}
					a_{11} & a_{12} & \dots & a_{1N} \\
					a_{21} & a_{22} & \dots & a_{2N} \\
					\vdots & \vdots & \ddots & \vdots \\
					a_{N1} & a_{N2} & \dots & a_{NN} \\
				\end{bmatrix} \begin{bmatrix}
					x_{1} \\ x_{2} \\ \vdots \\ x_{N}
				\end{bmatrix} \right) \\
			%
			&= \frac{\partial}{\partial \vtX} \left(
			\begin{bmatrix}
				\displaystyle \sum_{i = 1}^{N} x_{i}a_{i1} & 
				\displaystyle \sum_{i = 1}^{N} x_{i}a_{i2} & 
				\dots & 
				\displaystyle \sum_{i = 1}^{N} x_{i}a_{iN}
			\end{bmatrix} \begin{bmatrix}
				x_{1} \\ x_{2} \\ \vdots \\ x_{N}
			\end{bmatrix} \right) \\
			%
			&= \dfrac{\partial}{\partial \vtX} \left(
				\sum_{i = 1}^{N}\sum_{j = 1}^{N} x_{i} a_{ij} x_{j}
			\right) 
		\end{align*}}
	\end{itemize}
\end{frame}

\begin{frame}
	\frametitle{\normalsize Derivada de $\Transp{\vtX} \mtA \vtX$}
	{\tiny
	\begin{align*}
		\frac{\partial \Transp{\vtX} \mtA \vtX}{\partial \vtX} &= \begin{bmatrix}
			\displaystyle \frac{\partial}{\partial x_1} \left( \sum_{i = 1}^{N}\sum_{j = 1}^{N} x_{i} a_{ij} x_{j} \right) \\ 
			\displaystyle  \frac{\partial}{\partial x_2} \left( \sum_{i = 1}^{N}\sum_{j = 1}^{N} x_{i} a_{ij} x_{j} \right) \\ 
			\vdots \\ 
			\displaystyle \frac{\partial}{\partial x_N} \left( \sum_{i = 1}^{N}\sum_{j = 1}^{N} x_{i} a_{ij} x_{j} \right) 
		\end{bmatrix} 
		%
		= \begin{bmatrix}
			\displaystyle 2x_1a_{11} + \sum_{\substack{j = 1 \\ j \neq 1}}^{N} a_{1j} x_{j} + \sum_{\substack{i = 1 \\ i \neq 1}}^{N} a_{i1} x_{i} \\
			\displaystyle 2x_2a_{22} + \sum_{\substack{j = 1 \\ j \neq 2}}^{N} a_{2j} x_{j} + \sum_{\substack{i = 1 \\ i \neq 2}}^{N} a_{i2} x_{i} \\
			\vdots \\
			\displaystyle 2x_Na_{NN} + \sum_{\substack{j = 1 \\ j \neq N}}^{N} a_{Nj} x_{j} + \sum_{\substack{i = 1 \\ i \neq N}}^{N} a_{iN} x_{i} 
		\end{bmatrix} \\
		%
		&= \begin{bmatrix}
			\displaystyle \sum_{j = 1}^{N} a_{1j} x_{j} + \sum_{i = 1}^{N} a_{i1} x_{i} \\
			\displaystyle \sum_{j = 1}^{N} a_{2j} x_{j} + \sum_{i = 1}^{N} a_{i2} x_{i} \\
			\vdots \\
			\displaystyle \sum_{j = 1}^{N} a_{Nj} x_{j} + \sum_{i = 1}^{N} a_{iN} x_{i} 
		\end{bmatrix} 
	\end{align*}}
\end{frame}

\begin{frame}
	\frametitle{\normalsize Derivada de $\Transp{\vtX} \mtA \vtX$}
	\begin{itemize}
		\item Que pode ser escrito matricialmente como:
		\[
			\frac{\partial \Transp{\vtX} \mtA \vtX}{\partial \vtX} = \Transp{\mtA} \vtX + \mtA \vtX
		\]
		ou seja,
		\[
			\boxed{\frac{\partial \Transp{\vtX} \mtA \vtX}{\partial \vtX} = (\Transp{\mtA} + \mtA) \vtX}
		\]
		\item No caso de \underline{$\mtA$ ser uma matriz simétrica}, então $\mtA = \Transp{\mtA}$, logo:
		\[
			\boxed{\frac{\partial \Transp{\vtX} \mtA \vtX}{\partial \vtX} = 2\mtA \vtX}
		\]
	\end{itemize}
\end{frame}

\subsubsection{Derivada do Traço de uma matriz $\mtX$}
\begin{frame}
	\frametitle{\normalsize Derivada do Traço de uma matriz $\mtX$}
	\begin{itemize}
		\item Seja uma matriz $\mtA \in \mathbb{C}^{N \times M}$ e uma matriz $\vtX \in \mathbb{C}^{M \times N}$, temos que a derivada do traço de $\mtA \mtX$ é dada por:
		{\tiny
		\begin{align*}
			\frac{\partial \Trace{\mtA \mtX}}{\partial \mtX} &= \frac{\partial}{\partial \mtX} \left( \Trace{\begin{bmatrix}
				a_{11} & a_{12} & \dots & a_{1M} \\
				a_{21} & a_{22} & \dots & a_{2M} \\
				\vdots & \vdots & \ddots & \vdots \\
				a_{N1} & a_{N2} & \dots & a_{NM} \\
			\end{bmatrix}
			\begin{bmatrix}
				x_{11} & x_{12} & \dots & x_{1N} \\
				x_{21} & x_{22} & \dots & x_{2N} \\
				\vdots & \vdots & \ddots & \vdots \\
				x_{M1} & x_{M2} & \dots & x_{MN} \\
			\end{bmatrix}} \right) \\
			%
			&= \frac{\partial}{\partial \mtX} \left( \sum_{i = 1}^{N} \sum_{j = 1}^{N} a_{ij}x_{ji} \right) \\
			%
			&= \begin{bmatrix}
				\displaystyle \frac{\partial}{\partial x_{11}} \left( \sum_{i = 1}^{N} \sum_{j = 1}^{N} a_{ij}x_{ji} \right) & 
				\displaystyle \frac{\partial}{\partial x_{12}} \left( \sum_{i = 1}^{N} \sum_{j = 1}^{N} a_{ij}x_{ji} \right) & 
				\dots & 
				\displaystyle \frac{\partial}{\partial x_{1N}} \left( \sum_{i = 1}^{N} \sum_{j = 1}^{N} a_{ij}x_{ji} \right) \\
				\displaystyle \frac{\partial}{\partial x_{21}} \left( \sum_{i = 1}^{N} \sum_{j = 1}^{N} a_{ij}x_{ji} \right) & 
				\displaystyle \frac{\partial}{\partial x_{22}} \left( \sum_{i = 1}^{N} \sum_{j = 1}^{N} a_{ij}x_{ji} \right) & 
				\dots & 
				\displaystyle \frac{\partial}{\partial x_{2N}} \left( \sum_{i = 1}^{N} \sum_{j = 1}^{N} a_{ij}x_{ji} \right) \\
				\vdots & \vdots & \ddots & \vdots \\
				\displaystyle \frac{\partial}{\partial x_{M1}} \left( \sum_{i = 1}^{N} \sum_{j = 1}^{N} a_{ij}x_{ji} \right) & 
				\displaystyle \frac{\partial}{\partial x_{M2}} \left( \sum_{i = 1}^{N} \sum_{j = 1}^{N} a_{ij}x_{ji} \right) & 
				\dots & 
				\displaystyle \frac{\partial}{\partial x_{MN}} \left( \sum_{i = 1}^{N} \sum_{j = 1}^{N} a_{ij}x_{ji} \right) \\
			\end{bmatrix} 
		\end{align*}}
	\end{itemize}
\end{frame}


\begin{frame}
	\frametitle{\normalsize Derivada do Traço de uma matriz $\mtX$}
	\begin{itemize}
		\item Resultando em
		{\tiny
		\begin{align*}
			\frac{\partial \Trace{\mtA \mtX}}{\partial \mtX} &= \begin{bmatrix}
				a_{11} & a_{21} & \dots & a_{N1} \\
				a_{12} & a_{22} & \dots & a_{N2} \\
				\vdots & \vdots & \ddots & \vdots \\
				a_{1M} & a_{2M} & \dots & a_{NM} \\
			\end{bmatrix}
		\end{align*}}
		
		\[
			\boxed{\frac{\partial \Trace{\mtA \mtX}}{\partial \mtX} = \Transp{\mtA}}
		\]
	\end{itemize}
\end{frame}

\subsubsection{Derivada do Determinante de uma matriz $\mtX$}
\begin{frame}
	\frametitle{\normalsize Derivada do Determinante de uma matriz $\mtX$}
	\begin{itemize}
		\item Seja uma matriz $\mtX \in \mathbb{C}^{N \times N}$ quadrada.
		\item Através da expansão de Laplace (expansão em cofatores), podemos reescrever o determinante de $\mtX$ como a soma dos cofatores de uma linha ou coluna qualquer, multiplicada pelo seu elemento gerador, ou seja,
		\[
			\Det{\mtX} = \sum_{i = 1}^{N} x_{ki} \Det{\mtC_{ki}} = \sum_{i = 1}^{N} x_{ik} \Det{\mtC_{ik}} \,\,\,\,\,\, \forall k \in (1, N),
		\]
		em que $\mtC_{ij}$ representa o cofator da matriz $\mtX$ gerado a partir do elemento $x_{ij}$.
		\item Vale salientar que o cofator $\Det{\mtC_{ij}}$ independe do valor de qualquer elemento da linha $i$ ou da coluna $j$ de $\mtX$.
	\end{itemize}
\end{frame}

\begin{frame}
	\frametitle{\normalsize Derivada do Determinante de uma matriz $\mtX$}
	\begin{itemize}
		\item Assim, temos que:
		{\tiny
		\begin{align*}
			\frac{\partial \Det{\mtX}}{\partial \mtX} &= \frac{\partial}{\partial \mtX} \left( \sum_{i = 1}^{N} x_{ki} \Det{\mtC_{ki}} \right) \,\,\,\,\,\, \forall k \in (1, N) \\
			%
			&\hspace*{-10ex} = \frac{\partial}{\partial \mtX} \left( \begin{bmatrix}
				\displaystyle \sum_{i = 1}^{N} x_{1i} \Det{\mtC_{1i}} & 
				\displaystyle \sum_{i = 1}^{N} x_{1i} \Det{\mtC_{1i}} & 
				\dots & 
				\displaystyle \sum_{i = 1}^{N} x_{1i} \Det{\mtC_{1i}} \\
				\displaystyle \sum_{i = 1}^{N} x_{2i} \Det{\mtC_{2i}} & 
				\displaystyle \sum_{i = 1}^{N} x_{2i} \Det{\mtC_{2i}} & 
				\dots & 
				\displaystyle \sum_{i = 1}^{N} x_{2i} \Det{\mtC_{2i}} \\
				\vdots & \vdots & \ddots & \vdots \\
				\displaystyle \sum_{i = 1}^{N} x_{Ni} \Det{\mtC_{Ni}} & 
				\displaystyle \sum_{i = 1}^{N} x_{Ni} \Det{\mtC_{Ni}} & 
				\dots & 
				\displaystyle \sum_{i = 1}^{N} x_{Ni} \Det{\mtC_{Ni}}
			\end{bmatrix} \right) \\
			%
			&\hspace*{-10ex} = \begin{bmatrix}
				\displaystyle \frac{\partial}{\partial x_{11}} \left( \sum_{i = 1}^{N} x_{1i} \Det{\mtC_{1i}} \right) & 
				\displaystyle \frac{\partial}{\partial x_{12}} \left( \sum_{i = 1}^{N} x_{1i} \Det{\mtC_{1i}} \right) & 
				\dots & 
				\displaystyle \frac{\partial}{\partial x_{13}} \left( \sum_{i = 1}^{N} x_{1i} \Det{\mtC_{1i}} \right) \\
				\displaystyle \frac{\partial}{\partial x_{21}} \left( \sum_{i = 1}^{N} x_{2i} \Det{\mtC_{2i}} \right) & 
				\displaystyle \frac{\partial}{\partial x_{22}} \left( \sum_{i = 1}^{N} x_{2i} \Det{\mtC_{2i}} \right) & 
				\dots & 
				\displaystyle \frac{\partial}{\partial x_{33}} \left( \sum_{i = 1}^{N} x_{2i} \Det{\mtC_{2i}} \right) \\
				\vdots & \vdots & \ddots & \vdots \\
				\displaystyle \frac{\partial}{\partial x_{N1}} \left( \sum_{i = 1}^{N} x_{Ni} \Det{\mtC_{Ni}} \right) & 
				\displaystyle \frac{\partial}{\partial x_{N2}} \left( \sum_{i = 1}^{N} x_{Ni} \Det{\mtC_{Ni}} \right) & 
				\dots & 
				\displaystyle \frac{\partial}{\partial x_{N3}} \left( \sum_{i = 1}^{N} x_{Ni} \Det{\mtC_{Ni}} \right) \\
			\end{bmatrix}
		\end{align*}}
	\end{itemize}
\end{frame}

\begin{frame}
	\frametitle{\normalsize Derivada do Determinante de uma matriz $\mtX$}
	\begin{itemize}
		\item Resultando em:
		{\tiny
		\begin{align*}
			\frac{\partial \Det{\mtX}}{\partial \mtX} &= \begin{bmatrix}
				\Det{\mtC_{11}} & \Det{\mtC_{12}} & \dots & \Det{\mtC_{1N}} \\
				\Det{\mtC_{21}} & \Det{\mtC_{22}} & \dots & \Det{\mtC_{2N}} \\
				\vdots & \vdots & \ddots & \vdots \\
				\Det{\mtC_{N1}} & \Det{\mtC_{N2}} & \dots & \Det{\mtC_{NN}} \\
			\end{bmatrix}
		\end{align*}}
		\item A matriz $N \times N$ composta pelo determinante dos cofatores de $\mtX$ é denominada matriz adjunta de $\mtX$, ou simplesmente $\Adj{\mtX}$.
		
		\[
			\boxed{\frac{\partial \Det{\mtX}}{\partial \mtX} = \Adj{\mtX}}
		\]
	\end{itemize}
\end{frame}

% \documentclass[xcolor={svgnames},11pt]{beamer}
% \documentclass[xcolor={svgnames},handout]{beamer}
% \usepackage{pgfpages}
% \pgfpagesuselayout{2 on 1}[a4paper,border shrink=5mm]

\usepackage[brazil]{babel}
\usepackage[none]{hyphenat}

% \usepackage[default]{sourcesanspro}
\usepackage{inconsolata}
\usepackage{pifont}

\usepackage[latin1]{inputenc}
\usepackage[T1]{fontenc}

\usepackage{amsmath}
\usepackage{amsfonts}
\usepackage{amssymb}
\usepackage{amstext}

\usepackage{tabularx}

\usepackage{multicol}

\usepackage{graphicx}                           % Allow graphic usage
\graphicspath{{.}{./Figs/}}

\usepackage[nolist]{acronym}

\usepackage[ddmmyy]{datetime}
\ddmmyyyydate

\usepackage{calc}

\hypersetup{
pdfauthor = {Tarcisio F. Maciel, Diego A. Sousa, Jos� Mairton B. da Silva Jr., Francisco Hugo C. Neto, e Yuri Victor L de Melo}, %
pdftitle = {Otimiza��o convexa},%
pdfsubject = {Curso de otimiza��o convexa},%
pdfkeywords = {Otimiza��o, programa��o matem�tica},%
}

\usepackage{tikz}
\usetikzlibrary{arrows}
\usetikzlibrary{positioning}
\usetikzlibrary{shapes.misc}
\usetikzlibrary{shapes.symbols}
\usetikzlibrary{shapes.geometric}

% \usepackage{pgfpages}
% \pgfpagesuselayout{2 on 1}[a4paper,border shrink=5mm]

%%%%%%%%%%%%%%%%%%%%%%%%%%%%%%%%%

\usepackage{algorithmic}
\renewcommand{\algorithmicrequire}{\textbf{Entrada:}}
\renewcommand{\algorithmicensure}{\textbf{Sa�da:}}
\renewcommand{\algorithmicend}{\textbf{fim}}
\renewcommand{\algorithmicif}{\textbf{se}}
\renewcommand{\algorithmicthen}{\textbf{ent�o}}
\renewcommand{\algorithmicelse}{\textbf{sen�o}}
\renewcommand{\algorithmicelsif}{\algorithmicelse\ \algorithmicif}
\renewcommand{\algorithmicendif}{\algorithmicend-\algorithmicif;}
\renewcommand{\algorithmicfor}{\textbf{para}}
\renewcommand{\algorithmicforall}{\textbf{para todo}}
\renewcommand{\algorithmicdo}{\textbf{fa�a}}
\renewcommand{\algorithmicendfor}{\algorithmicend-\algorithmicfor;}
\renewcommand{\algorithmicwhile}{\textbf{enquanto}}
\renewcommand{\algorithmicendwhile}{\algorithmicend-\algorithmicwhile;}
\renewcommand{\algorithmicrepeat}{\textbf{repita}}
\renewcommand{\algorithmicuntil}{\textbf{at�}}
\renewcommand{\algorithmicprint}{\textbf{imprima}}
\renewcommand{\algorithmicreturn}{\textbf{retorne}}
\renewcommand{\algorithmictrue}{\textbf{verdadeiro}}
\renewcommand{\algorithmicfalse}{\textbf{falso}}
\renewcommand{\algorithmiccomment}[1]{\{#1\}}
%%%%%%%%%%%%%%%%%%%%%%%%%%%%%%%%%

\usecolortheme{structure}
\useoutertheme{infolines}
\useinnertheme[shadow]{rounded}
\usefonttheme{structurebold}
\usefonttheme{professionalfonts}
\usefonttheme[onlymath]{serif}

% Printing
% \usepackage{pgfpages}\in\fdR^n
% \pgfpagesuselayout{2 on 1}[a4paper,border shrink=5mm]
% \setbeamercolor{structure}{bg=White,fg=Black}
% \setbeamercolor{alerted text}{fg=Black}

% Screen
\setbeamercolor{structure}{bg=White,fg=Sienna!80!Black}
\setbeamercolor{palette primary}{fg=Black,bg=structure.bg}
\setbeamercolor{palette secondary}{fg=Black,bg=structure.fg!30!White}
\setbeamercolor{palette tertiary}{fg=structure.bg,bg=structure.fg!90!White}
\setbeamercolor{title}{fg=structure.bg,bg=structure.fg}
\setbeamercolor{part page}{fg=structure.bg,bg=structure.fg}
\setbeamercolor{frametitle}{fg=structure.bg,bg=structure.fg}
\setbeamercolor{block title}{fg=structure.bg,bg=structure.fg}
\setbeamercolor{block body}{fg=Black,bg=structure.fg!05!White}
\setbeamercolor{alerted text}{fg=Red}

\setbeamerfont{part page}{size=\Large,series=\bfseries}
\setbeamerfont{title}{size=\Large,series=\bfseries}
\setbeamerfont{frametitle}{size=\large,series=\bfseries}
\setbeamerfont{block title}{size=\small,series=\bfseries}
\setbeamerfont{block body}{size=\footnotesize}
\setbeamerfont{section in head/foot}{series=\bfseries,size=\tiny}
\setbeamerfont{subsection in head/foot}{series=\bfseries,size=\tiny}
\setbeamerfont{institute in head/foot}{series=\bfseries,size=\tiny}
\setbeamerfont{author in head/foot}{series=\bfseries,size=\tiny}
\setbeamerfont{date in head/foot}{series=\bfseries,size=\tiny}
\setbeamerfont{footnote}{size=\tiny}
\setbeamerfont{bibliography entry author}{size=\scriptsize,series=\bfseries}
\setbeamerfont{bibliography entry title}{size=\scriptsize,series=\bfseries}
\setbeamerfont{bibliography entry journal}{size=\scriptsize,series=\bfseries}
\setbeamerfont{bibliography entry note}{size=\scriptsize,series=\bfseries}

\setbeamertemplate{bibliography item}[text]

\setbeamertemplate{theorems}[numbered]
\newtheorem{teorema}[theorem]{Teorema}
\newtheorem{propriedade}[theorem]{Propriedade}
\newtheorem{corolario}[theorem]{Corol�rio}
\newtheorem{atividade}[theorem]{Atividade}


\setlength{\leftmargini}{1em}
\setlength{\leftmarginii}{1em}
\setlength{\leftmarginiii}{1em}

\setlength{\abovedisplayskip}{-\baselineskip}

\setlength{\arraycolsep}{3pt}

\newlength{\AuxWidth}

\title{Otimiza��o Convexa}
\author[T. Maciel et al]{Tarcisio F. Maciel, Diego A. Sousa, Jos� Mairton B. da Silva Jr., Francisco Hugo C. Neto, e Yuri Victor L de Melo}
\institute[PPGETI-CT-UFC]{\normalsize Universidade Federal do Cear� \\ Centro de Tecnologia \\ Programa de P�s-Gradua��o em Engenharia de Teleinform�tica}
\date{\today}

% Math stuff
% Alphabets (new fonts)
\DeclareMathAlphabet{\mathppl}{T1}{ppl}{m}{it}
\DeclareMathAlphabet{\mathphv}{T1}{phv}{m}{it}
\DeclareMathAlphabet{\mathpzc}{T1}{pzc}{m}{it}
% Operators
\DeclareMathOperator{\abs}{abs}
\DeclareMathOperator{\blockdiag}{blockdiag}
\DeclareMathOperator{\card}{card}
\DeclareMathOperator{\conv}{conv}
\DeclareMathOperator{\diag}{diag}
\DeclareMathOperator{\rank}{rank}
\DeclareMathOperator{\re}{Re}
\DeclareMathOperator{\subto}{\text{s.t.: }}
\DeclareMathOperator{\SubTo}{\text{subject to: }}
\DeclareMathOperator{\Sujeito}{\text{sujeito a}}
\DeclareMathOperator{\Vect}{vec}
\DeclareMathOperator{\tr}{tr}
\DeclareMathOperator{\adj}{adj}
\DeclareMathOperator{\nullity}{nullity}
\DeclareMathOperator{\sen}{sen}
\DeclareMathOperator{\dom}{dom}
\DeclareMathOperator{\epi}{epi}
\DeclareMathOperator{\hypo}{hypo}
\DeclareMathOperator{\minimize}{\text{minimize}}
\DeclareMathOperator{\maximize}{\text{maximize}}
\DeclareMathOperator{\setspan}{span}
% Paired delimiters
\DeclarePairedDelimiterX{\Abs}[1]{\lvert}{\rvert}{#1}
\DeclarePairedDelimiterX{\Norm}[1]{\lVert}{\rVert}{#1}
\DeclarePairedDelimiterX{\Floor}[1]{\lfloor}{\rfloor}{#1}
\DeclarePairedDelimiterX{\Ceil}[1]{\lceil}{\rceil}{#1}
\DeclarePairedDelimiterX{\InnerProd}[2]{\langle}{\rangle}{{#1},{#2}}
\DeclarePairedDelimiterX{\Round}[1]{\lceil}{\rfloor}{#1}
% Commands
\newcommand{\ArgMaxMin}[2]{\underset{#1}{\arg\max\!.\!\min\!.\!}\left\{#2\right\}}
\newcommand{\ArgMax}[2]{\underset{#1}{\arg\max\!.\!}\left\{#2\right\}}
\newcommand{\ArgMinMax}[2]{\underset{#1}{\arg\min\!.\!\max\!.\!}\left\{#2\right\}}
\newcommand{\ArgMin}[2]{\underset{#1}{\arg\min\!.\!}\left\{#2\right\}}
\newcommand{\Adj}[1]{\adj\left( #1 \right)}
\newcommand{\Avg}[1]{\overline{#1}}
\newcommand{\Card}[1]{\card\left\{#1\right\}}
\newcommand{\Conv}[1]{\conv\left\{#1\right\}}
\newcommand{\Conj}[1]{{#1}^*}
\newcommand{\Cos}[1]{\cos\left( #1 \right)}
\newcommand{\Deg}[1]{\deg\left(#1\right)}
\newcommand{\Det}[1]{\det\left(#1\right)}
\newcommand{\Diag}[1]{\mathcal{D}\left\{ #1 \right\}}
\newcommand{\Dim}[1]{\dim\left(#1 \right)}
\newcommand{\Elem}[2]{\left[{#1}\right]_{#2}}
\newcommand{\Exp}[1]{\text{e}^{#1}}
\newcommand{\Field}[1]{\mathbb{\uppercase{#1}}}
\newcommand{\First}{1^{\text{st}}}
\newcommand{\Herm}[1]{{#1}^{\mathrm{H}}}
\newcommand{\HInv}[1]{{#1}^{-\mathrm{H}}}
\newcommand{\Inv}[1]{{#1}^{-1}}
\newcommand{\TInv}[1]{{#1}^{-\mathrm{T}}}
\newcommand{\Lagrange}{\calL}
\newcommand{\LogTen}{\log_{10}}
\newcommand{\LogTwo}{\log_{2}}
\newcommand{\Max}[1]{\max\left\{ #1 \right\}}
\newcommand{\Maximize}[2]{\underset{#1}{\maximize}\left\{#2\right\}}
\newcommand{\Mean}[1]{\mathcal{E}\left\{ #1 \right\}}
\newcommand{\Min}[1]{\min\left\{ #1 \right\}}
\newcommand{\Minimize}[2]{\underset{#1}{\minimize}\;#2}
\newcommand{\Mod}[1]{\left\vert #1 \right\vert}
\newcommand{\MOD}[1]{\vert #1 \vert}
\newcommand{\Mt}[1]{\boldsymbol{#1}}
% \newcommand{\Norm}[1]{\left\Vert #1 \right\Vert}
\newcommand{\NormF}[1]{\Norm{#1}_{\mathrm{F}}}
\newcommand{\NormInf}[1]{\Norm{#1}_{\infty}}
\newcommand{\NormOne}[1]{\Norm{#1}_{1}}
\newcommand{\NormP}[1]{\Norm{#1}_{p}}
\newcommand{\NormTwo}[1]{\Norm{#1}_{2}}
\newcommand{\NORM}[1]{\Vert #1 \Vert}
\newcommand{\NORMF}[1]{\NORM{#1}_{\mathrm{F}}}
\newcommand{\NORMINF}[1]{\NORM{#1}_{\infty}}
\newcommand{\NORMONE}[1]{\NORM{#1}_{1}}
\newcommand{\NORMP}[1]{\NORM{#1}_{\text{p}}}
\newcommand{\NORMTWO}[1]{\NORM{#1}_{2}}
\newcommand{\Null}[1]{\calN\left(#1\right)}
\newcommand{\Nullity}[1]{\nullity\left(#1\right)}
\newcommand{\Order}[1]{\mathcal{O}\!\left(#1\right)}
\newcommand{\Ord}[1]{{#1}^{\text{th}}}
\newcommand{\PInv}[1]{{#1}^{\dagger}}
\newcommand{\Range}[1]{\calR\left(#1\right)}
\newcommand{\Rank}[1]{\rank\left(#1\right)}
\newcommand{\Real}[1]{\re\left\{#1\right\}}
\newcommand{\Set}[1]{\mathcal{\uppercase{#1}}}
\newcommand{\Second}{2^{\text{nd}}}
\newcommand{\Sin}[1]{\sin\left( #1 \right)}
\newcommand{\Span}[1]{\setspan\left\{ #1 \right\}}
\newcommand{\Third}{3^{\text{rd}}}
\newcommand{\Trace}[1]{\tr\left(#1\right)}
\newcommand{\Transp}[1]{#1^{\mathrm{T}}}
\newcommand{\Vector}[1]{\Vect\left\{#1\right\}}
\newcommand{\Vt}[1]{\boldsymbol{\lowercase{#1}}}
% Matrices
\newcommand{\mtA}{\Mt{A}}
\newcommand{\mtB}{\Mt{B}}
\newcommand{\mtC}{\Mt{C}}
\newcommand{\mtD}{\Mt{D}}
\newcommand{\mtE}{\Mt{E}}
\newcommand{\mtF}{\Mt{F}}
\newcommand{\mtG}{\Mt{G}}
\newcommand{\mtH}{\Mt{H}}
\newcommand{\mtI}{\Mt{I}}
\newcommand{\mtJ}{\Mt{J}}
\newcommand{\mtK}{\Mt{K}}
\newcommand{\mtL}{\Mt{L}}
\newcommand{\mtM}{\Mt{M}}
\newcommand{\mtN}{\Mt{N}}
\newcommand{\mtO}{\Mt{O}}
\newcommand{\mtP}{\Mt{P}}
\newcommand{\mtQ}{\Mt{Q}}
\newcommand{\mtR}{\Mt{R}}
\newcommand{\mtS}{\Mt{S}}
\newcommand{\mtT}{\Mt{T}}
\newcommand{\mtU}{\Mt{U}}
\newcommand{\mtV}{\Mt{V}}
\newcommand{\mtW}{\Mt{W}}
\newcommand{\mtX}{\Mt{X}}
\newcommand{\mtY}{\Mt{Y}}
\newcommand{\mtZ}{\Mt{Z}}
% Transposed matrices
\newcommand{\mtAt}{\Transp{\mtA}}
\newcommand{\mtBt}{\Transp{\mtB}}
\newcommand{\mtCt}{\Transp{\mtC}}
\newcommand{\mtDt}{\Transp{\mtD}}
\newcommand{\mtEt}{\Transp{\mtE}}
\newcommand{\mtFt}{\Transp{\mtF}}
\newcommand{\mtGt}{\Transp{\mtG}}
\newcommand{\mtHt}{\Transp{\mtH}}
\newcommand{\mtIt}{\Transp{\mtI}}
\newcommand{\mtJt}{\Transp{\mtJ}}
\newcommand{\mtKt}{\Transp{\mtK}}
\newcommand{\mtLt}{\Transp{\mtL}}
\newcommand{\mtMt}{\Transp{\mtM}}
\newcommand{\mtNt}{\Transp{\mtN}}
\newcommand{\mtOt}{\Transp{\mtP}}
\newcommand{\mtPt}{\Transp{\mtP}}
\newcommand{\mtQt}{\Transp{\mtQ}}
\newcommand{\mtRt}{\Transp{\mtR}}
\newcommand{\mtSt}{\Transp{\mtS}}
\newcommand{\mtTt}{\Transp{\mtT}}
\newcommand{\mtUt}{\Transp{\mtU}}
\newcommand{\mtVt}{\Transp{\mtV}}
\newcommand{\mtWt}{\Transp{\mtW}}
\newcommand{\mtXt}{\Transp{\mtX}}
\newcommand{\mtYt}{\Transp{\mtY}}
\newcommand{\mtZt}{\Transp{\mtZ}}
% Hermitian matrices
\newcommand{\mtAh}{\Herm{\mtA}}
\newcommand{\mtBh}{\Herm{\mtB}}
\newcommand{\mtCh}{\Herm{\mtC}}
\newcommand{\mtDh}{\Herm{\mtD}}
\newcommand{\mtEh}{\Herm{\mtE}}
\newcommand{\mtFh}{\Herm{\mtF}}
\newcommand{\mtGh}{\Herm{\mtG}}
\newcommand{\mtHh}{\Herm{\mtH}}
\newcommand{\mtIh}{\Herm{\mtI}}
\newcommand{\mtJh}{\Herm{\mtJ}}
\newcommand{\mtKh}{\Herm{\mtK}}
\newcommand{\mtLh}{\Herm{\mtL}}
\newcommand{\mtMh}{\Herm{\mtM}}
\newcommand{\mtNh}{\Herm{\mtN}}
\newcommand{\mtOh}{\Herm{\mtP}}
\newcommand{\mtPh}{\Herm{\mtP}}
\newcommand{\mtQh}{\Herm{\mtQ}}
\newcommand{\mtRh}{\Herm{\mtR}}
\newcommand{\mtSh}{\Herm{\mtS}}
\newcommand{\mtTh}{\Herm{\mtT}}
\newcommand{\mtUh}{\Herm{\mtU}}
\newcommand{\mtVh}{\Herm{\mtV}}
\newcommand{\mtWh}{\Herm{\mtW}}
\newcommand{\mtXh}{\Herm{\mtX}}
\newcommand{\mtYh}{\Herm{\mtY}}
\newcommand{\mtZh}{\Herm{\mtZ}}
% Inverse matrices
\newcommand{\mtAi}{\Inv{\mtA}}
\newcommand{\mtBi}{\Inv{\mtB}}
\newcommand{\mtCi}{\Inv{\mtC}}
\newcommand{\mtDi}{\Inv{\mtD}}
\newcommand{\mtEi}{\Inv{\mtE}}
\newcommand{\mtFi}{\Inv{\mtF}}
\newcommand{\mtGi}{\Inv{\mtG}}
\newcommand{\mtHi}{\Inv{\mtH}}
\newcommand{\mtIi}{\Inv{\mtI}}
\newcommand{\mtJi}{\Inv{\mtJ}}
\newcommand{\mtKi}{\Inv{\mtK}}
\newcommand{\mtLi}{\Inv{\mtL}}
\newcommand{\mtMi}{\Inv{\mtM}}
\newcommand{\mtNi}{\Inv{\mtN}}
\newcommand{\mtOi}{\Inv{\mtP}}
\newcommand{\mtPi}{\Inv{\mtP}}
\newcommand{\mtQi}{\Inv{\mtQ}}
\newcommand{\mtRi}{\Inv{\mtR}}
\newcommand{\mtSi}{\Inv{\mtS}}
\newcommand{\mtTi}{\Inv{\mtT}}
\newcommand{\mtUi}{\Inv{\mtU}}
\newcommand{\mtVi}{\Inv{\mtV}}
\newcommand{\mtWi}{\Inv{\mtW}}
% \newcommand{\mtXi}{\Inv{\mtX}} % Conflicts with the greek letter matrix \mtXi
\newcommand{\mtYi}{\Inv{\mtY}}
\newcommand{\mtZi}{\Inv{\mtZ}}
% Bar matrices
\newcommand{\mtAb}{\bar{\mtA}}
\newcommand{\mtBb}{\bar{\mtB}}
\newcommand{\mtCb}{\bar{\mtC}}
\newcommand{\mtDb}{\bar{\mtD}}
\newcommand{\mtEb}{\bar{\mtE}}
\newcommand{\mtFb}{\bar{\mtF}}
\newcommand{\mtGb}{\bar{\mtG}}
\newcommand{\mtHb}{\bar{\mtH}}
\newcommand{\mtIb}{\bar{\mtI}}
\newcommand{\mtJb}{\bar{\mtJ}}
\newcommand{\mtKb}{\bar{\mtK}}
\newcommand{\mtLb}{\bar{\mtL}}
\newcommand{\mtMb}{\bar{\mtM}}
\newcommand{\mtNb}{\bar{\mtN}}
\newcommand{\mtOb}{\bar{\mtP}}
\newcommand{\mtPb}{\bar{\mtP}}
\newcommand{\mtQb}{\bar{\mtQ}}
\newcommand{\mtRb}{\bar{\mtR}}
\newcommand{\mtSb}{\bar{\mtS}}
\newcommand{\mtTb}{\bar{\mtT}}
\newcommand{\mtUb}{\bar{\mtU}}
\newcommand{\mtVb}{\bar{\mtV}}
\newcommand{\mtWb}{\bar{\mtW}}
\newcommand{\mtXb}{\bar{\mtX}}
\newcommand{\mtYb}{\bar{\mtY}}
\newcommand{\mtZb}{\bar{\mtZ}}
% Underlined matrices
\newcommand{\mtAu}{\underline{\mtA}}
\newcommand{\mtBu}{\underline{\mtB}}
\newcommand{\mtCu}{\underline{\mtC}}
\newcommand{\mtDu}{\underline{\mtD}}
\newcommand{\mtEu}{\underline{\mtE}}
\newcommand{\mtFu}{\underline{\mtF}}
\newcommand{\mtGu}{\underline{\mtG}}
\newcommand{\mtHu}{\underline{\mtH}}
\newcommand{\mtIu}{\underline{\mtI}}
\newcommand{\mtJu}{\underline{\mtJ}}
\newcommand{\mtKu}{\underline{\mtK}}
\newcommand{\mtLu}{\underline{\mtL}}
\newcommand{\mtMu}{\underline{\mtM}}
\newcommand{\mtNu}{\underline{\mtN}}
\newcommand{\mtOu}{\underline{\mtP}}
\newcommand{\mtPu}{\underline{\mtP}}
\newcommand{\mtQu}{\underline{\mtQ}}
\newcommand{\mtRu}{\underline{\mtR}}
\newcommand{\mtSu}{\underline{\mtS}}
\newcommand{\mtTu}{\underline{\mtT}}
\newcommand{\mtUu}{\underline{\mtU}}
\newcommand{\mtVu}{\underline{\mtV}}
\newcommand{\mtWu}{\underline{\mtW}}
\newcommand{\mtXu}{\underline{\mtX}}
\newcommand{\mtYu}{\underline{\mtY}}
\newcommand{\mtZu}{\underline{\mtZ}}
% Dotted matrices
\newcommand{\mtAd}{\dot{\mtA}}
\newcommand{\mtBd}{\dot{\mtB}}
\newcommand{\mtCd}{\dot{\mtC}}
\newcommand{\mtDd}{\dot{\mtD}}
\newcommand{\mtEd}{\dot{\mtE}}
\newcommand{\mtFd}{\dot{\mtF}}
\newcommand{\mtGd}{\dot{\mtG}}
\newcommand{\mtHd}{\dot{\mtH}}
\newcommand{\mtId}{\dot{\mtI}}
\newcommand{\mtJd}{\dot{\mtJ}}
\newcommand{\mtKd}{\dot{\mtK}}
\newcommand{\mtLd}{\dot{\mtL}}
\newcommand{\mtMd}{\dot{\mtM}}
\newcommand{\mtNd}{\dot{\mtN}}
\newcommand{\mtOd}{\dot{\mtP}}
\newcommand{\mtPd}{\dot{\mtP}}
\newcommand{\mtQd}{\dot{\mtQ}}
\newcommand{\mtRd}{\dot{\mtR}}
\newcommand{\mtSd}{\dot{\mtS}}
\newcommand{\mtTd}{\dot{\mtT}}
\newcommand{\mtUd}{\dot{\mtU}}
\newcommand{\mtVd}{\dot{\mtV}}
\newcommand{\mtWd}{\dot{\mtW}}
\newcommand{\mtXd}{\dot{\mtX}}
\newcommand{\mtYd}{\dot{\mtY}}
\newcommand{\mtZd}{\dot{\mtZ}}
% Double dotted matrices
\newcommand{\mtAdd}{\ddot{\mtA}}
\newcommand{\mtBdd}{\ddot{\mtB}}
\newcommand{\mtCdd}{\ddot{\mtC}}
\newcommand{\mtDdd}{\ddot{\mtD}}
\newcommand{\mtEdd}{\ddot{\mtE}}
\newcommand{\mtFdd}{\ddot{\mtF}}
\newcommand{\mtGdd}{\ddot{\mtG}}
\newcommand{\mtHdd}{\ddot{\mtH}}
\newcommand{\mtIdd}{\ddot{\mtI}}
\newcommand{\mtJdd}{\ddot{\mtJ}}
\newcommand{\mtKdd}{\ddot{\mtK}}
\newcommand{\mtLdd}{\ddot{\mtL}}
\newcommand{\mtMdd}{\ddot{\mtM}}
\newcommand{\mtNdd}{\ddot{\mtN}}
\newcommand{\mtOdd}{\ddot{\mtP}}
\newcommand{\mtPdd}{\ddot{\mtP}}
\newcommand{\mtQdd}{\ddot{\mtQ}}
\newcommand{\mtRdd}{\ddot{\mtR}}
\newcommand{\mtSdd}{\ddot{\mtS}}
\newcommand{\mtTdd}{\ddot{\mtT}}
\newcommand{\mtUdd}{\ddot{\mtU}}
\newcommand{\mtVdd}{\ddot{\mtV}}
\newcommand{\mtWdd}{\ddot{\mtW}}
\newcommand{\mtXdd}{\ddot{\mtX}}
\newcommand{\mtYdd}{\ddot{\mtY}}
\newcommand{\mtZdd}{\ddot{\mtZ}}

% Special matrices
\newcommand{\mtLambda}{\Mt{\Lambda}}
\newcommand{\mtPhi}{\Mt{\Phi}}
\newcommand{\mtPsi}{\Mt{\Psi}}
\newcommand{\mtSigma}{\Mt{\Sigma}}
\newcommand{\mtGamma}{\Mt{\Gamma}}
\newcommand{\mtXi}{\Mt{\Xi}}
\newcommand{\mtZero}{\Mt{0}}
\newcommand{\mtOne}{\Mt{1}}
\newcommand{\mtUpsilon}{\Mt{\Upsilon}}
% Vectors
\newcommand{\vtA}{\Vt{A}}
\newcommand{\vtB}{\Vt{B}}
\newcommand{\vtC}{\Vt{C}}
\newcommand{\vtD}{\Vt{D}}
\newcommand{\vtE}{\Vt{E}}
\newcommand{\vtF}{\Vt{F}}
\newcommand{\vtG}{\Vt{G}}
\newcommand{\vtH}{\Vt{H}}
\newcommand{\vtI}{\Vt{I}}
\newcommand{\vtJ}{\Vt{J}}
\newcommand{\vtK}{\Vt{K}}
\newcommand{\vtL}{\Vt{L}}
\newcommand{\vtM}{\Vt{M}}
\newcommand{\vtN}{\Vt{N}}
\newcommand{\vtO}{\Vt{P}}
\newcommand{\vtP}{\Vt{P}}
\newcommand{\vtQ}{\Vt{Q}}
\newcommand{\vtR}{\Vt{R}}
\newcommand{\vtS}{\Vt{S}}
\newcommand{\vtT}{\Vt{T}}
\newcommand{\vtU}{\Vt{U}}
\newcommand{\vtV}{\Vt{V}}
\newcommand{\vtW}{\Vt{W}}
\newcommand{\vtX}{\Vt{X}}
\newcommand{\vtY}{\Vt{Y}}
\newcommand{\vtZ}{\Vt{Z}}
% Transposed vectors
\newcommand{\vtAt}{\Transp{\vtA}}
\newcommand{\vtBt}{\Transp{\vtB}}
\newcommand{\vtCt}{\Transp{\vtC}}
\newcommand{\vtDt}{\Transp{\vtD}}
\newcommand{\vtEt}{\Transp{\vtE}}
\newcommand{\vtFt}{\Transp{\vtF}}
\newcommand{\vtGt}{\Transp{\vtG}}
\newcommand{\vtHt}{\Transp{\vtH}}
\newcommand{\vtIt}{\Transp{\vtI}}
\newcommand{\vtJt}{\Transp{\vtJ}}
\newcommand{\vtKt}{\Transp{\vtK}}
\newcommand{\vtLt}{\Transp{\vtL}}
\newcommand{\vtMt}{\Transp{\vtM}}
\newcommand{\vtNt}{\Transp{\vtN}}
\newcommand{\vtOt}{\Transp{\vtP}}
\newcommand{\vtPt}{\Transp{\vtP}}
\newcommand{\vtQt}{\Transp{\vtQ}}
\newcommand{\vtRt}{\Transp{\vtR}}
\newcommand{\vtSt}{\Transp{\vtS}}
\newcommand{\vtTt}{\Transp{\vtT}}
\newcommand{\vtUt}{\Transp{\vtU}}
\newcommand{\vtVt}{\Transp{\vtV}}
\newcommand{\vtWt}{\Transp{\vtW}}
\newcommand{\vtXt}{\Transp{\vtX}}
\newcommand{\vtYt}{\Transp{\vtY}}
\newcommand{\vtZt}{\Transp{\vtZ}}
% Hermitian vectors
\newcommand{\vtAh}{\Herm{\vtA}}
\newcommand{\vtBh}{\Herm{\vtB}}
\newcommand{\vtCh}{\Herm{\vtC}}
\newcommand{\vtDh}{\Herm{\vtD}}
\newcommand{\vtEh}{\Herm{\vtE}}
\newcommand{\vtFh}{\Herm{\vtF}}
\newcommand{\vtGh}{\Herm{\vtG}}
\newcommand{\vtHh}{\Herm{\vtH}}
\newcommand{\vtIh}{\Herm{\vtI}}
\newcommand{\vtJh}{\Herm{\vtJ}}
\newcommand{\vtKh}{\Herm{\vtK}}
\newcommand{\vtLh}{\Herm{\vtL}}
\newcommand{\vtMh}{\Herm{\vtM}}
\newcommand{\vtNh}{\Herm{\vtN}}
\newcommand{\vtOh}{\Herm{\vtP}}
\newcommand{\vtPh}{\Herm{\vtP}}
\newcommand{\vtQh}{\Herm{\vtQ}}
\newcommand{\vtRh}{\Herm{\vtR}}
\newcommand{\vtSh}{\Herm{\vtS}}
\newcommand{\vtTh}{\Herm{\vtT}}
\newcommand{\vtUh}{\Herm{\vtU}}
\newcommand{\vtVh}{\Herm{\vtV}}
\newcommand{\vtWh}{\Herm{\vtW}}
\newcommand{\vtXh}{\Herm{\vtX}}
\newcommand{\vtYh}{\Herm{\vtY}}
\newcommand{\vtZh}{\Herm{\vtZ}}
% Bar vectors
\newcommand{\vtAb}{\bar{\vtA}}
\newcommand{\vtBb}{\bar{\vtB}}
\newcommand{\vtCb}{\bar{\vtC}}
\newcommand{\vtDb}{\bar{\vtD}}
\newcommand{\vtEb}{\bar{\vtE}}
\newcommand{\vtFb}{\bar{\vtF}}
\newcommand{\vtGb}{\bar{\vtG}}
\newcommand{\vtHb}{\bar{\vtH}}
\newcommand{\vtIb}{\bar{\vtI}}
\newcommand{\vtJb}{\bar{\vtJ}}
\newcommand{\vtKb}{\bar{\vtK}}
\newcommand{\vtLb}{\bar{\vtL}}
\newcommand{\vtMb}{\bar{\vtM}}
\newcommand{\vtNb}{\bar{\vtN}}
\newcommand{\vtOb}{\bar{\vtP}}
\newcommand{\vtPb}{\bar{\vtP}}
\newcommand{\vtQb}{\bar{\vtQ}}
\newcommand{\vtRb}{\bar{\vtR}}
\newcommand{\vtSb}{\bar{\vtS}}
\newcommand{\vtTb}{\bar{\vtT}}
\newcommand{\vtUb}{\bar{\vtU}}
\newcommand{\vtVb}{\bar{\vtV}}
\newcommand{\vtWb}{\bar{\vtW}}
\newcommand{\vtXb}{\bar{\vtX}}
\newcommand{\vtYb}{\bar{\vtY}}
\newcommand{\vtZb}{\bar{\vtZ}}
% Vectors underlined
\newcommand{\vtAu}{\underline{\Vt{A}}}
\newcommand{\vtBu}{\underline{\Vt{B}}}
\newcommand{\vtCu}{\underline{\Vt{C}}}
\newcommand{\vtDu}{\underline{\Vt{D}}}
\newcommand{\vtEu}{\underline{\Vt{E}}}
\newcommand{\vtFu}{\underline{\Vt{F}}}
\newcommand{\vtGu}{\underline{\Vt{G}}}
\newcommand{\vtHu}{\underline{\Vt{H}}}
\newcommand{\vtIu}{\underline{\Vt{I}}}
\newcommand{\vtJu}{\underline{\Vt{J}}}
\newcommand{\vtKu}{\underline{\Vt{K}}}
\newcommand{\vtLu}{\underline{\Vt{L}}}
\newcommand{\vtMu}{\underline{\Vt{M}}}
\newcommand{\vtNu}{\underline{\Vt{N}}}
\newcommand{\vtOu}{\underline{\Vt{P}}}
\newcommand{\vtPu}{\underline{\Vt{P}}}
\newcommand{\vtQu}{\underline{\Vt{Q}}}
\newcommand{\vtRu}{\underline{\Vt{R}}}
\newcommand{\vtSu}{\underline{\Vt{S}}}
\newcommand{\vtTu}{\underline{\Vt{T}}}
\newcommand{\vtUu}{\underline{\Vt{U}}}
\newcommand{\vtVu}{\underline{\Vt{V}}}
\newcommand{\vtWu}{\underline{\Vt{W}}}
\newcommand{\vtXu}{\underline{\Vt{X}}}
\newcommand{\vtYu}{\underline{\Vt{Y}}}
\newcommand{\vtZu}{\underline{\Vt{Z}}}
% Vectors dotted
\newcommand{\vtAd}{\dot{\Vt{A}}}
\newcommand{\vtBd}{\dot{\Vt{B}}}
\newcommand{\vtCd}{\dot{\Vt{C}}}
\newcommand{\vtDd}{\dot{\Vt{D}}}
\newcommand{\vtEd}{\dot{\Vt{E}}}
\newcommand{\vtFd}{\dot{\Vt{F}}}
\newcommand{\vtGd}{\dot{\Vt{G}}}
\newcommand{\vtHd}{\dot{\Vt{H}}}
\newcommand{\vtId}{\dot{\Vt{I}}}
\newcommand{\vtJd}{\dot{\Vt{J}}}
\newcommand{\vtKd}{\dot{\Vt{K}}}
\newcommand{\vtLd}{\dot{\Vt{L}}}
\newcommand{\vtMd}{\dot{\Vt{M}}}
\newcommand{\vtNd}{\dot{\Vt{N}}}
\newcommand{\vtOd}{\dot{\Vt{P}}}
\newcommand{\vtPd}{\dot{\Vt{P}}}
\newcommand{\vtQd}{\dot{\Vt{Q}}}
\newcommand{\vtRd}{\dot{\Vt{R}}}
\newcommand{\vtSd}{\dot{\Vt{S}}}
\newcommand{\vtTd}{\dot{\Vt{T}}}
\newcommand{\vtUd}{\dot{\Vt{U}}}
\newcommand{\vtVd}{\dot{\Vt{V}}}
\newcommand{\vtWd}{\dot{\Vt{W}}}
\newcommand{\vtXd}{\dot{\Vt{X}}}
\newcommand{\vtYd}{\dot{\Vt{Y}}}
\newcommand{\vtZd}{\dot{\Vt{Z}}}
% Vectors double dotted
\newcommand{\vtAdd}{\ddot{\Vt{A}}}
\newcommand{\vtBdd}{\ddot{\Vt{B}}}
\newcommand{\vtCdd}{\ddot{\Vt{C}}}
\newcommand{\vtDdd}{\ddot{\Vt{D}}}
\newcommand{\vtEdd}{\ddot{\Vt{E}}}
\newcommand{\vtFdd}{\ddot{\Vt{F}}}
\newcommand{\vtGdd}{\ddot{\Vt{G}}}
\newcommand{\vtHdd}{\ddot{\Vt{H}}}
\newcommand{\vtIdd}{\ddot{\Vt{I}}}
\newcommand{\vtJdd}{\ddot{\Vt{J}}}
\newcommand{\vtKdd}{\ddot{\Vt{K}}}
\newcommand{\vtLdd}{\ddot{\Vt{L}}}
\newcommand{\vtMdd}{\ddot{\Vt{M}}}
\newcommand{\vtNdd}{\ddot{\Vt{N}}}
\newcommand{\vtOdd}{\ddot{\Vt{P}}}
\newcommand{\vtPdd}{\ddot{\Vt{P}}}
\newcommand{\vtQdd}{\ddot{\Vt{Q}}}
\newcommand{\vtRdd}{\ddot{\Vt{R}}}
\newcommand{\vtSdd}{\ddot{\Vt{S}}}
\newcommand{\vtTdd}{\ddot{\Vt{T}}}
\newcommand{\vtUdd}{\ddot{\Vt{U}}}
\newcommand{\vtVdd}{\ddot{\Vt{V}}}
\newcommand{\vtWdd}{\ddot{\Vt{W}}}
\newcommand{\vtXdd}{\ddot{\Vt{X}}}
\newcommand{\vtYdd}{\ddot{\Vt{Y}}}
\newcommand{\vtZdd}{\ddot{\Vt{Z}}}
% Special vectors
\newcommand{\vtAlpha}{\Vt{\boldsymbol{\alpha}}}
\newcommand{\vtBeta}{\Vt{\boldsymbol{\beta}}}
\newcommand{\vtDelta}{\Vt{\boldsymbol{\Delta}}}
\newcommand{\vtEta}{\Vt{\boldsymbol{\eta}}}
\newcommand{\vtLambda}{\Vt{\boldsymbol{\lambda}}}
\newcommand{\vtMy}{\Vt{\boldsymbol{\mu}}}
\newcommand{\vtNy}{\Vt{\boldsymbol{\nu}}}
\newcommand{\vtOne}{\Vt{1}}
\newcommand{\vtPsi}{\Vt{\boldsymbol{\psi}}}
\newcommand{\vtSigma}{\Vt{\boldsymbol{\sigma}}}
\newcommand{\vtTau}{\Vt{\boldsymbol{\tau}}}
\newcommand{\vtZero}{\Vt{0}}
% Fields
\newcommand{\fdA}{\Field{A}}
\newcommand{\fdB}{\Field{B}}
\newcommand{\fdC}{\Field{C}}
\newcommand{\fdD}{\Field{D}}
\newcommand{\fdE}{\Field{E}}
\newcommand{\fdF}{\Field{F}}
\newcommand{\fdG}{\Field{G}}
\newcommand{\fdH}{\Field{H}}
\newcommand{\fdI}{\Field{I}}
\newcommand{\fdJ}{\Field{J}}
\newcommand{\fdK}{\Field{K}}
\newcommand{\fdL}{\Field{L}}
\newcommand{\fdM}{\Field{M}}
\newcommand{\fdN}{\Field{N}}
\newcommand{\fdO}{\Field{O}}
\newcommand{\fdP}{\Field{P}}
\newcommand{\fdQ}{\Field{Q}}
\newcommand{\fdR}{\Field{R}}
\newcommand{\fdS}{\Field{S}}
\newcommand{\fdT}{\Field{T}}
\newcommand{\fdU}{\Field{U}}
\newcommand{\fdV}{\Field{V}}
\newcommand{\fdW}{\Field{W}}
\newcommand{\fdX}{\Field{X}}
\newcommand{\fdY}{\Field{Y}}
\newcommand{\fdZ}{\Field{Z}}
% Sets
\newcommand{\stA}{\Set{A}}
\newcommand{\stB}{\Set{B}}
\newcommand{\stC}{\Set{C}}
\newcommand{\stD}{\Set{D}}
\newcommand{\stE}{\Set{E}}
\newcommand{\stF}{\Set{F}}
\newcommand{\stG}{\Set{G}}
\newcommand{\stH}{\Set{H}}
\newcommand{\stI}{\Set{I}}
\newcommand{\stJ}{\Set{J}}
\newcommand{\stK}{\Set{K}}
\newcommand{\stL}{\Set{L}}
\newcommand{\stM}{\Set{M}}
\newcommand{\stN}{\Set{N}}
\newcommand{\stO}{\Set{O}}
\newcommand{\stP}{\Set{P}}
\newcommand{\stQ}{\Set{Q}}
\newcommand{\stR}{\Set{R}}
\newcommand{\stS}{\Set{S}}
\newcommand{\stT}{\Set{T}}
\newcommand{\stU}{\Set{U}}
\newcommand{\stV}{\Set{V}}
\newcommand{\stW}{\Set{W}}
\newcommand{\stX}{\Set{X}}
\newcommand{\stY}{\Set{Y}}
\newcommand{\stZ}{\Set{Z}}
% Alias for math fonts
%%%%%%%%%%%%%%%%%%%%%%%%%%%%%%%
% rm
%%%%%%%%%%%%%%%%%%%%%%%%%%%%%%%
\newcommand{\rmA}{\mathrm{A}}
\newcommand{\rmB}{\mathrm{B}}
\newcommand{\rmC}{\mathrm{C}}
\newcommand{\rmD}{\mathrm{D}}
\newcommand{\rmE}{\mathrm{E}}
\newcommand{\rmF}{\mathrm{F}}
\newcommand{\rmG}{\mathrm{G}}
\newcommand{\rmH}{\mathrm{H}}
\newcommand{\rmI}{\mathrm{I}}
\newcommand{\rmJ}{\mathrm{J}}
\newcommand{\rmK}{\mathrm{K}}
\newcommand{\rmL}{\mathrm{L}}
\newcommand{\rmM}{\mathrm{M}}
\newcommand{\rmN}{\mathrm{N}}
\newcommand{\rmO}{\mathrm{O}}
\newcommand{\rmP}{\mathrm{P}}
\newcommand{\rmQ}{\mathrm{Q}}
\newcommand{\rmR}{\mathrm{R}}
\newcommand{\rmS}{\mathrm{S}}
\newcommand{\rmT}{\mathrm{T}}
\newcommand{\rmU}{\mathrm{U}}
\newcommand{\rmV}{\mathrm{V}}
\newcommand{\rmW}{\mathrm{W}}
\newcommand{\rmX}{\mathrm{X}}
\newcommand{\rmY}{\mathrm{Y}}
\newcommand{\rmZ}{\mathrm{Z}}
%%%%%%%%%%%%%%%%%%%%%%%%%%%%%%%
\newcommand{\rma}{\mathrm{a}}
\newcommand{\rmb}{\mathrm{b}}
\newcommand{\rmc}{\mathrm{c}}
\newcommand{\rmd}{\mathrm{d}}
\newcommand{\rme}{\mathrm{e}}
\newcommand{\rmf}{\mathrm{f}}
\newcommand{\rmg}{\mathrm{g}}
\newcommand{\rmh}{\mathrm{h}}
\newcommand{\rmi}{\mathrm{i}}
\newcommand{\rmj}{\mathrm{j}}
\newcommand{\rmk}{\mathrm{k}}
\newcommand{\rml}{\mathrm{l}}
\newcommand{\rmm}{\mathrm{m}}
\newcommand{\rmn}{\mathrm{n}}
\newcommand{\rmo}{\mathrm{o}}
\newcommand{\rmp}{\mathrm{p}}
\newcommand{\rmq}{\mathrm{q}}
\newcommand{\rmr}{\mathrm{r}}
\newcommand{\rms}{\mathrm{s}}
\newcommand{\rmt}{\mathrm{t}}
\newcommand{\rmu}{\mathrm{u}}
\newcommand{\rmv}{\mathrm{v}}
\newcommand{\rmw}{\mathrm{w}}
\newcommand{\rmx}{\mathrm{x}}
\newcommand{\rmy}{\mathrm{y}}
\newcommand{\rmz}{\mathrm{z}}
%%%%%%%%%%%%%%%%%%%%%%%%%%%%%%%
%%%%%%%%%%%%%%%%%%%%%%%%%%%%%%%
% sf
%%%%%%%%%%%%%%%%%%%%%%%%%%%%%%%
\newcommand{\sfA}{\mathsf{A}}
\newcommand{\sfB}{\mathsf{B}}
\newcommand{\sfC}{\mathsf{C}}
\newcommand{\sfD}{\mathsf{D}}
\newcommand{\sfE}{\mathsf{E}}
\newcommand{\sfF}{\mathsf{F}}
\newcommand{\sfG}{\mathsf{G}}
\newcommand{\sfH}{\mathsf{H}}
\newcommand{\sfI}{\mathsf{I}}
\newcommand{\sfJ}{\mathsf{J}}
\newcommand{\sfK}{\mathsf{K}}
\newcommand{\sfL}{\mathsf{L}}
\newcommand{\sfM}{\mathsf{M}}
\newcommand{\sfN}{\mathsf{N}}
\newcommand{\sfO}{\mathsf{O}}
\newcommand{\sfP}{\mathsf{P}}
\newcommand{\sfQ}{\mathsf{Q}}
\newcommand{\sfR}{\mathsf{R}}
\newcommand{\sfS}{\mathsf{S}}
\newcommand{\sfT}{\mathsf{T}}
\newcommand{\sfU}{\mathsf{U}}
\newcommand{\sfV}{\mathsf{V}}
\newcommand{\sfW}{\mathsf{W}}
\newcommand{\sfX}{\mathsf{X}}
\newcommand{\sfY}{\mathsf{Y}}
\newcommand{\sfZ}{\mathsf{Z}}
%%%%%%%%%%%%%%%%%%%%%%%%%%%%%%%
%\newcommand{\sfa}{\mathsf{a}}
%\newcommand{\sfb}{\mathsf{b}}
%\newcommand{\sfc}{\mathsf{c}}
%\newcommand{\sfd}{\mathsf{d}}
%\newcommand{\sfe}{\mathsf{e}}
%\newcommand{\sff}{\mathsf{f}}
%\newcommand{\sfg}{\mathsf{g}}
%\newcommand{\sfh}{\mathsf{h}}
%\newcommand{\sfi}{\mathsf{i}}
%\newcommand{\sfj}{\mathsf{j}}
%\newcommand{\sfk}{\mathsf{k}}
%\newcommand{\sfl}{\mathsf{l}}
%\newcommand{\sfm}{\mathsf{m}}
%\newcommand{\sfn}{\mathsf{n}}
%\newcommand{\sfo}{\mathsf{o}}
%\newcommand{\sfp}{\mathsf{p}}
%\newcommand{\sfq}{\mathsf{q}}
%\newcommand{\sfr}{\mathsf{r}}
%\newcommand{\sfs}{\mathsf{s}}
%\newcommand{\sft}{\mathsf{t}}
%\newcommand{\sfu}{\mathsf{u}}
%\newcommand{\sfv}{\mathsf{v}}
%\newcommand{\sfw}{\mathsf{w}}
%\newcommand{\sfx}{\mathsf{x}}
%\newcommand{\sfy}{\mathsf{y}}
%\newcommand{\sfz}{\mathsf{z}}
%%%%%%%%%%%%%%%%%%%%%%%%%%%%%%%
%%%%%%%%%%%%%%%%%%%%%%%%%%%%%%%
% bf
%%%%%%%%%%%%%%%%%%%%%%%%%%%%%%%
\newcommand{\bfA}{\mathbf{A}}
\newcommand{\bfB}{\mathbf{B}}
\newcommand{\bfC}{\mathbf{C}}
\newcommand{\bfD}{\mathbf{D}}
\newcommand{\bfE}{\mathbf{E}}
\newcommand{\bfF}{\mathbf{F}}
\newcommand{\bfG}{\mathbf{G}}
\newcommand{\bfH}{\mathbf{H}}
\newcommand{\bfI}{\mathbf{I}}
\newcommand{\bfJ}{\mathbf{J}}
\newcommand{\bfK}{\mathbf{K}}
\newcommand{\bfL}{\mathbf{L}}
\newcommand{\bfM}{\mathbf{M}}
\newcommand{\bfN}{\mathbf{N}}
\newcommand{\bfO}{\mathbf{O}}
\newcommand{\bfP}{\mathbf{P}}
\newcommand{\bfQ}{\mathbf{Q}}
\newcommand{\bfR}{\mathbf{R}}
\newcommand{\bfS}{\mathbf{S}}
\newcommand{\bfT}{\mathbf{T}}
\newcommand{\bfU}{\mathbf{U}}
\newcommand{\bfV}{\mathbf{V}}
\newcommand{\bfW}{\mathbf{W}}
\newcommand{\bfX}{\mathbf{X}}
\newcommand{\bfY}{\mathbf{Y}}
\newcommand{\bfZ}{\mathbf{Z}}
%%%%%%%%%%%%%%%%%%%%%%%%%%%%%%%
\newcommand{\bfa}{\mathbf{a}}
\newcommand{\bfb}{\mathbf{b}}
\newcommand{\bfc}{\mathbf{c}}
\newcommand{\bfd}{\mathbf{d}}
\newcommand{\bfe}{\mathbf{e}}
\newcommand{\bff}{\mathbf{f}}
\newcommand{\bfg}{\mathbf{g}}
\newcommand{\bfh}{\mathbf{h}}
\newcommand{\bfi}{\mathbf{i}}
\newcommand{\bfj}{\mathbf{j}}
\newcommand{\bfk}{\mathbf{k}}
\newcommand{\bfl}{\mathbf{l}}
\newcommand{\bfm}{\mathbf{m}}
\newcommand{\bfn}{\mathbf{n}}
\newcommand{\bfo}{\mathbf{o}}
\newcommand{\bfp}{\mathbf{p}}
\newcommand{\bfq}{\mathbf{q}}
\newcommand{\bfr}{\mathbf{r}}
\newcommand{\bfs}{\mathbf{s}}
\newcommand{\bft}{\mathbf{t}}
\newcommand{\bfu}{\mathbf{u}}
\newcommand{\bfv}{\mathbf{v}}
\newcommand{\bfw}{\mathbf{w}}
\newcommand{\bfx}{\mathbf{x}}
\newcommand{\bfy}{\mathbf{y}}
\newcommand{\bfz}{\mathbf{z}}
%%%%%%%%%%%%%%%%%%%%%%%%%%%%%%%
%%%%%%%%%%%%%%%%%%%%%%%%%%%%%%%
% it
%%%%%%%%%%%%%%%%%%%%%%%%%%%%%%%
\newcommand{\itA}{\mathit{A}}
\newcommand{\itB}{\mathit{B}}
\newcommand{\itC}{\mathit{C}}
\newcommand{\itD}{\mathit{D}}
\newcommand{\itE}{\mathit{E}}
\newcommand{\itF}{\mathit{F}}
\newcommand{\itG}{\mathit{G}}
\newcommand{\itH}{\mathit{H}}
\newcommand{\itI}{\mathit{I}}
\newcommand{\itJ}{\mathit{J}}
\newcommand{\itK}{\mathit{K}}
\newcommand{\itL}{\mathit{L}}
\newcommand{\itM}{\mathit{M}}
\newcommand{\itN}{\mathit{N}}
\newcommand{\itO}{\mathit{O}}
\newcommand{\itP}{\mathit{P}}
\newcommand{\itQ}{\mathit{Q}}
\newcommand{\itR}{\mathit{R}}
\newcommand{\itS}{\mathit{S}}
\newcommand{\itT}{\mathit{T}}
\newcommand{\itU}{\mathit{U}}
\newcommand{\itV}{\mathit{V}}
\newcommand{\itW}{\mathit{W}}
\newcommand{\itX}{\mathit{X}}
\newcommand{\itY}{\mathit{Y}}
\newcommand{\itZ}{\mathit{Z}}
%%%%%%%%%%%%%%%%%%%%%%%%%%%%%%%
\newcommand{\ita}{\mathit{a}}
\newcommand{\itb}{\mathit{b}}
\newcommand{\itc}{\mathit{c}}
\newcommand{\itd}{\mathit{d}}
\newcommand{\ite}{\mathit{e}}
\newcommand{\itf}{\mathit{f}}
\newcommand{\itg}{\mathit{g}}
\newcommand{\ith}{\mathit{h}}
\newcommand{\iti}{\mathit{i}}
\newcommand{\itj}{\mathit{j}}
\newcommand{\itk}{\mathit{k}}
\newcommand{\itl}{\mathit{l}}
\newcommand{\itm}{\mathit{m}}
\newcommand{\itn}{\mathit{n}}
\newcommand{\ito}{\mathit{o}}
\newcommand{\itp}{\mathit{p}}
\newcommand{\itq}{\mathit{q}}
\newcommand{\itr}{\mathit{r}}
\newcommand{\its}{\mathit{s}}
\newcommand{\itt}{\mathit{t}}
\newcommand{\itu}{\mathit{u}}
\newcommand{\itv}{\mathit{v}}
\newcommand{\itw}{\mathit{w}}
\newcommand{\itx}{\mathit{x}}
\newcommand{\ity}{\mathit{y}}
\newcommand{\itz}{\mathit{z}}
%%%%%%%%%%%%%%%%%%%%%%%%%%%%%%%
%%%%%%%%%%%%%%%%%%%%%%%%%%%%%%%
% frak
%%%%%%%%%%%%%%%%%%%%%%%%%%%%%%%
\newcommand{\fkA}{\mathfrak{A}}
\newcommand{\fkB}{\mathfrak{B}}
\newcommand{\fkC}{\mathfrak{C}}
\newcommand{\fkD}{\mathfrak{D}}
\newcommand{\fkE}{\mathfrak{E}}
\newcommand{\fkF}{\mathfrak{F}}
\newcommand{\fkG}{\mathfrak{G}}
\newcommand{\fkH}{\mathfrak{H}}
\newcommand{\fkI}{\mathfrak{I}}
\newcommand{\fkJ}{\mathfrak{J}}
\newcommand{\fkK}{\mathfrak{K}}
\newcommand{\fkL}{\mathfrak{L}}
\newcommand{\fkM}{\mathfrak{M}}
\newcommand{\fkN}{\mathfrak{N}}
\newcommand{\fkO}{\mathfrak{O}}
\newcommand{\fkP}{\mathfrak{P}}
\newcommand{\fkQ}{\mathfrak{Q}}
\newcommand{\fkR}{\mathfrak{R}}
\newcommand{\fkS}{\mathfrak{S}}
\newcommand{\fkT}{\mathfrak{T}}
\newcommand{\fkU}{\mathfrak{U}}
\newcommand{\fkV}{\mathfrak{V}}
\newcommand{\fkW}{\mathfrak{W}}
\newcommand{\fkX}{\mathfrak{X}}
\newcommand{\fkY}{\mathfrak{Y}}
\newcommand{\fkZ}{\mathfrak{Z}}
%%%%%%%%%%%%%%%%%%%%%%%%%%%%%%%
\newcommand{\fka}{\mathfrak{a}}
\newcommand{\fkb}{\mathfrak{b}}
\newcommand{\fkc}{\mathfrak{c}}
\newcommand{\fkd}{\mathfrak{d}}
\newcommand{\fke}{\mathfrak{e}}
\newcommand{\fkf}{\mathfrak{f}}
\newcommand{\fkg}{\mathfrak{g}}
\newcommand{\fkh}{\mathfrak{h}}
\newcommand{\fki}{\mathfrak{i}}
\newcommand{\fkj}{\mathfrak{j}}
\newcommand{\fkk}{\mathfrak{k}}
\newcommand{\fkl}{\mathfrak{l}}
\newcommand{\fkm}{\mathfrak{m}}
\newcommand{\fkn}{\mathfrak{n}}
\newcommand{\fko}{\mathfrak{o}}
\newcommand{\fkp}{\mathfrak{p}}
\newcommand{\fkq}{\mathfrak{q}}
\newcommand{\fkr}{\mathfrak{r}}
\newcommand{\fks}{\mathfrak{s}}
\newcommand{\fkt}{\mathfrak{t}}
\newcommand{\fku}{\mathfrak{u}}
\newcommand{\fkv}{\mathfrak{v}}
\newcommand{\fkw}{\mathfrak{w}}
\newcommand{\fkx}{\mathfrak{x}}
\newcommand{\fky}{\mathfrak{y}}
\newcommand{\fkz}{\mathfrak{z}}
%%%%%%%%%%%%%%%%%%%%%%%%%%%%%%%
% Eufrak matrices
\newcommand{\mtkA}{\boldsymbol{\fkA}}
\newcommand{\mtkB}{\boldsymbol{\fkB}}
\newcommand{\mtkC}{\boldsymbol{\fkC}}
\newcommand{\mtkD}{\boldsymbol{\fkD}}
\newcommand{\mtkE}{\boldsymbol{\fkE}}
\newcommand{\mtkF}{\boldsymbol{\fkF}}
\newcommand{\mtkG}{\boldsymbol{\fkG}}
\newcommand{\mtkH}{\boldsymbol{\fkH}}
\newcommand{\mtkI}{\boldsymbol{\fkI}}
\newcommand{\mtkJ}{\boldsymbol{\fkJ}}
\newcommand{\mtkK}{\boldsymbol{\fkK}}
\newcommand{\mtkL}{\boldsymbol{\fkL}}
\newcommand{\mtkM}{\boldsymbol{\fkM}}
\newcommand{\mtkN}{\boldsymbol{\fkN}}
\newcommand{\mtkO}{\boldsymbol{\fkO}}
\newcommand{\mtkP}{\boldsymbol{\fkP}}
\newcommand{\mtkQ}{\boldsymbol{\fkQ}}
\newcommand{\mtkR}{\boldsymbol{\fkR}}
\newcommand{\mtkS}{\boldsymbol{\fkS}}
\newcommand{\mtkT}{\boldsymbol{\fkT}}
\newcommand{\mtkU}{\boldsymbol{\fkU}}
\newcommand{\mtkV}{\boldsymbol{\fkV}}
\newcommand{\mtkW}{\boldsymbol{\fkW}}
\newcommand{\mtkX}{\boldsymbol{\fkX}}
\newcommand{\mtkY}{\boldsymbol{\fkY}}
\newcommand{\mtkZ}{\boldsymbol{\fkZ}}
%%%%%%%%%%%%%%%%%%%%%%%%%%%%%%%
%%%%%%%%%%%%%%%%%%%%%%%%%%%%%%%
% ppl
%%%%%%%%%%%%%%%%%%%%%%%%%%%%%%%
\newcommand{\pplA}{\mathppl{A}}
\newcommand{\pplB}{\mathppl{B}}
\newcommand{\pplC}{\mathppl{C}}
\newcommand{\pplD}{\mathppl{D}}
\newcommand{\pplE}{\mathppl{E}}
\newcommand{\pplF}{\mathppl{F}}
\newcommand{\pplG}{\mathppl{G}}
\newcommand{\pplH}{\mathppl{H}}
\newcommand{\pplI}{\mathppl{I}}
\newcommand{\pplJ}{\mathppl{J}}
\newcommand{\pplK}{\mathppl{K}}
\newcommand{\pplL}{\mathppl{L}}
\newcommand{\pplM}{\mathppl{M}}
\newcommand{\pplN}{\mathppl{N}}
\newcommand{\pplO}{\mathppl{O}}
\newcommand{\pplP}{\mathppl{P}}
\newcommand{\pplQ}{\mathppl{Q}}
\newcommand{\pplR}{\mathppl{R}}
\newcommand{\pplS}{\mathppl{S}}
\newcommand{\pplT}{\mathppl{T}}
\newcommand{\pplU}{\mathppl{U}}
\newcommand{\pplV}{\mathppl{V}}
\newcommand{\pplW}{\mathppl{W}}
\newcommand{\pplX}{\mathppl{X}}
\newcommand{\pplY}{\mathppl{Y}}
\newcommand{\pplZ}{\mathppl{Z}}
%%%%%%%%%%%%%%%%%%%%%%%%%%%%%%%
\newcommand{\ppla}{\mathppl{a}}
\newcommand{\pplb}{\mathppl{b}}
\newcommand{\pplc}{\mathppl{c}}
\newcommand{\ppld}{\mathppl{d}}
\newcommand{\pple}{\mathppl{e}}
\newcommand{\pplf}{\mathppl{f}}
\newcommand{\pplg}{\mathppl{g}}
\newcommand{\pplh}{\mathppl{h}}
\newcommand{\ppli}{\mathppl{i}}
\newcommand{\pplj}{\mathppl{j}}
\newcommand{\pplk}{\mathppl{k}}
\newcommand{\ppll}{\mathppl{l}}
\newcommand{\pplm}{\mathppl{m}}
\newcommand{\ppln}{\mathppl{n}}
\newcommand{\pplo}{\mathppl{o}}
\newcommand{\pplp}{\mathppl{p}}
\newcommand{\pplq}{\mathppl{q}}
\newcommand{\pplr}{\mathppl{r}}
\newcommand{\ppls}{\mathppl{s}}
\newcommand{\pplt}{\mathppl{t}}
\newcommand{\pplu}{\mathppl{u}}
\newcommand{\pplv}{\mathppl{v}}
\newcommand{\pplw}{\mathppl{w}}
\newcommand{\pplx}{\mathppl{x}}
\newcommand{\pply}{\mathppl{y}}
\newcommand{\pplz}{\mathppl{z}}
%%%%%%%%%%%%%%%%%%%%%%%%%%%%%%%
%%%%%%%%%%%%%%%%%%%%%%%%%%%%%%%
% phv
%%%%%%%%%%%%%%%%%%%%%%%%%%%%%%%
\newcommand{\phvA}{\mathphv{A}}
\newcommand{\phvB}{\mathphv{B}}
\newcommand{\phvC}{\mathphv{C}}
\newcommand{\phvD}{\mathphv{D}}
\newcommand{\phvE}{\mathphv{E}}
\newcommand{\phvF}{\mathphv{F}}
\newcommand{\phvG}{\mathphv{G}}
\newcommand{\phvH}{\mathphv{H}}
\newcommand{\phvI}{\mathphv{I}}
\newcommand{\phvJ}{\mathphv{J}}
\newcommand{\phvK}{\mathphv{K}}
\newcommand{\phvL}{\mathphv{L}}
\newcommand{\phvM}{\mathphv{M}}
\newcommand{\phvN}{\mathphv{N}}
\newcommand{\phvO}{\mathphv{O}}
\newcommand{\phvP}{\mathphv{P}}
\newcommand{\phvQ}{\mathphv{Q}}
\newcommand{\phvR}{\mathphv{R}}
\newcommand{\phvS}{\mathphv{S}}
\newcommand{\phvT}{\mathphv{T}}
\newcommand{\phvU}{\mathphv{U}}
\newcommand{\phvV}{\mathphv{V}}
\newcommand{\phvW}{\mathphv{W}}
\newcommand{\phvX}{\mathphv{X}}
\newcommand{\phvY}{\mathphv{Y}}
\newcommand{\phvZ}{\mathphv{Z}}
%%%%%%%%%%%%%%%%%%%%%%%%%%%%%%%
\newcommand{\phva}{\mathphv{a}}
\newcommand{\phvb}{\mathphv{b}}
\newcommand{\phvc}{\mathphv{c}}
\newcommand{\phvd}{\mathphv{d}}
\newcommand{\phve}{\mathphv{e}}
\newcommand{\phvf}{\mathphv{f}}
\newcommand{\phvg}{\mathphv{g}}
\newcommand{\phvh}{\mathphv{h}}
\newcommand{\phvi}{\mathphv{i}}
\newcommand{\phvj}{\mathphv{j}}
\newcommand{\phvk}{\mathphv{k}}
\newcommand{\phvl}{\mathphv{l}}
\newcommand{\phvm}{\mathphv{m}}
\newcommand{\phvn}{\mathphv{n}}
\newcommand{\phvo}{\mathphv{o}}
\newcommand{\phvp}{\mathphv{p}}
\newcommand{\phvq}{\mathphv{q}}
\newcommand{\phvr}{\mathphv{r}}
\newcommand{\phvs}{\mathphv{s}}
\newcommand{\phvt}{\mathphv{t}}
\newcommand{\phvu}{\mathphv{u}}
\newcommand{\phvv}{\mathphv{v}}
\newcommand{\phvw}{\mathphv{w}}
\newcommand{\phvx}{\mathphv{x}}
\newcommand{\phvy}{\mathphv{y}}
\newcommand{\phvz}{\mathphv{z}}
%%%%%%%%%%%%%%%%%%%%%%%%%%%%%%%
%%%%%%%%%%%%%%%%%%%%%%%%%%%%%%%
% pzc
%%%%%%%%%%%%%%%%%%%%%%%%%%%%%%%
\newcommand{\pzcA}{\mathpzc{A}}
\newcommand{\pzcB}{\mathpzc{B}}
\newcommand{\pzcC}{\mathpzc{C}}
\newcommand{\pzcD}{\mathpzc{D}}
\newcommand{\pzcE}{\mathpzc{E}}
\newcommand{\pzcF}{\mathpzc{F}}
\newcommand{\pzcG}{\mathpzc{G}}
\newcommand{\pzcH}{\mathpzc{H}}
\newcommand{\pzcI}{\mathpzc{I}}
\newcommand{\pzcJ}{\mathpzc{J}}
\newcommand{\pzcK}{\mathpzc{K}}
\newcommand{\pzcL}{\mathpzc{L}}
\newcommand{\pzcM}{\mathpzc{M}}
\newcommand{\pzcN}{\mathpzc{N}}
\newcommand{\pzcO}{\mathpzc{O}}
\newcommand{\pzcP}{\mathpzc{P}}
\newcommand{\pzcQ}{\mathpzc{Q}}
\newcommand{\pzcR}{\mathpzc{R}}
\newcommand{\pzcS}{\mathpzc{S}}
\newcommand{\pzcT}{\mathpzc{T}}
\newcommand{\pzcU}{\mathpzc{U}}
\newcommand{\pzcV}{\mathpzc{V}}
\newcommand{\pzcW}{\mathpzc{W}}
\newcommand{\pzcX}{\mathpzc{X}}
\newcommand{\pzcY}{\mathpzc{Y}}
\newcommand{\pzcZ}{\mathpzc{Z}}
%%%%%%%%%%%%%%%%%%%%%%%%%%%%%%%
\newcommand{\pzca}{\mathpzc{a}}
\newcommand{\pzcb}{\mathpzc{b}}
\newcommand{\pzcc}{\mathpzc{c}}
\newcommand{\pzcd}{\mathpzc{d}}
\newcommand{\pzce}{\mathpzc{e}}
\newcommand{\pzcf}{\mathpzc{f}}
\newcommand{\pzcg}{\mathpzc{g}}
\newcommand{\pzch}{\mathpzc{h}}
\newcommand{\pzci}{\mathpzc{i}}
\newcommand{\pzcj}{\mathpzc{j}}
\newcommand{\pzck}{\mathpzc{k}}
\newcommand{\pzcl}{\mathpzc{l}}
\newcommand{\pzcm}{\mathpzc{m}}
\newcommand{\pzcn}{\mathpzc{n}}
\newcommand{\pzco}{\mathpzc{o}}
\newcommand{\pzcp}{\mathpzc{p}}
\newcommand{\pzcq}{\mathpzc{q}}
\newcommand{\pzcr}{\mathpzc{r}}
\newcommand{\pzcs}{\mathpzc{s}}
\newcommand{\pzct}{\mathpzc{t}}
\newcommand{\pzcu}{\mathpzc{u}}
\newcommand{\pzcv}{\mathpzc{v}}
\newcommand{\pzcw}{\mathpzc{w}}
\newcommand{\pzcx}{\mathpzc{x}}
\newcommand{\pzcy}{\mathpzc{y}}
\newcommand{\pzcz}{\mathpzc{z}}
%%%%%%%%%%%%%%%%%%%%%%%%%%%%%%%
%%%%%%%%%%%%%%%%%%%%%%%%%%%%%%%
% bb
%%%%%%%%%%%%%%%%%%%%%%%%%%%%%%%
\newcommand{\bbA}{\mathbb{A}}
\newcommand{\bbB}{\mathbb{B}}
\newcommand{\bbC}{\mathbb{C}}
\newcommand{\bbD}{\mathbb{D}}
\newcommand{\bbE}{\mathbb{E}}
\newcommand{\bbF}{\mathbb{F}}
\newcommand{\bbG}{\mathbb{G}}
\newcommand{\bbH}{\mathbb{H}}
\newcommand{\bbI}{\mathbb{I}}
\newcommand{\bbJ}{\mathbb{J}}
\newcommand{\bbK}{\mathbb{K}}
\newcommand{\bbL}{\mathbb{L}}
\newcommand{\bbM}{\mathbb{M}}
\newcommand{\bbN}{\mathbb{N}}
\newcommand{\bbO}{\mathbb{O}}
\newcommand{\bbP}{\mathbb{P}}
\newcommand{\bbQ}{\mathbb{Q}}
\newcommand{\bbR}{\mathbb{R}}
\newcommand{\bbS}{\mathbb{S}}
\newcommand{\bbT}{\mathbb{T}}
\newcommand{\bbU}{\mathbb{U}}
\newcommand{\bbV}{\mathbb{V}}
\newcommand{\bbW}{\mathbb{W}}
\newcommand{\bbX}{\mathbb{X}}
\newcommand{\bbY}{\mathbb{Y}}
\newcommand{\bbZ}{\mathbb{Z}}
%%%%%%%%%%%%%%%%%%%%%%%%%%%%%%%
\newcommand{\bba}{\mathbb{a}}
\newcommand{\bbb}{\mathbb{b}}
\newcommand{\bbc}{\mathbb{c}}
\newcommand{\bbd}{\mathbb{d}}
\newcommand{\bbe}{\mathbb{e}}
\newcommand{\bbf}{\mathbb{f}}
\newcommand{\bbg}{\mathbb{g}}
\newcommand{\bbh}{\mathbb{h}}
\newcommand{\bbi}{\mathbb{i}}
\newcommand{\bbj}{\mathbb{j}}
\newcommand{\bbk}{\mathbb{k}}
\newcommand{\bbl}{\mathbb{l}}
\newcommand{\bbm}{\mathbb{m}}
\newcommand{\bbn}{\mathbb{n}}
\newcommand{\bbo}{\mathbb{o}}
\newcommand{\bbp}{\mathbb{p}}
\newcommand{\bbq}{\mathbb{q}}
\newcommand{\bbr}{\mathbb{r}}
\newcommand{\bbs}{\mathbb{s}}
\newcommand{\bbt}{\mathbb{t}}
\newcommand{\bbu}{\mathbb{u}}
\newcommand{\bbv}{\mathbb{v}}
\newcommand{\bbw}{\mathbb{w}}
\newcommand{\bbx}{\mathbb{x}}
\newcommand{\bby}{\mathbb{y}}
\newcommand{\bbz}{\mathbb{z}}
%%%%%%%%%%%%%%%%%%%%%%%%%%%%%%%
%%%%%%%%%%%%%%%%%%%%%%%%%%%%%%%
% sc
%%%%%%%%%%%%%%%%%%%%%%%%%%%%%%%
\newcommand{\scA}{\mathscr{A}}
\newcommand{\scB}{\mathscr{B}}
\newcommand{\scC}{\mathscr{C}}
\newcommand{\scD}{\mathscr{D}}
\newcommand{\scE}{\mathscr{E}}
\newcommand{\scF}{\mathscr{F}}
\newcommand{\scG}{\mathscr{G}}
\newcommand{\scH}{\mathscr{H}}
\newcommand{\scI}{\mathscr{I}}
\newcommand{\scJ}{\mathscr{J}}
\newcommand{\scK}{\mathscr{K}}
\newcommand{\scL}{\mathscr{L}}
\newcommand{\scM}{\mathscr{M}}
\newcommand{\scN}{\mathscr{N}}
\newcommand{\scO}{\mathscr{O}}
\newcommand{\scP}{\mathscr{P}}
\newcommand{\scQ}{\mathscr{Q}}
\newcommand{\scR}{\mathscr{R}}
\newcommand{\scS}{\mathscr{S}}
\newcommand{\scT}{\mathscr{T}}
\newcommand{\scU}{\mathscr{U}}
\newcommand{\scV}{\mathscr{V}}
\newcommand{\scW}{\mathscr{W}}
\newcommand{\scX}{\mathscr{X}}
\newcommand{\scY}{\mathscr{Y}}
\newcommand{\scZ}{\mathscr{Z}}
%%%%%%%%%%%%%%%%%%%%%%%%%%%%%%%
%%%%%%%%%%%%%%%%%%%%%%%%%%%%%%%
% cal
%%%%%%%%%%%%%%%%%%%%%%%%%%%%%%%
\newcommand{\calA}{\mathcal{A}}
\newcommand{\calB}{\mathcal{B}}
\newcommand{\calC}{\mathcal{C}}
\newcommand{\calD}{\mathcal{D}}
\newcommand{\calE}{\mathcal{E}}
\newcommand{\calF}{\mathcal{F}}
\newcommand{\calG}{\mathcal{G}}
\newcommand{\calH}{\mathcal{H}}
\newcommand{\calI}{\mathcal{I}}
\newcommand{\calJ}{\mathcal{J}}
\newcommand{\calK}{\mathcal{K}}
\newcommand{\calL}{\mathcal{L}}
\newcommand{\calM}{\mathcal{M}}
\newcommand{\calN}{\mathcal{N}}
\newcommand{\calO}{\mathcal{O}}
\newcommand{\calP}{\mathcal{P}}
\newcommand{\calQ}{\mathcal{Q}}
\newcommand{\calR}{\mathcal{R}}
\newcommand{\calS}{\mathcal{S}}
\newcommand{\calT}{\mathcal{T}}
\newcommand{\calU}{\mathcal{U}}
\newcommand{\calV}{\mathcal{V}}
\newcommand{\calW}{\mathcal{W}}
\newcommand{\calX}{\mathcal{X}}
\newcommand{\calY}{\mathcal{Y}}
\newcommand{\calZ}{\mathcal{Z}}
%%%%%%%%%%%%%%%%%%%%%%%%%%%%%%%
%%%%%%%%%%%%%%%%%%%%%%%%%%%%%%%
% txt
%%%%%%%%%%%%%%%%%%%%%%%%%%%%%%%
\newcommand{\txtA}{\text{A}}
\newcommand{\txtB}{\text{B}}
\newcommand{\txtC}{\text{C}}
\newcommand{\txtD}{\text{D}}
\newcommand{\txtE}{\text{E}}
\newcommand{\txtF}{\text{F}}
\newcommand{\txtG}{\text{G}}
\newcommand{\txtH}{\text{H}}
\newcommand{\txtI}{\text{I}}
\newcommand{\txtJ}{\text{J}}
\newcommand{\txtK}{\text{K}}
\newcommand{\txtL}{\text{L}}
\newcommand{\txtM}{\text{M}}
\newcommand{\txtN}{\text{N}}
\newcommand{\txtO}{\text{O}}
\newcommand{\txtP}{\text{P}}
\newcommand{\txtQ}{\text{Q}}
\newcommand{\txtR}{\text{R}}
\newcommand{\txtS}{\text{S}}
\newcommand{\txtT}{\text{T}}
\newcommand{\txtU}{\text{U}}
\newcommand{\txtV}{\text{V}}
\newcommand{\txtW}{\text{W}}
\newcommand{\txtX}{\text{X}}
\newcommand{\txtY}{\text{Y}}
\newcommand{\txtZ}{\text{Z}}
%%%%%%%%%%%%%%%%%%%%%%%%%%%%%%%
\newcommand{\txta}{\text{a}}
\newcommand{\txtb}{\text{b}}
\newcommand{\txtc}{\text{c}}
\newcommand{\txtd}{\text{d}}
\newcommand{\txte}{\text{e}}
\newcommand{\txtf}{\text{f}}
\newcommand{\txtg}{\text{g}}
\newcommand{\txth}{\text{h}}
\newcommand{\txti}{\text{i}}
\newcommand{\txtj}{\text{j}}
\newcommand{\txtk}{\text{k}}
\newcommand{\txtl}{\text{l}}
\newcommand{\txtm}{\text{m}}
\newcommand{\txtn}{\text{n}}
\newcommand{\txto}{\text{o}}
\newcommand{\txtp}{\text{p}}
\newcommand{\txtq}{\text{q}}
\newcommand{\txtr}{\text{r}}
\newcommand{\txts}{\text{s}}
\newcommand{\txtt}{\text{t}}
\newcommand{\txtu}{\text{u}}
\newcommand{\txtv}{\text{v}}
\newcommand{\txtw}{\text{w}}
\newcommand{\txtx}{\text{x}}
\newcommand{\txty}{\text{y}}
\newcommand{\txtz}{\text{z}}
%%%%%%%%%%%%%%%%%%%%%%%%%%%%%%%
%%%%%%%%%%%%%%%%%%%%%%%%%%%%%%%
% tt
%%%%%%%%%%%%%%%%%%%%%%%%%%%%%%%
\newcommand{\ttA}{\mathtt{A}}
\newcommand{\ttB}{\mathtt{B}}
\newcommand{\ttC}{\mathtt{C}}
\newcommand{\ttD}{\mathtt{D}}
\newcommand{\ttE}{\mathtt{E}}
\newcommand{\ttF}{\mathtt{F}}
\newcommand{\ttG}{\mathtt{G}}
\newcommand{\ttH}{\mathtt{H}}
\newcommand{\ttI}{\mathtt{I}}
\newcommand{\ttJ}{\mathtt{J}}
\newcommand{\ttK}{\mathtt{K}}
\newcommand{\ttL}{\mathtt{L}}
\newcommand{\ttM}{\mathtt{M}}
\newcommand{\ttN}{\mathtt{N}}
\newcommand{\ttO}{\mathtt{O}}
\newcommand{\ttP}{\mathtt{P}}
\newcommand{\ttQ}{\mathtt{Q}}
\newcommand{\ttR}{\mathtt{R}}
\newcommand{\ttS}{\mathtt{S}}
\newcommand{\ttT}{\mathtt{T}}
\newcommand{\ttU}{\mathtt{U}}
\newcommand{\ttV}{\mathtt{V}}
\newcommand{\ttW}{\mathtt{W}}
\newcommand{\ttX}{\mathtt{X}}
\newcommand{\ttY}{\mathtt{Y}}
\newcommand{\ttZ}{\mathtt{Z}}
%%%%%%%%%%%%%%%%%%%%%%%%%%%%%%%
\newcommand{\tta}{\mathtt{a}}
\newcommand{\ttb}{\mathtt{b}}
\newcommand{\ttc}{\mathtt{c}}
\newcommand{\ttd}{\mathtt{d}}
\newcommand{\tte}{\mathtt{e}}
\newcommand{\ttf}{\mathtt{f}}
\newcommand{\ttg}{\mathtt{g}}
\newcommand{\tth}{\mathtt{h}}
\newcommand{\tti}{\mathtt{i}}
\newcommand{\ttj}{\mathtt{j}}
\newcommand{\ttk}{\mathtt{k}}
\newcommand{\ttl}{\mathtt{l}}
\newcommand{\ttm}{\mathtt{m}}
\newcommand{\ttn}{\mathtt{n}}
\newcommand{\tto}{\mathtt{o}}
\newcommand{\ttp}{\mathtt{p}}
\newcommand{\ttq}{\mathtt{q}}
\newcommand{\ttr}{\mathtt{r}}
\newcommand{\tts}{\mathtt{s}}
\newcommand{\ttt}{\mathtt{t}}
\newcommand{\ttu}{\mathtt{u}}
\newcommand{\ttv}{\mathtt{v}}
\newcommand{\ttw}{\mathtt{w}}
\newcommand{\ttx}{\mathtt{x}}
\newcommand{\tty}{\mathtt{y}}
\newcommand{\ttz}{\mathtt{z}}
%%%%%%%%%%%%%%%%%%%%%%%%%%%%%%%


% Local Variables:
% ispell-local-dictionary: "en_US"
% End:

\input{Acros.tex}

\begin{document}

\begin{frame}
	\titlepage
\end{frame}

\AtBeginPart{
\begin{frame}<handout:0>
	\begin{block}{\centering\large\bfseries Parte \insertpartnumber}
		\centering\large\insertpart
	\end{block}
\end{frame}

\begin{frame}
	\frametitle{Conte�do}
	\tableofcontents
\end{frame}
}
%
% \AtBeginLecture{
% \begin{frame}<handout:0>
% 	\begin{block}{\centering\large\bfseries Tema da aula}
% 		\centering\large\insertlecture
% 	\end{block}
% \end{frame}
% }

% \AtBeginSection{
% \begin{frame}
% 	\frametitle{Conte�do}
% 	\tableofcontents[currentsection,hideothersubsections]
% \end{frame}
% }

\AtBeginSubsection{
\begin{frame}<handout:0>
	\frametitle{Conte�do}
	\tableofcontents[subsectionstyle=show/shaded/shaded]
% 	\tableofcontents[subsectionstyle=show/shaded/hide]
\end{frame}
}

\part{Conjuntos Convexos}


\begin{frame}{Conjuntos Convexos}

 \begin{itemize}
	\item Conjuntos Afins 
            \begin{itemize}
            \item Um conjunto $ C \subseteq \fdR^n$ � afim caso um segmento entre $x_{1}, x_{2} \in C$ e $ \theta \in \fdR^n$, perten�a a $ C $, ou seja, $ \theta x_{1}+(1-\theta) x_{2} \in C$.
	     \end{itemize}
	
	\end{itemize}

 \begin{itemize}
	\item Conjuntos Convexos
            \begin{itemize}
            \item Um conjunto $ C \subseteq \fdR^n$ � convexo caso um segmento entre $x_{1}, x_{2} \in C$ e $ 0 \leq \theta \leq 1 $, perten�a a $ C $, ou seja, $ \theta x_{1}+(1-\theta) x_{2} \in C$.
	     \end{itemize}
 \end{itemize}
% 
\end{frame}

\begin{frame}{Conjuntos Convexos}

 \begin{itemize}
	\item Covex Hull 
            \begin{itemize}
            \item Dado um conjunto $ C $, o convex hull � o menor conjunto convexo que engloba $ C $, denotado por $\textbf{conv } C$. 
               
               \begin{equation}\label{eq_convex_hull}
			\textbf{conv } C = \{ \theta_{1} x_{1} + \cdots + \theta_{k} x_{k} \mid x_{1} \in C, \theta_{i} \geq 0, i=1, \cdots, k, \theta_{1} + \cdots + \theta_{k} = 1\}
		\end{equation}
         
	    
	    \end{itemize}
	
	\end{itemize}
% 
\end{frame}


\begin{frame}{Conjuntos Convexos}

 \begin{itemize}
	\item Cone
            \begin{itemize}
            \item Um conjunto � chamado cone, caso para todo $ x \in C$ e $ \theta \geq 0$ (n�o negativo), ou seja:
            
                \begin{equation}\label{eq_cone}
			\theta x \in C
		\end{equation}
		
	    \end{itemize}
	
	\item Cone Convexo
            \begin{itemize}
            \item Um conjunto � chamado cone convexo, caso seja cone e convexo, para qualquer $x_{1}, x_{2} \in C$ e $ \theta_{1}, \theta_{2} \geq 0$, ou seja:  
                \begin{equation}\label{eq_cone_conv}
	         \theta_{1} x_{1} + \theta_{2} x_{2} \in C
		\end{equation}
	    \end{itemize}
	
	\end{itemize}
\end{frame}



\begin{frame}{Conjuntos Convexos}

 \begin{itemize}
	\item Hiperplano
            \begin{itemize}
            \item Hiperplano � um conjunto de pontos, podendo ser escrito como:
            
                \begin{equation}\label{eq_hiperplano}
			\{ x \mid a^{T}(x-x_{0} = 0) \}
		\end{equation}
		onde $ a \in \fdR^n$, $ a \neq 0$ e $x_{0}$ determina o offset do hiperplano. Um hiperplano divide o espa�o em dois semi-espa�os.
	    \end{itemize}
	
	\item Semi-espa�o
            \begin{itemize}
            \item Um semi-espa�o � um conjunto da forma:  
                \begin{equation}\label{eq_halfspace}
	         \{ x \mid a^{T}(x-x_{0} \leq 0) \}
		\end{equation}
		onde $ a \in \fdR^n$, $ a \neq 0$.
	    \end{itemize}
	\end{itemize}
\end{frame}


\begin{frame}{Conjuntos Convexos}

 \begin{itemize}
	\item Bola Euclidiana
            \begin{itemize}
            \item A bola euclidiana com o centro em $x_{c}$ e raio $r$ � representado por:
            
                \begin{equation}\label{eq_ball}
			B(x_{c},r) = \{ x_{c} + ru \mid \Norm{u}_2 \leq 1 \}
		\end{equation}
	    \end{itemize}
	
	\item Elipsoide
            \begin{itemize}
            \item  O elipsoide � um conjunto que pode ser representado na forma:   
                \begin{equation}\label{eq_elipsoide}
	        \varepsilon = \{ x \mid (x-x_{c})^{T} P^{-1} (x-x_{c}) \leq 1) \}
		\end{equation}
		onde $ P \in \fdS^n_{++}$(matriz simetrica e positiva definida).
	    \end{itemize}
	\end{itemize}
\end{frame}


\begin{frame}{Conjuntos Convexos}

 \begin{itemize}
	\item Poliedro
            \begin{itemize}
            \item O poliedro � definido como um conjunto de igualdades lineares e inequa��es.
            
                \begin{equation}\label{eq_poliedro}
			A x \preceq b, C x = d
		\end{equation}
		onde $\preceq$ � o s�mbolo que representa desiguadade componente a componente entre vetores.
	  
	     \item Dessa forma, a representa��o do poliedro pode escrita como: 
	  
	        \begin{equation}\label{eq_poliedro_2}
	        P = \{ x \mid A x \preceq b, C x = d \}
		\end{equation}
	  \end{itemize}
	
	\end{itemize}
\end{frame}


\begin{frame}{Conjuntos Convexos}

 \begin{itemize}
	\item Norma da Bola com centro $x_{c}$ e raio $r$ � representado por:
	
	\begin{equation}\label{eq_norm_ball}
			\{ x \mid \Norm{x-x_c} \leq r \}
		\end{equation}
  onde $\Norm{\cdot}$ � a norma.
  
            \item A norma do cone pode ser definido como:
            
                \begin{equation}\label{eq_norm_cone}
			\{ (x,t) \in  \fdR^n+1 \mid \Norm{x} \leq t \}
		\end{equation}
	\end{itemize}
	
\end{frame}



\begin{frame}{Conjuntos Convexos}

 \begin{itemize}
	\item PSD (Cone positivo semidefinido) � um cone formado a parti das matrizes positivas semidefinidas que utiliza a nota��o:
	
	\begin{equation}\label{eq_psd}
			\fdS^n_+ = \{ X \in \fdS^n \mid X \succ 0 \}
		\end{equation}
  onde $\fdS^n$ denota o conjunto de matrizes simetricas. Dessa forma o conjunto $\fdS^n_+$ � um cone positivo semidefinido, se $ \theta_{1}, \theta_{2} \geqslant 0$ e $A, B \in \fdS^n_+$, assim $ \theta_{1} A + \theta_{2} B \in \fdS^n_+$.	
  \end{itemize}
\end{frame}


\begin{frame}{Conjuntos Convexos}

 \begin{itemize} 
	\item Opera��es que preserva a convexidade
            \begin{itemize}
              \item Interse��o
                  \begin{itemize}
                    \item Sendo $ S_{\alpha}$ convexo para todo $\alpha \in A$:
                    	        \begin{equation}\label{eq_oper_inter}
				   \cap_{\alpha \in A} S_{\alpha} \textbf{ (Convexo)}
				\end{equation}
                    
                  \end{itemize}
              
              \item Multiplica��o por escalar
                  \begin{itemize}
                    \item Se $ S \subseteq \fdR^n$ � convexa, $\alpha \in \fdR$:
                               \begin{equation}\label{eq_oper_escalar}
				   \alpha S = \{ \alpha x \mid x \in S \} \textbf{ (Convexo)}
				\end{equation}
                    
                  \end{itemize}
      
	                  \item Transla��o
                  \begin{itemize}
                    \item Se $ S \subseteq \fdR^n$ � convexa, $a \in \fdR^n$:
                               \begin{equation}\label{eq_oper_trans}
				   S + a = \{ x + a \mid x \in S \} \textbf{ (Convexo)}
				\end{equation}
                    
                  \end{itemize}
	    
	    \end{itemize}
	
	\end{itemize}
\end{frame}


\begin{frame}{Conjuntos Convexos}

 \begin{itemize} 
	\item Opera��es que preserva a convexidade
            \begin{itemize}
              \item Soma
                  \begin{itemize}
                    \item Se $S_{1}$ e $S_{2}$ s�o convexos:
                    	        \begin{equation}\label{eq_oper_soma}
				  S_{1} + S_{2} = \{ x + y \mid x \in S_{1}, y \in S_{2}\} \textbf{ (Convexo)}
				\end{equation}
                    
                  \end{itemize}
              
              \item Produto Cartesiano
                  \begin{itemize}
                     \item Se $S_{1}$ e $S_{2}$ s�o convexos:
                               \begin{equation}\label{eq_oper_pcart}
				   S_{1} \times S_{2} = \{ (x,y) \mid x \in S_{1}, y \in S_{2}\} \textbf{ (Convexo)}
				\end{equation}
                    
                  \end{itemize}
      
	    
	    \end{itemize}
	
	\end{itemize}
\end{frame}


\begin{frame}{Conjuntos Convexos}

 \begin{itemize} 
	\item Desigualdade generalizadas
            \begin{itemize}
              \item Um cone pode ser usado para definir a desigualdade generalizada, que � similar a rela��o de ordem apresentada em $\fdR$. Desta forma, para um cone $ K $ onde $ K \subseteq \fdR^n$, as desigualdades n�o estritas e estritas $ \preceq_{K}$ e $\prec_{K}$, respectivamente s�o definidas da seguinte maneira:
              
			\begin{subequations}
			\begin{align}
                               x \preceq_{K} y \Leftrightarrow y - x \in K\\
                               x \prec_{K} y \Leftrightarrow y - x \in \textbf{int(}K\textbf{)}
			\end{align}
		\end{subequations}
		
		onde $\textbf{int($K$)}$ representa o interior do conjunto $K$.
			
		\item Propriedades
                  \begin{itemize}
                     \item Se $x \preceq_{K} y$ e $u \preceq_{K} v$, ent�o $ x + u \preceq_{K} y + v$
                      \item  Se $x \preceq_{K} y$ e $y \preceq_{K} u$, ent�o $ x \preceq_{K} u$
                       \item  Se $x \preceq_{K} y$ e $y \preceq_{K} x$, ent�o $ x = y$
                        \item  Se $x \prec_{K} y$ e $u \preceq_{K} v$, ent�o $ x + u \prec_{K} y + v$
                  \end{itemize}
	    
	    \end{itemize}
	
	\end{itemize}
\end{frame}


\begin{frame}{Water filling}

 \begin{itemize}
	\item O objetivo do water filling � distribuir pot�ncias entre os links de uma sistema telecomunica��o objetivando a maximiza�ao da vaz�o de dados, uma vez que a qualidade do link dependente da pot�ncia utilizada na comunica��o, dessa forma:
		\begin{equation}
		\min - \sum\limits_{i = 1}^n \log(\alpha_{i} + x_{i}) \text{sujeito a}\begin{cases}
				x  \succeq 0, \\
				\textbf{1}^{T}x = 1
			\end{cases}
		\end{equation}
		onde $\alpha_{i} > 0$, $x_{i}$ representa a pot�ncia alocada ao canal $i$.
  \end{itemize}

  \end{frame}

  \part{Water Filling}
  
  \begin{frame}{Water Filling}

 \begin{itemize}
	\item Adotando o multiplicador lagrangeano $ \lambda^{*} \in \fdR^n$ para $x \succeq 0$ e o multiplicador $v^{*} \in \fdR$ para $\textbf{1}^{T}x = 1$, o lagrangeano apresenta a seguinte forma:
		\begin{equation}
		H(x^{*},\lambda^{*}, v^{*}) = - \sum\limits_{i = 1}^n \log(\alpha_{i} + x_{i}) - \lambda^{*} x + v^{*}(\textbf{1}^{T}x - 1)
		\end{equation}

    \item Derivando em seguida utilizando as KKT: 
  
			\begin{subequations}
			\begin{align}		
                               \frac{H(x^{*}, \lambda^{*}, v^{*})}{\partial x^{*}} = - \frac{1}{(\alpha_{i}+x_{i})} - \lambda^{*} + v^{*} = 0\\
                                \frac{H(x^{*}, \lambda^{*}, v^{*})}{\partial \lambda^{*}} = x = 0\\
                                 \frac{H(x^{*}, \lambda^{*}, v^{*})}{\partial v^{*}} = \textbf{1}^{T}x - 1 = 0
			\end{align}
		\end{subequations}
  
  
    \end{itemize}
  \end{frame}

  
    \begin{frame}{Water Filling}

 \begin{itemize}
	\item Utilizando as KKT:
		\begin{subequations}
			\begin{align}		
                              x_{*} \succeq 0, \\
                              \textbf{1}^{T}x^{*} = 1, \\
                              \lambda^{*} \succeq 0,\\
                              \lambda^{*}_{i}x^{*}_{i} = 0, \\
                              - \frac{1}{(\alpha_{i}+x^{*}_{i})} - \lambda^{*}_{i} + v^{*} = 0
			\end{align}
		\end{subequations}

 \item Isolando $\lambda^{*}_{i}$ e substituindo em $\lambda^{*}_{i}x^{*}_{i}$:
  
			\begin{subequations}
			\begin{align}		
                              x^{*}_{i}( v^{*} - \frac{1}{\alpha_{i} + x^{*}_{i}}) = 0, \\
                              v^{*} - \frac{1}{\alpha_{i} + x^{*}_{i}} \succeq 0
			\end{align}
		\end{subequations}
  
  
    \end{itemize}
  \end{frame}
  
  
  
    \begin{frame}{Water Filling}

 \begin{itemize}
	\item Isolando $x^{*}_{i}$ da express�o $v^{*} \succeq  \frac{1}{\alpha_{i} + x^{*}_{i}}$ temos:
	
			\begin{equation}
                             x^{*}_{i} \geq \frac{1}{v_{*}} - \alpha_{i} 
		        \end{equation}
		

 \item Entretando $x^{*}_{i}$ n�o pode ser negativo no caso $ v^{*} > \frac{1}{\alpha_{i}}$, assim:
  
		\begin{equation}\label{eq_vec_orthonormal}
			x^{*}_{i} = \begin{cases}
				\frac{1}{v^{*}} - \alpha_{i}, & v^{*} < \frac{1}{\alpha_{i}} \\
				0, & v^{*} \geq \frac{1}{\alpha_{i}}
			\end{cases}
		\end{equation}
  
  
    \end{itemize}
  \end{frame}
  
 

\end{document}


\section{Bibliografia}
\begin{frame}
	\frametitle{Bibliografia}
	\footnotesize\bfseries
	\bibliographystyle{IEEEtranSA}
	\bibliography{IEEEfull,Full,MyBib}
\end{frame}

\end{document}

% Local Variables:
% fill-column: 120
% ispell-local-dictionary: "pt_BR"
% TeX-master: "Slides"
% End:
