\documentclass{article}
\usepackage[margin=2mm,textwidth=155mm,paperheight=100cm,paperwidth=30cm]{geometry}

% table
\usepackage{tabularx}
\newcolumntype{A}{>{\setlength\hsize{1\hsize}\setlength\linewidth{\hsize}}X}
\newcolumntype{B}{>{\setlength\hsize{.5\hsize}\setlength\linewidth{\hsize}\centering\arraybackslash}X}
\newcolumntype{C}{>{\setlength\hsize{1\hsize}\setlength\linewidth{\hsize}}X}

% itemize
\usepackage{enumitem}

% math
\usepackage{amsmath}
\usepackage{amsfonts}
\usepackage{amssymb}
\usepackage{newtxmath} % for Greek variants (bold, nonitalic, etc...)

% commands
\newcommand{\trans}{\mathsf{T}}
\newcommand{\hermit}{\mathsf{H}}
\newcommand\norm[1]{\left\lVert#1\right\rVert}
\newcommand\abs[1]{\left\lvert#1\right\rvert}

\begin{document}

\begin{table}[ht!]
\begin{tabularx}{\textwidth}{|A|C|}
\hline
\multicolumn{2}{|c|}{Convex sets}\\
\hline
\multicolumn{1}{|c|}{Set} & \multicolumn{1}{|c|}{Comments}\\
\hline
Convex hull:
\begin{itemize}
\item $\textnormal{conv } C = \left\{ \sum_{i=1}^{k} \theta_i\mathbf{x}_i \mid \mathbf{x}_i \in C, \mathbf{0} \preceq \boldsymbol{\thetaup} \preceq \mathbf{1}, \mathbf{1}^\trans\boldsymbol{\thetaup} = 1  \right\}$
\end{itemize} & \vspace{-3.5ex}
\begin{itemize}[leftmargin=*]
    \item $\textnormal{conv } C$ will be the smallest convex set that contains $C$.
    \item $\textnormal{conv } C$ will be a finite set as long as $C$ is also finite.
\end{itemize}\\
\hline
Affine hull: 
\begin{itemize}[leftmargin=*]
    \item $\textnormal{aff } C = \left\{ \sum_{i=1}^{k} \theta_i\mathbf{x}_i \mid \mathbf{x}_i \in C \textnormal{ for } i=1,\cdots, k, \mathbf{1}^\trans\boldsymbol{\thetaup} = 1  \right\}$
\end{itemize} & \vspace{-3.5ex}
\begin{itemize}[leftmargin=*]
    \item $A$ will be the smallest affine set that contains $C$.
    \item Different from the convex set, \(\theta_i\) is not restricted between 0 and 1
    \item $\textnormal{aff } C$ will always be an infinite set. If $\textnormal{aff } C$ contains the origin, it is also a subspace.
\end{itemize}\\
\hline
Conic hull:
\begin{itemize}[leftmargin=*]
    \item $A = \left\{ \sum_{i=1}^{k} \theta_i\mathbf{x}_i \mid \mathbf{x}_i \in C, \theta_i > 0 \textnormal{ for } i=1,\cdots, k \right\}$
\end{itemize} & \vspace{-3.5ex}
\begin{itemize}[leftmargin=*]
    \item $A$ will be the smallest convex conic that contains $C$.
    \item Different from the convex and affine sets, \(\theta_i\) does not need to sum up 1.
\end{itemize}\\
\hline
Ray:
\begin{itemize}[leftmargin=*]
    \item \(\mathcal{R} = \left\{ \mathbf{x}_0 + \theta \mathbf{v} \mid \theta \geq 0 \right\}\)
\end{itemize} & \vspace{-3.5ex} \begin{itemize}[leftmargin=*]
    \item The ray is an infinite set that begins in \(\mathbf{x}_0\) and extends infinitely in direction of \(\mathbf{v}\). In other words, it has a beginning, but it has no end.
\end{itemize} \\
\hline
Hyperplane:
\begin{itemize}[leftmargin=*]
    \item \( \mathcal{H} = \left\{ \mathbf{x} \mid \mathbf{a}^\trans \mathbf{x} = b \right\}\)
    \item \(\mathcal{H} = \left\{ \mathbf{x} \mid \mathbf{a}^\trans (\mathbf{x} - \mathbf{x}_{0}) = \mathbf{0} \right\}\)
    \item \(\mathcal{H} = \mathbf{x}_0 + a^{\perp} \)
\end{itemize} & \vspace{-3.5ex}
\begin{itemize}[leftmargin=*]
    \item It is an infinite set \(\mathbb{R}^{n-1} \subset \mathbb{R}^{n}\) that divides the space into two halfspaces.
    \item \(a^{\perp} = \left\{ \mathbf{v} \mid \mathbf{a}^\trans \mathbf{v} = 0 \right\}\) is the set of vectors perpendicular to \(\mathbf{a}\). It passes through the origin.
    \item \(a^{\perp}\) is offset from the origin by \(\mathbf{x}_0\), which is any vector in \(\mathcal{H}\).
\end{itemize} \\
\hline
Halfspaces:
\begin{itemize}[leftmargin=*]
    \item \(\mathcal{H}_{-} = \left\{ \mathbf{x} \mid \mathbf{a}^\trans \mathbf{x} \leq b \right\}\)
    \item \(\mathcal{H}_{+} = \left\{ \mathbf{x} \mid \mathbf{a}^\trans \mathbf{x} \geq b \right\}\)
\end{itemize} & \vspace{-3.5ex}
\begin{itemize}[leftmargin=*]
    \item They are infinite sets of the parts divided by \(\mathcal{H}\).
\end{itemize}\\
\hline
Euclidean ball:
\begin{itemize}[leftmargin=*]
    \item \(B(\mathbf{x}_c, r) = \left\{ \mathbf{x} \mid \norm{\mathbf{x}-\mathbf{x}_c}_2 \leq r \right\}\)
    \item \(B(\mathbf{x}_c, r) = \left\{ \mathbf{x} \mid \left( \mathbf{x}-\mathbf{x}_c \right)^\trans \left( \mathbf{x}-\mathbf{x}_c \right) \leq r \right\}\)
    \item \(B(\mathbf{x}_c, r) = \left\{ \mathbf{x}_c + r \norm{\mathbf{u}} \mid \norm{\mathbf{u}} \leq 1 \right\}\)
\end{itemize} & \vspace{-3.5ex}
\begin{itemize}[leftmargin=*]
    \item \(B(\mathbf{x}_c, r)\) is a finite set as long as \(r < \infty\).
    \item \(\mathbf{x}_c\) is the center of the ball.
    \item \(r\) is its radius.
\end{itemize}\\
\hline
Ellipsoid:
\begin{itemize}[leftmargin=*]
    \item \(\mathcal{E} = \left\{ \mathbf{x} \mid (\mathbf{x}-\mathbf{x}_c)^\trans\mathbf{P}^{-1}(\mathbf{x}-\mathbf{x}_c) \leq 1 \right\}\)
    \item \(\mathcal{E} = \left\{ \mathbf{x}_{c} + \mathbf{Au} \mid \norm{\mathbf{u}} \leq 1 \right\}\), where \(\mathbf{A} = \mathbf{P}^{1/2}\).
\end{itemize} & \vspace{-3.5ex}
\begin{itemize}[leftmargin=*]
    \item \(\mathcal{E}\) is a finite set as long as \(\mathbf{P}\) is a finite matrix.
    \item \(\mathbf{P}\) is symmetric and positive definite, that is, \(\mathbf{P}=\mathbf{P}^\trans \succ 0\).
    \item \(\mathbf{x}_{c}\) is the center of the ellipsoid.
    \item The lengths of the semi-axes are given by \(\sqrt{\lambda_i}\).
    \item \(\mathbf{A}\) is invertible. When it is not, we say that \(\mathcal{E}\) is a degenerated ellipsoid (degenerated ellipsoids are also convex).
\end{itemize}\\
\hline
Norm cone:
\begin{itemize}[leftmargin=*]
    \item \(C = \left\{ [x_1, x_2, \cdots, x_n, t]^\trans \in \mathbb{R}^{n+1} \mid \mathbf{x} \in \mathbb{R}^{n}, \norm{\mathbf{x}}_{p} \leq t \right\} \subseteq \mathbb{R}^{n+1}\)
\end{itemize} & \vspace{-3.5ex}
\begin{itemize}[leftmargin=*]
    \item Although it is named ``Norm cone'', it is a set, not a scalar.
    \item The cone norm increases the dimension of \(\mathbf{x}\) in 1.
    \item For \(p=2\), it is called the second-order cone, quadratic cone,  Lorentz cone or ice-cream cone.
\end{itemize} \\
\hline
Proper cone: \(K \subset \mathbb{R}^{n}\) is a proper cone when it has the following properties
\begin{itemize}
    \item \(K\) is a convex cone, i.e., \(\alpha K \equiv K, \alpha > 0\).
    \item \(K\) is closed.
    \item \(K\) is solid.
    \item \(K\) is pointed, i.e., \(-K \cap K = \left\{ \mathbf{0} \right\}\).
\end{itemize} & \vspace{-3.5ex} \begin{itemize}[leftmargin=*]
    \item The proper cone \(K\) is used to define the \emph{generalized inequality} (or \emph{partial ordering}) in some set \(S\). For the generalized inequality, one must define both the proper cone \(K\) and the set \(S\).
    \item \(\mathbf{x} \preceq \mathbf{y} \iff \mathbf{y} - \mathbf{x} \in K\) for \(\mathbf{x}, \mathbf{y} \in S\) (generalized inequality)
    \item \(\mathbf{x} \prec \mathbf{y} \iff \mathbf{y} - \mathbf{x} \in \textnormal{int }K\) for \(\mathbf{x}, \mathbf{y} \in S\) (strict generalized inequality).
    \item There are two cases where \(K\) and \(S\) are understood from context and the subscript \(K\) is dropped out:
        \begin{itemize}[label={$\triangleright$}]
            \item When \(S = \mathbb{R}^{n}\) and \(K = \mathbb{R}^{n}_{+}\) (the nonnegative orthant). In this case, \(\mathbf{x} \preceq \mathbf{y}\) means that \(x_i \leq y_i\).
            \item When \(S = \mathcal{S}^{n}\) and \(K = \mathcal{S}^{n}_{+}\) or \(K = \mathcal{S}^{n}_{++}\), where \(\mathcal{S}^{n}\) denotes the set of symmetric \(n\times n\) matrices, \(\mathcal{S}^{n}_{+}\) is the space of the positive semidefinite matrices, and \(\mathcal{S}^{n}_{++}\) is the space of the positive definite matrices. \(\mathcal{S}^{n}_{+}\) is a proper cone in \(\mathcal{S}^{n}\) (??). In this case, the generalized inequality \(\mathbf{Y} \succeq \mathbf{X}\) means that \(\mathbf{Y}-\mathbf{X}\) is a positive semidefinite matrix belonging to the positive semidefinite cone \(\mathcal{S}^{n}_{+}\) in the subspace of symmetric matrices \(\mathcal{S}^{n}\). It is usual to denote \(\mathbf{X} \succ \mathbf{0}\) and \(\mathbf{X} \succeq \mathbf{0}\) to mean than \(\mathbf{X}\) is a positive definite and semidefinite matrix, respectively, where \(\mathbf{0} \in \mathbb{R}^{n\times n}\) is a zero matrix.
        \end{itemize}
    \item Another common usage is when \(S = \mathbb{R}^{n}\) and \(K = \left\{ \mathbf{c} \in \mathbb{R}^{n} \mid c_1 + c_2 t + \dots + c_n t^{n-1} \geq 0, \textnormal{ for } 0\leq t\leq 1 \right\}\). In this case, \(\mathbf{x} \preceq_K \mathbf{y}\) means that \(x_1 + x_2 t + \dots + x_n t^{n-1} \leq y_1 + y_2 t + \dots + y_n t^{n-1}\).
    \item The generalized inequality has the following properties:
        \begin{itemize}[label={$\triangleright$}]
            \item If \(\mathbf{x} \preceq_K \mathbf{y}\) and \(\mathbf{u} \preceq_K \mathbf{v}\), then \(\mathbf{x} + \mathbf{u} \preceq_k \mathbf{y} + \mathbf{v}\) (preserve under addition).
            \item If \(\mathbf{x} \preceq_K \mathbf{y}\) and \(\mathbf{y} \preceq_K \mathbf{z}\), then \(\mathbf{x} \preceq_K \mathbf{z}\) (transitivity).
            \item If \(\mathbf{x}\preceq_K \mathbf{y}\), then \(\alpha\mathbf{x}\preceq_K \mathbf{y}\) for \(\alpha\geq0\) (preserve under nonnegative scaling).
            \item \(\mathbf{x}\preceq_K \mathbf{x}\) (reflexivity).
            \item If \(\mathbf{x}\preceq_K \mathbf{y}\) and \(\mathbf{y}\preceq_K \mathbf{x}\), then \(\mathbf{x} = \mathbf{y}\) (antisymmetric).
            \item If \(\mathbf{x}_i\preceq_K \mathbf{y}_i\), for \(i = 1, 2, \dots\), and \(\mathbf{x}_i \rightarrow \mathbf{x}\) and \(\mathbf{y}_i \rightarrow \mathbf{y}\) as \(i \rightarrow \infty\), then \(\mathbf{x} \preceq_K \mathbf{y}\).
        \end{itemize}
    \item It is called partial ordering because \(\mathbf{x} \nsucceq_K \mathbf{y}\) and \(\mathbf{y} \nsucceq_K \mathbf{x}\) for many \(\mathbf{x}, \mathbf{y} \in S\). When it happens, we say that \(\mathbf{x}\) and \(\mathbf{y}\) are not comparable (this case does not happen in ordinary inequality, \(<\) and \(>\)).
    \item \(\mathbf{x} \in S\) is the \emph{minimum} element of \(S\) if \(\mathbf{x} \preceq_K \mathbf{y}\) for every \(\mathbf{y} \in S\). The set does not necessarily have a minimum, but the minimum is unique if it does. The same is true for \emph{maximum}. The mathematical notation for that is \(S \subseteq \mathbf{x} + K\), where \(\mathbf{x} + K\) denotes all points that are comparable to \(\mathbf{x}\) and greater than or equal to \(\mathbf{x}\) (for the maximum, we have \(S \subseteq \mathbf{x} - K\)).
    \item \(\mathbf{x} \in S\) is the \emph{minimal} element of \(S\) if \(\mathbf{y} \preceq_K \mathbf{x}\) only when \(\mathbf{y} = \mathbf{x}\). The same is true for \emph{maximal}. We can have many different minimal (maximal) elements. The mathematical notation for that is \((\mathbf{x} - K) \cap S = \left\{ \mathbf{x} \right\}\), where \(\mathbf{x} - K\) denotes all points that are comparable to \(\mathbf{x}\) and less than or equal to \(\mathbf{x}\) (for the maximal, we have \((\mathbf{x} + K) \cap S = \left\{ \mathbf{x} \right\}\)).
    \item When \(K = \mathbb{R}_{+}\) and \(S = \mathbb{R}\) (ordinary inequality), the minimal is equal to the minimum and the maximal is equal to the maximum.
\end{itemize} \\
\hline
Dual cone:
\begin{itemize}
    \item \(K^* = \left\{ \mathbf{y}\mid \mathbf{x}^\mathsf{T}\mathbf{y} \geq 0, \;\forall\; \mathbf{x} \in K \right\}\)
\end{itemize} & \vspace{-3.5ex} \begin{itemize}[leftmargin=*]
    \item \(K^*\) is a cone, and it is convex even when the original cone \(K\) is nonconvex.
    \item \(K^*\) has the following properties:
    \begin{itemize}[label={$\triangleright$}]
        \item \(K^*\) is closed and convex.
        \item \(K_1 \subseteq K_2\) implies \(K_1^* \subseteq K_2^*\).
        \item If \(K\) has a nonempty interior, then \(K^*\) is pointed.
        \item If the closure of \(K\) is pointed then \(K^*\) has a nonempty interior.
        \item \(K^{**}\) is the closure of the convex hull of \(K\). Hence, if \(K\) is convex and closed, \(K^{**}=K\).
    \end{itemize}
\end{itemize} \\
\hline
Polyhedra:
\begin{itemize}[leftmargin=*]
    \item $\mathcal{P} = \left\{ \mathbf{x} \mid \mathbf{a}_j^\trans \mathbf{x} \leq b_j, j=1, \dots, m, \mathbf{a}_j^\trans \mathbf{x} = d_j, j=1,\cdots, p  \right\}$
    \item \(\mathcal{P} = \left\{ \mathbf{x} \mid \mathbf{Ax} \preceq \mathbf{b}, \mathbf{Cx} = \mathbf{d} \right\}\), where \(\mathbf{A} = \begin{bmatrix}
            \mathbf{a}_1 & \mathbf{a}_2 & \dots & \mathbf{a}_m
        \end{bmatrix}^\trans\) and \(\mathbf{C} = \begin{bmatrix}
            \mathbf{c}_1 & \mathbf{c}_2 & \dots & \mathbf{c}_m
        \end{bmatrix}^\trans\)
\end{itemize} & \vspace{-3.5ex}
\begin{itemize}[leftmargin=*]
    \item The polyhedron may or may not be an infinite set.
    \item Polyhedron is the result of the intersection of \(m\) halfspaces and \(p\) hyperplanes.
    \item Subspaces, hyperplanes, lines, rays line segments, and
    halfspaces are all polyhedra.
    \item The \emph{nonnegative orthant}, \(\mathbb{R}_{+}^{n} = \left\{ \mathbf{x} \in \mathbb{R}^n \mid x_i \leq 0 \textnormal{ for } i=1,\dots n \right\} = \left\{ \mathbf{x} \in \mathbb{R}^n \mid \mathbf{Ix}  \succeq \mathbf{0}\right\}\), is a special polyhedron.
\end{itemize}\\
\hline
Simplex:
\begin{itemize}[leftmargin=*]
    \item \(\mathcal{S} = \textnormal{conv }\left\{ \mathbf{v}_m \right\}_{m=0}^{k} = \left\{ \sum_{i=0}^{k} \theta_i \mathbf{v}_i \mid \mathbf{0} \preceq \boldsymbol{\thetaup} \preceq \mathbf{1}, \mathbf{1}^\trans \boldsymbol{\thetaup} = 1 \right\}\)
    \item \(\mathcal{S} = \left\{ \mathbf{x} \mid \mathbf{x} = \mathbf{v}_0 + \mathbf{V} \boldsymbol{\thetaup} \right\}\), where \(\mathbf{V} = \begin{bmatrix}
        \mathbf{v}_1 - \mathbf{v}_0 & \dots & \mathbf{v}_n - \mathbf{v}_0
    \end{bmatrix} \in \mathbb{R}^{n \times k}\)
    \item \(\mathcal{S} = \{ \mathbf{x} \mid \underbrace{\mathbf{A}_1 \mathbf{x} \preceq \mathbf{A}_1 \mathbf{v}_0, \, \mathbf{1}^\trans \mathbf{A}_1 \mathbf{x} \leq 1 + \mathbf{1}^\trans\mathbf{A}_1 \mathbf{v}_0}_{\textnormal{Linear inequalities in } x}, \, \underbrace{\mathbf{A}_2 \mathbf{x} = \mathbf{A}_2 \mathbf{v}_0}_{\substack{\text{\textnormal{Linear equalities}} \\\textnormal{in } x}} \}\) (Polyhedra form), where \(\mathbf{A} = \begin{bmatrix}
        \mathbf{A}_1 \\ \mathbf{A}_2
    \end{bmatrix}\) and \(\mathbf{AV} = \begin{bmatrix}
        \mathbf{I}_{k\times k}\\
        \mathbf{0}_{n-k \times n-k}
    \end{bmatrix}\)
\end{itemize} & \vspace{-3.5ex}
\begin{itemize}[leftmargin=*]
    \item Simplexes are a subfamily of the polyhedra set.
    \item Also called k-dimensional Simplex in \(\mathbb{R}^{n}\).
    \item The set \(\left\{ \mathbf{v}_m \right\}_{m=0}^{k}\) is a affinely independent, which means \(\left\{ \mathbf{v}_1-\mathbf{v}_0, \dots, \mathbf{v}_k-\mathbf{v}_0 \right\}\) are linearly independent.
    \item \(\mathbf{V} \in \mathbb{R}^{n\times k}\) is a full-rank tall matrix, i.e., \(\textnormal{rank}(\mathbf{V}) = k\). All its column vectors are independent. The matrix \(\mathbf{A}\) is its left pseudoinverse.
\end{itemize}\\
\hline
\end{tabularx}
% \begin{tabular}{||c | c | c||}
%     \hline
%     \multicolumn{3}{|c|}{Functions} \\
%     \hline
%     Function & Convex? & Proof \\ [0.5ex] 
%     \hline\hline
%     $\mathbf{y} = \max(f_1, f_2)$ & Yes, if $f_1$ and $f_2$ are convex functions & \\
%     \hline
%     $\mathbf{y} = \min(f_1, f_2)$ & Not always & \\
%     \hline
%     $y = \mathbf{c}^\trans \mathbf{x}$ (linear function) & Yes (but not strictly convex) & \\
%     \hline
%     $y = \norm{\mathbf{x}}_p $ (p-norm) & Yes (for any $p \in \mathbb{N}_+$) & $\norm{\theta\mathbf{x} + (1-\theta)\mathbf{y}} \leq \theta\norm{\mathbf{x}} + (1-\theta)\norm{\mathbf{y}}$ (triangular inequality) \\
%     \hline
%     $f(g(\mathbf{x}))$ & Yes, if $f, g$ are convex & \\
%     \hline
% \end{tabular}
    \begin{tabularx}{\textwidth}{|>{\setlength\hsize{1\hsize}\setlength\linewidth{\hsize}}X|>{\setlength\hsize{.9\hsize}\setlength\linewidth{\hsize}}X|>{\setlength\hsize{1.1\hsize}\setlength\linewidth{\hsize}}X|}%{| >{\hsize=.5\hsize}X | >{\hsize=1.5\hsize}X |}
        \hline
        \multicolumn{3}{|c|}{Functions (or operators) and their implications regarding convexity} \\
        \hline
        \multicolumn{1}{|c|}{Function} & \multicolumn{1}{|c|}{Convex?} & \multicolumn{1}{|c|}{Comments} \\
        \hline
        Union: $C = A \cup B $ & Not in most of the cases. & \\
        \hline
        Intersection: $C = A \cap B $ & Yes, if $A$ and $B$ are convex sets. & \\
        \hline
        Convex function: \(f: \mathbb{R}^{n} \rightarrow \mathbb{R}\)
        \begin{itemize}[leftmargin=*]
            \item \(f(\theta\mathbf{x}+(1-\theta)\mathbf{y}) \leq \theta f(\mathbf{x}) + (1-\theta)f(\mathbf{y})\), where \(0\leq\theta\leq 1\).
            \item \(\textnormal{dom } f\subseteq \mathbb{R}^{n}\) shall be a convex set to \(f\) be a convex function.
        \end{itemize} & Yes. & \vspace{-3.5ex}
        \begin{itemize}[leftmargin=*]
            \item Graphically, the line segment between \((\mathbf{x}, f(\mathbf{x}))\) and \((\mathbf{y}, f(\mathbf{y}))\) lies always above the graph \(f\).
            \item In terms of sets, a function is convex iff a line segment within \(\textnormal{dom } f\), which is a convex set, gives an image set that is also convex.
            \item \(\textnormal{dom} f\) is convex iff all points for any line segment within \(\textnormal{dom } f\) belong to it.
            \item \emph{First-order condition}: \(f\) is convex iff \(\textnormal{dom } f\) is convex and \(f(\mathbf{y}) \geq f(\mathbf{x}) + \nabla f (\mathbf{x})^\mathsf{T} (\mathbf{y} - \mathbf{x}),\forall\;\mathbf{x},\mathbf{y} \in \textnormal{dom } f, \mathbf{x}\neq\mathbf{y}\), being \(\nabla f (\mathbf{x})\) the gradient vector. This inequation says that the first-order Taylor approximation is a \emph{underestimator} for convex functions. The first-order condition requires that \(f\) is differentiable.
            \item If \(\nabla f(\mathbf{x}) = \mathbf{0}\), then \(f(\mathbf{y})\geq f(\mathbf{x}),\forall\;\mathbf{y}\in\textnormal{dom }f\) and \(\mathbf{x}\) is a global minimum.
            \item \emph{Second-order condition}: \(f\) is convex iff \(\textnormal{dom } f\) is convex and \(\mathbf{H}\succeq \mathbf{0}\), that is, the Hessian matrix \(\mathbf{H}\) is a positive semidefinite matrix. It means that the graphic of the curvature has a positive (upward) curvature at \(\mathbf{x}\). It is important to note that, if \(\mathbf{H}\succ\mathbf{0}, \forall\; \mathbf{x} \in \textnormal{dom }f\), then \(f\) is strictly convex. But is \(f\) is strictly convex, not necessarily that \(\mathbf{H}\succ\mathbf{0},\forall\; \mathbf{x} \in \textnormal{dom }f\). Therefore, strict convexity can only be partially characterized.
        \end{itemize}\\
        \hline
        Affine function \(f: \mathbb{R}^n \rightarrow \mathbb{R}^m\)
        \begin{itemize}[leftmargin=*]
            \item $f(\mathbf{x}) = \mathbf{Ax} + \mathbf{b}$, where \(\mathbf{A} \in \mathbb{R}^{m\times n}, \mathbf{b} \in \mathbb{R}^{m}, \mathbf{x} \in \mathbb{R}^{n}\)
        \end{itemize} & Yes, if the domain \(S \subseteq \mathbb{R}^{n}\) is a convex set, then its image \(f(S) = \left\{ f(\mathbf{x})|\mathbf{x}\in S \right\} \subseteq \mathbb{R}^{m}\) is also convex. & \vspace{-3.5ex} \begin{itemize}[leftmargin=*]
            \item The affine function, \(f(\mathbf{x}) = \mathbf{Ax} + \mathbf{b}\), is a broader category that encompasses the linear function, \(f(\mathbf{x}) = \mathbf{Ax}\). The linear function has its origin fixed at \(\mathbf{0}\) after the transformation, whereas the affine function does not necessarily have it (when not, this makes the affine function nonlinear). Graphically, we can think of an affine function as a linear transformation plus a shift from the origin of \(\mathbf{b}\).
            \item Similarly, the inverse image of \(C\), \(f^{-1}(C) = \left\{ \mathbf{x} \mid f(\mathbf{x}) \in C \right\}\), is also convex.
            \item The \emph{linear matrix inequality} (LMI), \(\mathbf{A}(\mathbf{x}) = x_1\mathbf{A}_1 + \dots + x_n\mathbf{A}_n \preceq \mathbf{B}\), is a special case of affine function. In other words, \(f(S) = \left\{ \mathbf{x} \mid \mathbf{A}(\mathbf{x}) \preceq \mathbf{B} \right\}\) is a convex set if \(S\) is convex. Many optimization problems can be formulated as LMI problems and solved optimally.
        \end{itemize} \\
        \hline
        Exponential function \(f: \mathbb{R} \rightarrow \mathbb{R}\)
        \begin{itemize}[leftmargin=*]
            \item \(f(x)=e^{ax} \in \mathbb{R}\), where \(a \in \mathbb{R}\)
        \end{itemize} & Yes. & \\
        \hline
        Quadratic function \(f: \mathbb{R}^{n} \rightarrow \mathbb{R}\)
        \begin{itemize}[leftmargin=*]
            \item \(f(\mathbf{x}) = a \mathbf{x}^\mathsf{T}\mathbf{P} \mathbf{x} + \mathbf{p}^\mathsf{T} \mathbf{x} + r \in \mathbb{R}\), where \(\mathbf{x},\mathbf{p} \in \mathbb{R}^{n}, \mathbf{P} \in \mathbb{R}^{n\times n}\), and \(a,b \in \mathbb{R}\)
        \end{itemize} & It depends on the matrix \(\mathbf{P}\): \begin{itemize}[leftmargin=*]
            \item \(f\) is convex iff \(\mathbf{P} \succeq \mathbf{0}\).
            \item \(f\) is strictly convex iff \(\mathbf{P} \succ \mathbf{0}\).
            \item \(f\) is concave iff \(\mathbf{P} \preceq \mathbf{0}\).
            \item \(f\) is strictly concave iff \(\mathbf{P} \prec \mathbf{0}\).
        \end{itemize} & \\
        \hline
        Power function \(f: \mathbb{R}_{++} \rightarrow \mathbb{R} \) \begin{itemize}[leftmargin=*]
            \item \(f(x) = x^{a}\)
        \end{itemize} & It depends on \(a\) \begin{itemize}[leftmargin=*]
            \item \(f\) is convex iff \(a\geq 1\) or \(a\leq 0\).
            \item \(f\) is concave iff \(0\leq a \leq 1\).
        \end{itemize} & \\
        \hline
        Power of absolute value: \(f: \mathbb{R} \rightarrow \mathbb{R}\) \begin{itemize}[leftmargin=*]
            \item \(f(x) = \abs{x}^p\), where \(p\leq 1\).
        \end{itemize} & Yes. & \\
        \hline
        Logarithm function: \(f: \mathbb{R}_{++} \rightarrow \mathbb{R}\) \begin{itemize}
            \item \(f(x) = \log x\)
        \end{itemize} & Yes. & \\
        \hline
        Perspective function \(f: \mathbb{R}^{n} \times \mathbb{R}_{++} \rightarrow \mathbb{R}^{n}\)
        \begin{itemize}[leftmargin=*]
            \item \(f(\mathbf{x}, t) = \mathbf{x}/t\), where \(\mathbf{x} \in \mathbb{R}^{n}, t \in \mathbb{R}\).
        \end{itemize} & Yes, if the domain \(S \subseteq \textnormal{dom } f\) is a convex set, then its image \(f(S) = \left\{ f(\mathbf{x})|\mathbf{x}\in S \right\} \subseteq \mathbb{R}^{n}\) is also convex. & \vspace{-3.5ex} \begin{itemize}[leftmargin=*]
            \item The perspective function decreases the dimension of the function domain since \(\textnormal{dim}(\textnormal{dom } f) = n+1\).
            \item Its effect is similar to the camera zoom.
            \item The inverse image is also convex, that is, if \(C \subseteq \mathbb{R}^{n}\) is convex, then \(f^{-1}(C) = \left\{ (\mathbf{x}, t) \in \mathbb{R}^{n+1} \mid \mathbf{x}/t \in C, t>0 \right\}\) is also convex.
        \end{itemize} \\
        \hline
        Projective (or linear-fractional) function, \(f: \mathbb{R}^{n} \rightarrow \mathbb{R}^{m}\)
        \begin{itemize}[leftmargin=*]
            \item \(f = p \circ g\), i.e., \(f(\mathbf{x}) = (p\circ g)(\mathbf{x}) = p(g(\mathbf{x}))\), where
                \begin{itemize}[label={$\triangleright$}]
                    \item \(g: \mathbb{R}^{n} \rightarrow \mathbb{R}^{m+1}\) is an affine function given by \(g(\mathbf{x}) = \begin{bmatrix}
                        \mathbf{A}\\
                        \mathbf{c}^\mathsf{T}
                    \end{bmatrix} \mathbf{x} + \begin{bmatrix}
                        \mathbf{b} \\
                        d
                    \end{bmatrix}\), being \(\mathbf{A}\in \mathbb{R}^{m \times n}, \mathbf{b} \in \mathbb{R}^{m}, \mathbf{c} \in \mathbb{R}^{n}\), and \(d \in \mathbb{R}\).
                    \item \(p: \mathbb{R}^{m+1} \rightarrow \mathbb{R}^{m}\) is the perspective function.
                \end{itemize}
            \item \(f(\mathbf{x}) = \mathcal{P}^{-1}(\mathbf{Q}\mathcal{P}(\mathbf{x}))\)
                \begin{itemize}[label={$\triangleright$}]
                    \item \(\mathcal{P}(\mathbf{x}) = \left\{ (t\mathbf{x}, t) \mid t \geq 0 \right\} \subset \mathbb{R}^{n+1}\)
                    \item \(\mathbf{Q} = \begin{bmatrix}
                        \mathbf{A} & \mathbf{b} \\
                        \mathbf{c}^\mathsf{T} & d
                    \end{bmatrix} \in \mathbb{R}^{(m+1)\times(n+1)}\)
                \end{itemize}
        \end{itemize} & Yes, if the domain \(S \subseteq \textnormal{dom } f\) is a convex set, then its image \(f(S) = \left\{ f(\mathbf{x})|\mathbf{x}\in S \right\} \subseteq \mathbb{R}^{m}\) is also convex. & \vspace{-3.5ex} \begin{itemize}[leftmargin=*]
            \item The linear and affine functions are special cases of the linear-fractional function.
            \item \(\textnormal{dom } f = \left\{ \mathbf{x} \in \mathbb{R}^{n} \mid \mathbf{c}^\mathsf{T} \mathbf{x} + d > 0 \right\}\)
            \item \(\mathcal{P}(\mathbf{x}) \subset \mathbb{R}^{n+1}\) is a ray set that begins at the origin and its last component takes only positive values. For each \(\mathbf{x} \in \textnormal{dom } f\), it is associated a ray set in \(\mathbb{R}^{n+1}\) in this form. This (projective) correspondence between all points in \(\textnormal{dom } f\) and their respective sets \(\mathcal{P}\) is a biunivocal mapping.
            \item The linear transformation \(\mathbf{Q}\) acts on these rays, forming another set of rays.
            \item Finally we take the inverse projective transformation to recover \(f(\mathbf{x})\).
        \end{itemize}\\
        \hline
    \end{tabularx}
\end{table}

% \begin{itemize}
%     \item All convex set is quasiconvex, but not all quasiconvex is convex.
%     \item It is possible to solve quasiconvex functions, even if it is not convex (see Algorithm 4.1). But not all quasiconvex functions that are nonconvex can be solved(?).
%     \item Superlevel set (a set) 3.3.6, all convex functions have all convex \(\alpha\) sub-level set, but not all functions that have convex \(\alpha\) sub-level set are convex (see slide 3.11).
% \end{itemize}

\end{document}