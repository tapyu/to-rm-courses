\documentclass{article}
\usepackage[margin=2mm,textwidth=155mm,paperheight=170cm,paperwidth=30cm]{geometry}

% items, table and figs
\usepackage{enumitem}
\usepackage{tabularx}
\usepackage{tikz}
\usepackage{standalone} % place tikz environments or other material in own source files
\usetikzlibrary{positioning}
\usetikzlibrary{arrows}

\newcolumntype{A}{>{\setlength\hsize{1\hsize}\setlength\linewidth{\hsize}}X}
\newcolumntype{B}{>{\setlength\hsize{.5\hsize}\setlength\linewidth{\hsize}\centering\arraybackslash}X}
\newcolumntype{C}{>{\setlength\hsize{1\hsize}\setlength\linewidth{\hsize}}X}

% commands
\usepackage{xparse} % make command behave differently depending on the number of arguments

% math
\usepackage{amsmath}
\usepackage{amsfonts}
\usepackage{amssymb}
\usepackage{newtxmath} % for Greek variants (bold, nonitalic, etc...)

% commands
\newcommand{\trans}{\mathsf{T}}
\newcommand{\hermit}{\mathsf{H}}
\newcommand{\norm}[1]{\left\lVert#1\right\rVert}
\newcommand{\abs}[1]{\left\lvert#1\right\rvert}
\newcommand{\eval}[2]{\left.#1\right|_{#2}} % x|_x=a -> evaluation bar
\newcommand{\dom}[1]{\ensuremath{\textnormal{dom}\left(#1\right)}} % domain of the function f
\newcommand{\tr}[1]{\ensuremath{\textnormal{tr}\left(#1\right)}} % trace
\newcommand{\adj}[1]{\ensuremath{\textnormal{adj}\left(#1\right)}} % adjugate matrix
\newcommand{\intersection}{\bigcap\limits}
\NewDocumentCommand{\diff}{o}{\mathop{}\!\mathrm{d}\IfValueT{#1}{^{#1}}}

\begin{document}

\begin{table}[ht!]
\begin{tabularx}{\textwidth}{|A|C|}
\hline
\multicolumn{2}{|c|}{Convex sets}\\
\hline
\multicolumn{1}{|c|}{Set} & \multicolumn{1}{|c|}{Comments}\\
\hline
Convex hull:
\begin{itemize}[leftmargin=*]
\item $\textnormal{conv } C = \left\{ \sum_{i=1}^{k} \theta_i\mathbf{x}_i \mid \mathbf{x}_i \in C \textnormal{ for } i=1,\cdots, k, \mathbf{0} \preceq \boldsymbol{\thetaup} \preceq \mathbf{1}, \mathbf{1}^\trans\boldsymbol{\thetaup} = 1  \right\}$
\end{itemize} & \vspace{-3.5ex}
\begin{itemize}[leftmargin=*]
    \item $\textnormal{conv } C$ is the smallest convex set that contains $C$.
    \item $\textnormal{conv } C$ is a finite set as long as $C$ is also finite.
\end{itemize}\\
\hline
Affine hull:
\begin{itemize}[leftmargin=*]
    \item $\textnormal{aff } C = \left\{ \sum_{i=1}^{k} \theta_i\mathbf{x}_i \mid \mathbf{x}_i \in C \textnormal{ for } i=1,\cdots, k, \mathbf{1}^\trans\boldsymbol{\thetaup} = 1  \right\}$
\end{itemize} & \vspace{-3.5ex}
\begin{itemize}[leftmargin=*]
    \item $\textnormal{aff } C$ is the smallest affine set that contains $C$.
    \item $\textnormal{aff } C$ is always an infinite set. If $\textnormal{aff } C$ contains the origin, it is also a subspace.
    \item Different from the convex set, \(\theta_i\) is not restricted between 0 and 1
\end{itemize}\\
\hline
Conic hull:
\begin{itemize}[leftmargin=*]
    \item $A = \left\{ \sum_{i=1}^{k} \theta_i\mathbf{x}_i \mid \mathbf{x}_i \in C, \theta_i \geq 0 \textnormal{ for } i=1,\cdots, k \right\}$
\end{itemize} & \vspace{-3.5ex}
\begin{itemize}[leftmargin=*]
    \item $A$ is the smallest convex conic that contains $C$.
    \item Different from the convex and affine sets, \(\theta_i\) does not need to sum up 1.
\end{itemize}\\
\hline
Ray:
\begin{itemize}[leftmargin=*]
    \item \(\mathcal{R} = \left\{ \mathbf{x}_0 + \theta \mathbf{v} \mid \theta \geq 0 \right\}\)
\end{itemize} & \vspace{-3.5ex} \begin{itemize}[leftmargin=*]
    \item The ray is an infinite set that begins in \(\mathbf{x}_0\) and extends infinitely in direction of \(\mathbf{v}\). In other words, it has a beginning, but it has no end.
    \item The ray becomes a convex cone if \(\mathbf{x}_0 = \mathbf{0}.\)
\end{itemize} \\
\hline
Hyperplane:
\begin{itemize}[leftmargin=*]
    \item \( \mathcal{H} = \left\{ \mathbf{x} \mid \mathbf{a}^\trans \mathbf{x} = b \right\}\)
    \item \(\mathcal{H} = \left\{ \mathbf{x} \mid \mathbf{a}^\trans (\mathbf{x} - \mathbf{x}_{0}) = \mathbf{0} \right\}\)
    \item \(\mathcal{H} = \mathbf{x}_0 + a^{\perp} \)
\end{itemize} & \vspace{-3.5ex}
\begin{itemize}[leftmargin=*]
    \item It is an infinite set \(\mathbb{R}^{n-1} \subset \mathbb{R}^{n}\) that divides the space into two halfspaces.
    \item The inner product between \(\mathbf{a}\) and any vector in \(\mathcal{H}\) yields the constant value \(b\).
    \item \(a^{\perp} = \left\{ \mathbf{v} \mid \mathbf{a}^\trans \mathbf{v} = 0 \right\}\) is the infinite set of vectors perpendicular to \(\mathbf{a}\). It passes through the origin.
    \item \(a^{\perp}\) is offset from the origin by \(\mathbf{x}_0\), which is any vector in \(\mathcal{H}\).
\end{itemize} \\
\hline
Halfspaces:
\begin{itemize}[leftmargin=*]
    \item \(\mathcal{H}_{-} = \left\{ \mathbf{x} \mid \mathbf{a}^\trans \mathbf{x} \leq b \right\}\)
    \item \(\mathcal{H}_{+} = \left\{ \mathbf{x} \mid \mathbf{a}^\trans \mathbf{x} \geq b \right\}\)
\end{itemize} & \vspace{-3.5ex}
\begin{itemize}[leftmargin=*]
    \item They are infinite sets of the parts divided by \(\mathcal{H}\).
\end{itemize}\\
\hline
Euclidean ball:
\begin{itemize}[leftmargin=*]
    \item \(B(\mathbf{x}_c, r) = \left\{ \mathbf{x} \mid \norm{\mathbf{x}-\mathbf{x}_c} \leq r \right\}\)
    \item \(B(\mathbf{x}_c, r) = \left\{ \mathbf{x} \mid \left( \mathbf{x}-\mathbf{x}_c \right)^\trans \left( \mathbf{x}-\mathbf{x}_c \right) \leq r^2 \right\}\)
    \item \(B(\mathbf{x}_c, r) = \left\{ \mathbf{x}_c + r \norm{\mathbf{u}} \mid \norm{\mathbf{u}} \leq 1 \right\}\)
\end{itemize} & \vspace{-3.5ex}
\begin{itemize}[leftmargin=*]
    \item \(B(\mathbf{x}_c, r)\) is a finite set as long as \(r < \infty\).
    \item \(\mathbf{x}_c\) is the center of the ball.
    \item \(r\) is its radius.
\end{itemize}\\
\hline
Ellipsoid:
\begin{itemize}[leftmargin=*]
    \item \(\mathcal{E} = \left\{ \mathbf{x} \mid (\mathbf{x}-\mathbf{x}_c)^\trans\mathbf{P}^{-1}(\mathbf{x}-\mathbf{x}_c) \leq 1 \right\}\)
    \item \(\mathcal{E} = \left\{ \mathbf{x}_{c} + \mathbf{P}^{1/2}\mathbf{u} \mid \norm{\mathbf{u}} \leq 1 \right\}\)
\end{itemize} & \vspace{-3.5ex}
\begin{itemize}[leftmargin=*]
    \item \(\mathcal{E}\) is a finite set as long as \(\mathbf{P}\) is a finite matrix.
    \item \(\mathbf{P}\) is symmetric and positive definite, that is, \(\mathbf{P}=\mathbf{P}^\trans \succ \mathbf{0}\). It determines how far the ellipsoid extends in every direction from \(\mathbf{x}_c\).
    \item \(\mathbf{x}_{c}\) is the center of the ellipsoid.
    \item The lengths of the semi-axes are given by \(\sqrt{\lambda_i}\).
    \item When \(\mathbf{P}^{1/2} \succeq \mathbf{0}\) but singular, we say that \(\mathcal{E}\) is a degenerated ellipsoid (degenerated ellipsoids are also convex).
\end{itemize}\\
\hline
Norm cone:
\begin{itemize}[leftmargin=*]
    \item \(C = \left\{ (x_1, x_2, \cdots, x_n, t) \in \mathbb{R}^{n+1} \mid \mathbf{x} \in \mathbb{R}^{n}, \norm{\mathbf{x}}_{p} \leq t \right\} \subseteq \mathbb{R}^{n+1}\)
\end{itemize} & \vspace{-3.5ex}
\begin{itemize}[leftmargin=*]
    \item Although it is named ``Norm cone'', it is a set, not a scalar.
    \item The cone norm increases the dimension of \(\mathbf{x}\) in 1.
    \item For \(p=2\), it is called the second-order cone, quadratic cone,  Lorentz cone or ice-cream cone.
\end{itemize} \\
\hline
Proper cone: \(K \subset \mathbb{R}^{n}\) is a proper cone when it has the following properties
\begin{itemize}
    \item \(K\) is a convex cone, i.e., \(\alpha K \equiv K, \alpha > 0\).
    \item \(K\) is closed.
    \item \(K\) is solid.
    \item \(K\) is pointed, i.e., \(-K \cap K = \left\{ \mathbf{0} \right\}\).
\end{itemize} & \vspace{-3.5ex} \begin{itemize}[leftmargin=*]
    \item The proper cone \(K\) is used to define the \emph{generalized inequality} (or \emph{partial ordering}) in some set \(S\). For the generalized inequality, one must define both the proper cone \(K\) and the set \(S\).
    \item \(\mathbf{x} \preceq \mathbf{y} \iff \mathbf{y} - \mathbf{x} \in K\) for \(\mathbf{x}, \mathbf{y} \in S\) (generalized inequality)
    \item \(\mathbf{x} \prec \mathbf{y} \iff \mathbf{y} - \mathbf{x} \in \textnormal{int }K\) for \(\mathbf{x}, \mathbf{y} \in S\) (strict generalized inequality).
    \item There are two cases where \(K\) and \(S\) are understood from context and the subscript \(K\) is dropped out:
        \begin{itemize}[label={$\triangleright$}]
            \item When \(S = \mathbb{R}^{n}\) and \(K = \mathbb{R}^{n}_{+}\) (the nonnegative orthant). In this case, \(\mathbf{x} \preceq \mathbf{y}\) means that \(x_i \leq y_i\).
            \item When \(S = \mathcal{S}^{n}\) and \(K = \mathcal{S}^{n}_{+}\) or \(K = \mathcal{S}^{n}_{++}\), where \(\mathcal{S}^{n}\) denotes the set of symmetric \(n\times n\) matrices, \(\mathcal{S}^{n}_{+}\) is the space of the positive semidefinite matrices, and \(\mathcal{S}^{n}_{++}\) is the space of the positive definite matrices. \(\mathcal{S}^{n}_{+}\) is a proper cone in \(\mathcal{S}^{n}\) (??). In this case, the generalized inequality \(\mathbf{Y} \succeq \mathbf{X}\) means that \(\mathbf{Y}-\mathbf{X}\) is a positive semidefinite matrix belonging to the positive semidefinite cone \(\mathcal{S}^{n}_{+}\) in the subspace of symmetric matrices \(\mathcal{S}^{n}\). It is usual to denote \(\mathbf{X} \succ \mathbf{0}\) and \(\mathbf{X} \succeq \mathbf{0}\) to mean than \(\mathbf{X}\) is a positive definite and semidefinite matrix, respectively, where \(\mathbf{0} \in \mathbb{R}^{n\times n}\) is a zero matrix.
        \end{itemize}
    \item Another common usage is when \(S = \mathbb{R}^{n}\) and \(K = \left\{ \mathbf{c} \in \mathbb{R}^{n} \mid c_1 + c_2 t + \dots + c_n t^{n-1} \geq 0, \textnormal{ for } 0\leq t\leq 1 \right\}\). In this case, \(\mathbf{x} \preceq_K \mathbf{y}\) means that \(x_1 + x_2 t + \dots + x_n t^{n-1} \leq y_1 + y_2 t + \dots + y_n t^{n-1}\).
    \item The generalized inequality has the following properties:
        \begin{itemize}[label={$\triangleright$}]
            \item If \(\mathbf{x} \preceq_K \mathbf{y}\) and \(\mathbf{u} \preceq_K \mathbf{v}\), then \(\mathbf{x} + \mathbf{u} \preceq_k \mathbf{y} + \mathbf{v}\) (preserve under addition).
            \item If \(\mathbf{x} \preceq_K \mathbf{y}\) and \(\mathbf{y} \preceq_K \mathbf{z}\), then \(\mathbf{x} \preceq_K \mathbf{z}\) (transitivity).
            \item If \(\mathbf{x}\preceq_K \mathbf{y}\), then \(\alpha\mathbf{x}\preceq_K \mathbf{y}\) for \(\alpha\geq0\) (preserve under nonnegative scaling).
            \item \(\mathbf{x}\preceq_K \mathbf{x}\) (reflexivity).
            \item If \(\mathbf{x}\preceq_K \mathbf{y}\) and \(\mathbf{y}\preceq_K \mathbf{x}\), then \(\mathbf{x} = \mathbf{y}\) (antisymmetric).
            \item If \(\mathbf{x}_i\preceq_K \mathbf{y}_i\), for \(i = 1, 2, \dots\), and \(\mathbf{x}_i \rightarrow \mathbf{x}\) and \(\mathbf{y}_i \rightarrow \mathbf{y}\) as \(i \rightarrow \infty\), then \(\mathbf{x} \preceq_K \mathbf{y}\).
        \end{itemize}
    \item It is called partial ordering because \(\mathbf{x} \nsucceq_K \mathbf{y}\) and \(\mathbf{y} \nsucceq_K \mathbf{x}\) for many \(\mathbf{x}, \mathbf{y} \in S\). When it happens, we say that \(\mathbf{x}\) and \(\mathbf{y}\) are not comparable (this case does not happen in ordinary inequality, \(<\) and \(>\)).
    \item \(\mathbf{x} \in S\) is the \emph{minimum} element of \(S\) with respect to the proper cone \(K\) if \(\mathbf{x} \preceq_K \mathbf{y}, \;\forall\;\mathbf{y} \in S\) (for \emph{maximum}, \(\mathbf{x} \succeq_K \mathbf{y}, \;\forall\;\mathbf{y} \in S\)). It means that \(S \subseteq \mathbf{x} + K\) (for the maximum, \(S \subseteq \mathbf{x} - K\)), where \(\mathbf{x} + K\) denotes the set \(K\) shifted from the origin by \(\mathbf{x}\). Note that any point in \(K+\mathbf{x}\) is comparable with \(\mathbf{x}\) and is greater or equal to \(\mathbf{x}\) in the generalized inequality mean. The set \(S\) does not necessarily have a minimum (maximum), but the minimum (maximum) is unique if it does.
    \item \(\mathbf{x} \in S\) is the \emph{minimal} element of \(S\) with respect to the proper cone \(K\) if \(\mathbf{y} \preceq_K \mathbf{x}\) only when \(\mathbf{y} = \mathbf{x}\) (for the \emph{maximal}, \(\mathbf{y} \succeq_K \mathbf{x}\) only when \(\mathbf{y} = \mathbf{x}\)). It means that \((\mathbf{x} - K) \cap S = \left\{ \mathbf{x} \right\}\) for minimal (for the maximal \((\mathbf{x} + K) \cap S = \left\{ \mathbf{x} \right\}\)), where \(\mathbf{x} - K\) denotes the reflected set \(K\) shift by \(\mathbf{x}\). Note that any point in \(\mathbf{x} - K\) is comparable with \(\mathbf{x}\) and is less than or equal to \(\mathbf{x}\) in the generalized inequality mean. The set \(S\) can have many different minimal (maximal) elements.
    \item When \(K = \mathbb{R}_{+}\) and \(S = \mathbb{R}\) (ordinary inequality), the minimal is equal to the minimum and the maximal is equal to the maximum.
    \item When we say that a scalar-valued function \(f: \mathbb{R}^{n} \rightarrow \mathbb{R}\) is nondecreasing, it means that whenever \(\mathbf{u}\preceq \mathbf{v}\), we have \(\tilde{h}(\mathbf{u})\leq \tilde{h}(\mathbf{v})\). Similar results hold for decreasing, increasing, and nonincreasing scalar functions.
\end{itemize} \\
\hline
Subspace (cone set?) of the symmetric matrices:
\begin{itemize}
    \item \(\mathcal{S}^n = \left\{ \mathbf{X} \in \mathbb{R}^{n\times n} \mid \mathbf{X} = \mathbf{X}^\mathsf{T}\right\}\)
\end{itemize} & \vspace{-3.5ex} \begin{itemize}[leftmargin=*]
    \item The positive semidefinite cone is given by \(\mathcal{S}^n_+ = \left\{ \mathbf{X} \in \mathbb{R}^{n\times n} \mid \mathbf{X} \succeq \mathbf{0} \right\} \subset \mathcal{S}^n\). This is the proper cone used to define the generalized inequalities between matrices, e.g., \(\mathbf{A} \preceq \mathbf{B}\).
    \item The positive definite cone is given by \(\mathcal{S}^n_{++} = \left\{ \mathbf{X} \in \mathbb{R}^{n\times n} \mid \mathbf{X} \succ \mathbf{0} \right\}\subseteq \mathcal{S}^n_+ \). This is the proper cone used to define the generalized inequalities between matrices, e.g., \(\mathbf{A} \prec \mathbf{B}\).
\end{itemize} \\
\hline
Dual cone:
\begin{itemize}
    \item \(K^* = \left\{ \mathbf{y}\mid \mathbf{x}^\mathsf{T}\mathbf{y} \geq 0, \;\forall\; \mathbf{x} \in K \right\}\)
\end{itemize} & \vspace{-3.5ex} \begin{itemize}[leftmargin=*]
    \item \(K^*\) is a cone, and it is convex even when the original cone \(K\) is nonconvex.
    \item \(K^*\) has the following properties:
    \begin{itemize}[label={$\triangleright$}]
        \item \(K^*\) is closed and convex.
        \item \(K_1 \subseteq K_2\) implies \(K_1^* \subseteq K_2^*\).
        \item If \(K\) has a nonempty interior, then \(K^*\) is pointed.
        \item If the closure of \(K\) is pointed then \(K^*\) has a nonempty interior.
        \item \(K^{**}\) is the closure of the convex hull of \(K\). Hence, if \(K\) is convex and closed, \(K^{**}=K\).
    \end{itemize}
\end{itemize} \\
\hline
Polyhedra:
\begin{itemize}[leftmargin=*]
    \item $\mathcal{P} = \left\{ \mathbf{x} \mid \mathbf{a}_j^\trans \mathbf{x} \leq b_j, j=1, \dots, m, \mathbf{a}_j^\trans \mathbf{x} = d_j, j=1,\cdots, p  \right\}$
    \item \(\mathcal{P} = \left\{ \mathbf{x} \mid \mathbf{Ax} \preceq \mathbf{b}, \mathbf{Cx} = \mathbf{d} \right\}\), where \(\mathbf{A} = \begin{bmatrix}
            \mathbf{a}_1 & \mathbf{a}_2 & \dots & \mathbf{a}_m
        \end{bmatrix}^\trans\) and \(\mathbf{C} = \begin{bmatrix}
            \mathbf{c}_1 & \mathbf{c}_2 & \dots & \mathbf{c}_m
        \end{bmatrix}^\trans\)
\end{itemize} & \vspace{-3.5ex}
\begin{itemize}[leftmargin=*]
    \item The polyhedron may or may not be an infinite set.
    \item Polyhedron is the result of the intersection of \(m\) halfspaces and \(p\) hyperplanes.
    \item Subspaces, hyperplanes, lines, rays line segments, and halfspaces are all special cases of polyhedra.
    \item The \emph{nonnegative orthant}, \(\mathbb{R}_{+}^{n} = \left\{ \mathbf{x} \in \mathbb{R}^n \mid x_i \leq 0 \textnormal{ for } i=1,\dots n \right\} = \left\{ \mathbf{x} \in \mathbb{R}^n \mid \mathbf{Ix}  \succeq \mathbf{0}\right\}\), is a special polyhedron.
\end{itemize}\\
\hline
Simplex:
\begin{itemize}[leftmargin=*]
    \item \(S = \textnormal{conv }\left\{ \mathbf{v}_m \right\}_{m=0}^{k} = \left\{ \sum_{i=0}^{k} \theta_i \mathbf{v}_i \mid \mathbf{0} \preceq \boldsymbol{\thetaup} \preceq \mathbf{1}, \mathbf{1}^\trans \boldsymbol{\thetaup} = 1 \right\}\)
    \item \(S = \left\{ \mathbf{x} \mid \mathbf{x} = \mathbf{v}_0 + \mathbf{V} \boldsymbol{\thetaup} \right\}\), where \(\mathbf{V} = \begin{bmatrix}
        \mathbf{v}_1 - \mathbf{v}_0 & \dots & \mathbf{v}_n - \mathbf{v}_0
    \end{bmatrix} \in \mathbb{R}^{n \times k}\)
    \item \(S = \{ \mathbf{x} \mid \underbrace{\mathbf{A}_1 \mathbf{x} \preceq \mathbf{A}_1 \mathbf{v}_0, \, \mathbf{1}^\trans \mathbf{A}_1 \mathbf{x} \leq 1 + \mathbf{1}^\trans\mathbf{A}_1 \mathbf{v}_0}_{\textnormal{Linear inequalities in } x}, \, \underbrace{\mathbf{A}_2 \mathbf{x} = \mathbf{A}_2 \mathbf{v}_0}_{\substack{\text{\textnormal{Linear equalities}} \\\textnormal{in } x}} \}\) (Polyhedra form), where \(\mathbf{A} = \begin{bmatrix}
        \mathbf{A}_1 \\ \mathbf{A}_2
    \end{bmatrix}\) and \(\mathbf{AV} = \begin{bmatrix}
        \mathbf{I}_{k\times k}\\
        \mathbf{0}_{n-k \times n-k}
    \end{bmatrix}\)
\end{itemize} & \vspace{-3.5ex}
\begin{itemize}[leftmargin=*]
    \item Simplexes are a subfamily of the polyhedra set.
    \item Also called k-dimensional Simplex in \(\mathbb{R}^{n}\).
    \item The set \(\left\{ \mathbf{v}_m \right\}_{m=0}^{k}\) is a affinely independent, which means \(\left\{ \mathbf{v}_1-\mathbf{v}_0, \dots, \mathbf{v}_k-\mathbf{v}_0 \right\}\) are linearly independent.
    \item \(\mathbf{V} \in \mathbb{R}^{n\times k}\) is a full-rank tall matrix, i.e., \(\textnormal{rank}(\mathbf{V}) = k\). All its column vectors are independent. The matrix \(\mathbf{A}\) is its left pseudoinverse.
\end{itemize}\\
\hline
\(\alpha\)-sublevel set:
\begin{itemize}[leftmargin=*]
    \item \(C_\alpha = \{\mathbf{x} \in \dom{f} \mid f(\mathbf{x}) \leq \alpha\}\) (regarding convexity), where \(f: \mathbb{R}^{n} \rightarrow \mathbb{R}\)
    \item \(C_\alpha = \{\mathbf{x} \in \dom{f} \mid f(\mathbf{x}) \geq \alpha\}\) (regarding concavity), where \(f: \mathbb{R}^{n} \rightarrow \mathbb{R}\)
\end{itemize} & \vspace{-3.5ex}
\begin{itemize}[leftmargin=*]
    \item If \(f\) is a convex (concave) function, then sublevel sets of \(f\) are convexes (concaves) for any \(\alpha\in \mathbb{R}\).
    \item The converse is not true: a function can have all its sublevel set convex and not be a convex function.
    \item \(C_\alpha \subseteq \dom{f}\)
\end{itemize}\\
\hline
\end{tabularx}
    \begin{tabularx}{\textwidth}{|>{\setlength\hsize{1\hsize}\setlength\linewidth{\hsize}}X|>{\setlength\hsize{.9\hsize}\setlength\linewidth{\hsize}}X|>{\setlength\hsize{1.1\hsize}\setlength\linewidth{\hsize}}X|}%{| >{\hsize=.5\hsize}X | >{\hsize=1.5\hsize}X |}
        \hline
        \multicolumn{3}{|c|}{Functions (or operators) and their implications regarding convexity} \\
        \hline
        \multicolumn{1}{|c|}{Function} & \multicolumn{1}{|c|}{Convexity} & \multicolumn{1}{|c|}{Comments} \\
        \hline
        Union: $C = A \cup B $ & Not in most of the cases. & \\
        \hline
        Intersection: $C = A \cap B $ & Yes, if $A$ and $B$ are convex sets. & \\
        \hline
        Convex function: \(f: \dom{f} \rightarrow \mathbb{R}\)
        \begin{itemize}[leftmargin=*]
            \item \(f(\theta\mathbf{x}+(1-\theta)\mathbf{y}) \leq \theta f(\mathbf{x}) + (1-\theta)f(\mathbf{y})\), where \(0\leq\theta\leq 1\).
            \item \(\dom{f}\) shall be a convex set to \(f\) be considered a convex function.
        \end{itemize} & Yes. & \vspace{-3.5ex}
        \begin{itemize}[leftmargin=*]
            \item Graphically, the line segment between \((\mathbf{x}, f(\mathbf{x}))\) and \((\mathbf{y}, f(\mathbf{y}))\) lies always above the graph \(f\).
            \item In terms of sets, a function is convex iff a line segment within \(\dom{f}\), which is a convex set, gives an image set that is also convex.
            \item \(\textnormal{dom} f\) is convex iff all points for any line segment within \(\dom{f}\) belong to it.
            \item \emph{First-order condition}: \(f\) is convex iff \(\dom{f}\) is convex and \(f(\mathbf{y}) \geq f(\mathbf{x}) + \nabla f (\mathbf{x})^\mathsf{T} (\mathbf{y} - \mathbf{x}),\forall\;\mathbf{x},\mathbf{y} \in \dom{f}, \mathbf{x}\neq\mathbf{y}\), being \(\nabla f (\mathbf{x})\) the gradient vector. This inequation says that the first-order Taylor approximation is a \emph{underestimator} for convex functions. The first-order condition requires that \(f\) is differentiable.
            \item If \(\nabla f(\mathbf{x}) = \mathbf{0}\), then \(f(\mathbf{y})\geq f(\mathbf{x}),\forall\;\mathbf{y}\in\dom{f}\) and \(\mathbf{x}\) is a global minimum.
            \item \emph{Second-order condition}: \(f\) is convex iff \(\dom{f}\) is convex and \(\mathbf{H}\succeq \mathbf{0}\), that is, the Hessian matrix \(\mathbf{H}\) is a positive semidefinite matrix. It means that the graphic of the curvature has a positive (upward) curvature at \(\mathbf{x}\). It is important to note that, if \(\mathbf{H}\succ\mathbf{0}, \forall\; \mathbf{x} \in \dom{f}\), then \(f\) is strictly convex. But is \(f\) is strictly convex, not necessarily that \(\mathbf{H}\succ\mathbf{0},\forall\; \mathbf{x} \in \dom{f}\). Therefore, strict convexity can only be partially characterized.
        \end{itemize}\\
        \hline
        Convex function: \(f: \dom{f} \rightarrow \mathbb{R}\)
        \begin{itemize}[leftmargin=*]
            \item \(f(\theta\mathbf{x}+(1-\theta)\mathbf{y}) \geq \theta f(\mathbf{x}) + (1-\theta)f(\mathbf{y})\), where \(0\leq\theta\leq 1\).
            \item \(\dom{f}\) shall be a concave set to \(f\) be considered a concave function.
        \end{itemize} & Concave & yes \\
        \hline
        Affine function \(f: \mathbb{R}^n \rightarrow \mathbb{R}^m\)
        \begin{itemize}[leftmargin=*]
            \item $f(\mathbf{x}) = \mathbf{Ax} + \mathbf{b}$, where \(\mathbf{A} \in \mathbb{R}^{m\times n}, \mathbf{b} \in \mathbb{R}^{m}, \mathbf{x} \in \mathbb{R}^{n}\)
        \end{itemize} & Yes, if the domain \(S \subseteq \mathbb{R}^{n}\) is a convex set, then its image \(f(S) = \left\{ f(\mathbf{x})|\mathbf{x}\in S \right\} \subseteq \mathbb{R}^{m}\) is also convex. & \vspace{-3.5ex} \begin{itemize}[leftmargin=*]
            \item $f$ is an affine function iff $f(\theta \mathbf{x} + (1-\theta)\mathbf{y}) = \theta f(\mathbf{x}) + (1-\theta)f(\mathbf{y})$, where \(\theta \in \mathbb{R}\).
            \item The affine function, \(f(\mathbf{x}) = \mathbf{Ax} + \mathbf{b}\), is a broader category that encompasses the linear function, \(f(\mathbf{x}) = \mathbf{Ax}\). The linear function has its origin fixed at \(\mathbf{0}\) after the transformation, whereas the affine function does not necessarily have it (when not, this makes the affine function nonlinear). Graphically, we can think of an affine function as a linear transformation plus a shift from the origin of \(\mathbf{b}\).
            \item A special case of the linear function is when \(\mathbf{A} = \mathbf{c}^\mathsf{T}\). In this case, we have \(f(\mathbf{x}) = \mathbf{c}^\mathsf{T}\mathbf{x}\), which is the inner product between the vector \(\mathbf{c}\) and \(\mathbf{x}\).
            \item The inverse image of \(C\), \(f^{-1}(C) = \left\{ \mathbf{x} \mid f(\mathbf{x}) \in C \right\}\), is also convex.
            \item The \emph{linear matrix inequality} (LMI), \(\mathbf{A}(\mathbf{x}) = x_1\mathbf{A}_1 + \dots + x_n\mathbf{A}_n \preceq \mathbf{B}\), is a special case of affine function. In other words, \(f(S) = \left\{ \mathbf{x} \mid \mathbf{A}(\mathbf{x}) \preceq \mathbf{B} \right\}\) is a convex set if \(S\) is convex. Many optimization problems can be formulated as LMI problems and solved optimally.
        \end{itemize} \\
        \hline
        Constant function \(f: \mathbb{R} \rightarrow \mathbb{R}\)
        \begin{itemize}[leftmargin=*]
            \item \(f(\theta \mathbf{x} + (1-\theta)\mathbf{y}) = f(\mathbf{x})\), where \(\theta \in \mathbb{R}\).
        \end{itemize} & Convex and concave. & \\
        \hline
        Exponential function \(f: \mathbb{R} \rightarrow \mathbb{R}\)
        \begin{itemize}[leftmargin=*]
            \item \(f(x)=e^{ax} \in \mathbb{R}\), where \(a \in \mathbb{R}\)
        \end{itemize} & Convex. & \\
        \hline
        Quadratic function \(f: \mathbb{R}^{n} \rightarrow \mathbb{R}\)
        \begin{itemize}[leftmargin=*]
            \item \(f(\mathbf{x}) = a \mathbf{x}^\mathsf{T}\mathbf{P} \mathbf{x} + \mathbf{p}^\mathsf{T} \mathbf{x} + r \in \mathbb{R}\), where \(\mathbf{x},\mathbf{p} \in \mathbb{R}^{n}, \mathbf{P} \in \mathbb{R}^{n\times n}\), and \(a,b \in \mathbb{R}\)
        \end{itemize} & It depends on the matrix \(\mathbf{P}\): \begin{itemize}[leftmargin=*]
            \item \(f\) is convex iff \(\mathbf{P} \succeq \mathbf{0}\).
            \item \(f\) is strictly convex iff \(\mathbf{P} \succ \mathbf{0}\).
            \item \(f\) is concave iff \(\mathbf{P} \preceq \mathbf{0}\).
            \item \(f\) is strictly concave iff \(\mathbf{P} \prec \mathbf{0}\).
        \end{itemize} & \\
        \hline
        Power function \(f: \mathbb{R}_{++} \rightarrow \mathbb{R} \) \begin{itemize}[leftmargin=*]
            \item \(f(x) = x^{a}\)
        \end{itemize} & It depends on \(a\) \begin{itemize}[leftmargin=*]
            \item \(f\) is convex iff \(a\geq 1\) or \(a\leq 0\).
            \item \(f\) is concave iff \(0\leq a \leq 1\).
        \end{itemize} & \\
        \hline
        Power of absolute value: \(f: \mathbb{R} \rightarrow \mathbb{R}\) \begin{itemize}[leftmargin=*]
            \item \(f(x) = \abs{x}^p\), where \(p\leq 1\).
        \end{itemize} & Convex. & \\
        \hline
        Logarithm function: \(f: \mathbb{R}_{++} \rightarrow \mathbb{R}\) \begin{itemize}[leftmargin=*]
            \item \(f(x) = \log x\)
        \end{itemize} & Concave. & \\
        \hline
        Negative entropy function: \(f: \mathbb{R}_{+} \rightarrow \mathbb{R}\)
        \begin{itemize}[leftmargin=*]
            \item \(f(x) = x\log x \)
        \end{itemize} & Convex. &
        \begin{itemize}[leftmargin=*]
            \item When it is defined \(\eval{f(x)}{x=0} = 0 \), \(\dom{f} = \mathbb{R}\).
        \end{itemize} \\
        \hline
        Minkowski distance, \(p\)-norm function, or \(l_p\) norm function: \(f: \mathbb{R}^{n} \rightarrow \mathbb{R}\)
        \begin{itemize}[leftmargin=*]
            \item \(f(\mathbf{x}) = \norm{\mathbf{x}}_{p}\), where \(p \in \mathbb{N}_{++}\).
        \end{itemize} & Convex. & \vspace{-3.5ex} \begin{itemize}[leftmargin=*]
            \item It can be proved by triangular inequality.
        \end{itemize} \\
        \hline
        Maximum element: \(f: \mathbb{R}^{n} \rightarrow \mathbb{R}\)
        \begin{itemize}[leftmargin=*]
            \item \(f(\mathbf{x}) = \max\left\{ x_1, \dots, x_n \right\}\).
        \end{itemize} & Convex. & \\
        \hline
        Pointwise maximum (maximum function): \(f: \mathbb{R}^{n} \rightarrow \mathbb{R}\)
        \begin{itemize}[leftmargin=*]
            \item \(f(\mathbf{x}) = \max\left\{ f_1(\mathbf{x}), \dots, f_n(\mathbf{x}) \right\}\).
        \end{itemize} & \(f\) is convex if \(f_1, \dots, f_n\) are convex functions. &
        \vspace{-3.5ex} \begin{itemize}[leftmargin=*]
            \item Its domain \(\dom{f} = \bigcap\limits_{i=1}^{n} \dom{f_i}\) is also convex.
        \end{itemize} \\
        \hline
        Pointwise infimum:
        \begin{itemize}[leftmargin=*]
            \item \(f(\mathbf{x}) = \underset{\mathbf{y} \in \mathcal{A}}{\textnormal{inf }} g(\mathbf{x},\mathbf{y})\).
        \end{itemize} & \(f\) is concave if \(g\) is concave for each \(\mathbf{y}\in \mathcal{A}\). &
        \vspace{-3.5ex} \begin{itemize}[leftmargin=*]
            \item For each value of \(x\), we have an infinite set of points \(\eval{g(x,y)}{y\in \mathcal{A}}\). The value \(f(x)\) will be the greatest value in the codomain of \(f\) that is less than or equal this set.
            \item \(\dom{f} = \left\{ x \mid (x,y) \in \dom{g} \;\forall\; y \in \mathcal{A}, \underset{y \in \mathcal{A}}{\textnormal{ inf }}g(x,y)> -\infty \right\}\).
        \end{itemize} \\
        \hline
        Pointwise supremum:
        \begin{itemize}[leftmargin=*]
            \item \(f(\mathbf{x}) = \underset{\mathbf{y} \in \mathcal{A}}{\textnormal{sup }} g(\mathbf{x},\mathbf{y})\).
        \end{itemize} & \(f\) is convex if \(g\) is convex for each \(\mathbf{y}\in \mathcal{A}\). &
        \vspace{-3.5ex} \begin{itemize}[leftmargin=*]
            \item For each value of \(x\), we have an infinite set of points \(\eval{g(x,y)}{y\in \mathcal{A}}\). The value \(f(x)\) will be the least value in the codomain of \(f\) that is greater than or equal this set.
            \item \(\dom{f} = \left\{ x \mid (x,y) \in \dom{g} \;\forall\; y \in \mathcal{A}, \underset{y \in \mathcal{A}}{\textnormal{ sup }}g(x,y)<\infty \right\}\).
            \item In terms of epigraphs, the pointwise supremum of the infinite set of functions \(\eval{g(x,y)}{y\in \mathcal{A}}\) corresponds to the intersection of the following epigraphs: \(\textnormal{epi } f = \intersection_{y \in \mathcal{A}} \textnormal{epi } g(\cdot, y)\)
        \end{itemize} \\
        \hline
        Minimum function: \(f: \mathbb{R}^{n} \rightarrow \mathbb{R}\)
        \begin{itemize}[leftmargin=*]
            \item \(f(\mathbf{x}) = \min\left\{ f_1(\mathbf{x}), \dots, f_n(\mathbf{x}) \right\}\).
        \end{itemize} & Nonconvex and nonconcave in most of the cases. & \\
        \hline
        Log-sum-exp function: \(f: \mathbb{R}^{n} \rightarrow \mathbb{R}\)
        \begin{itemize}[leftmargin=*]
            \item \(f(\mathbf{x}) = \log\left( e^{x_1} + \dots+ e^{x_n} \right)\)
        \end{itemize} & Convex. & \vspace{-3.5ex}
        \begin{itemize}[leftmargin=*]
            \item This function is interpreted as the approximation of the maximum element function, since \(\max\left\{ x_1, \dots, x_n \right\} \leq f(\mathbf{x}) \leq \max \left\{ x_1, \dots, x_n \right\} + \log n\)
        \end{itemize} \\
        \hline
        Geometric mean function \(f: \mathbb{R}^{n}\rightarrow \mathbb{R}\)
        \begin{itemize}[leftmargin=*]
            \item \(f(\mathbf{x}) = \left( \Pi_{i=1}^{n} x_i \right)^{1/n}\)
        \end{itemize} & Convex. & \\
        \hline
        Log-determinant function \(f: \mathcal{S}^{n}_{++}\rightarrow \mathbb{R}\)
        \begin{itemize}[leftmargin=*]
            \item \(f(\mathbf{X}) = \log \abs{\mathbf{X}}\)
        \end{itemize} & Convex. & \vspace{-3.5ex}
        \begin{itemize}[leftmargin=*]
            \item \(\mathbf{X}\) is positive semidefinite, i.e., \(\mathbf{X} \succ \mathbf{0} \therefore \mathbf{X}\in \mathcal{S}^{n}_{++}\).
        \end{itemize} \\
        \hline
        Composite function \(f = h\circ g : \mathbb{R}^{n}\rightarrow \mathbb{R}\)
        \begin{itemize}[leftmargin=*]
            \item \(f = g \circ h\), i.e., \(f(\mathbf{x}) = (h\circ g)(\mathbf{x}) = h(g(\mathbf{x}))\), where:
            \begin{itemize}[label=\(\triangleright\)]
                \item \(g: \mathbb{R}^{n}\rightarrow \mathbb{R}^{k}\).
                \item \(h: \mathbb{R}^{k}\rightarrow \mathbb{R}\).
                \item \(\dom{f} = \left\{ \mathbf{x} \in \dom{g}\mid g(\mathbf{x}) \in \dom{h} \right\}\).
            \end{itemize}  %\(\mathbf{x} \in S \subseteq \mathbb{R}^{p}\)
        \end{itemize} & \vspace{-3.5ex}
        \begin{itemize}[leftmargin=*]
            \item Scalar composition: the following statements hold for \(k=1\) and \(n\geq 1\), i.e., \(h: \mathbb{R}\rightarrow \mathbb{R}\) and \(g: \mathbb{R}^{n} \rightarrow \mathbb{R}\):
                \begin{itemize}[label=\(\triangleright\)]
                    \item \(f\) is convex if \(h\) is convex, \(\tilde{h}\) is nondecreasing, and \(g\) is convex. In this case, \(\dom{h}\) is either \((-\infty, a]\) or \((-\infty, a)\).
                    \item \(f\) is convex if \(h\) is convex, \(\tilde{h}\) is nonincreasing, and \(g\) is concave. In this case, \(\dom{h}\) is either \([a, \infty)\) or \((a, \infty)\).
                    \item \(f\) is concave if \(h\) is concave, \(\tilde{h}\) is nondecreasing, and \(g\) is concave.
                    \item \(f\) is concave if \(h\) is concave, \(\tilde{h}\) is nonincreasing, and \(g\) is convex.
                \end{itemize}
        \end{itemize}
        \begin{itemize}[leftmargin=*]
            \item Vector composition: the following statements hold for \(k\geq 1\) and \(n\geq 1\), i.e., \(h: \mathbb{R}^{k} \rightarrow \mathbb{R}\) and \(\boldsymbol{g}: \mathbb{R}^{n} \rightarrow \mathbb{R}^{k}\). Hence, \(g(\mathbf{x}) = (g_1 (\mathbf{x}), g_2(\mathbf{x}), \dots, g_k(\mathbf{x}))\) is a vector-valued function (or simply, vector function), where \(g_i: \mathbb{R}^{k} \rightarrow \mathbb{R}\) for \(1 \leq i \leq k\).
            \begin{itemize}[label=\(\triangleright\)]
                \item \(f\) is convex if \(h\) is is convex, \(\tilde{h}\) is nondecreasing in each argument of \(\mathbf{x}\), and \(\left\{ g_i \right\}_{i=1}^{k}\) is a set of convex functions.
                \item \(f\) is convex if \(h\) is is convex, \(\tilde{h}\) is nonincreasing in each argument of \(\mathbf{x}\), and \(\left\{ g_i \right\}_{i=1}^{k}\) is a set of concave functions.
                \item \(f\) is concave if \(h\) is is concave, \(\tilde{h}\) is nondecreasing in each argument of \(\mathbf{x}\), and \(\left\{ g_i \right\}_{i=1}^{k}\) is a set of concave functions.
            \end{itemize}
        \end{itemize}
        Where \(\tilde{h}\) is the extended-value extension of the function \(h\), which assigns the value \(\infty\) (\(-\infty\)) to the point not in \(\dom{h}\) for \(h\) convex (concave).
        & \vspace{-3.5ex}
        \begin{itemize}[leftmargin=*]
        \item The composition function allows us to see a large class of functions as convex (or concave).
        \item For scale composition, the remarkable ones are:
            \begin{itemize}[label=\(\triangleright\)]
                \item If \(g\) is convex then \(f(x) = h(g(\mathbf{x})) = \exp{g(\mathbf{x})}\) is convex.
                \item If \(g\) is concave and \(\dom{g} \subseteq \mathbb{R}_{++}\), then \(f(\mathbf{x}) = h(g(\mathbf{x})) = \log{g(\mathbf{x})}\) is concave.
                \item If \(g\) is concave and \(\dom{g} \subseteq \mathbb{R}_{++}\), then \(f(\mathbf{x}) = h(g(\mathbf{x})) = 1/g(\mathbf{x})\) is convex.
                \item If \(g\) is convex and \(\dom{g} \subseteq \mathbb{R}_{+}\), then \(f(\mathbf{x}) = h(g(\mathbf{x})) = g^{p}(\mathbf{x})\) is convex, where \(p\geq 1\).
                \item If \(g\) is convex then \(f(\mathbf{x}) = h(g(\mathbf{x})) = - \log{\left( -g(x) \right)}\) is convex, where \(\dom{f} = \left\{ \mathbf{x} \mid g(\mathbf{x})<0 \right\}\).
            \end{itemize}
        \item For vector composition, we have the following examples:
            \begin{itemize}[label=\(\triangleright\)]
                \item If \(g(\mathbf{x}) = \mathbf{Ax} + \mathbf{b}\) is an affine function, then \(f = h \circ g\) is convex (concave) if \(h\) is convex (concave).
                \item Let \(h(\mathbf{x}) = x_{\left[ 1 \right]} + \dots + x_{\left[ r \right]}\) be the sum of the \(r\) largest components of \(\mathbf{x} \in \mathbb{R}^{k}\). If \(g_1, g_2, \dots, g_k\) are convex, where \(\dom{g_i} = \mathbb{R}^{n}\), then \(f = h\circ g\), which is the pointwise sum of the largest \(g_i\)'s, is convex.
                \item \(f = h\circ g\) is a convex function when \(h(\mathbf{x}) = \log\left( \sum_{i=1}^{k} e^{x_i} \right)\) and \(g_1, g_2, \dots, g_k\) are convex functions.
                \item For \(0<p \leq 1\), the function \(h(\mathbf{x}) = \left( \sum_{i=1}^{k} x_i^p \right)^{1/p}\), where \(\dom{h} = \mathbb{R}_+^{n}\), is concave. If \(g_1, g_2, \dots, g_k\) are concaves (convexes) and nonnegatives, then \(f = h \circ g\) is concave (convex).
            \end{itemize}
        \end{itemize}\\
        \hline
        Nonnegative weighted sum: \(f: \dom{f} \rightarrow \mathbb{R}\)
        \begin{itemize}
            \item \(f(\mathbf{x}) = \sum_{i=1}^{m} w_if_i(\mathbf{x})\), where \(w\geq 0\).
        \end{itemize} & \vspace{-3.5ex} \begin{itemize}[leftmargin=*]
            \item If \(f_1, f_2, \dots, f_m\) are convex or concave functions, then \(f\) is a convex or concave function, respectively.
            \item If \(f_1, f_2, \dots, f_m\) are strictly convex or concave functions, then \(f\) is a strictly convex or concave function, respectively.
        \end{itemize} & \vspace{-3.5ex} \begin{itemize}[leftmargin=*]
            \item Special cases is when \(f = w f\) (a nonnegative scaling) and \(f = f_1 + f_2\) (sum).
        \end{itemize}\\
        \hline
        Integral function \(f: \mathbb{R}^{n}\rightarrow \mathbb{R}\):
        \begin{itemize}
            \item \(f(\mathbf{x}) = \int_\mathcal{A} w(\mathbf{y}) g(\mathbf{x},\mathbf{y}) \diff \mathbf{y}\), where \(\mathbf{y} \in \mathcal{A} \subseteq \mathbb{R}^{m}\), and \(w: \mathbb{R}^{m} \rightarrow \mathbb{R}\).
        \end{itemize} & If \(g\) is convex in \(\mathbf{x}\) for each \(\mathbf{y}\in \mathcal{A}\) and if \(w(\mathbf{y}) \geq 0, \;\forall\; \mathbf{y}\in \mathcal{A}\), then \(f\) is convex (provided the integral exists). & \\
        \hline
        Perspective function \(f: \mathbb{R}^{n} \times \mathbb{R}_{++} \rightarrow \mathbb{R}^{n}\)
        \begin{itemize}[leftmargin=*]
            \item \(f(\mathbf{x}, t) = \mathbf{x}/t\), where \(\mathbf{x} \in \mathbb{R}^{n}, t \in \mathbb{R}\).
        \end{itemize} & Yes, if \(S \subseteq \dom{f}\) is a convex set, then its image, \(f(S) = \left\{ f(\mathbf{x})|\mathbf{x}\in S \right\} \subseteq \mathbb{R}^{n}\), is also convex. & \vspace{-3.5ex} \begin{itemize}[leftmargin=*]
            \item The perspective function decreases the dimension of the function domain since \(\textnormal{dim}(\dom{f}) = n+1\).
            \item Its effect is similar to the camera zoom.
            \item The inverse image is also convex, that is, if \(C \subseteq \mathbb{R}^{n}\) is convex, then \(f^{-1}(C) = \left\{ (\mathbf{x}, t) \in \mathbb{R}^{n+1} \mid \mathbf{x}/t \in C, t>0 \right\}\) is also convex.
            \item A special case is when \(n=1\), which is called \emph{quadratic-over-linear function}.
        \end{itemize} \\
        \hline
        Projective (or linear-fractional) function, \(f: \mathbb{R}^{n} \rightarrow \mathbb{R}^{m}\)
        \begin{itemize}[leftmargin=*]
            \item \(f = p \circ g\), i.e., \(f(\mathbf{x}) = (p\circ g)(\mathbf{x}) = p(g(\mathbf{x}))\), where
                \begin{itemize}[label={$\triangleright$}]
                    \item \(g: \mathbb{R}^{n} \rightarrow \mathbb{R}^{m+1}\) is an affine function given by \(g(\mathbf{x}) = \begin{bmatrix}
                        \mathbf{A}\\
                        \mathbf{c}^\mathsf{T}
                    \end{bmatrix} \mathbf{x} + \begin{bmatrix}
                        \mathbf{b} \\
                        d
                    \end{bmatrix}\), being \(\mathbf{A}\in \mathbb{R}^{m \times n}, \mathbf{b} \in \mathbb{R}^{m}, \mathbf{c} \in \mathbb{R}^{n}\), and \(d \in \mathbb{R}\).
                    \item \(p: \mathbb{R}^{m+1} \rightarrow \mathbb{R}^{m}\) is the perspective function.
                \end{itemize}
            \item \(f(\mathbf{x}) = \mathcal{P}^{-1}(\mathbf{Q}\mathcal{P}(\mathbf{x}))\)
                \begin{itemize}[label={$\triangleright$}]
                    \item \(\mathcal{P}(\mathbf{x}) = \left\{ (t\mathbf{x}, t) \mid t \geq 0 \right\} \subset \mathbb{R}^{n+1}\)
                    \item \(\mathbf{Q} = \begin{bmatrix}
                        \mathbf{A} & \mathbf{b} \\
                        \mathbf{c}^\mathsf{T} & d
                    \end{bmatrix} \in \mathbb{R}^{(m+1)\times(n+1)}\)
                \end{itemize}
        \end{itemize} & Yes, if \(S \subseteq \dom{f}\) is a convex set, then its image, \(f(S) = \left\{ f(\mathbf{x})|\mathbf{x}\in S \right\} \subseteq \mathbb{R}^{n}\), is also convex. & \vspace{-3.5ex} \begin{itemize}[leftmargin=*]
            \item The linear and affine functions are special cases of the linear-fractional function.
            \item \(\dom{f} = \left\{ \mathbf{x} \in \mathbb{R}^{n} \mid \mathbf{c}^\mathsf{T} \mathbf{x} + d > 0 \right\}\)
            \item \(\mathcal{P}(\mathbf{x}) \subset \mathbb{R}^{n+1}\) is a ray set that begins at the origin and its last component takes only positive values. For each \(\mathbf{x} \in \dom{f}\), it is associated a ray set in \(\mathbb{R}^{n+1}\) in this form. This (projective) correspondence between all points in \(\dom{f}\) and their respective sets \(\mathcal{P}\) is a biunivocal mapping.
            \item The linear transformation \(\mathbf{Q}\) acts on these rays, forming another set of rays.
            \item Finally we take the inverse projective transformation to recover \(f(\mathbf{x})\).
        \end{itemize}\\
        \hline
        Epigraph:
        \begin{itemize}[leftmargin=*]
            \item \(\textnormal{epi } f = \{(\mathbf{x}, t)\mid \mathbf{x} \in \dom{f}, t\geq f(\mathbf{x})\}\)
        \end{itemize} & \vspace{-3.5ex}
        \begin{itemize}[leftmargin=*]
            \item The function \(f\) is convex iff its epigraph is convex.
        \end{itemize} & \vspace{-3.5ex}
        \begin{itemize}[leftmargin=*]
        \item Visually, it is the graph above the \((\mathbf{x}, f(\mathbf{x}))\) curve.
    \end{itemize}\\
        \hline
        Hypograph:
        \begin{itemize}[leftmargin=*]
            \item \(\textnormal{hypo } f = \{(\mathbf{x}, t)\mid \mathbf{x} \in \dom{f}, t\geq f(\mathbf{x})\}\)
        \end{itemize} & \vspace{-3.5ex}
        \begin{itemize}[leftmargin=*]
            \item The function \(f\) is concave iff its hypograph is convex.
        \end{itemize} & \vspace{-3.5ex}
        \begin{itemize}[leftmargin=*]
            \item Visually, it is the graph below the \((\mathbf{x}, f(\mathbf{x}))\) curve.
        \end{itemize}\\
        \hline
    \end{tabularx}
\end{table}

\clearpage
\edef\hmm{\pdfpagewidth=\the\pdfpagewidth \pdfpageheight=\the\pdfpageheight\relax}
\pdfpagewidth=80cm
\pdfpageheight=40cm
\pagenumbering{gobble}
\newgeometry{top=1in,left=1in,textwidth=15in,textheight=9in}

\begin{figure}
    \centering
    \includestandalone{figs/optmization_types}
  \end{figure}


% \restoregeometry
% \hmm
% Back to a standard page

% \begin{itemize}
%     \item All convex set is quasiconvex, but not all quasiconvex is convex.
%     \item It is possible to solve quasiconvex functions, even if it is not convex (see Algorithm 4.1). But not all quasiconvex functions that are nonconvex can be solved(?).
%     \item Superlevel set (a set) 3.3.6, all convex functions have all convex \(\alpha\) sub-level set, but not all functions that have convex \(\alpha\) sub-level set are convex (see slide 3.11).
% \end{itemize}

\end{document}