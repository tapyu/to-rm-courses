\documentclass{article}
\usepackage[margin=2mm,textwidth=155mm,paperheight=60cm,paperwidth=30cm]{geometry}

% table
\usepackage{tabularx}
\newcolumntype{A}{>{\setlength\hsize{1.4\hsize}\setlength\linewidth{\hsize}}X}
\newcolumntype{B}{>{\setlength\hsize{.5\hsize}\setlength\linewidth{\hsize}\centering\arraybackslash}X}
\newcolumntype{C}{>{\setlength\hsize{1.1\hsize}\setlength\linewidth{\hsize}}X}

% itemize
\usepackage{enumitem}

% math
\usepackage{amsmath}
\usepackage{amsfonts}
\usepackage{amssymb}

% commands
\newcommand{\trans}{\mathsf{T}}
\newcommand{\hermit}{\mathsf{H}}
\newcommand\norm[1]{\left\lVert#1\right\rVert}

\begin{document}

\begin{table}
\begin{tabularx}{\textwidth}{|A|B|C|}
\hline
\multicolumn{3}{|c|}{Sets}\\
\hline
Set & Convex? & \multicolumn{1}{|c|}{Comments}\\
\hline
Convex hull:
\begin{itemize}
\item $\textnormal{conv } C = \left\{ \sum_{i=1}^{k} \theta_i\mathbf{x}_i \mid \mathbf{x}_i \in C, \mathbf{0} \preceq \boldsymbol{\theta} \preceq \mathbf{1}, \mathbf{1}^\trans\boldsymbol{\theta} = 1  \right\}$
\end{itemize} &
Yes &
\begin{itemize}[leftmargin=*]
    \item $\textnormal{conv } C$ will be the smallest convex set that contains $C$.
    \item $\textnormal{conv } C$ will be a finite set as long as $C$ is also finite.
\end{itemize}\\
\hline
Affine hull: 
\begin{itemize}[leftmargin=*]
    \item $\textnormal{aff } C = \left\{ \sum_{i=1}^{k} \theta_i\mathbf{x}_i \mid \mathbf{x}_i \in C \textnormal{ for } i=1,\cdots, k, \mathbf{1}^\trans\boldsymbol{\theta} = 1  \right\}$
\end{itemize} &
Yes. &
\begin{itemize}[leftmargin=*]
    \item $A$ will be the smallest affine set that contains $C$.
    \item Different from the convex set, \(\theta_i\) is not restricted between 0 and 1
    \item $\textnormal{aff } C$ will always be an infinite set. If $\textnormal{aff } C$ contains the origin, it is also a subspace.
\end{itemize}\\
\hline
Conic hull:
\begin{itemize}[leftmargin=*]
    \item $A = \left\{ \sum_{i=1}^{k} \theta_i\mathbf{x}_i \mid \mathbf{x}_i \in C, \theta_i > 0 \textnormal{ for } i=1,\cdots, k \right\}$
\end{itemize} &
Yes. &
\begin{itemize}[leftmargin=*]
    \item $A$ will be the smallest convex conic that contains $C$.
    \item Different from the convex and affine sets, \(\theta_i\) does not need to sum up 1.
\end{itemize}\\
\hline
Hyperplane:
\begin{itemize}[leftmargin=*]
    \item \( \mathcal{H} = \left\{ \mathbf{x} \mid \mathbf{a}^\trans \mathbf{x} = b \right\}\)
    \item \(\mathcal{H} = \left\{ \mathbf{x} \mid \mathbf{a}^\trans (\mathbf{x} - \mathbf{x}_{0}) = \mathbf{0} \right\}\)
    \item \(\mathcal{H} = \mathbf{x}_0 + a^{\perp} \)
\end{itemize} & Yes. &
\begin{itemize}[leftmargin=*]
    \item It is an infinite set \(\mathbb{R}^{n-1} \subset \mathbb{R}^{n}\) that divides the space into two halfspaces.
    \item \(a^{\perp} = \left\{ \mathbf{v} \mid \mathbf{a}^\trans \mathbf{v} = 0 \right\}\) is the set of vectors perpendicular to \(\mathbf{a}\). It passes through the origin.
    \item \(a^{\perp}\) is offset from the origin by \(\mathbf{x}_0\), which is any vector in \(\mathcal{H}\).
\end{itemize} \\
\hline
Halfspaces:
\begin{itemize}[leftmargin=*]
    \item \(\mathcal{H}_{-} = \left\{ \mathbf{x} \mid \mathbf{a}^\trans \mathbf{x} \leq b \right\}\)
    \item \(\mathcal{H}_{+} = \left\{ \mathbf{x} \mid \mathbf{a}^\trans \mathbf{x} \geq b \right\}\)
\end{itemize} & Yes. &
\begin{itemize}[leftmargin=*]
    \item They are infinite sets of the parts divided by \(\mathcal{H}\).
\end{itemize}\\
\hline
Euclidean ball:
\begin{itemize}[leftmargin=*]
    \item \(B(\mathbf{x}_c, r) = \left\{ \mathbf{x} \mid \norm{\mathbf{x}-\mathbf{x}_c}_2 \leq r \right\}\)
    \item \(B(\mathbf{x}_c, r) = \left\{ \mathbf{x} \mid \left( \mathbf{x}-\mathbf{x}_c \right)^\trans \left( \mathbf{x}-\mathbf{x}_c \right) \leq r \right\}\)
    \item \(B(\mathbf{x}_c, r) = \left\{ \mathbf{x}_c + r \norm{\mathbf{u}} \mid \norm{\mathbf{u}} \leq 1 \right\}\)
\end{itemize} & Yes. &
\begin{itemize}[leftmargin=*]
    \item \(B(\mathbf{x}_c, r)\) is a finite set as long as \(r < \infty\).
    \item \(\mathbf{x}_c\) is the center of the ball.
    \item \(r\) is its radius.
\end{itemize}\\
\hline
Ellipsoid:
\begin{itemize}[leftmargin=*]
    \item \(\mathcal{E} = \left\{ \mathbf{x} \mid (\mathbf{x}-\mathbf{x}_c)^\trans\mathbf{P}^{-1}(\mathbf{x}-\mathbf{x}_c) \leq 1 \right\}\)
    \item \(\mathcal{E} = \left\{ \mathbf{x}_{c} + \mathbf{Au} \mid \norm{\mathbf{u}} \leq 1 \right\}\), where \(\mathbf{A} = \mathbf{P}^{1/2}\).
\end{itemize} & Yes. &
\begin{itemize}[leftmargin=*]
    \item \(\mathcal{E}\) is a finite set as long as \(\mathbf{P}\) is a finite matrix.
    \item \(\mathbf{P}\) is symmetric and positive definite, that is, \(\mathbf{P}=\mathbf{P}^\trans \succ 0\).
    \item \(\mathbf{x}_{c}\) is the center of the ellipsoid.
    \item The lengths of the semi-axes are given by \(\sqrt{\lambda_i}\).
    \item \(\mathbf{A}\) is invertible. When it is not, we say that \(\mathcal{E}\) is a degenerated ellipsoid (degenerated ellipsoids are also convex).
\end{itemize}\\
\hline
Norm cone:
\begin{itemize}[leftmargin=*]
    \item \(C = \left\{ [x_1, x_2, \cdots, x_n, t]^\trans \in \mathbb{R}^{n+1} \mid \mathbf{x} \in \mathbb{R}^{n}, \norm{\mathbf{x}}_{p} \leq t \right\} \subseteq \mathbb{R}^{n+1}\)
\end{itemize} & Yes. &
\begin{itemize}[leftmargin=*]
    \item Although it is named ``Norm cone'', it is a set, not a scalar.
    \item The cone norm increases the dimension of \(\mathbf{x}\) in 1.
    \item For \(p=2\), it is called the second-order cone, quadratic cone,  Lorentz cone or ice-cream cone.
\end{itemize} \\
\hline
Polyhedra:
\begin{itemize}[leftmargin=*]
    \item $\mathcal{P} = \left\{ \mathbf{x} \mid \mathbf{a}_j^\trans \mathbf{x} \leq b_j, j=1, \dots, m, \mathbf{a}_j^\trans \mathbf{x} = d_j, j=1,\cdots, p  \right\}$
    \item \(\mathcal{P} = \left\{ \mathbf{x} \mid \mathbf{Ax} \preceq \mathbf{b}, \mathbf{Cx} = \mathbf{d} \right\}\), where \(\mathbf{A} = \begin{bmatrix}
            \mathbf{a}_1 & \mathbf{a}_2 & \dots & \mathbf{a}_m
        \end{bmatrix}^\trans\) and \(\mathbf{C} = \begin{bmatrix}
            \mathbf{c}_1 & \mathbf{c}_2 & \dots & \mathbf{c}_m
        \end{bmatrix}^\trans\)
\end{itemize} & Yes. &
\begin{itemize}[leftmargin=*]
    \item Polyhedron is the result of the intersection of \(m\) halfspaces and \(p\) hyperplanes.
    \item The polyhedron may or may not be an infinite set.
    \item Subspaces, hyperplanes, lines, rays line segments, and
    halfspaces are all polyhedra.
    \item The \emph{nonnegative orthant}, \(\mathbb{R}_{+}^{n} = \left\{ \mathbf{x} \in \mathbb{R}^n \mid x_i \leq 0 \textnormal{ for } i=1,\dots n \right\} = \left\{ \mathbf{x} \in \mathbb{R}^n \mid \mathbf{Ix}  \succeq \mathbf{0}\right\}\), is a special polyhedron.
\end{itemize}\\
\hline
Simplex:
\begin{itemize}[leftmargin=*]
    \item \(\mathcal{S} = \textnormal{conv }\left\{ \mathbf{v}_m \right\}_{m=0}^{k} = \left\{ \sum_{i=0}^{k} \theta_i \mathbf{v}_i \mid \mathbf{0} \preceq \boldsymbol{\theta} \preceq \mathbf{1}, \mathbf{1}^\trans \boldsymbol{\theta} = 1 \right\}\)
    \item \(\mathcal{S} = \left\{ \mathbf{x} \mid \mathbf{x} = \mathbf{v}_0 + \mathbf{V} \boldsymbol{\theta} \right\}\), where \(\mathbf{V} = \begin{bmatrix}
        \mathbf{v}_1 - \mathbf{v}_0 & \dots & \mathbf{v}_n - \mathbf{v}_0
    \end{bmatrix} \in \mathbb{R}^{n \times k}\)
    \item \(\mathcal{S} = \{ \mathbf{x} \mid \underbrace{\mathbf{A}_1 \mathbf{x} \preceq \mathbf{A}_1 \mathbf{v}_0, \, \mathbf{1}^\trans \mathbf{A}_1 \mathbf{x} \leq 1 + \mathbf{1}^\trans\mathbf{A}_1 \mathbf{v}_0}_{\textnormal{Linear inequalities in } x}, \, \underbrace{\mathbf{A}_2 \mathbf{x} = \mathbf{A}_2 \mathbf{v}_0}_{\substack{\text{\textnormal{Linear equalities}} \\\textnormal{in } x}} \}\) (Polyhedra form), where \(\mathbf{A} = \begin{bmatrix}
        \mathbf{A}_1 \\ \mathbf{A}_2
    \end{bmatrix}\) and \(\mathbf{AV} = \begin{bmatrix}
        \mathbf{I}_{k\times k}\\
        \mathbf{0}_{n-k \times n-k}
    \end{bmatrix}\)
\end{itemize} & Yes. &
\begin{itemize}[leftmargin=*]
    \item Simplexes are a subfamily of the polyhedra set.
    \item Also called k-dimensional Simplex in \(\mathbb{R}^{n}\).
    \item The set \(\left\{ \mathbf{v}_m \right\}_{m=0}^{k}\) is a affinely independent, which means \(\left\{ \mathbf{v}_1-\mathbf{v}_0, \dots, \mathbf{v}_k-\mathbf{v}_0 \right\}\) are linearly independent.
    \item \(\mathbf{V} \in \mathbb{R}^{n\times k}\) is a full-rank tall matrix, i.e., \(\textnormal{rank}(\mathbf{V}) = k\). All its column vectors are independent. The matrix \(\mathbf{A}\) is its left pseudoinverse.
\end{itemize}\\
\hline
\end{tabularx}
% \begin{tabular}{|c|c|c|p{5cm}|}
%     Function & Domain & Codomain & Comments \\
%     \hline
%     System of linear equation: $\mathbf{b} = f(\mathbf{x}) = \mathbf{Ax}$ & $D = \left\{ \mathbf{x} \in \mathbb{R}^{n} \mid \mathbf{Ax} = \mathbf{b} \in C, \mathbf{A} \in \mathbb{R}^{m \times n} \right\}$ & $C = \left\{ \mathbf{b} \in \mathbb{R}^{m} | \mathbf{b} = \mathbf{Ax},\: \forall\: \mathbf{x} \in D  \right\}$ & If $D$ is an affine set, so $C$ is also affine set which, in turn, is a convex set. \\ \hline
% \end{tabular} \\
% \begin{tabular}{||c | c | c||}
%     \hline
%     \multicolumn{3}{|c|}{Functions} \\
%     \hline
%     Function & Convex? & Proof \\ [0.5ex] 
%     \hline\hline
%     $\mathbf{y} = \max(f_1, f_2)$ & Yes, if $f_1$ and $f_2$ are convex functions & \\
%     \hline
%     $\mathbf{y} = \min(f_1, f_2)$ & Not always & \\
%     \hline
%     $C = \left\{ \mathbf{x} \in \mathbb{R}^{n} \mid \mathbf{Ax} = \mathbf{b}, \mathbf{A} \in \mathbb{R}^{m \times n}, \mathbf{b} \in \mathbb{R}^{m} \right\}$ & It is an affine set (all affine set is a convex set)  & \\
%     \hline
%     $y = \mathbf{c}^\trans \mathbf{x}$ (linear function) & Yes (but not strictly convex) & \\
%     \hline
%     $y = \norm{\mathbf{x}}_p $ (p-norm) & Yes (for any $p \in \mathbb{N}_+$) & $\norm{\theta\mathbf{x} + (1-\theta)\mathbf{y}} \leq \theta\norm{\mathbf{x}} + (1-\theta)\norm{\mathbf{y}}$ (triangular inequality) \\
%     \hline
%     $f(g(\mathbf{x}))$ & Yes, if $f, g$ are convex & \\
%     \hline
% \end{tabular}
\end{table}

\begin{table}
    \begin{tabularx}{\textwidth}{|>{\setlength\hsize{1\hsize}\setlength\linewidth{\hsize}}X|>{\setlength\hsize{.9\hsize}\setlength\linewidth{\hsize}}X|>{\setlength\hsize{1.1\hsize}\setlength\linewidth{\hsize}}X|}%{| >{\hsize=.5\hsize}X | >{\hsize=1.5\hsize}X |}
        \hline
        \multicolumn{3}{|c|}{Functions (or operators) and their implications regarding convexity} \\
        \hline
        \multicolumn{1}{|c|}{Function} & \multicolumn{1}{|c|}{Convex?} & \multicolumn{1}{|c|}{Comments} \\
        \hline
        Union: $C = A \cup B $ & Not always. & \\
        \hline
        Intersection: $C = A \cap B $ & Yes, if $A$ and $B$ are convex sets. & \\
        \hline
    \end{tabularx}
\end{table}

\begin{itemize}
    \item All convex set is quasiconvex, but not all quasiconvex is convex.
    \item It is possible to solve quasiconvex functions, even if it is not convex (see Algorithm 4.1). But not all quasiconvex functions that are nonconvex can be solved(?).
    \item Superlevel set (a set) 3.3.6, all convex functions have all convex \(\alpha\) sub-level set, but not all functions that have convex \(\alpha\) sub-level set are convex (see slide 3.11).
\end{itemize}

\end{document}