\documentclass{article}
\usepackage[margin=25mm]{geometry}
\usepackage{pdflscape}
\usepackage{booktabs, makecell}

\usepackage{siunitx}
\usepackage{multirow}

\usepackage{amsfonts}
\usepackage{amsmath}

\newcommand{\trans}{\mathsf{T}}
\newcommand{\hermit}{\mathsf{H}}
\newcommand{\mc}[1]{\ensuremath{\mathcal{#1}}}
\newcommand{\mbb}[1]{\ensuremath{\mathbb{#1}}}
\newcommand{\Natural}{\mathbb{N}}
\newcommand{\Integer}{\mathbb{Z}}
\newcommand{\Irrational}{\mathbb{I}}
\newcommand{\Rational}{\mathbb{Q}}
\newcommand{\Real}{\mathbb{R}}
\newcommand{\Complex}{\mathbb{C}}
\newcommand{\obs}[1]{\textcolor{red}{(#1)}}
\newcommand{\sizecorr}[1]{\makebox[0cm]{\phantom{$\displaystyle #1$}}} % Used to seize the height of equation
%\newcommand{\iff}{\mbox{$\Longleftrightarrow$}}    % biimplication
% \newcommand{\and}{\wedge}
% \newcommand{\or}{\vee}
\newcommand*\diff{\mathop{}\!\mathrm{d}}
\newcommand*\Diff[1]{\mathop{}\!\mathrm{d^#1}}
\newcommand{\ensureoperation}{\negmedspace {}} % To ensure that a new line symbol is an operation instead of a sign

\setlength{\textwidth}{14cm}

\newcommand\norm[1]{\left\lVert#1\right\rVert}

\begin{document}
\begin{landscape}
    \vspace{-10ex}
    \begin{table}
        \centering
    \begin{tabular}{||c | c | c||}
        \hline
        \multicolumn{3}{|c|}{Sets} \\
        \hline
        Set & Convex? & Proof \\ [0.5ex] 
        \hline\hline
        $C = A \cup B $ & Not always & \\
        \hline
        $C = A \cap B $ & Yes, if $A$ and $B$ are convex sets. & \\
        \hline
        \hline
        \multicolumn{3}{|c|}{Functions} \\
        \hline
        Function & Convex? & Proof \\ [0.5ex] 
        \hline\hline
        $\mathbf{y} = \max(f_1, f_2)$ & Yes, if $f_1$ and $f_2$ are convex functions & \\
        \hline
        $\mathbf{y} = \min(f_1, f_2)$ & Not always & \\
        \hline
        $C = \left\{ \mathbf{x} \in \mathbb{R}^{n} \mid \mathbf{Ax} = \mathbf{b}, \mathbf{A} \in \mathbb{R}^{m \times n}, \mathbf{b} \in \mathbb{R}^{m} \right\}$ & It is an affine set (all affine set is a convex set)  & \\
        \hline
        $y = \mathbf{c}^\trans \mathbf{x}$ (linear function) & Yes (but not strictly convex) & \\
        \hline
        $y = \norm{\mathbf{x}}_p $ (p-norm) & Yes (for any $p \in \mathbb{N}_+$) & $\norm{\theta\mathbf{x} + (1-\theta)\mathbf{y}} \leq \theta\norm{\mathbf{x}} + (1-\theta)\norm{\mathbf{y}}$ (triangular inequality) \\
        \hline
        $f(g(\mathbf{x}))$ & Yes, if $f, g$ are convex & \\
        \hline
    \end{tabular}
    \begin{tabular}{|c|c|c|p{5cm}|}
        Function & Domain & Codomain & Comments \\
        \hline
        System of linear equation: $\mathbf{b} = f(\mathbf{x}) = \mathbf{Ax}$ & $D = \left\{ \mathbf{x} \in \mathbb{R}^{n} \mid \mathbf{Ax} = \mathbf{b} \in \mathbb{R}^{m}, \mathbf{A} \in \mathbb{R}^{m \times n} \right\}$ & $C = \left\{ \mathbf{b} \in \mathbb{R}^{m} | \mathbf{b} = \mathbf{Ax},\: \forall\: \mathbf{x} \in D  \right\}$ & If $D$ is an affine set, so $C$ is also affine set which, in turn, is a convex set.
    \end{tabular}
\end{table}
\end{landscape}

Remarks:
\begin{enumerate}
    \item All affine set is a convex set, but with infinite extension.
    \item If the affine set happens to have the origin, it is also a subspace of that space.
    \item An affine set contains every affine combination of its points: If \(C\) is an affine set, $x_1, \cdots , x_k \in C$, and \(\sum_{i=1}^{k} \theta_i = 1\), then the point \(\sum_{i=1}^{k} \theta_ix_1 \) also belongs to \(C\).
    
\end{enumerate}
\end{document}