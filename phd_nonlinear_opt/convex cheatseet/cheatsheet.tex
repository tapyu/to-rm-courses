\documentclass{article}
\usepackage[margin=2mm,paperheight=11in,paperwidth=25in]{geometry}
\usepackage{pdflscape}
\usepackage{booktabs, makecell}

\usepackage{siunitx}
\usepackage{multirow}
\usepackage{enumitem}
\usepackage{tabularx}

\usepackage{amsfonts}
\usepackage{amsmath}
\usepackage{amssymb}

\newcommand{\trans}{\mathsf{T}}
\newcommand{\hermit}{\mathsf{H}}
\newcommand{\mc}[1]{\ensuremath{\mathcal{#1}}}
\newcommand{\mbb}[1]{\ensuremath{\mathbb{#1}}}
\newcommand{\Natural}{\mathbb{N}}
\newcommand{\Integer}{\mathbb{Z}}
\newcommand{\Irrational}{\mathbb{I}}
\newcommand{\Rational}{\mathbb{Q}}
\newcommand{\Real}{\mathbb{R}}
\newcommand{\Complex}{\mathbb{C}}
\newcommand{\obs}[1]{\textcolor{red}{(#1)}}
\newcommand{\sizecorr}[1]{\makebox[0cm]{\phantom{$\displaystyle #1$}}} % Used to seize the height of equation
%\newcommand{\iff}{\mbox{$\Longleftrightarrow$}}    % biimplication
% \newcommand{\and}{\wedge}
% \newcommand{\or}{\vee}
\newcommand*\diff{\mathop{}\!\mathrm{d}}
\newcommand*\Diff[1]{\mathop{}\!\mathrm{d^#1}}
\newcommand{\ensureoperation}{\negmedspace {}} % To ensure that a new line symbol is an operation instead of a sign

\newcommand\norm[1]{\left\lVert#1\right\rVert}

\usepackage{booktabs}% http://ctan.org/pkg/booktabs
\newcommand{\tabitem}{~~\llap{\textbullet}~~}

\begin{document}
\begin{landscape}
    \vspace{-10ex}
    \begin{table}
        \centering
    \begin{tabular}{||c | c | c||}
        \hline
        \multicolumn{3}{|c|}{Functions} \\
        \hline
        Function & Convex? & Proof \\ [0.5ex] 
        \hline\hline
        $\mathbf{y} = \max(f_1, f_2)$ & Yes, if $f_1$ and $f_2$ are convex functions & \\
        \hline
        $\mathbf{y} = \min(f_1, f_2)$ & Not always & \\
        \hline
        $C = \left\{ \mathbf{x} \in \mathbb{R}^{n} \mid \mathbf{Ax} = \mathbf{b}, \mathbf{A} \in \mathbb{R}^{m \times n}, \mathbf{b} \in \mathbb{R}^{m} \right\}$ & It is an affine set (all affine set is a convex set)  & \\
        \hline
        $y = \mathbf{c}^\trans \mathbf{x}$ (linear function) & Yes (but not strictly convex) & \\
        \hline
        $y = \norm{\mathbf{x}}_p $ (p-norm) & Yes (for any $p \in \mathbb{N}_+$) & $\norm{\theta\mathbf{x} + (1-\theta)\mathbf{y}} \leq \theta\norm{\mathbf{x}} + (1-\theta)\norm{\mathbf{y}}$ (triangular inequality) \\
        \hline
        $f(g(\mathbf{x}))$ & Yes, if $f, g$ are convex & \\
        \hline
    \end{tabular}
    \begin{tabular}{|c|c|c|p{5cm}|}
        Function & Domain & Codomain & Comments \\
        \hline
        System of linear equation: $\mathbf{b} = f(\mathbf{x}) = \mathbf{Ax}$ & $D = \left\{ \mathbf{x} \in \mathbb{R}^{n} \mid \mathbf{Ax} = \mathbf{b} \in C, \mathbf{A} \in \mathbb{R}^{m \times n} \right\}$ & $C = \left\{ \mathbf{b} \in \mathbb{R}^{m} | \mathbf{b} = \mathbf{Ax},\: \forall\: \mathbf{x} \in D  \right\}$ & If $D$ is an affine set, so $C$ is also affine set which, in turn, is a convex set. \\ \hline
    \end{tabular}
    \begin{tabular}{|l|c|p{7cm}|}
        \hline
        \multicolumn{3}{|c|}{Sets}\\
        \hline
        Set & Convex? & Commens\\
        \hline
        \makecell{Convex hull: \\ $\textnormal{conv } C = \left\{ \sum_{i=1}^{k} \theta_i\mathbf{x}_i \mid \mathbf{x}_i \in C, 0 \le \theta_i \le 1 \textnormal{ for } i=1,\cdots, k, \mathbf{1}^\trans\boldsymbol{\theta} = 1  \right\}$} & Yes. & \begin{itemize}[leftmargin=*]
            \item $A$ will be the smallest convex set that contains $C$.
            \item $\textnormal{conv } C$ will be a finite set as long as $C$ is also finite.
        \end{itemize} \\
        \hline
        \makecell{Affine hull: \\ $\textnormal{aff } C = \left\{ \sum_{i=1}^{k} \theta_i\mathbf{x}_i \mid \mathbf{x}_i \in C \textnormal{ for } i=1,\cdots, k, \mathbf{1}^\trans\boldsymbol{\theta} = 1  \right\}$} & Yes. & \begin{itemize}[leftmargin=*]
            \item $A$ will be the smallest affine set that contains $C$.
            \item Different from the convex set, \(\theta_i\) is not restricted between 0 and 1
            \item $\textnormal{aff } C$ will always be an infinite set. If $\textnormal{aff } C$ contains the origin, it is also a subspace.
        \end{itemize}\\
        \hline
        \makecell{Conic hull: \\ $A = \left\{ \sum_{i=1}^{k} \theta_i\mathbf{x}_i \mid \mathbf{x}_i \in C, \theta_i > 0 \textnormal{ for } i=1,\cdots, k \right\}$} & Yes. & \begin{itemize}[leftmargin=*]
            \item $A$ will be the smallest convex conic that contains $C$.
            \item Different from the convex and affine sets, \(\theta_i\) does not need to sum up 1.
        \end{itemize} \\
        \hline
        \makecell{Hyperplane: \\ \( \mathcal{H} = \left\{ \mathbf{x} \mid \mathbf{a}^\trans \mathbf{x} = b \right\}\) \\ \(\mathcal{H} = \left\{ \mathbf{x} \mid \mathbf{a}^\trans (\mathbf{x} - \mathbf{x}_{0}) = \mathbf{0} \right\}\) \\ \(\mathcal{H} = \mathbf{x}_0 + a^{\perp} \) } & Yes. & \begin{itemize}[leftmargin=*]
            \item It is an infinite set \(\mathbb{R}^{n-1} \subset \mathbb{R}^{n}\) that divides the space into two halfspaces.
            \item \(a^{\perp} = \left\{ \mathbf{v} \mid \mathbf{a}^\trans \mathbf{v} = 0 \right\}\) is the set of vectors perpendicular to \(\mathbf{a}\). It passes through the origin.
            \item \(a^{\perp}\) is offset from the origin by \(\mathbf{x}_0\), which is any vector in \(\mathcal{H}\).
        \end{itemize} \\
        \hline
        \makecell{Halfspace: \\ \(\mathcal{H}_{-}\textnormal{ or }\mathcal{H}_{+}\left\{ \mathbf{x} \mid \mathbf{a}^\trans \mathbf{x} \lessgtr b \right\}\)} & Yes. & \begin{itemize}[leftmargin=*]
            \item They are infinite sets of the parts divided by \(\mathcal{H}\).
        \end{itemize}\\
        \hline
        \makecell{Euclidean ball: \\ \(B(\mathbf{x}_c, r) = \left\{ \mathbf{x} \mid \norm{\mathbf{x}-\mathbf{x}_c}_2 \leq r \right\}\) \\ \(B(\mathbf{x}_c, r) = \left\{ \mathbf{x} \mid \left( \mathbf{x}-\mathbf{x}_c \right)^\trans \left( \mathbf{x}-\mathbf{x}_c \right) \leq r \right\}\) \\ \(B(\mathbf{x}_c, r) = \left\{ \mathbf{x}_c + r \norm{\mathbf{u}} \mid \norm{\mathbf{u}} \leq 1 \right\}\)} & Yes & \begin{itemize}[leftmargin=*]
            \item \(B(\mathbf{x}_c, r)\) as long as \(r < \infty\).
            \item \(\mathbf{x}_c\) is the center of the ball.
            \item \(r\) is its radius.
        \end{itemize} \\
        \hline
        \makecell{Ellipsoid:\\\(\mathcal{E} = \left\{ \mathbf{x} \mid (\mathbf{x}-\mathbf{x}_c)^\trans\mathbf{P}^{-1}(\mathbf{x}-\mathbf{x}_c) \leq 1 \right\}\) \\ \(\mathcal{E} = \left\{ \mathbf{x}_{c} + \mathbf{Au} \mid \norm{\mathbf{u}} \leq 1 \right\}\)} & Yes & \begin{itemize}[leftmargin=*]
            \item \(\mathcal{E}\) is a finite set as long as \(\mathbf{P}\) is a finite matrix.
            \item \(\mathbf{P}\) is symmetric and positive definite, that is, \(\mathbf{P}=\mathbf{P}^\trans \succ 0\).
            \item \(\mathbf{x}_{c}\) is the center of the ellipsoid.
            \item The lengths of the semi-axes are given by \(\sqrt{\lambda_i}\).
            \item \(\mathbf{A} = \mathbf{P}^{1/2}\).
            \item \(\mathbf{A}\) is invertible. When it is not, we say that \(\mathcal{E}\) is a degenerated ellipsoid (degenerated ellipsoids are also convex).
        \end{itemize} \\
        \hline
        \makecell{Norm cone: \\ \(C = \left\{ [x_1, x_2, \cdots, x_n, t]^\trans \in \mathbb{R}^{n+1} \mid \mathbf{x} \in \mathbb{R}^{n}, \norm{\mathbf{x}}_{p} \leq t \right\} \subseteq \mathbb{R}^{n+1}\)} & Yes. & \begin{itemize}[leftmargin=*]
            \item Although it is named ``Norm cone'', it is a set, not a scalar.
            \item The cone norm increases the dimension of \(\mathbf{x}\) in 1.
            \item For \(p=2\), it is called the second-order cone, quadratic cone,  Lorentz cone or ice-cream cone.
        \end{itemize}\\
        \hline
        \makecell{Polyhedra: \\
        $\mathcal{P} = \left\{ \mathbf{x} \mid \mathbf{a}_j^\trans \mathbf{x} \leq b_j, j=1, \dots, m, \mathbf{a}_j^\trans \mathbf{x} = d_j, j=1,\cdots, p  \right\}$ \\ \(\mathcal{P} = \left\{ \mathbf{x} \mid \mathbf{Ax} \preceq \mathbf{b}, \mathbf{Cx} = \mathbf{d} \right\}\),\\where \(\mathbf{A} = \begin{bmatrix}
            \mathbf{a}_1 & \mathbf{a}_2 & \dots & \mathbf{a}_m
        \end{bmatrix}^\trans\)\\and \(\mathbf{C} = \begin{bmatrix}
            \mathbf{c}_1 & \mathbf{c}_2 & \dots & \mathbf{c}_m
        \end{bmatrix}^\trans\)} & Yes. & \begin{itemize}[leftmargin=*]
            \item Polyhedron is the intersection of \(m\) halfspaces.
            \item The polyhedron may or may not be an infinite set.
            \item Subspaces, hyperplanes, lines, rays line segments, and
            halfspaces are all polyhedra.
            \item The \emph{nonnegative orthant}, \(\mathbb{R}_{+}^{n} = \left\{ \mathbf{x} \in \mathbb{R}^n \mid x_i \leq 0 \textnormal{ for } i=1,\dots n \right\} = \left\{ \mathbf{x} \in \mathbb{R}^n \mid \mathbf{Ix}  \succeq \mathbf{0}\right\}\), is a special polyhedron
            \item \emph{Simplexes}, , are another family of polyhedra, where \(\left\{ \mathbf{v}_m \right\}_{m=0}^{k}\) is a affinely independent set. It forms a k-dimensional shape in \(\mathbb{R}^{n}\), thus being called k-dimensional simplex in \(\mathbb{R}^{n}\).
        \end{itemize} \\
        \hline
            \multicolumn{1}{|c|}{Simplex:} \\
            \tabitem \(\mathcal{S} = \textnormal{conv }\left\{ \mathbf{v}_m \right\}_{m=0}^{k} = \left\{ \sum_{i=0}^{k} \theta_i \mathbf{v}_i \mid \mathbf{0} \preceq \boldsymbol{\theta} \preceq \mathbf{1}, \mathbf{1}^\trans \boldsymbol{\theta} = 1 \right\}\) \\
            \tabitem \(\mathcal{S} = \left\{ \mathbf{x} \mid \mathbf{A}_1 \mathbf{x} \preceq \mathbf{A}_1 \mathbf{v}_0, \mathbf{A}_2 \mathbf{x} = \mathbf{A}_2 \mathbf{v}_0, \mathbf{1}^\trans \mathbf{A}_1 \mathbf{x} \leq 1 + \mathbf{1}^\trans\mathbf{A}_1 \mathbf{v}_0 \right\}\) \\
            \tabitem \(\mathcal{S} = \left\{ \mathbf{x} \mid \mathbf{A}_1 \mathbf{x} \prec \mathbf{b}, \mathbf{Cx} = \mathbf{d} \right\}\) (Polyhedra form),\\ where \(\mathbf{b} = \mathbf{A}_1 \mathbf{v}_0, \mathbf{C}=\mathbf{A}_2, \mathbf{d} = \mathbf{A}_2 \mathbf{v}_0\) & Yes. & \begin{itemize}[leftmargin=*]
            \item Also called k-dimensional Simplex in \(\mathbb{R}^{n}\).
            \end{itemize} \\
        \hline
        $C = A \cup B $ & Not always. &  \\
        \hline
        $C = A \cap B $ & Yes, if $A$ and $B$ are convex sets. & \\
        \hline
    \end{tabular}
    
\end{table}
\end{landscape}
\begin{landscape}
    \begin{itemize}
        \item All convex set is quasiconvex, but not all quasiconvex is convex.
        \item It is possible to solve quasiconvex functions, even if it is not convex (see Algorithm 4.1). But not all quasiconvex functions that are nonconvex can be solved(?).
        \item Superlevel set (a set) 3.3.6, all convex functions have all convex \(\alpha\) sub-level set, but not all functions that have convex \(\alpha\) sub-level set are convex (see slide 3.11).

    \end{itemize}
\end{landscape}
\end{document}